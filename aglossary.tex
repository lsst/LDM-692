% DO NOT EDIT - generated by lsst-texmf/texmf/../bin/generateAcronyms.py from https://lsst-texmf.lsst.io/.
\newacronym{API} {API} {Application Programming Interface}
\newacronym{ATM} {ATM} {Adaptavist Test Management}
\newglossaryentry{AURA} {name={AURA}, description={\gls{Association of Universities for Research in Astronomy}}}
\newglossaryentry{Archive} {name={Archive}, description={The repository for documents required by the NSF to be kept. These include documents related to design and development, construction, integration, test, and operations of the LSST observatory system. The archive is maintained using the enterprise content management system DocuShare, which is accessible through a link on the project website www.project.lsst.org.}}
\newglossaryentry{Association of Universities for Research in Astronomy} {name={Association of Universities for Research in Astronomy}, description={ consortium of US institutions and international affiliates that operates world-class astronomical observatories, AURA is the legal entity responsible for managing what it calls independent operating Centers, including LSST, under respective cooperative agreements with the National Science Foundation. AURA assumes fiducial responsibility for the funds provided through those cooperative agreements. AURA also is the legal owner of the AURA Observatory properties in Chile.}}
\newglossaryentry{CA-DM-CON-ICD} {name={CA-DM-CON-ICD}, description={Requirements for Interface between the Camera and Data Management (\citeds{LSE-69})}}
\newglossaryentry{CA-DM-DAQ-ICD} {name={CA-DM-DAQ-ICD}, description={Requirements for Camera Data Acquisition Interface (\citeds{LSE-68})}}
\newglossaryentry{CA-DM-SUP-ICD} {name={CA-DM-SUP-ICD}, description={Requirements for Support-Data Exchanges between Data Management and Camera (\citeds{LSE-130})}}
\newglossaryentry{CPT-OCS-INT-ICD} {name={CPT-OCS-INT-ICD}, description={Summit Computer Room Requirements (\citeds{LSE-209})}}
\newglossaryentry{Camera} {name={Camera}, description={The LSST subsystem responsible for the 3.2-gigapixel LSST camera, which will take more than 800 panoramic images of the sky every night. SLAC leads a consortium of Department of Energy laboratories to design and build the camera sensors, optics, electronics, cryostat, filters and filter exchange mechanism, and camera control system.}}
\newglossaryentry{Center} {name={Center}, description={An entity managed by AURA that is responsible for execution of a federally funded project}}
\newacronym{DAQ} {DAQ} {Data Acquisition System}
\newacronym{DM} {DM} {\gls{Data Management}}
\newglossaryentry{DM-TS-AUX-ICD} {name={DM-TS-AUX-ICD}, description={Requirements for Auxiliary Instrumentation Interface between Data Management and Telescope (\citeds{LSE-140})}}
\newglossaryentry{DM-TS-CON-ICD} {name={DM-TS-CON-ICD}, description={Requirements for Control System Interfaces between the Telescope \& Data Management (\citeds{LSE-75})}}
\newacronym{DMS} {DMS} {Data Management Subsystem}
\newglossaryentry{DMS-API-REQ} {name={DMS-API-REQ}, description={LSP API Aspect requirements (\citeds{LDM-554})}}
\newglossaryentry{DMS-LSP-REQ} {name={DMS-LSP-REQ}, description={Top level LSP requirements (\citeds{LDM-554})}}
\newglossaryentry{DMS-NB-REQ} {name={DMS-NB-REQ}, description={LSP Notebook Aspect requirements (\citeds{LDM-554})}}
\newglossaryentry{DMS-PRTL-REQ} {name={DMS-PRTL-REQ}, description={LSP Portal Aspect requirements (\citeds{LDM-554})}}
\newglossaryentry{DMS-REQ} {name={DMS-REQ}, description={Data Management top level requirements (\citeds{LSE-61})}}
\newglossaryentry{DMTR} {name={DMTR}, description={DM Test (Plan and) Report}}
\newacronym{DOE} {DOE} {\gls{Department of Energy}}
\newglossaryentry{Data Management} {name={Data Management}, description={The LSST Subsystem responsible for the Data Management System (DMS), which will capture, store, catalog, and serve the LSST dataset to the scientific community and public. The DM team is responsible for the DMS architecture, applications, middleware, infrastructure, algorithms, and Observatory Network Design. DM is a distributed team working at LSST and partner institutions, with the DM Subsystem Manager located at LSST headquarters in Tucson.}}
\newglossaryentry{Data Management Subsystem} {name={Data Management Subsystem}, description={The subsystems within Data Management may contain a defined combination of hardware, a software stack, a set of running processes, and the people who manage them: they are a major component of the DM System operations. Examples include the 'Archive Operations Subsystem' and the 'Data Processing Subsystem'"."}}
\newglossaryentry{Data Management System} {name={Data Management System}, description={The computing infrastructure, middleware, and applications that process, store, and enable information extraction from the LSST dataset; the DMS will process peta-scale data volume, convert raw images into a faithful representation of the universe, and archive the results in a useful form. The infrastructure layer consists of the computing, storage, networking hardware, and system software. The middleware layer handles distributed processing, data access, user interface, and system operations services. The applications layer includes the data pipelines and the science data archives' products and services.}}
\newglossaryentry{Department of Energy} {name={Department of Energy}, description={cabinet department of the United States federal government; the DOE has assumed technical and financial responsibility for providing the LSST camera. The DOE's responsibilities are executed by a collaboration led by SLAC National Accelerator Laboratory.}}
\newglossaryentry{DocuShare} {name={DocuShare}, description={The trade name for the enterprise management software used by LSST to archive and manage documents}}
\newglossaryentry{Document} {name={Document}, description={Any object (in any application supported by DocuShare or design archives such as PDMWorks or GIT) that supports project management or records milestones and deliverables of the LSST Project}}
\newglossaryentry{EP-DM-CON-ICD} {name={EP-DM-CON-ICD}, description={Requirements for Interface between Data Management and EPO (\citeds{LSE-131})}}
\newglossaryentry{EPO} {name={EPO}, description={Education and Public Outreach}}
\newglossaryentry{Handle} {name={Handle}, description={The unique identifier assigned to a document uploaded to DocuShare}}
\newacronym{ICD} {ICD} {Interface Control Document}
\newglossaryentry{LDM} {name={LDM}, description={LSST Data Management (Document Handle)}}
\newglossaryentry{LSE} {name={LSE}, description={LSST Systems Engineering (Document Handle)}}
\newacronym{LSP} {LSP} {LSST Science Platform}
\newacronym{LSST} {LSST} {Large Synoptic Survey Telescope}
\newglossaryentry{LVV} {name={LVV}, description={LSST Verification and Validation (Jira project)}}
\newacronym{NSF} {NSF} {\gls{National Science Foundation}}
\newglossaryentry{National Science Foundation} {name={National Science Foundation}, description={primary federal agency supporting research in all fields of fundamental science and engineering; NSF selects and funds projects through competitive, merit-based review}}
\newacronym{OCS} {OCS} {Observatory Control System}
\newglossaryentry{OCS-DM-COM-ICD} {name={OCS-DM-COM-ICD}, description={Requirements for OCS - Data Management Software Communication Interface (\citeds{LSE-72})}}
\newglossaryentry{Operations} {name={Operations}, description={The 10-year period following construction and commissioning during which the LSST Observatory conducts its survey}}
\newglossaryentry{Project Manager} {name={Project Manager}, description={The person responsible for exercising leadership and oversight over the entire LSST project; he or she controls schedule, budget, and all contingency funds}}
\newglossaryentry{Requirement} {name={Requirement}, description={A declaration of a specified function or quantitative performance that the delivered system or subsystem must meet.  It is a statement that identifies a necessary attribute, capability, characteristic, or quality of a system in order for the delivered system or subsystem to meet a derived or higher requirement, constraint, or function.}}
\newglossaryentry{SLAC} {name={SLAC}, description={No longer an acronym; formerly Stanford Linear Accelerator Center}}
\newglossaryentry{SYS-ALL-COM-ICD} {name={SYS-ALL-COM-ICD}, description={Requirements for LSST Observatory Control System Communication Architecture and Protocol (\citeds{LSE-70})}}
\newglossaryentry{Science Platform} {name={Science Platform}, description={A set of integrated web applications and services deployed at the LSST Data Access Centers (DACs) through which the scientific community will access, visualize, and perform next-to-the-data analysis of the LSST data products.}}
\newglossaryentry{Specification} {name={Specification}, description={One or more performance parameter(s) being established by a requirement that the delivered system or subsystem must meet}}
\newglossaryentry{Subsystem} {name={Subsystem}, description={A set of elements comprising a system within the larger LSST system that is responsible for a key technical deliverable of the project.}}
\newglossaryentry{Subsystem Manager} {name={Subsystem Manager}, description={responsible manager for an LSST subsystem; he or she exercises authority, within prescribed limits and under scrutiny of the Project Manager, over the relevant subsystem's cost, schedule, and work plans}}
\newglossaryentry{Summit} {name={Summit}, description={The site on the Cerro Pachón, Chile mountaintop where the LSST observatory, support facilities, and infrastructure will be built.}}
\newglossaryentry{Systems Engineering} {name={Systems Engineering}, description={an interdisciplinary field of engineering that focuses on how to design and manage complex engineering systems over their life cycles. Issues such as requirements engineering, reliability, logistics, coordination of different teams, testing and evaluation, maintainability and many other disciplines necessary for successful system development, design, implementation, and ultimate decommission become more difficult when dealing with large or complex projects. Systems engineering deals with work-processes, optimization methods, and risk management tools in such projects. It overlaps technical and human-centered disciplines such as industrial engineering, control engineering, software engineering, organizational studies, and project management. Systems engineering ensures that all likely aspects of a project or system are considered, and integrated into a whole.}}
\newacronym{TS} {TS} {Test Specification}
\newacronym{US} {US} {United States}
\newacronym{VCD} {VCD} {Verification Control Document}
\newglossaryentry{Validation} {name={Validation}, description={A process of confirming that the delivered system will provide its desired functionality; overall, a validation process includes the evaluation, integration, and test activities carried out at the system level to ensure that the final developed system satisfies the intent and performance of that system in operations}}
\newglossaryentry{Verification} {name={Verification}, description={The process of evaluating the design, including hardware and software - to ensure the requirements have been met;  verification (of requirements) is performed by test, analysis, inspection, and/or demonstration}}
\newglossaryentry{camera} {name={camera}, description={An imaging device mounted at a telescope focal plane, composed of optics, a shutter, a set of filters, and one or more sensors arranged in a focal plane array.}}
\newglossaryentry{stack} {name={stack}, description={A record of all versions of a document uploaded to a particular DocuShare handle}}
