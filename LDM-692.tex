\documentclass[DM,lsstdraft,STS,toc]{lsstdoc}
\usepackage{enumitem}
\usepackage{booktabs}
\usepackage{arydshln}
\usepackage{afterpage}
\usepackage{pdflscape}
\usepackage{graphicx}
\colorlet{dmyellow}{yellow!50!white}
\colorlet{dmblue}{cyan!50!white}
\colorlet{dmorange}{orange!50!white}
\colorlet{dmgreen}{green!50!white}
\colorlet{dmred}{red!50!white}
\colorlet{dmgray}{gray!50!white}
\colorlet{dmpink}{pink!50!white}


\input meta.tex

%% DO NOT EDIT - generated by lsst-texmf/texmf/../bin/generateAcronyms.py from https://lsst-texmf.lsst.io/.
\newacronym{API} {API} {Application Programming Interface}
\newglossaryentry{AURA} {name={AURA}, description={\gls{Association of Universities for Research in Astronomy}}}
\newglossaryentry{Archive} {name={Archive}, description={The repository for documents required by the NSF to be kept. These include documents related to design and development, construction, integration, test, and operations of the LSST observatory system. The archive is maintained using the enterprise content management system DocuShare, which is accessible through a link on the project website www.project.lsst.org.}}
\newglossaryentry{Association of Universities for Research in Astronomy} {name={Association of Universities for Research in Astronomy}, description={ consortium of US institutions and international affiliates that operates world-class astronomical observatories, AURA is the legal entity responsible for managing what it calls independent operating Centers, including LSST, under respective cooperative agreements with the National Science Foundation. AURA assumes fiducial responsibility for the funds provided through those cooperative agreements. AURA also is the legal owner of the AURA Observatory properties in Chile.}}
\newglossaryentry{CA-DM-CON-ICD} {name={CA-DM-CON-ICD}, description={Requirements for Interface between the Camera and Data Management (\citeds{LSE-69})}}
\newglossaryentry{CA-DM-DAQ-ICD} {name={CA-DM-DAQ-ICD}, description={Requirements for Camera Data Acquisition Interface (\citeds{LSE-68})}}
\newglossaryentry{CA-DM-SUP-ICD} {name={CA-DM-SUP-ICD}, description={Requirements for Support-Data Exchanges between Data Management and Camera (\citeds{LSE-130})}}
\newglossaryentry{CPT-OCS-INT-ICD} {name={CPT-OCS-INT-ICD}, description={Summit Computer Room Requirements (\citeds{LSE-209})}}
\newglossaryentry{Camera} {name={Camera}, description={The LSST subsystem responsible for the 3.2-gigapixel LSST camera, which will take more than 800 panoramic images of the sky every night. SLAC leads a consortium of Department of Energy laboratories to design and build the camera sensors, optics, electronics, cryostat, filters and filter exchange mechanism, and camera control system.}}
\newglossaryentry{Center} {name={Center}, description={An entity managed by AURA that is responsible for execution of a federally funded project}}
\newacronym{DAQ} {DAQ} {Data Acquisition System}
\newacronym{DM} {DM} {\gls{Data Management}}
\newglossaryentry{DM-TS-AUX-ICD} {name={DM-TS-AUX-ICD}, description={Requirements for Auxiliary Instrumentation Interface between Data Management and Telescope (\citeds{LSE-140})}}
\newglossaryentry{DM-TS-CON-ICD} {name={DM-TS-CON-ICD}, description={Requirements for Control System Interfaces between the Telescope \& Data Management (\citeds{LSE-75})}}
\newacronym{DMS} {DMS} {Data Management Subsystem}
\newglossaryentry{DMS-API-REQ} {name={DMS-API-REQ}, description={LSP API Aspect requirements (\citeds{LDM-554})}}
\newglossaryentry{DMS-LSP-REQ} {name={DMS-LSP-REQ}, description={Top level LSP requirements (\citeds{LDM-554})}}
\newglossaryentry{DMS-NB-REQ} {name={DMS-NB-REQ}, description={LSP Notebook Aspect requirements (\citeds{LDM-554})}}
\newglossaryentry{DMS-PRTL-REQ} {name={DMS-PRTL-REQ}, description={LSP Portal Aspect requirements (\citeds{LDM-554})}}
\newglossaryentry{DMS-REQ} {name={DMS-REQ}, description={Data Management top level requirements (\citeds{LSE-61})}}
\newglossaryentry{DMTR} {name={DMTR}, description={DM Test (Plan and) Report}}
\newacronym{DOE} {DOE} {\gls{Department of Energy}}
\newglossaryentry{Data Management} {name={Data Management}, description={The LSST Subsystem responsible for the Data Management System (DMS), which will capture, store, catalog, and serve the LSST dataset to the scientific community and public. The DM team is responsible for the DMS architecture, applications, middleware, infrastructure, algorithms, and Observatory Network Design. DM is a distributed team working at LSST and partner institutions, with the DM Subsystem Manager located at LSST headquarters in Tucson.}}
\newglossaryentry{Data Management Subsystem} {name={Data Management Subsystem}, description={The subsystems within Data Management may contain a defined combination of hardware, a software stack, a set of running processes, and the people who manage them: they are a major component of the DM System operations. Examples include the 'Archive Operations Subsystem' and the 'Data Processing Subsystem'"."}}
\newglossaryentry{Data Management System} {name={Data Management System}, description={The computing infrastructure, middleware, and applications that process, store, and enable information extraction from the LSST dataset; the DMS will process peta-scale data volume, convert raw images into a faithful representation of the universe, and archive the results in a useful form. The infrastructure layer consists of the computing, storage, networking hardware, and system software. The middleware layer handles distributed processing, data access, user interface, and system operations services. The applications layer includes the data pipelines and the science data archives' products and services.}}
\newglossaryentry{Department of Energy} {name={Department of Energy}, description={cabinet department of the United States federal government; the DOE has assumed technical and financial responsibility for providing the LSST camera. The DOE's responsibilities are executed by a collaboration led by SLAC National Accelerator Laboratory.}}
\newglossaryentry{DocuShare} {name={DocuShare}, description={The trade name for the enterprise management software used by LSST to archive and manage documents}}
\newglossaryentry{Document} {name={Document}, description={Any object (in any application supported by DocuShare or design archives such as PDMWorks or GIT) that supports project management or records milestones and deliverables of the LSST Project}}
\newglossaryentry{EP-DM-CON-ICD} {name={EP-DM-CON-ICD}, description={Requirements for Interface between Data Management and EPO (\citeds{LSE-131})}}
\newglossaryentry{EPO} {name={EPO}, description={Education and Public Outreach}}
\newglossaryentry{Handle} {name={Handle}, description={The unique identifier assigned to a document uploaded to DocuShare}}
\newacronym{ICD} {ICD} {Interface Control Document}
\newglossaryentry{LDM} {name={LDM}, description={LSST Data Management (Document Handle)}}
\newglossaryentry{LSE} {name={LSE}, description={LSST Systems Engineering (Document Handle)}}
\newacronym{LSP} {LSP} {LSST Science Platform}
\newacronym{LSST} {LSST} {Large Synoptic Survey Telescope}
\newglossaryentry{LVV} {name={LVV}, description={LSST Verification and Validation (Jira project)}}
\newacronym{NSF} {NSF} {\gls{National Science Foundation}}
\newglossaryentry{National Science Foundation} {name={National Science Foundation}, description={primary federal agency supporting research in all fields of fundamental science and engineering; NSF selects and funds projects through competitive, merit-based review}}
\newacronym{OCS} {OCS} {Observatory Control System}
\newglossaryentry{OCS-DM-COM-ICD} {name={OCS-DM-COM-ICD}, description={Requirements for OCS - Data Management Software Communication Interface (\citeds{LSE-72})}}
\newglossaryentry{Operations} {name={Operations}, description={The 10-year period following construction and commissioning during which the LSST Observatory conducts its survey}}
\newglossaryentry{Project Manager} {name={Project Manager}, description={The person responsible for exercising leadership and oversight over the entire LSST project; he or she controls schedule, budget, and all contingency funds}}
\newglossaryentry{Requirement} {name={Requirement}, description={A declaration of a specified function or quantitative performance that the delivered system or subsystem must meet.  It is a statement that identifies a necessary attribute, capability, characteristic, or quality of a system in order for the delivered system or subsystem to meet a derived or higher requirement, constraint, or function.}}
\newglossaryentry{SLAC} {name={SLAC}, description={No longer an acronym; formerly Stanford Linear Accelerator Center}}
\newglossaryentry{SYS-ALL-COM-ICD} {name={SYS-ALL-COM-ICD}, description={Requirements for LSST Observatory Control System Communication Architecture and Protocol (\citeds{LSE-70})}}
\newglossaryentry{Science Platform} {name={Science Platform}, description={A set of integrated web applications and services deployed at the LSST Data Access Centers (DACs) through which the scientific community will access, visualize, and perform next-to-the-data analysis of the LSST data products.}}
\newglossaryentry{Specification} {name={Specification}, description={One or more performance parameter(s) being established by a requirement that the delivered system or subsystem must meet}}
\newglossaryentry{Subsystem} {name={Subsystem}, description={A set of elements comprising a system within the larger LSST system that is responsible for a key technical deliverable of the project.}}
\newglossaryentry{Subsystem Manager} {name={Subsystem Manager}, description={responsible manager for an LSST subsystem; he or she exercises authority, within prescribed limits and under scrutiny of the Project Manager, over the relevant subsystem's cost, schedule, and work plans}}
\newglossaryentry{Summit} {name={Summit}, description={The site on the Cerro Pachón, Chile mountaintop where the LSST observatory, support facilities, and infrastructure will be built.}}
\newglossaryentry{Systems Engineering} {name={Systems Engineering}, description={an interdisciplinary field of engineering that focuses on how to design and manage complex engineering systems over their life cycles. Issues such as requirements engineering, reliability, logistics, coordination of different teams, testing and evaluation, maintainability and many other disciplines necessary for successful system development, design, implementation, and ultimate decommission become more difficult when dealing with large or complex projects. Systems engineering deals with work-processes, optimization methods, and risk management tools in such projects. It overlaps technical and human-centered disciplines such as industrial engineering, control engineering, software engineering, organizational studies, and project management. Systems engineering ensures that all likely aspects of a project or system are considered, and integrated into a whole.}}
\newacronym{TS} {TS} {Test Specification}
\newacronym{US} {US} {United States}
\newacronym{VCD} {VCD} {Verification Control Document}
\newglossaryentry{Validation} {name={Validation}, description={A process of confirming that the delivered system will provide its desired functionality; overall, a validation process includes the evaluation, integration, and test activities carried out at the system level to ensure that the final developed system satisfies the intent and performance of that system in operations}}
\newglossaryentry{Verification} {name={Verification}, description={The process of evaluating the design, including hardware and software - to ensure the requirements have been met;  verification (of requirements) is performed by test, analysis, inspection, and/or demonstration}}
\newglossaryentry{camera} {name={camera}, description={An imaging device mounted at a telescope focal plane, composed of optics, a shutter, a set of filters, and one or more sensors arranged in a focal plane array.}}
\newglossaryentry{stack} {name={stack}, description={A record of all versions of a document uploaded to a particular DocuShare handle}}

%\makeglossaries

\begin{document}

\providecommand{\tightlist}{%
  \setlength{\itemsep}{0pt}\setlength{\parskip}{0pt}}

\def\product{LSST Data Management}

\setDocCompact{true}

\title[VCD \product]{DM Verification Control Document}

\author{Gabriele Comoretto}
\setDocRef{\lsstDocType-\lsstDocNum}
\setDocDate{\vcsdate}

\setDocAbstract {
The DM Verification Control Document is an extraction from the LSST wide system engineering verification database (Jira)
which gives an overview of DMS verification state with respect to the DM Requirements.
}

% Most recent last
\setDocChangeRecord{%
	\addtohist{}{2019-03-28}{First draft}{G.~Comoretto}
}

\setDocCurator{Gabriele Comoretto}
\setDocUpstreamLocation{\url{https://github.com/lsst/ldm-692}}
\setDocUpstreamVersion{\vcsrevision}

\maketitle


% Status
\newcommand{\notexec}{\cellcolor{dmgray} \textbf{Not Executed}}
\newcommand{\inprog}{\cellcolor{dmorange} \textbf{In Progress}}
\newcommand{\passed}{\cellcolor{dmgreen} \textbf{Passed}}
\newcommand{\cndpass}{\cellcolor{dmpink} \textbf{Conditionally Pass}}
\newcommand{\failed}{\cellcolor{dmred} \textbf{Failed}}
\newcommand{\blocked}{\cellcolor{dmblue} \textbf{Blocked}}

% References
\newcommand{\vcdJiraRef}[1]{\scriptsize{\jira{#1}}}
\newcommand{\vcdDocRef}[1]{\scriptsize{\citeds{#1}}}



\section{Introduction}
\label{sec:intro}

The Verification Control Document (VCD) is a detailed look at where we are in verifying our requirements \citeds{LSE-61}.
We verify requirements by
running tests, the results of which are recorded in Jira.  A description of how the verification testing is set up is given in Section 2 of \citeds{LDM-503). Specifically section 2.3.2 of \citeds{LDM-503} describes the verification elements, test cases which appear in \secref{sec:vcd}.

In general the summary provided in \secref{sec:summary} will give the reader the overview required, the remainder of the document is a large matrix of all the DM requirements , their verification elements, tests and if those tests have run.

The matrix is fully linked - any give Test Case mentioned in the matrix such as \jira{LVV-T362} provide a link to the Jira element defining that test case. Similarly relevant documentation such as test reports like \citeds{DMTR-111} are linked to the bibliography which provides a link to the document.  If you are browsing with Acrobat it is extremely useful to know that if you follow a hyper link in the document the   $\leftarrow$\footnote{I think this is ctrl $\leftarrow$ on windows/Linux} takes you back to where you clicked.

Hence this is not a document you want to read but provides a good view on the Jira database in which all of the same information may be found (apart from the summaries).




\subsection{Applicable Documents}
\label{sec:docs}

\addtocounter{table}{-1}

\begin{tabular}[htb]{l l}
\citeds{LSE-61}  & LSST DM Subsystem Requirements \\
\citeds{LDM-503} & LSST DM Test Plan \\
\citeds{LDM-639} & DM Acceptance Test Specification \\
\end{tabular}


\newpage
% generated from JIRA project LVV
% using template at <template>.
% Collecting data for component: "DM"
% using docsteady version 1.2rc12
% Please do not edit -- update information in Jira instead
%
% This file is meant to be included in LaTeX document in order to provide:
%   - section 3: Summary Information
%   - section 4: VCD
%   - appendix A: Summary Explanations

\section{Summary Information}\label{sec:summary}

Table \ref{table:summary} provides an overview of the requirements and verification elements coverage.

% Summary of Summaries
\begin{longtable}{rp{2cm}p{1cm}p{1cm}p{1cm}p{1cm}p{1cm}p{1cm}}
 & \rotatebox[origin=l]{60}{ \textbf{Priority}  }
 & \rotatebox[origin=l]{60}{ \textbf{ Fully Verified } {\scriptsize \ref{sec:fullyverified} } }
 & \rotatebox[origin=l]{60}{ \textbf{ Partially Verified } {\scriptsize \ref{sec:partiallyverified} } }
 & \rotatebox[origin=l]{60}{ \textbf{ With Failures } {\scriptsize \ref{sec:withfaulres} } }
 & \rotatebox[origin=l]{60}{ \textbf{ Not Verified } {\scriptsize \ref{sec:notverified} } }
 & \rotatebox[origin=l]{60}{ \textbf{ Not Covered } {\scriptsize \ref{sec:notcovered} } }
 & \rotatebox[origin=l]{60}{ \textbf{ Total } }
\\ \toprule
\textbf{ DM Requirements} & (All)
 & 11
 & 104
 & 3
 & 332
 & 260
 & \textbf{ 710 }
\\ \toprule
\citeds{LSE-61} & 1a  & 2  & 25  & 1  & 27  &   & \textbf{ 55 }
 \\ \cdashline{2-8}
 & 1b  & 7  & 23  & 2  & 75  & 4  & \textbf{ 111 }
 \\ \cdashline{2-8}
 & 2  &   & 4  &   & 42  & 6  & \textbf{ 52 }
 \\ \cdashline{2-8}
 & 3  &   & 1  &   & 5  &   & \textbf{ 6 }
 \\ \cdashline{2-8}
 & (All)  & 9  & 53  & 3  & 149  & 10  & \textbf{ 224 }
 \\ \hline
\citeds{LSE-68} & Not Set  &   &   &   &   & 19  & \textbf{ 19 }
 \\ \hline
\citeds{LSE-69} & Not Set  &   &   &   &   & 16  & \textbf{ 16 }
 \\ \hline
\citeds{LSE-72} & Not Set  &   &   &   &   & 52  & \textbf{ 52 }
 \\ \hline
\citeds{LSE-75} & Not Set  &   &   &   &   & 9  & \textbf{ 9 }
 \\ \hline
\citeds{LSE-130} & Not Set  &   &   &   &   & 31  & \textbf{ 31 }
 \\ \hline
\citeds{LSE-131} & Not Set  &   &   &   &   & 13  & \textbf{ 13 }
 \\ \hline
\citeds{LSE-140} & Not Set  &   &   &   &   & 23  & \textbf{ 23 }
 \\ \hline
\citeds{LSE-70} & Not Set  &   &   &   &   & 26  & \textbf{ 26 }
 \\ \hline
\citeds{LSE-209} & Not Set  &   &   &   &   & 46  & \textbf{ 46 }
 \\ \hline
\citeds{LDM-554} & Not Set  & 2  & 51  &   & 183  &   & \textbf{ 236 }
 \\ \hline
\citeds{LSE-400} & Not Set  &   &   &   &   & 15  & \textbf{ 15 }
 \\ \hline
\textbf{ DM Verification E.} & (All)
 & 17
 & 110
 & 4
 & 361
 & 511
 & \textbf{ 1003 }
\\ \hline
\caption{Summary overview of all DM~requirements and verification elements.}
\label{table:summary}
\end{longtable}



Table \ref{table:testsummary} provides the Test Cases result summary.

\begin{longtable}{rp{1cm}p{1cm}p{1cm}p{1cm}p{1cm}}
& \rotatebox[origin=l]{60}{ \textbf{ Passed }{\scriptsize \ref{sec:pass} } }
& \rotatebox[origin=l]{60}{ \textbf{ P. w/Dev. }{\scriptsize \ref{sec:condpass} } }
& \rotatebox[origin=l]{60}{ \textbf{ Failed }{\scriptsize \ref{sec:fail} } }
& \rotatebox[origin=l]{60}{ \textbf{ Not Ex. }{\scriptsize \ref{sec:notexec} } }
& \rotatebox[origin=l]{60}{ \textbf{Total}}
\\ \toprule
Test Cases Results
& 47
& 14
& 2
& 467
 & \textbf{ 530 } \\
\bottomrule
\caption{Summary overview of DM~test cases executions.}
\label{table:testsummary}
\end{longtable}

Note that test cases may be associated with multiple requirements or verification elements,
and requirements or verification elements may be associated with multiple test cases.


\subsection{Coverage Description}

\subsubsection{Fully Verified}\label{sec:fullyverified}

These are the Verification Elements and Requirements for which all of the associated Test Cases have been
successfully executed. The last execution status for all test cases shall be
\textbf{Passed} or \textbf{Passed w/Deviation}.


\subsubsection{Partially Verified}\label{sec:partiallyverified}

These are the Verification Elements and Requirements for which at least one of the associated Test Cases have been
successfully executed. The last execution status of the test cases shall be
\textbf{Not Executed}, \textbf{Passed} or \textbf{Passed w/Deviation}.


\subsubsection{With Failures}\label{sec:withfaulres}

These are the Verification Elements and Requirements for which at least one of the associated Test Cases
has a \textbf{Failure} as execution result.


\subsubsection{Not Verified}\label{sec:notverified}

These are the Verification Elements and Requirements for which none of the associated Test Cases have been executed.


\subsubsection{Not Covered}\label{sec:notcovered}

These are the Verification Elements and Requirements for which there are not any associated Test Cases.


\subsection{Test Executions Description}


\subsubsection{Passed}\label{sec:pass}

Test cases that have been executed without any problems.
Issues may have been found during the execution, and linked to the test case, but they do not affect
the verification process.


\subsubsection{Passed with Deviation}\label{sec:condpass}

Test cases for which the execution can be considered successful, but a deviation to the requirement is needed.
The deviation shall be recorded in a Jira issue, type \textit{Deviation} and linked to the test.


\subsubsection{Failed}\label{sec:fail}

The test case execution failed. One or more Jira issues shall be filed and related with the test.


\subsubsection{Not Executed}\label{sec:notexec}

The test case has not been executed yet.


\newpage
\section{Verification Control} \label{sec:vcd}

In the following subsections, a detailed overview of the requirements coverage for each specification is provided.

Where available, the priority is reported in round brackets, under the requirement or verification element identifier.
For example $(p. 1a)$ denotes a 1a priority requirement.
Requirements and verification elements for which no priority (p.) is specified inherit the priority for
the higher-level requirement from which they are derived.
For a full description of the prioritization levels, see \cite{DMSR}.


\subsection{LSE-61 Requirements Coverage}

\setlength\LTleft{-0.25in}
\setlength\LTright{-0.5in}
{\small
\begin{longtable}{lllll}
\caption{ DM LSE-61 Requirements.} \\
\toprule
\textbf{Requirement} & \textbf{Verification Element} & \textbf{Test Case} & \textbf{Last Run} & \textbf{Test Status} \\
\toprule
\endhead
  \begin{tabular}{@{}l@{}}
  DMS-REQ-0002\\\vcdDocRef{LSE-61}~{\tiny
 (p. 1b)   }
  \end{tabular} &
    \begin{tabular}{@{}l@{}}
    \hypertarget{dms-req-0002-v-01}{DMS-REQ-0002-V-01}
    \\\vcdJiraRef{LVV-3}~{\tiny
 (p. 1b)     }
    \end{tabular} &
        \begin{tabular}{@{}l@{}}
        \href{https://jira.lsstcorp.org/secure/Tests.jspa\#/testCase/LVV-T101}{LVV-T101} \\
        \vcdDocRef{LDM-639}
        \end{tabular} &
          & \notexec{} \\
          \cmidrule{3-5}
          & &
        \begin{tabular}{@{}l@{}}
        \href{https://jira.lsstcorp.org/secure/Tests.jspa\#/testCase/LVV-T217}{LVV-T217} \\
        \vcdDocRef{LDM-533}
        \end{tabular} &
          \begin{tabular}{@{}l@{}}
          2018-07-04 \\
            \vcdDocRef{DMTR-91}
            {\scriptsize \href{https://jira.lsstcorp.org/secure/Tests.jspa\#/testPlan/LVV-P1}{LVV-P1} }
          \end{tabular} &
          \cndpass \\
  \midrule
  \begin{tabular}{@{}l@{}}
  DMS-REQ-0008\\\vcdDocRef{LSE-61}~{\tiny
 (p. 1b)   }
  \end{tabular} &
    \begin{tabular}{@{}l@{}}
    \hypertarget{dms-req-0008-v-01}{DMS-REQ-0008-V-01}
    \\\vcdJiraRef{LVV-5}~{\tiny
 (p. 1b)     }
    \end{tabular} &
        \begin{tabular}{@{}l@{}}
        \href{https://jira.lsstcorp.org/secure/Tests.jspa\#/testCase/LVV-T171}{LVV-T171} \\
        \vcdDocRef{LDM-639}
        \end{tabular} &
          & \notexec{} \\
          \cmidrule{3-5}
          & &
        \begin{tabular}{@{}l@{}}
        \href{https://jira.lsstcorp.org/secure/Tests.jspa\#/testCase/LVV-T287}{LVV-T287} \\
        \vcdDocRef{LDM-538}
        \end{tabular} &
          & \notexec{} \\
  \midrule
  \begin{tabular}{@{}l@{}}
  DMS-REQ-0009\\\vcdDocRef{LSE-61}~{\tiny
 (p. 1b)   }
  \end{tabular} &
    \begin{tabular}{@{}l@{}}
    \hypertarget{dms-req-0009-v-01}{DMS-REQ-0009-V-01}
    \\\vcdJiraRef{LVV-6}~{\tiny
 (p. 1b)     }
    \end{tabular} &
        \begin{tabular}{@{}l@{}}
        \href{https://jira.lsstcorp.org/secure/Tests.jspa\#/testCase/LVV-T125}{LVV-T125} \\
        \vcdDocRef{LDM-639}
        \end{tabular} &
          & \notexec{} \\
  \midrule
  \begin{tabular}{@{}l@{}}
  DMS-REQ-0010\\\vcdDocRef{LSE-61}~{\tiny
 (p. 1b)   }
  \end{tabular} &
    \begin{tabular}{@{}l@{}}
    \hypertarget{dms-req-0010-v-01}{DMS-REQ-0010-V-01}
    \\\vcdJiraRef{LVV-7}~{\tiny
 (p. 1b)     }
    \end{tabular} &
        \begin{tabular}{@{}l@{}}
        \href{https://jira.lsstcorp.org/secure/Tests.jspa\#/testCase/LVV-T18}{LVV-T18} \\
        \vcdDocRef{LDM-533}
        \end{tabular} &
          \begin{tabular}{@{}l@{}}
          2019-05-22 \\
            \vcdDocRef{DMTR-53}
            {\scriptsize \href{https://jira.lsstcorp.org/secure/Tests.jspa\#/testPlan/LVV-P44}{LVV-P44} }
          \end{tabular} &
          \passed \\
          \cmidrule{3-5}
          & &
        \begin{tabular}{@{}l@{}}
        \href{https://jira.lsstcorp.org/secure/Tests.jspa\#/testCase/LVV-T20}{LVV-T20} \\
        \vcdDocRef{LDM-533}
        \end{tabular} &
          \begin{tabular}{@{}l@{}}
          2019-05-22 \\
            \vcdDocRef{DMTR-53}
            {\scriptsize \href{https://jira.lsstcorp.org/secure/Tests.jspa\#/testPlan/LVV-P44}{LVV-P44} }
          \end{tabular} &
          \failed \\
          \cmidrule{3-5}
          & &
        \begin{tabular}{@{}l@{}}
        \href{https://jira.lsstcorp.org/secure/Tests.jspa\#/testCase/LVV-T36}{LVV-T36} \\
        \vcdDocRef{LDM-639}
        \end{tabular} &
          & \notexec{} \\
  \midrule
  \begin{tabular}{@{}l@{}}
  DMS-REQ-0018\\\vcdDocRef{LSE-61}~{\tiny
 (p. 1a)   }
  \end{tabular} &
    \begin{tabular}{@{}l@{}}
    \hypertarget{dms-req-0018-v-01}{DMS-REQ-0018-V-01}
    \\\vcdJiraRef{LVV-8}~{\tiny
 (p. 1a)     }
    \end{tabular} &
        \begin{tabular}{@{}l@{}}
        \href{https://jira.lsstcorp.org/secure/Tests.jspa\#/testCase/LVV-T29}{LVV-T29} \\
        \vcdDocRef{LDM-639}
        \end{tabular} &
          & \notexec{} \\
          \cmidrule{3-5}
          & &
        \begin{tabular}{@{}l@{}}
        \href{https://jira.lsstcorp.org/secure/Tests.jspa\#/testCase/LVV-T283}{LVV-T283} \\
        \vcdDocRef{LDM-538}
        \end{tabular} &
          \begin{tabular}{@{}l@{}}
          2019-05-22 \\
            \vcdDocRef{DMTR-61}
            {\scriptsize \href{https://jira.lsstcorp.org/secure/Tests.jspa\#/testPlan/LVV-P45}{LVV-P45} }
          \end{tabular} &
          \passed \\
          \cmidrule{3-5}
          & &
        \begin{tabular}{@{}l@{}}
        \href{https://jira.lsstcorp.org/secure/Tests.jspa\#/testCase/LVV-T284}{LVV-T284} \\
        \vcdDocRef{LDM-538}
        \end{tabular} &
          \begin{tabular}{@{}l@{}}
          2019-06-24 \\
            \vcdDocRef{DMTR-102}
            {\scriptsize \href{https://jira.lsstcorp.org/secure/Tests.jspa\#/testPlan/LVV-P10}{LVV-P10} }
          \end{tabular} &
          \passed \\
          \cmidrule{3-5}
          & &
        \begin{tabular}{@{}l@{}}
        \href{https://jira.lsstcorp.org/secure/Tests.jspa\#/testCase/LVV-T1549}{LVV-T1549} \\
        \vcdDocRef{}
        \end{tabular} &
          & \notexec{} \\
          \cmidrule{3-5}
          & &
        \begin{tabular}{@{}l@{}}
        \href{https://jira.lsstcorp.org/secure/Tests.jspa\#/testCase/LVV-T1550}{LVV-T1550} \\
        \vcdDocRef{}
        \end{tabular} &
          & \notexec{} \\
          \cmidrule{3-5}
          & &
        \begin{tabular}{@{}l@{}}
        \href{https://jira.lsstcorp.org/secure/Tests.jspa\#/testCase/LVV-T1556}{LVV-T1556} \\
        \vcdDocRef{}
        \end{tabular} &
          & \notexec{} \\
  \midrule
  \begin{tabular}{@{}l@{}}
  DMS-REQ-0020\\\vcdDocRef{LSE-61}~{\tiny
 (p. 1a)   }
  \end{tabular} &
    \begin{tabular}{@{}l@{}}
    \hypertarget{dms-req-0020-v-01}{DMS-REQ-0020-V-01}
    \\\vcdJiraRef{LVV-9}~{\tiny
 (p. 1a)     }
    \end{tabular} &
        \begin{tabular}{@{}l@{}}
        \href{https://jira.lsstcorp.org/secure/Tests.jspa\#/testCase/LVV-T30}{LVV-T30} \\
        \vcdDocRef{LDM-639}
        \end{tabular} &
          & \notexec{} \\
          \cmidrule{3-5}
          & &
        \begin{tabular}{@{}l@{}}
        \href{https://jira.lsstcorp.org/secure/Tests.jspa\#/testCase/LVV-T283}{LVV-T283} \\
        \vcdDocRef{LDM-538}
        \end{tabular} &
          \begin{tabular}{@{}l@{}}
          2019-05-22 \\
            \vcdDocRef{DMTR-61}
            {\scriptsize \href{https://jira.lsstcorp.org/secure/Tests.jspa\#/testPlan/LVV-P45}{LVV-P45} }
          \end{tabular} &
          \passed \\
          \cmidrule{3-5}
          & &
        \begin{tabular}{@{}l@{}}
        \href{https://jira.lsstcorp.org/secure/Tests.jspa\#/testCase/LVV-T284}{LVV-T284} \\
        \vcdDocRef{LDM-538}
        \end{tabular} &
          \begin{tabular}{@{}l@{}}
          2019-06-24 \\
            \vcdDocRef{DMTR-102}
            {\scriptsize \href{https://jira.lsstcorp.org/secure/Tests.jspa\#/testPlan/LVV-P10}{LVV-P10} }
          \end{tabular} &
          \passed \\
          \cmidrule{3-5}
          & &
        \begin{tabular}{@{}l@{}}
        \href{https://jira.lsstcorp.org/secure/Tests.jspa\#/testCase/LVV-T1549}{LVV-T1549} \\
        \vcdDocRef{}
        \end{tabular} &
          & \notexec{} \\
          \cmidrule{3-5}
          & &
        \begin{tabular}{@{}l@{}}
        \href{https://jira.lsstcorp.org/secure/Tests.jspa\#/testCase/LVV-T1556}{LVV-T1556} \\
        \vcdDocRef{}
        \end{tabular} &
          & \notexec{} \\
  \midrule
  \begin{tabular}{@{}l@{}}
  DMS-REQ-0024\\\vcdDocRef{LSE-61}~{\tiny
 (p. 1a)   }
  \end{tabular} &
    \begin{tabular}{@{}l@{}}
    \hypertarget{dms-req-0024-v-01}{DMS-REQ-0024-V-01}
    \\\vcdJiraRef{LVV-11}~{\tiny
 (p. 1a)     }
    \end{tabular} &
        \begin{tabular}{@{}l@{}}
        \href{https://jira.lsstcorp.org/secure/Tests.jspa\#/testCase/LVV-T32}{LVV-T32} \\
        \vcdDocRef{LDM-639}
        \end{tabular} &
          & \notexec{} \\
          \cmidrule{3-5}
          & &
        \begin{tabular}{@{}l@{}}
        \href{https://jira.lsstcorp.org/secure/Tests.jspa\#/testCase/LVV-T283}{LVV-T283} \\
        \vcdDocRef{LDM-538}
        \end{tabular} &
          \begin{tabular}{@{}l@{}}
          2019-05-22 \\
            \vcdDocRef{DMTR-61}
            {\scriptsize \href{https://jira.lsstcorp.org/secure/Tests.jspa\#/testPlan/LVV-P45}{LVV-P45} }
          \end{tabular} &
          \passed \\
          \cmidrule{3-5}
          & &
        \begin{tabular}{@{}l@{}}
        \href{https://jira.lsstcorp.org/secure/Tests.jspa\#/testCase/LVV-T284}{LVV-T284} \\
        \vcdDocRef{LDM-538}
        \end{tabular} &
          \begin{tabular}{@{}l@{}}
          2019-06-24 \\
            \vcdDocRef{DMTR-102}
            {\scriptsize \href{https://jira.lsstcorp.org/secure/Tests.jspa\#/testPlan/LVV-P10}{LVV-P10} }
          \end{tabular} &
          \passed \\
          \cmidrule{3-5}
          & &
        \begin{tabular}{@{}l@{}}
        \href{https://jira.lsstcorp.org/secure/Tests.jspa\#/testCase/LVV-T1549}{LVV-T1549} \\
        \vcdDocRef{}
        \end{tabular} &
          & \notexec{} \\
          \cmidrule{3-5}
          & &
        \begin{tabular}{@{}l@{}}
        \href{https://jira.lsstcorp.org/secure/Tests.jspa\#/testCase/LVV-T1550}{LVV-T1550} \\
        \vcdDocRef{}
        \end{tabular} &
          & \notexec{} \\
          \cmidrule{3-5}
          & &
        \begin{tabular}{@{}l@{}}
        \href{https://jira.lsstcorp.org/secure/Tests.jspa\#/testCase/LVV-T1556}{LVV-T1556} \\
        \vcdDocRef{}
        \end{tabular} &
          & \notexec{} \\
  \midrule
  \begin{tabular}{@{}l@{}}
  DMS-REQ-0029\\\vcdDocRef{LSE-61}~{\tiny
 (p. 1b)   }
  \end{tabular} &
    \begin{tabular}{@{}l@{}}
    \hypertarget{dms-req-0029-v-01}{DMS-REQ-0029-V-01}
    \\\vcdJiraRef{LVV-12}~{\tiny
 (p. 1b)     }
    \end{tabular} &
        \begin{tabular}{@{}l@{}}
        \href{https://jira.lsstcorp.org/secure/Tests.jspa\#/testCase/LVV-T15}{LVV-T15} \\
        \vcdDocRef{LDM-534}
        \end{tabular} &
          \begin{tabular}{@{}l@{}}
          2019-05-22 \\
            \vcdDocRef{DMTR-51}
            {\scriptsize \href{https://jira.lsstcorp.org/secure/Tests.jspa\#/testPlan/LVV-P43}{LVV-P43} }
          \end{tabular} &
          \passed \\
          \cmidrule{3-5}
          & &
        \begin{tabular}{@{}l@{}}
        \href{https://jira.lsstcorp.org/secure/Tests.jspa\#/testCase/LVV-T19}{LVV-T19} \\
        \vcdDocRef{LDM-533}
        \end{tabular} &
          \begin{tabular}{@{}l@{}}
          2019-05-22 \\
            \vcdDocRef{DMTR-53}
            {\scriptsize \href{https://jira.lsstcorp.org/secure/Tests.jspa\#/testPlan/LVV-P44}{LVV-P44} }
          \end{tabular} &
          \passed \\
          \cmidrule{3-5}
          & &
        \begin{tabular}{@{}l@{}}
        \href{https://jira.lsstcorp.org/secure/Tests.jspa\#/testCase/LVV-T39}{LVV-T39} \\
        \vcdDocRef{LDM-639}
        \end{tabular} &
          & \notexec{} \\
  \midrule
  \begin{tabular}{@{}l@{}}
  DMS-REQ-0030\\\vcdDocRef{LSE-61}~{\tiny
 (p. 1a)   }
  \end{tabular} &
    \begin{tabular}{@{}l@{}}
    \hypertarget{dms-req-0030-v-01}{DMS-REQ-0030-V-01}
    \\\vcdJiraRef{LVV-13}~{\tiny
 (p. 1a)     }
    \end{tabular} &
        \begin{tabular}{@{}l@{}}
        \href{https://jira.lsstcorp.org/secure/Tests.jspa\#/testCase/LVV-T15}{LVV-T15} \\
        \vcdDocRef{LDM-534}
        \end{tabular} &
          \begin{tabular}{@{}l@{}}
          2019-05-22 \\
            \vcdDocRef{DMTR-51}
            {\scriptsize \href{https://jira.lsstcorp.org/secure/Tests.jspa\#/testPlan/LVV-P43}{LVV-P43} }
          \end{tabular} &
          \passed \\
          \cmidrule{3-5}
          & &
        \begin{tabular}{@{}l@{}}
        \href{https://jira.lsstcorp.org/secure/Tests.jspa\#/testCase/LVV-T19}{LVV-T19} \\
        \vcdDocRef{LDM-533}
        \end{tabular} &
          \begin{tabular}{@{}l@{}}
          2019-05-22 \\
            \vcdDocRef{DMTR-53}
            {\scriptsize \href{https://jira.lsstcorp.org/secure/Tests.jspa\#/testPlan/LVV-P44}{LVV-P44} }
          \end{tabular} &
          \passed \\
          \cmidrule{3-5}
          & &
        \begin{tabular}{@{}l@{}}
        \href{https://jira.lsstcorp.org/secure/Tests.jspa\#/testCase/LVV-T40}{LVV-T40} \\
        \vcdDocRef{LDM-639}
        \end{tabular} &
          \begin{tabular}{@{}l@{}}
          2020-01-28 \\
            \vcdDocRef{DMTR-201}
            {\scriptsize \href{https://jira.lsstcorp.org/secure/Tests.jspa\#/testPlan/LVV-P65}{LVV-P65} }
          \end{tabular} &
          \cndpass \\
      \cmidrule{2-5}
      &
    \begin{tabular}{@{}l@{}}
    \hypertarget{dms-req-0030-v-02}{DMS-REQ-0030-V-02}
    \\\vcdJiraRef{LVV-9741}~{\tiny
    }
    \end{tabular} &
        \begin{tabular}{@{}l@{}}
        \href{https://jira.lsstcorp.org/secure/Tests.jspa\#/testCase/LVV-T1240}{LVV-T1240} \\
        \vcdDocRef{LDM-639}
        \end{tabular} &
          \begin{tabular}{@{}l@{}}
          2020-01-28 \\
            \vcdDocRef{DMTR-201}
            {\scriptsize \href{https://jira.lsstcorp.org/secure/Tests.jspa\#/testPlan/LVV-P65}{LVV-P65} }
          \end{tabular} &
          \passed \\
  \midrule
  \begin{tabular}{@{}l@{}}
  DMS-REQ-0032\\\vcdDocRef{LSE-61}~{\tiny
 (p. 1b)   }
  \end{tabular} &
    \begin{tabular}{@{}l@{}}
    \hypertarget{dms-req-0032-v-01}{DMS-REQ-0032-V-01}
    \\\vcdJiraRef{LVV-14}~{\tiny
 (p. 1b)     }
    \end{tabular} &
        \begin{tabular}{@{}l@{}}
        \href{https://jira.lsstcorp.org/secure/Tests.jspa\#/testCase/LVV-T126}{LVV-T126} \\
        \vcdDocRef{LDM-639}
        \end{tabular} &
          & \notexec{} \\
  \midrule
  \begin{tabular}{@{}l@{}}
  DMS-REQ-0033\\\vcdDocRef{LSE-61}~{\tiny
 (p. 1a)   }
  \end{tabular} &
    \begin{tabular}{@{}l@{}}
    \hypertarget{dms-req-0033-v-01}{DMS-REQ-0033-V-01}
    \\\vcdJiraRef{LVV-15}~{\tiny
 (p. 1a)     }
    \end{tabular} &
        \begin{tabular}{@{}l@{}}
        \href{https://jira.lsstcorp.org/secure/Tests.jspa\#/testCase/LVV-T127}{LVV-T127} \\
        \vcdDocRef{LDM-639}
        \end{tabular} &
          & \notexec{} \\
          \cmidrule{3-5}
          & &
        \begin{tabular}{@{}l@{}}
        \href{https://jira.lsstcorp.org/secure/Tests.jspa\#/testCase/LVV-T362}{LVV-T362} \\
        \vcdDocRef{}
        \end{tabular} &
          \begin{tabular}{@{}l@{}}
          2019-03-31 \\
            \vcdDocRef{DMTR-111}
            {\scriptsize \href{https://jira.lsstcorp.org/secure/Tests.jspa\#/testPlan/LVV-P15}{LVV-P15} }
          \end{tabular} &
          \passed \\
  \midrule
  \begin{tabular}{@{}l@{}}
  DMS-REQ-0034\\\vcdDocRef{LSE-61}~{\tiny
 (p. 1a)   }
  \end{tabular} &
    \begin{tabular}{@{}l@{}}
    \hypertarget{dms-req-0034-v-01}{DMS-REQ-0034-V-01}
    \\\vcdJiraRef{LVV-16}~{\tiny
 (p. 1a)     }
    \end{tabular} &
        \begin{tabular}{@{}l@{}}
        \href{https://jira.lsstcorp.org/secure/Tests.jspa\#/testCase/LVV-T61}{LVV-T61} \\
        \vcdDocRef{LDM-639}
        \end{tabular} &
          & \notexec{} \\
  \midrule
  \begin{tabular}{@{}l@{}}
  DMS-REQ-0042\\\vcdDocRef{LSE-61}~{\tiny
 (p. 1b)   }
  \end{tabular} &
    \begin{tabular}{@{}l@{}}
    \hypertarget{dms-req-0042-v-01}{DMS-REQ-0042-V-01}
    \\\vcdJiraRef{LVV-17}~{\tiny
 (p. 1b)     }
    \end{tabular} &
        \begin{tabular}{@{}l@{}}
        \href{https://jira.lsstcorp.org/secure/Tests.jspa\#/testCase/LVV-T128}{LVV-T128} \\
        \vcdDocRef{LDM-639}
        \end{tabular} &
          & \notexec{} \\
  \midrule
  \begin{tabular}{@{}l@{}}
  DMS-REQ-0043\\\vcdDocRef{LSE-61}~{\tiny
 (p. 1a)   }
  \end{tabular} &
    \begin{tabular}{@{}l@{}}
    \hypertarget{dms-req-0043-v-01}{DMS-REQ-0043-V-01}
    \\\vcdJiraRef{LVV-18}~{\tiny
 (p. 1a)     }
    \end{tabular} &
        \begin{tabular}{@{}l@{}}
        \href{https://jira.lsstcorp.org/secure/Tests.jspa\#/testCase/LVV-T21}{LVV-T21} \\
        \vcdDocRef{LDM-533}
        \end{tabular} &
          \begin{tabular}{@{}l@{}}
          2019-05-22 \\
            \vcdDocRef{DMTR-53}
            {\scriptsize \href{https://jira.lsstcorp.org/secure/Tests.jspa\#/testPlan/LVV-P44}{LVV-P44} }
          \end{tabular} &
          \passed \\
          \cmidrule{3-5}
          & &
        \begin{tabular}{@{}l@{}}
        \href{https://jira.lsstcorp.org/secure/Tests.jspa\#/testCase/LVV-T22}{LVV-T22} \\
        \vcdDocRef{LDM-533}
        \end{tabular} &
          \begin{tabular}{@{}l@{}}
          2019-05-22 \\
            \vcdDocRef{DMTR-53}
            {\scriptsize \href{https://jira.lsstcorp.org/secure/Tests.jspa\#/testPlan/LVV-P44}{LVV-P44} }
          \end{tabular} &
          \passed \\
          \cmidrule{3-5}
          & &
        \begin{tabular}{@{}l@{}}
        \href{https://jira.lsstcorp.org/secure/Tests.jspa\#/testCase/LVV-T129}{LVV-T129} \\
        \vcdDocRef{LDM-639}
        \end{tabular} &
          & \notexec{} \\
  \midrule
  \begin{tabular}{@{}l@{}}
  DMS-REQ-0046\\\vcdDocRef{LSE-61}~{\tiny
 (p. 2)   }
  \end{tabular} &
    \begin{tabular}{@{}l@{}}
    \hypertarget{dms-req-0046-v-01}{DMS-REQ-0046-V-01}
    \\\vcdJiraRef{LVV-19}~{\tiny
 (p. 2)     }
    \end{tabular} &
        \begin{tabular}{@{}l@{}}
        \href{https://jira.lsstcorp.org/secure/Tests.jspa\#/testCase/LVV-T68}{LVV-T68} \\
        \vcdDocRef{LDM-639}
        \end{tabular} &
          & \notexec{} \\
  \midrule
  \begin{tabular}{@{}l@{}}
  DMS-REQ-0047\\\vcdDocRef{LSE-61}~{\tiny
 (p. 1b)   }
  \end{tabular} &
    \begin{tabular}{@{}l@{}}
    \hypertarget{dms-req-0047-v-01}{DMS-REQ-0047-V-01}
    \\\vcdJiraRef{LVV-20}~{\tiny
 (p. 1b)     }
    \end{tabular} &
        \begin{tabular}{@{}l@{}}
        \href{https://jira.lsstcorp.org/secure/Tests.jspa\#/testCase/LVV-T16}{LVV-T16} \\
        \vcdDocRef{LDM-534}
        \end{tabular} &
          \begin{tabular}{@{}l@{}}
          2019-05-22 \\
            \vcdDocRef{DMTR-51}
            {\scriptsize \href{https://jira.lsstcorp.org/secure/Tests.jspa\#/testPlan/LVV-P43}{LVV-P43} }
          \end{tabular} &
          \passed \\
          \cmidrule{3-5}
          & &
        \begin{tabular}{@{}l@{}}
        \href{https://jira.lsstcorp.org/secure/Tests.jspa\#/testCase/LVV-T62}{LVV-T62} \\
        \vcdDocRef{LDM-639}
        \end{tabular} &
          \begin{tabular}{@{}l@{}}
          2020-01-23 \\
            \vcdDocRef{DMTR-201}
            {\scriptsize \href{https://jira.lsstcorp.org/secure/Tests.jspa\#/testPlan/LVV-P65}{LVV-P65} }
          \end{tabular} &
          \passed \\
  \midrule
  \begin{tabular}{@{}l@{}}
  DMS-REQ-0052\\\vcdDocRef{LSE-61}~{\tiny
 (p. 1b)   }
  \end{tabular} &
    \begin{tabular}{@{}l@{}}
    \hypertarget{dms-req-0052-v-01}{DMS-REQ-0052-V-01}
    \\\vcdJiraRef{LVV-21}~{\tiny
 (p. 1b)     }
    \end{tabular} &
        \begin{tabular}{@{}l@{}}
        \href{https://jira.lsstcorp.org/secure/Tests.jspa\#/testCase/LVV-T130}{LVV-T130} \\
        \vcdDocRef{LDM-639}
        \end{tabular} &
          & \notexec{} \\
  \midrule
  \begin{tabular}{@{}l@{}}
  DMS-REQ-0059\\\vcdDocRef{LSE-61}~{\tiny
 (p. 1a)   }
  \end{tabular} &
    \begin{tabular}{@{}l@{}}
    \hypertarget{dms-req-0059-v-01}{DMS-REQ-0059-V-01}
    \\\vcdJiraRef{LVV-22}~{\tiny
 (p. 1a)     }
    \end{tabular} &
        \begin{tabular}{@{}l@{}}
        \href{https://jira.lsstcorp.org/secure/Tests.jspa\#/testCase/LVV-T83}{LVV-T83} \\
        \vcdDocRef{LDM-639}
        \end{tabular} &
          & \notexec{} \\
  \midrule
  \begin{tabular}{@{}l@{}}
  DMS-REQ-0060\\\vcdDocRef{LSE-61}~{\tiny
 (p. 1a)   }
  \end{tabular} &
    \begin{tabular}{@{}l@{}}
    \hypertarget{dms-req-0060-v-01}{DMS-REQ-0060-V-01}
    \\\vcdJiraRef{LVV-23}~{\tiny
 (p. 1a)     }
    \end{tabular} &
        \begin{tabular}{@{}l@{}}
        \href{https://jira.lsstcorp.org/secure/Tests.jspa\#/testCase/LVV-T84}{LVV-T84} \\
        \vcdDocRef{LDM-639}
        \end{tabular} &
          & \notexec{} \\
          \cmidrule{3-5}
          & &
        \begin{tabular}{@{}l@{}}
        \href{https://jira.lsstcorp.org/secure/Tests.jspa\#/testCase/LVV-T368}{LVV-T368} \\
        \vcdDocRef{}
        \end{tabular} &
          \begin{tabular}{@{}l@{}}
          2018-12-06 \\
            \vcdDocRef{DMTR-112}
            {\scriptsize \href{https://jira.lsstcorp.org/secure/Tests.jspa\#/testPlan/LVV-P16}{LVV-P16} }
          \end{tabular} &
          \passed \\
  \midrule
  \begin{tabular}{@{}l@{}}
  DMS-REQ-0061\\\vcdDocRef{LSE-61}~{\tiny
 (p. 1a)   }
  \end{tabular} &
    \begin{tabular}{@{}l@{}}
    \hypertarget{dms-req-0061-v-01}{DMS-REQ-0061-V-01}
    \\\vcdJiraRef{LVV-24}~{\tiny
 (p. 1a)     }
    \end{tabular} &
        \begin{tabular}{@{}l@{}}
        \href{https://jira.lsstcorp.org/secure/Tests.jspa\#/testCase/LVV-T85}{LVV-T85} \\
        \vcdDocRef{LDM-639}
        \end{tabular} &
          & \notexec{} \\
  \midrule
  \begin{tabular}{@{}l@{}}
  DMS-REQ-0062\\\vcdDocRef{LSE-61}~{\tiny
 (p. 1b)   }
  \end{tabular} &
    \begin{tabular}{@{}l@{}}
    \hypertarget{dms-req-0062-v-01}{DMS-REQ-0062-V-01}
    \\\vcdJiraRef{LVV-25}~{\tiny
 (p. 1b)     }
    \end{tabular} &
        \begin{tabular}{@{}l@{}}
        \href{https://jira.lsstcorp.org/secure/Tests.jspa\#/testCase/LVV-T86}{LVV-T86} \\
        \vcdDocRef{LDM-639}
        \end{tabular} &
          & \notexec{} \\
  \midrule
  \begin{tabular}{@{}l@{}}
  DMS-REQ-0063\\\vcdDocRef{LSE-61}~{\tiny
 (p. 1b)   }
  \end{tabular} &
    \begin{tabular}{@{}l@{}}
    \hypertarget{dms-req-0063-v-01}{DMS-REQ-0063-V-01}
    \\\vcdJiraRef{LVV-26}~{\tiny
 (p. 1b)     }
    \end{tabular} &
        \begin{tabular}{@{}l@{}}
        \href{https://jira.lsstcorp.org/secure/Tests.jspa\#/testCase/LVV-T87}{LVV-T87} \\
        \vcdDocRef{LDM-639}
        \end{tabular} &
          & \notexec{} \\
  \midrule
  \begin{tabular}{@{}l@{}}
  DMS-REQ-0065\\\vcdDocRef{LSE-61}~{\tiny
 (p. 1b)   }
  \end{tabular} &
    \begin{tabular}{@{}l@{}}
    \hypertarget{dms-req-0065-v-01}{DMS-REQ-0065-V-01}
    \\\vcdJiraRef{LVV-27}~{\tiny
 (p. 1b)     }
    \end{tabular} &
        \multicolumn{3}{c}{
        \begin{tabular}{ r l }
        Verified in: &
            \hyperlink{dms-api-req-0028-v-01}{DMS-API-REQ-0028-V-01}(\vcdJiraRef{LVV-10004})\\
               &
            \hyperlink{dms-api-req-0016-v-01}{DMS-API-REQ-0016-V-01}(\vcdJiraRef{LVV-10016})\\
               &
            \hyperlink{dms-api-req-0017-v-01}{DMS-API-REQ-0017-V-01}(\vcdJiraRef{LVV-10018})\\
               &
            \hyperlink{dms-api-req-0018-v-01}{DMS-API-REQ-0018-V-01}(\vcdJiraRef{LVV-10017})\\
        \end{tabular}
        } \\
          \cmidrule{3-5}
          & &
        \begin{tabular}{@{}l@{}}
        \href{https://jira.lsstcorp.org/secure/Tests.jspa\#/testCase/LVV-T134}{LVV-T134} \\
        \vcdDocRef{LDM-639}
        \end{tabular} &
          & \notexec{} \\
  \midrule
  \begin{tabular}{@{}l@{}}
  DMS-REQ-0068\\\vcdDocRef{LSE-61}~{\tiny
 (p. 1a)   }
  \end{tabular} &
    \begin{tabular}{@{}l@{}}
    \hypertarget{dms-req-0068-v-01}{DMS-REQ-0068-V-01}
    \\\vcdJiraRef{LVV-28}~{\tiny
 (p. 1a)     }
    \end{tabular} &
        \begin{tabular}{@{}l@{}}
        \href{https://jira.lsstcorp.org/secure/Tests.jspa\#/testCase/LVV-T33}{LVV-T33} \\
        \vcdDocRef{LDM-639}
        \end{tabular} &
          & \notexec{} \\
          \cmidrule{3-5}
          & &
        \begin{tabular}{@{}l@{}}
        \href{https://jira.lsstcorp.org/secure/Tests.jspa\#/testCase/LVV-T283}{LVV-T283} \\
        \vcdDocRef{LDM-538}
        \end{tabular} &
          \begin{tabular}{@{}l@{}}
          2019-05-22 \\
            \vcdDocRef{DMTR-61}
            {\scriptsize \href{https://jira.lsstcorp.org/secure/Tests.jspa\#/testPlan/LVV-P45}{LVV-P45} }
          \end{tabular} &
          \passed \\
          \cmidrule{3-5}
          & &
        \begin{tabular}{@{}l@{}}
        \href{https://jira.lsstcorp.org/secure/Tests.jspa\#/testCase/LVV-T284}{LVV-T284} \\
        \vcdDocRef{LDM-538}
        \end{tabular} &
          \begin{tabular}{@{}l@{}}
          2019-06-24 \\
            \vcdDocRef{DMTR-102}
            {\scriptsize \href{https://jira.lsstcorp.org/secure/Tests.jspa\#/testPlan/LVV-P10}{LVV-P10} }
          \end{tabular} &
          \passed \\
          \cmidrule{3-5}
          & &
        \begin{tabular}{@{}l@{}}
        \href{https://jira.lsstcorp.org/secure/Tests.jspa\#/testCase/LVV-T286}{LVV-T286} \\
        \vcdDocRef{LDM-538}
        \end{tabular} &
          \begin{tabular}{@{}l@{}}
          2019-05-22 \\
            \vcdDocRef{DMTR-61}
            {\scriptsize \href{https://jira.lsstcorp.org/secure/Tests.jspa\#/testPlan/LVV-P45}{LVV-P45} }
          \end{tabular} &
          \passed \\
          \cmidrule{3-5}
          & &
        \begin{tabular}{@{}l@{}}
        \href{https://jira.lsstcorp.org/secure/Tests.jspa\#/testCase/LVV-T1549}{LVV-T1549} \\
        \vcdDocRef{}
        \end{tabular} &
          & \notexec{} \\
          \cmidrule{3-5}
          & &
        \begin{tabular}{@{}l@{}}
        \href{https://jira.lsstcorp.org/secure/Tests.jspa\#/testCase/LVV-T1550}{LVV-T1550} \\
        \vcdDocRef{}
        \end{tabular} &
          & \notexec{} \\
          \cmidrule{3-5}
          & &
        \begin{tabular}{@{}l@{}}
        \href{https://jira.lsstcorp.org/secure/Tests.jspa\#/testCase/LVV-T1556}{LVV-T1556} \\
        \vcdDocRef{}
        \end{tabular} &
          & \notexec{} \\
  \midrule
  \begin{tabular}{@{}l@{}}
  DMS-REQ-0069\\\vcdDocRef{LSE-61}~{\tiny
 (p. 1a)   }
  \end{tabular} &
    \begin{tabular}{@{}l@{}}
    \hypertarget{dms-req-0069-v-01}{DMS-REQ-0069-V-01}
    \\\vcdJiraRef{LVV-29}~{\tiny
 (p. 1a)     }
    \end{tabular} &
        \begin{tabular}{@{}l@{}}
        \href{https://jira.lsstcorp.org/secure/Tests.jspa\#/testCase/LVV-T15}{LVV-T15} \\
        \vcdDocRef{LDM-534}
        \end{tabular} &
          \begin{tabular}{@{}l@{}}
          2019-05-22 \\
            \vcdDocRef{DMTR-51}
            {\scriptsize \href{https://jira.lsstcorp.org/secure/Tests.jspa\#/testPlan/LVV-P43}{LVV-P43} }
          \end{tabular} &
          \passed \\
          \cmidrule{3-5}
          & &
        \begin{tabular}{@{}l@{}}
        \href{https://jira.lsstcorp.org/secure/Tests.jspa\#/testCase/LVV-T18}{LVV-T18} \\
        \vcdDocRef{LDM-533}
        \end{tabular} &
          \begin{tabular}{@{}l@{}}
          2019-05-22 \\
            \vcdDocRef{DMTR-53}
            {\scriptsize \href{https://jira.lsstcorp.org/secure/Tests.jspa\#/testPlan/LVV-P44}{LVV-P44} }
          \end{tabular} &
          \passed \\
          \cmidrule{3-5}
          & &
        \begin{tabular}{@{}l@{}}
        \href{https://jira.lsstcorp.org/secure/Tests.jspa\#/testCase/LVV-T19}{LVV-T19} \\
        \vcdDocRef{LDM-533}
        \end{tabular} &
          \begin{tabular}{@{}l@{}}
          2019-05-22 \\
            \vcdDocRef{DMTR-53}
            {\scriptsize \href{https://jira.lsstcorp.org/secure/Tests.jspa\#/testPlan/LVV-P44}{LVV-P44} }
          \end{tabular} &
          \passed \\
          \cmidrule{3-5}
          & &
        \begin{tabular}{@{}l@{}}
        \href{https://jira.lsstcorp.org/secure/Tests.jspa\#/testCase/LVV-T38}{LVV-T38} \\
        \vcdDocRef{LDM-639}
        \end{tabular} &
          & \notexec{} \\
          \cmidrule{3-5}
          & &
        \begin{tabular}{@{}l@{}}
        \href{https://jira.lsstcorp.org/secure/Tests.jspa\#/testCase/LVV-T362}{LVV-T362} \\
        \vcdDocRef{}
        \end{tabular} &
          \begin{tabular}{@{}l@{}}
          2019-03-31 \\
            \vcdDocRef{DMTR-111}
            {\scriptsize \href{https://jira.lsstcorp.org/secure/Tests.jspa\#/testPlan/LVV-P15}{LVV-P15} }
          \end{tabular} &
          \passed \\
  \midrule
  \begin{tabular}{@{}l@{}}
  DMS-REQ-0070\\\vcdDocRef{LSE-61}~{\tiny
 (p. 1b)   }
  \end{tabular} &
    \begin{tabular}{@{}l@{}}
    \hypertarget{dms-req-0070-v-01}{DMS-REQ-0070-V-01}
    \\\vcdJiraRef{LVV-30}~{\tiny
 (p. 1b)     }
    \end{tabular} &
        \begin{tabular}{@{}l@{}}
        \href{https://jira.lsstcorp.org/secure/Tests.jspa\#/testCase/LVV-T15}{LVV-T15} \\
        \vcdDocRef{LDM-534}
        \end{tabular} &
          \begin{tabular}{@{}l@{}}
          2019-05-22 \\
            \vcdDocRef{DMTR-51}
            {\scriptsize \href{https://jira.lsstcorp.org/secure/Tests.jspa\#/testPlan/LVV-P43}{LVV-P43} }
          \end{tabular} &
          \passed \\
          \cmidrule{3-5}
          & &
        \begin{tabular}{@{}l@{}}
        \href{https://jira.lsstcorp.org/secure/Tests.jspa\#/testCase/LVV-T19}{LVV-T19} \\
        \vcdDocRef{LDM-533}
        \end{tabular} &
          \begin{tabular}{@{}l@{}}
          2019-05-22 \\
            \vcdDocRef{DMTR-53}
            {\scriptsize \href{https://jira.lsstcorp.org/secure/Tests.jspa\#/testPlan/LVV-P44}{LVV-P44} }
          \end{tabular} &
          \passed \\
          \cmidrule{3-5}
          & &
        \begin{tabular}{@{}l@{}}
        \href{https://jira.lsstcorp.org/secure/Tests.jspa\#/testCase/LVV-T41}{LVV-T41} \\
        \vcdDocRef{LDM-639}
        \end{tabular} &
          \begin{tabular}{@{}l@{}}
          2020-02-04 \\
            \vcdDocRef{DMTR-201}
            {\scriptsize \href{https://jira.lsstcorp.org/secure/Tests.jspa\#/testPlan/LVV-P65}{LVV-P65} }
          \end{tabular} &
          \passed \\
  \midrule
  \begin{tabular}{@{}l@{}}
  DMS-REQ-0072\\\vcdDocRef{LSE-61}~{\tiny
 (p. 1a)   }
  \end{tabular} &
    \begin{tabular}{@{}l@{}}
    \hypertarget{dms-req-0072-v-01}{DMS-REQ-0072-V-01}
    \\\vcdJiraRef{LVV-31}~{\tiny
 (p. 1a)     }
    \end{tabular} &
        \begin{tabular}{@{}l@{}}
        \href{https://jira.lsstcorp.org/secure/Tests.jspa\#/testCase/LVV-T15}{LVV-T15} \\
        \vcdDocRef{LDM-534}
        \end{tabular} &
          \begin{tabular}{@{}l@{}}
          2019-05-22 \\
            \vcdDocRef{DMTR-51}
            {\scriptsize \href{https://jira.lsstcorp.org/secure/Tests.jspa\#/testPlan/LVV-P43}{LVV-P43} }
          \end{tabular} &
          \passed \\
          \cmidrule{3-5}
          & &
        \begin{tabular}{@{}l@{}}
        \href{https://jira.lsstcorp.org/secure/Tests.jspa\#/testCase/LVV-T19}{LVV-T19} \\
        \vcdDocRef{LDM-533}
        \end{tabular} &
          \begin{tabular}{@{}l@{}}
          2019-05-22 \\
            \vcdDocRef{DMTR-53}
            {\scriptsize \href{https://jira.lsstcorp.org/secure/Tests.jspa\#/testPlan/LVV-P44}{LVV-P44} }
          \end{tabular} &
          \passed \\
          \cmidrule{3-5}
          & &
        \begin{tabular}{@{}l@{}}
        \href{https://jira.lsstcorp.org/secure/Tests.jspa\#/testCase/LVV-T42}{LVV-T42} \\
        \vcdDocRef{LDM-639}
        \end{tabular} &
          & \notexec{} \\
  \midrule
  \begin{tabular}{@{}l@{}}
  DMS-REQ-0074\\\vcdDocRef{LSE-61}~{\tiny
 (p. 1b)   }
  \end{tabular} &
    \begin{tabular}{@{}l@{}}
    \hypertarget{dms-req-0074-v-01}{DMS-REQ-0074-V-01}
    \\\vcdJiraRef{LVV-32}~{\tiny
 (p. 1b)     }
    \end{tabular} &
        \begin{tabular}{@{}l@{}}
        \href{https://jira.lsstcorp.org/secure/Tests.jspa\#/testCase/LVV-T20}{LVV-T20} \\
        \vcdDocRef{LDM-533}
        \end{tabular} &
          \begin{tabular}{@{}l@{}}
          2019-05-22 \\
            \vcdDocRef{DMTR-53}
            {\scriptsize \href{https://jira.lsstcorp.org/secure/Tests.jspa\#/testPlan/LVV-P44}{LVV-P44} }
          \end{tabular} &
          \failed \\
          \cmidrule{3-5}
          & &
        \begin{tabular}{@{}l@{}}
        \href{https://jira.lsstcorp.org/secure/Tests.jspa\#/testCase/LVV-T37}{LVV-T37} \\
        \vcdDocRef{LDM-639}
        \end{tabular} &
          & \notexec{} \\
  \midrule
  \begin{tabular}{@{}l@{}}
  DMS-REQ-0075\\\vcdDocRef{LSE-61}~{\tiny
 (p. 1a)   }
  \end{tabular} &
    \begin{tabular}{@{}l@{}}
    \hypertarget{dms-req-0075-v-01}{DMS-REQ-0075-V-01}
    \\\vcdJiraRef{LVV-33}~{\tiny
 (p. 1a)     }
    \end{tabular} &
        \begin{tabular}{@{}l@{}}
        \href{https://jira.lsstcorp.org/secure/Tests.jspa\#/testCase/LVV-T149}{LVV-T149} \\
        \vcdDocRef{LDM-639}
        \end{tabular} &
          & \notexec{} \\
          \cmidrule{3-5}
          & &
        \begin{tabular}{@{}l@{}}
        \href{https://jira.lsstcorp.org/secure/Tests.jspa\#/testCase/LVV-T1085}{LVV-T1085} \\
        \vcdDocRef{LDM-552}
        \end{tabular} &
          \begin{tabular}{@{}l@{}}
          2019-07-08 \\
            \vcdDocRef{DMTR-71}
            {\scriptsize \href{https://jira.lsstcorp.org/secure/Tests.jspa\#/testPlan/LVV-P46}{LVV-P46} }
          \end{tabular} &
          \passed \\
          \cmidrule{3-5}
          & &
        \begin{tabular}{@{}l@{}}
        \href{https://jira.lsstcorp.org/secure/Tests.jspa\#/testCase/LVV-T1086}{LVV-T1086} \\
        \vcdDocRef{LDM-552}
        \end{tabular} &
          \begin{tabular}{@{}l@{}}
          2019-07-08 \\
            \vcdDocRef{DMTR-71}
            {\scriptsize \href{https://jira.lsstcorp.org/secure/Tests.jspa\#/testPlan/LVV-P46}{LVV-P46} }
          \end{tabular} &
          \passed \\
          \cmidrule{3-5}
          & &
        \begin{tabular}{@{}l@{}}
        \href{https://jira.lsstcorp.org/secure/Tests.jspa\#/testCase/LVV-T1087}{LVV-T1087} \\
        \vcdDocRef{LDM-552}
        \end{tabular} &
          \begin{tabular}{@{}l@{}}
          2019-07-08 \\
            \vcdDocRef{DMTR-71}
            {\scriptsize \href{https://jira.lsstcorp.org/secure/Tests.jspa\#/testPlan/LVV-P46}{LVV-P46} }
          \end{tabular} &
          \passed \\
  \midrule
  \begin{tabular}{@{}l@{}}
  DMS-REQ-0077\\\vcdDocRef{LSE-61}~{\tiny
 (p. 1b)   }
  \end{tabular} &
    \begin{tabular}{@{}l@{}}
    \hypertarget{dms-req-0077-v-01}{DMS-REQ-0077-V-01}
    \\\vcdJiraRef{LVV-34}~{\tiny
 (p. 1b)     }
    \end{tabular} &
        \begin{tabular}{@{}l@{}}
        \href{https://jira.lsstcorp.org/secure/Tests.jspa\#/testCase/LVV-T150}{LVV-T150} \\
        \vcdDocRef{LDM-639}
        \end{tabular} &
          & \notexec{} \\
  \midrule
  \begin{tabular}{@{}l@{}}
  DMS-REQ-0078\\\vcdDocRef{LSE-61}~{\tiny
 (p. 1a)   }
  \end{tabular} &
    \begin{tabular}{@{}l@{}}
    \hypertarget{dms-req-0078-v-01}{DMS-REQ-0078-V-01}
    \\\vcdJiraRef{LVV-35}~{\tiny
 (p. 1a)     }
    \end{tabular} &
        \begin{tabular}{@{}l@{}}
        \href{https://jira.lsstcorp.org/secure/Tests.jspa\#/testCase/LVV-T151}{LVV-T151} \\
        \vcdDocRef{LDM-639}
        \end{tabular} &
          & \notexec{} \\
          \cmidrule{3-5}
          & &
        \begin{tabular}{@{}l@{}}
        \href{https://jira.lsstcorp.org/secure/Tests.jspa\#/testCase/LVV-T1232}{LVV-T1232} \\
        \vcdDocRef{LDM-639}
        \end{tabular} &
          & \notexec{} \\
  \midrule
  \begin{tabular}{@{}l@{}}
  DMS-REQ-0089\\\vcdDocRef{LSE-61}~{\tiny
 (p. 1b)   }
  \end{tabular} &
    \begin{tabular}{@{}l@{}}
    \hypertarget{dms-req-0089-v-01}{DMS-REQ-0089-V-01}
    \\\vcdJiraRef{LVV-36}~{\tiny
 (p. 1b)     }
    \end{tabular} &
        \begin{tabular}{@{}l@{}}
        \href{https://jira.lsstcorp.org/secure/Tests.jspa\#/testCase/LVV-T102}{LVV-T102} \\
        \vcdDocRef{LDM-639}
        \end{tabular} &
          & \notexec{} \\
  \midrule
  \begin{tabular}{@{}l@{}}
  DMS-REQ-0094\\\vcdDocRef{LSE-61}~{\tiny
 (p. 1b)   }
  \end{tabular} &
    \begin{tabular}{@{}l@{}}
    \hypertarget{dms-req-0094-v-01}{DMS-REQ-0094-V-01}
    \\\vcdJiraRef{LVV-37}~{\tiny
 (p. 1b)     }
    \end{tabular} &
        \begin{tabular}{@{}l@{}}
        \href{https://jira.lsstcorp.org/secure/Tests.jspa\#/testCase/LVV-T152}{LVV-T152} \\
        \vcdDocRef{LDM-639}
        \end{tabular} &
          & \notexec{} \\
  \midrule
  \begin{tabular}{@{}l@{}}
  DMS-REQ-0096\\\vcdDocRef{LSE-61}~{\tiny
 (p. 1a)   }
  \end{tabular} &
    \begin{tabular}{@{}l@{}}
    \hypertarget{dms-req-0096-v-01}{DMS-REQ-0096-V-01}
    \\\vcdJiraRef{LVV-38}~{\tiny
 (p. 1a)     }
    \end{tabular} &
        \begin{tabular}{@{}l@{}}
        \href{https://jira.lsstcorp.org/secure/Tests.jspa\#/testCase/LVV-T103}{LVV-T103} \\
        \vcdDocRef{LDM-639}
        \end{tabular} &
          & \notexec{} \\
  \midrule
  \begin{tabular}{@{}l@{}}
  DMS-REQ-0097\\\vcdDocRef{LSE-61}~{\tiny
 (p. 1a)   }
  \end{tabular} &
    \begin{tabular}{@{}l@{}}
    \hypertarget{dms-req-0097-v-01}{DMS-REQ-0097-V-01}
    \\\vcdJiraRef{LVV-39}~{\tiny
 (p. 1a)     }
    \end{tabular} &
        \begin{tabular}{@{}l@{}}
        \href{https://jira.lsstcorp.org/secure/Tests.jspa\#/testCase/LVV-T45}{LVV-T45} \\
        \vcdDocRef{LDM-639}
        \end{tabular} &
          & \notexec{} \\
  \midrule
  \begin{tabular}{@{}l@{}}
  DMS-REQ-0098\\\vcdDocRef{LSE-61}~{\tiny
 (p. 1b)   }
  \end{tabular} &
    \begin{tabular}{@{}l@{}}
    \hypertarget{dms-req-0098-v-01}{DMS-REQ-0098-V-01}
    \\\vcdJiraRef{LVV-40}~{\tiny
 (p. 1b)     }
    \end{tabular} &
        \begin{tabular}{@{}l@{}}
        \href{https://jira.lsstcorp.org/secure/Tests.jspa\#/testCase/LVV-T104}{LVV-T104} \\
        \vcdDocRef{LDM-639}
        \end{tabular} &
          & \notexec{} \\
  \midrule
  \begin{tabular}{@{}l@{}}
  DMS-REQ-0099\\\vcdDocRef{LSE-61}~{\tiny
 (p. 1b)   }
  \end{tabular} &
    \begin{tabular}{@{}l@{}}
    \hypertarget{dms-req-0099-v-01}{DMS-REQ-0099-V-01}
    \\\vcdJiraRef{LVV-41}~{\tiny
 (p. 1b)     }
    \end{tabular} &
        \begin{tabular}{@{}l@{}}
        \href{https://jira.lsstcorp.org/secure/Tests.jspa\#/testCase/LVV-T46}{LVV-T46} \\
        \vcdDocRef{LDM-639}
        \end{tabular} &
          & \notexec{} \\
  \midrule
  \begin{tabular}{@{}l@{}}
  DMS-REQ-0100\\\vcdDocRef{LSE-61}~{\tiny
 (p. 1b)   }
  \end{tabular} &
    \begin{tabular}{@{}l@{}}
    \hypertarget{dms-req-0100-v-01}{DMS-REQ-0100-V-01}
    \\\vcdJiraRef{LVV-42}~{\tiny
 (p. 1b)     }
    \end{tabular} &
        \begin{tabular}{@{}l@{}}
        \href{https://jira.lsstcorp.org/secure/Tests.jspa\#/testCase/LVV-T105}{LVV-T105} \\
        \vcdDocRef{LDM-639}
        \end{tabular} &
          & \notexec{} \\
  \midrule
  \begin{tabular}{@{}l@{}}
  DMS-REQ-0101\\\vcdDocRef{LSE-61}~{\tiny
 (p. 1a)   }
  \end{tabular} &
    \begin{tabular}{@{}l@{}}
    \hypertarget{dms-req-0101-v-01}{DMS-REQ-0101-V-01}
    \\\vcdJiraRef{LVV-43}~{\tiny
 (p. 1a)     }
    \end{tabular} &
        \begin{tabular}{@{}l@{}}
        \href{https://jira.lsstcorp.org/secure/Tests.jspa\#/testCase/LVV-T47}{LVV-T47} \\
        \vcdDocRef{LDM-639}
        \end{tabular} &
          & \notexec{} \\
  \midrule
  \begin{tabular}{@{}l@{}}
  DMS-REQ-0102\\\vcdDocRef{LSE-61}~{\tiny
 (p. 1a)   }
  \end{tabular} &
    \begin{tabular}{@{}l@{}}
    \hypertarget{dms-req-0102-v-01}{DMS-REQ-0102-V-01}
    \\\vcdJiraRef{LVV-44}~{\tiny
 (p. 1a)     }
    \end{tabular} &
        \begin{tabular}{@{}l@{}}
        \href{https://jira.lsstcorp.org/secure/Tests.jspa\#/testCase/LVV-T153}{LVV-T153} \\
        \vcdDocRef{LDM-639}
        \end{tabular} &
          & \notexec{} \\
  \midrule
  \begin{tabular}{@{}l@{}}
  DMS-REQ-0103\\\vcdDocRef{LSE-61}~{\tiny
 (p. 1b)   }
  \end{tabular} &
    \begin{tabular}{@{}l@{}}
    \hypertarget{dms-req-0103-v-01}{DMS-REQ-0103-V-01}
    \\\vcdJiraRef{LVV-45}~{\tiny
 (p. 1b)     }
    \end{tabular} &
        \begin{tabular}{@{}l@{}}
        \href{https://jira.lsstcorp.org/secure/Tests.jspa\#/testCase/LVV-T63}{LVV-T63} \\
        \vcdDocRef{LDM-639}
        \end{tabular} &
          & \notexec{} \\
  \midrule
  \begin{tabular}{@{}l@{}}
  DMS-REQ-0106\\\vcdDocRef{LSE-61}~{\tiny
 (p. 1b)   }
  \end{tabular} &
    \begin{tabular}{@{}l@{}}
    \hypertarget{dms-req-0106-v-01}{DMS-REQ-0106-V-01}
    \\\vcdJiraRef{LVV-46}~{\tiny
 (p. 1b)     }
    \end{tabular} &
        \begin{tabular}{@{}l@{}}
        \href{https://jira.lsstcorp.org/secure/Tests.jspa\#/testCase/LVV-T11}{LVV-T11} \\
        \vcdDocRef{LDM-534}
        \end{tabular} &
          \begin{tabular}{@{}l@{}}
          2019-05-22 \\
            \vcdDocRef{DMTR-51}
            {\scriptsize \href{https://jira.lsstcorp.org/secure/Tests.jspa\#/testPlan/LVV-P43}{LVV-P43} }
          \end{tabular} &
          \passed \\
          \cmidrule{3-5}
          & &
        \begin{tabular}{@{}l@{}}
        \href{https://jira.lsstcorp.org/secure/Tests.jspa\#/testCase/LVV-T64}{LVV-T64} \\
        \vcdDocRef{LDM-639}
        \end{tabular} &
          & \notexec{} \\
  \midrule
  \begin{tabular}{@{}l@{}}
  DMS-REQ-0119\\\vcdDocRef{LSE-61}~{\tiny
 (p. 2)   }
  \end{tabular} &
    \begin{tabular}{@{}l@{}}
    \hypertarget{dms-req-0119-v-01}{DMS-REQ-0119-V-01}
    \\\vcdJiraRef{LVV-47}~{\tiny
 (p. 2)     }
    \end{tabular} &
        \begin{tabular}{@{}l@{}}
        \href{https://jira.lsstcorp.org/secure/Tests.jspa\#/testCase/LVV-T117}{LVV-T117} \\
        \vcdDocRef{LDM-639}
        \end{tabular} &
          & \notexec{} \\
  \midrule
  \begin{tabular}{@{}l@{}}
  DMS-REQ-0120\\\vcdDocRef{LSE-61}~{\tiny
 (p. 2)   }
  \end{tabular} &
    \begin{tabular}{@{}l@{}}
    \hypertarget{dms-req-0120-v-01}{DMS-REQ-0120-V-01}
    \\\vcdJiraRef{LVV-48}~{\tiny
 (p. 2)     }
    \end{tabular} &
        \begin{tabular}{@{}l@{}}
        \href{https://jira.lsstcorp.org/secure/Tests.jspa\#/testCase/LVV-T118}{LVV-T118} \\
        \vcdDocRef{LDM-639}
        \end{tabular} &
          & \notexec{} \\
  \midrule
  \begin{tabular}{@{}l@{}}
  DMS-REQ-0121\\\vcdDocRef{LSE-61}~{\tiny
 (p. 2)   }
  \end{tabular} &
    \begin{tabular}{@{}l@{}}
    \hypertarget{dms-req-0121-v-01}{DMS-REQ-0121-V-01}
    \\\vcdJiraRef{LVV-49}~{\tiny
 (p. 2)     }
    \end{tabular} &
        \begin{tabular}{@{}l@{}}
        \href{https://jira.lsstcorp.org/secure/Tests.jspa\#/testCase/LVV-T119}{LVV-T119} \\
        \vcdDocRef{LDM-639}
        \end{tabular} &
          & \notexec{} \\
  \midrule
  \begin{tabular}{@{}l@{}}
  DMS-REQ-0122\\\vcdDocRef{LSE-61}~{\tiny
 (p. 2)   }
  \end{tabular} &
    \begin{tabular}{@{}l@{}}
    \hypertarget{dms-req-0122-v-01}{DMS-REQ-0122-V-01}
    \\\vcdJiraRef{LVV-50}~{\tiny
 (p. 2)     }
    \end{tabular} &
        \begin{tabular}{@{}l@{}}
        \href{https://jira.lsstcorp.org/secure/Tests.jspa\#/testCase/LVV-T204}{LVV-T204} \\
        \vcdDocRef{LDM-639}
        \end{tabular} &
          & \notexec{} \\
  \midrule
  \begin{tabular}{@{}l@{}}
  DMS-REQ-0123\\\vcdDocRef{LSE-61}~{\tiny
 (p. 2)   }
  \end{tabular} &
    \begin{tabular}{@{}l@{}}
    \hypertarget{dms-req-0123-v-01}{DMS-REQ-0123-V-01}
    \\\vcdJiraRef{LVV-51}~{\tiny
 (p. 2)     }
    \end{tabular} &
        \begin{tabular}{@{}l@{}}
        \href{https://jira.lsstcorp.org/secure/Tests.jspa\#/testCase/LVV-T205}{LVV-T205} \\
        \vcdDocRef{LDM-639}
        \end{tabular} &
          & \notexec{} \\
  \midrule
  \begin{tabular}{@{}l@{}}
  DMS-REQ-0124\\\vcdDocRef{LSE-61}~{\tiny
 (p. 2)   }
  \end{tabular} &
    \begin{tabular}{@{}l@{}}
    \hypertarget{dms-req-0124-v-01}{DMS-REQ-0124-V-01}
    \\\vcdJiraRef{LVV-52}~{\tiny
 (p. 2)     }
    \end{tabular} &
        \begin{tabular}{@{}l@{}}
        \href{https://jira.lsstcorp.org/secure/Tests.jspa\#/testCase/LVV-T206}{LVV-T206} \\
        \vcdDocRef{LDM-639}
        \end{tabular} &
          & \notexec{} \\
  \midrule
  \begin{tabular}{@{}l@{}}
  DMS-REQ-0125\\\vcdDocRef{LSE-61}~{\tiny
 (p. 2)   }
  \end{tabular} &
    \begin{tabular}{@{}l@{}}
    \hypertarget{dms-req-0125-v-01}{DMS-REQ-0125-V-01}
    \\\vcdJiraRef{LVV-53}~{\tiny
 (p. 2)     }
    \end{tabular} &
        \begin{tabular}{@{}l@{}}
        \href{https://jira.lsstcorp.org/secure/Tests.jspa\#/testCase/LVV-T120}{LVV-T120} \\
        \vcdDocRef{LDM-639}
        \end{tabular} &
          & \notexec{} \\
  \midrule
  \begin{tabular}{@{}l@{}}
  DMS-REQ-0126\\\vcdDocRef{LSE-61}~{\tiny
 (p. 2)   }
  \end{tabular} &
    \begin{tabular}{@{}l@{}}
    \hypertarget{dms-req-0126-v-01}{DMS-REQ-0126-V-01}
    \\\vcdJiraRef{LVV-54}~{\tiny
 (p. 2)     }
    \end{tabular} &
        \begin{tabular}{@{}l@{}}
        \href{https://jira.lsstcorp.org/secure/Tests.jspa\#/testCase/LVV-T207}{LVV-T207} \\
        \vcdDocRef{LDM-639}
        \end{tabular} &
          & \notexec{} \\
  \midrule
  \begin{tabular}{@{}l@{}}
  DMS-REQ-0127\\\vcdDocRef{LSE-61}~{\tiny
 (p. 2)   }
  \end{tabular} &
    \begin{tabular}{@{}l@{}}
    \hypertarget{dms-req-0127-v-01}{DMS-REQ-0127-V-01}
    \\\vcdJiraRef{LVV-55}~{\tiny
 (p. 2)     }
    \end{tabular} &
        \begin{tabular}{@{}l@{}}
        \href{https://jira.lsstcorp.org/secure/Tests.jspa\#/testCase/LVV-T208}{LVV-T208} \\
        \vcdDocRef{LDM-639}
        \end{tabular} &
          & \notexec{} \\
  \midrule
  \begin{tabular}{@{}l@{}}
  DMS-REQ-0128\\\vcdDocRef{LSE-61}~{\tiny
 (p. 2)   }
  \end{tabular} &
    \begin{tabular}{@{}l@{}}
    \hypertarget{dms-req-0128-v-01}{DMS-REQ-0128-V-01}
    \\\vcdJiraRef{LVV-56}~{\tiny
 (p. 2)     }
    \end{tabular} &
        \begin{tabular}{@{}l@{}}
        \href{https://jira.lsstcorp.org/secure/Tests.jspa\#/testCase/LVV-T121}{LVV-T121} \\
        \vcdDocRef{LDM-639}
        \end{tabular} &
          & \notexec{} \\
  \midrule
  \begin{tabular}{@{}l@{}}
  DMS-REQ-0130\\\vcdDocRef{LSE-61}~{\tiny
 (p. 1a)   }
  \end{tabular} &
    \begin{tabular}{@{}l@{}}
    \hypertarget{dms-req-0130-v-01}{DMS-REQ-0130-V-01}
    \\\vcdJiraRef{LVV-57}~{\tiny
 (p. 1a)     }
    \end{tabular} &
        \begin{tabular}{@{}l@{}}
        \href{https://jira.lsstcorp.org/secure/Tests.jspa\#/testCase/LVV-T88}{LVV-T88} \\
        \vcdDocRef{LDM-639}
        \end{tabular} &
          & \notexec{} \\
  \midrule
  \begin{tabular}{@{}l@{}}
  DMS-REQ-0131\\\vcdDocRef{LSE-61}~{\tiny
 (p. 2)   }
  \end{tabular} &
    \begin{tabular}{@{}l@{}}
    \hypertarget{dms-req-0131-v-01}{DMS-REQ-0131-V-01}
    \\\vcdJiraRef{LVV-58}~{\tiny
 (p. 2)     }
    \end{tabular} &
        \begin{tabular}{@{}l@{}}
        \href{https://jira.lsstcorp.org/secure/Tests.jspa\#/testCase/LVV-T106}{LVV-T106} \\
        \vcdDocRef{LDM-639}
        \end{tabular} &
          & \notexec{} \\
      \cmidrule{2-5}
      &
    \begin{tabular}{@{}l@{}}
    \hypertarget{dms-req-0131-v-02}{DMS-REQ-0131-V-02}
    \\\vcdJiraRef{LVV-9745}~{\tiny
    }
    \end{tabular} &
        \begin{tabular}{@{}l@{}}
        \href{https://jira.lsstcorp.org/secure/Tests.jspa\#/testCase/LVV-T1277}{LVV-T1277} \\
        \vcdDocRef{LDM-639}
        \end{tabular} &
          & \notexec{} \\
  \midrule
  \begin{tabular}{@{}l@{}}
  DMS-REQ-0132\\\vcdDocRef{LSE-61}~{\tiny
 (p. 1a)   }
  \end{tabular} &
    \begin{tabular}{@{}l@{}}
    \hypertarget{dms-req-0132-v-01}{DMS-REQ-0132-V-01}
    \\\vcdJiraRef{LVV-59}~{\tiny
 (p. 1a)     }
    \end{tabular} &
        \begin{tabular}{@{}l@{}}
        \href{https://jira.lsstcorp.org/secure/Tests.jspa\#/testCase/LVV-T89}{LVV-T89} \\
        \vcdDocRef{LDM-639}
        \end{tabular} &
          & \notexec{} \\
  \midrule
  \begin{tabular}{@{}l@{}}
  DMS-REQ-0155\\\vcdDocRef{LSE-61}~{\tiny
 (p. 1a)   }
  \end{tabular} &
    \begin{tabular}{@{}l@{}}
    \hypertarget{dms-req-0155-v-01}{DMS-REQ-0155-V-01}
    \\\vcdJiraRef{LVV-60}~{\tiny
 (p. 1a)     }
    \end{tabular} &
        \multicolumn{3}{c}{
        \begin{tabular}{ r l }
        Verified in: &
            \hyperlink{dms-req-0298-v-01}{DMS-REQ-0298-V-01}(\vcdJiraRef{LVV-129})\\
               &
            \hyperlink{dms-req-0300-v-01}{DMS-REQ-0300-V-01}(\vcdJiraRef{LVV-131})\\
               &
            \hyperlink{dms-req-0299-v-01}{DMS-REQ-0299-V-01}(\vcdJiraRef{LVV-130})\\
        \end{tabular}
        } \\
  \midrule
  \begin{tabular}{@{}l@{}}
  DMS-REQ-0156\\\vcdDocRef{LSE-61}~{\tiny
 (p. 1a)   }
  \end{tabular} &
    \begin{tabular}{@{}l@{}}
    \hypertarget{dms-req-0156-v-01}{DMS-REQ-0156-V-01}
    \\\vcdJiraRef{LVV-61}~{\tiny
 (p. 1a)     }
    \end{tabular} &
        \multicolumn{3}{c}{
        \begin{tabular}{ r l }
        Verified in: &
            \hyperlink{dms-req-0304-v-01}{DMS-REQ-0304-V-01}(\vcdJiraRef{LVV-135})\\
               &
            \hyperlink{dms-req-0303-v-01}{DMS-REQ-0303-V-01}(\vcdJiraRef{LVV-134})\\
               &
            \hyperlink{dms-req-0302-v-01}{DMS-REQ-0302-V-01}(\vcdJiraRef{LVV-133})\\
        \end{tabular}
        } \\
  \midrule
  \begin{tabular}{@{}l@{}}
  DMS-REQ-0158\\\vcdDocRef{LSE-61}~{\tiny
 (p. 1a)   }
  \end{tabular} &
    \begin{tabular}{@{}l@{}}
    \hypertarget{dms-req-0158-v-01}{DMS-REQ-0158-V-01}
    \\\vcdJiraRef{LVV-62}~{\tiny
 (p. 1a)     }
    \end{tabular} &
        \multicolumn{3}{c}{
        \begin{tabular}{ r l }
        Verified in: &
            \hyperlink{dms-req-0305-v-01}{DMS-REQ-0305-V-01}(\vcdJiraRef{LVV-136})\\
               &
            \hyperlink{dms-req-0307-v-01}{DMS-REQ-0307-V-01}(\vcdJiraRef{LVV-138})\\
               &
            \hyperlink{dms-req-0306-v-01}{DMS-REQ-0306-V-01}(\vcdJiraRef{LVV-137})\\
        \end{tabular}
        } \\
          \cmidrule{3-5}
          & &
        \begin{tabular}{@{}l@{}}
        \href{https://jira.lsstcorp.org/secure/Tests.jspa\#/testCase/LVV-T11}{LVV-T11} \\
        \vcdDocRef{LDM-534}
        \end{tabular} &
          \begin{tabular}{@{}l@{}}
          2019-05-22 \\
            \vcdDocRef{DMTR-51}
            {\scriptsize \href{https://jira.lsstcorp.org/secure/Tests.jspa\#/testPlan/LVV-P43}{LVV-P43} }
          \end{tabular} &
          \passed \\
  \midrule
  \begin{tabular}{@{}l@{}}
  DMS-REQ-0160\\\vcdDocRef{LSE-61}~{\tiny
 (p. 1b)   }
  \end{tabular} &
    \begin{tabular}{@{}l@{}}
    \hypertarget{dms-req-0160-v-01}{DMS-REQ-0160-V-01}
    \\\vcdJiraRef{LVV-63}~{\tiny
 (p. 1b)     }
    \end{tabular} &
        \begin{tabular}{@{}l@{}}
        \href{https://jira.lsstcorp.org/secure/Tests.jspa\#/testCase/LVV-T131}{LVV-T131} \\
        \vcdDocRef{LDM-639}
        \end{tabular} &
          & \notexec{} \\
          \cmidrule{3-5}
          & &
        \begin{tabular}{@{}l@{}}
        \href{https://jira.lsstcorp.org/secure/Tests.jspa\#/testCase/LVV-T368}{LVV-T368} \\
        \vcdDocRef{}
        \end{tabular} &
          \begin{tabular}{@{}l@{}}
          2018-12-06 \\
            \vcdDocRef{DMTR-112}
            {\scriptsize \href{https://jira.lsstcorp.org/secure/Tests.jspa\#/testPlan/LVV-P16}{LVV-P16} }
          \end{tabular} &
          \passed \\
  \midrule
  \begin{tabular}{@{}l@{}}
  DMS-REQ-0161\\\vcdDocRef{LSE-61}~{\tiny
 (p. 1b)   }
  \end{tabular} &
    \begin{tabular}{@{}l@{}}
    \hypertarget{dms-req-0161-v-01}{DMS-REQ-0161-V-01}
    \\\vcdJiraRef{LVV-64}~{\tiny
 (p. 1b)     }
    \end{tabular} &
        \begin{tabular}{@{}l@{}}
        \href{https://jira.lsstcorp.org/secure/Tests.jspa\#/testCase/LVV-T172}{LVV-T172} \\
        \vcdDocRef{LDM-639}
        \end{tabular} &
          & \notexec{} \\
  \midrule
  \begin{tabular}{@{}l@{}}
  DMS-REQ-0162\\\vcdDocRef{LSE-61}~{\tiny
 (p. 1b)   }
  \end{tabular} &
    \begin{tabular}{@{}l@{}}
    \hypertarget{dms-req-0162-v-01}{DMS-REQ-0162-V-01}
    \\\vcdJiraRef{LVV-65}~{\tiny
 (p. 1b)     }
    \end{tabular} &
        \begin{tabular}{@{}l@{}}
        \href{https://jira.lsstcorp.org/secure/Tests.jspa\#/testCase/LVV-T173}{LVV-T173} \\
        \vcdDocRef{LDM-639}
        \end{tabular} &
          & \notexec{} \\
          \cmidrule{3-5}
          & &
        \begin{tabular}{@{}l@{}}
        \href{https://jira.lsstcorp.org/secure/Tests.jspa\#/testCase/LVV-T287}{LVV-T287} \\
        \vcdDocRef{LDM-538}
        \end{tabular} &
          & \notexec{} \\
  \midrule
  \begin{tabular}{@{}l@{}}
  DMS-REQ-0163\\\vcdDocRef{LSE-61}~{\tiny
 (p. 1b)   }
  \end{tabular} &
    \begin{tabular}{@{}l@{}}
    \hypertarget{dms-req-0163-v-01}{DMS-REQ-0163-V-01}
    \\\vcdJiraRef{LVV-66}~{\tiny
 (p. 1b)     }
    \end{tabular} &
        \begin{tabular}{@{}l@{}}
        \href{https://jira.lsstcorp.org/secure/Tests.jspa\#/testCase/LVV-T174}{LVV-T174} \\
        \vcdDocRef{LDM-639}
        \end{tabular} &
          & \notexec{} \\
  \midrule
  \begin{tabular}{@{}l@{}}
  DMS-REQ-0164\\\vcdDocRef{LSE-61}~{\tiny
 (p. 1b)   }
  \end{tabular} &
    \begin{tabular}{@{}l@{}}
    \hypertarget{dms-req-0164-v-01}{DMS-REQ-0164-V-01}
    \\\vcdJiraRef{LVV-67}~{\tiny
 (p. 1b)     }
    \end{tabular} &
        \begin{tabular}{@{}l@{}}
        \href{https://jira.lsstcorp.org/secure/Tests.jspa\#/testCase/LVV-T175}{LVV-T175} \\
        \vcdDocRef{LDM-639}
        \end{tabular} &
          & \notexec{} \\
  \midrule
  \begin{tabular}{@{}l@{}}
  DMS-REQ-0165\\\vcdDocRef{LSE-61}~{\tiny
 (p. 1b)   }
  \end{tabular} &
    \begin{tabular}{@{}l@{}}
    \hypertarget{dms-req-0165-v-01}{DMS-REQ-0165-V-01}
    \\\vcdJiraRef{LVV-68}~{\tiny
 (p. 1b)     }
    \end{tabular} &
        \begin{tabular}{@{}l@{}}
        \href{https://jira.lsstcorp.org/secure/Tests.jspa\#/testCase/LVV-T176}{LVV-T176} \\
        \vcdDocRef{LDM-639}
        \end{tabular} &
          & \notexec{} \\
          \cmidrule{3-5}
          & &
        \begin{tabular}{@{}l@{}}
        \href{https://jira.lsstcorp.org/secure/Tests.jspa\#/testCase/LVV-T287}{LVV-T287} \\
        \vcdDocRef{LDM-538}
        \end{tabular} &
          & \notexec{} \\
  \midrule
  \begin{tabular}{@{}l@{}}
  DMS-REQ-0166\\\vcdDocRef{LSE-61}~{\tiny
 (p. 1b)   }
  \end{tabular} &
    \begin{tabular}{@{}l@{}}
    \hypertarget{dms-req-0166-v-01}{DMS-REQ-0166-V-01}
    \\\vcdJiraRef{LVV-69}~{\tiny
 (p. 1b)     }
    \end{tabular} &
        \begin{tabular}{@{}l@{}}
        \href{https://jira.lsstcorp.org/secure/Tests.jspa\#/testCase/LVV-T177}{LVV-T177} \\
        \vcdDocRef{LDM-639}
        \end{tabular} &
          & \notexec{} \\
  \midrule
  \begin{tabular}{@{}l@{}}
  DMS-REQ-0167\\\vcdDocRef{LSE-61}~{\tiny
 (p. 2)   }
  \end{tabular} &
    \begin{tabular}{@{}l@{}}
    \hypertarget{dms-req-0167-v-01}{DMS-REQ-0167-V-01}
    \\\vcdJiraRef{LVV-70}~{\tiny
 (p. 2)     }
    \end{tabular} &
        \begin{tabular}{@{}l@{}}
        \href{https://jira.lsstcorp.org/secure/Tests.jspa\#/testCase/LVV-T178}{LVV-T178} \\
        \vcdDocRef{LDM-639}
        \end{tabular} &
          & \notexec{} \\
          \cmidrule{3-5}
          & &
        \begin{tabular}{@{}l@{}}
        \href{https://jira.lsstcorp.org/secure/Tests.jspa\#/testCase/LVV-T287}{LVV-T287} \\
        \vcdDocRef{LDM-538}
        \end{tabular} &
          & \notexec{} \\
  \midrule
  \begin{tabular}{@{}l@{}}
  DMS-REQ-0168\\\vcdDocRef{LSE-61}~{\tiny
 (p. 1a)   }
  \end{tabular} &
    \begin{tabular}{@{}l@{}}
    \hypertarget{dms-req-0168-v-01}{DMS-REQ-0168-V-01}
    \\\vcdJiraRef{LVV-71}~{\tiny
 (p. 1a)     }
    \end{tabular} &
        \begin{tabular}{@{}l@{}}
        \href{https://jira.lsstcorp.org/secure/Tests.jspa\#/testCase/LVV-T1097}{LVV-T1097} \\
        \vcdDocRef{LDM-639}
        \end{tabular} &
          & \notexec{} \\
  \midrule
  \begin{tabular}{@{}l@{}}
  DMS-REQ-0170\\\vcdDocRef{LSE-61}~{\tiny
 (p. 1b)   }
  \end{tabular} &
    \begin{tabular}{@{}l@{}}
    \hypertarget{dms-req-0170-v-01}{DMS-REQ-0170-V-01}
    \\\vcdJiraRef{LVV-72}~{\tiny
 (p. 1b)     }
    \end{tabular} &
        \begin{tabular}{@{}l@{}}
        \href{https://jira.lsstcorp.org/secure/Tests.jspa\#/testCase/LVV-T182}{LVV-T182} \\
        \vcdDocRef{LDM-639}
        \end{tabular} &
          & \notexec{} \\
  \midrule
  \begin{tabular}{@{}l@{}}
  DMS-REQ-0171\\\vcdDocRef{LSE-61}~{\tiny
 (p. 1a)   }
  \end{tabular} &
    \begin{tabular}{@{}l@{}}
    \hypertarget{dms-req-0171-v-01}{DMS-REQ-0171-V-01}
    \\\vcdJiraRef{LVV-73}~{\tiny
 (p. 1a)     }
    \end{tabular} &
        \begin{tabular}{@{}l@{}}
        \href{https://jira.lsstcorp.org/secure/Tests.jspa\#/testCase/LVV-T1168}{LVV-T1168} \\
        \vcdDocRef{LDM-639}
        \end{tabular} &
          \begin{tabular}{@{}l@{}}
          2019-07-09 \\
            \vcdDocRef{DMTR-151}
            {\scriptsize \href{https://jira.lsstcorp.org/secure/Tests.jspa\#/testPlan/LVV-P47}{LVV-P47} }
          \end{tabular} &
          \passed \\
          \cmidrule{3-5}
          & &
        \begin{tabular}{@{}l@{}}
        \href{https://jira.lsstcorp.org/secure/Tests.jspa\#/testCase/LVV-T1612}{LVV-T1612} \\
        \vcdDocRef{}
        \end{tabular} &
          & \notexec{} \\
  \midrule
  \begin{tabular}{@{}l@{}}
  DMS-REQ-0172\\\vcdDocRef{LSE-61}~{\tiny
 (p. 1b)   }
  \end{tabular} &
    \begin{tabular}{@{}l@{}}
    \hypertarget{dms-req-0172-v-01}{DMS-REQ-0172-V-01}
    \\\vcdJiraRef{LVV-74}~{\tiny
 (p. 1b)     }
    \end{tabular} &
        \begin{tabular}{@{}l@{}}
        \href{https://jira.lsstcorp.org/secure/Tests.jspa\#/testCase/LVV-T185}{LVV-T185} \\
        \vcdDocRef{LDM-639}
        \end{tabular} &
          & \notexec{} \\
  \midrule
  \begin{tabular}{@{}l@{}}
  DMS-REQ-0173\\\vcdDocRef{LSE-61}~{\tiny
 (p. 1b)   }
  \end{tabular} &
    \begin{tabular}{@{}l@{}}
    \hypertarget{dms-req-0173-v-01}{DMS-REQ-0173-V-01}
    \\\vcdJiraRef{LVV-75}~{\tiny
 (p. 1b)     }
    \end{tabular} &
        \begin{tabular}{@{}l@{}}
        \href{https://jira.lsstcorp.org/secure/Tests.jspa\#/testCase/LVV-T186}{LVV-T186} \\
        \vcdDocRef{LDM-639}
        \end{tabular} &
          & \notexec{} \\
  \midrule
  \begin{tabular}{@{}l@{}}
  DMS-REQ-0174\\\vcdDocRef{LSE-61}~{\tiny
 (p. 1b)   }
  \end{tabular} &
    \begin{tabular}{@{}l@{}}
    \hypertarget{dms-req-0174-v-01}{DMS-REQ-0174-V-01}
    \\\vcdJiraRef{LVV-76}~{\tiny
 (p. 1b)     }
    \end{tabular} &
        \begin{tabular}{@{}l@{}}
        \href{https://jira.lsstcorp.org/secure/Tests.jspa\#/testCase/LVV-T187}{LVV-T187} \\
        \vcdDocRef{LDM-639}
        \end{tabular} &
          & \notexec{} \\
  \midrule
  \begin{tabular}{@{}l@{}}
  DMS-REQ-0175\\\vcdDocRef{LSE-61}~{\tiny
 (p. 1b)   }
  \end{tabular} &
    \begin{tabular}{@{}l@{}}
    \hypertarget{dms-req-0175-v-01}{DMS-REQ-0175-V-01}
    \\\vcdJiraRef{LVV-77}~{\tiny
 (p. 1b)     }
    \end{tabular} &
        \begin{tabular}{@{}l@{}}
        \href{https://jira.lsstcorp.org/secure/Tests.jspa\#/testCase/LVV-T188}{LVV-T188} \\
        \vcdDocRef{LDM-639}
        \end{tabular} &
          & \notexec{} \\
  \midrule
  \begin{tabular}{@{}l@{}}
  DMS-REQ-0176\\\vcdDocRef{LSE-61}~{\tiny
 (p. 1a)   }
  \end{tabular} &
    \begin{tabular}{@{}l@{}}
    \hypertarget{dms-req-0176-v-01}{DMS-REQ-0176-V-01}
    \\\vcdJiraRef{LVV-78}~{\tiny
 (p. 1a)     }
    \end{tabular} &
        \begin{tabular}{@{}l@{}}
        \href{https://jira.lsstcorp.org/secure/Tests.jspa\#/testCase/LVV-T189}{LVV-T189} \\
        \vcdDocRef{LDM-639}
        \end{tabular} &
          & \notexec{} \\
  \midrule
  \begin{tabular}{@{}l@{}}
  DMS-REQ-0178\\\vcdDocRef{LSE-61}~{\tiny
 (p. 1b)   }
  \end{tabular} &
    \begin{tabular}{@{}l@{}}
    \hypertarget{dms-req-0178-v-01}{DMS-REQ-0178-V-01}
    \\\vcdJiraRef{LVV-80}~{\tiny
 (p. 1b)     }
    \end{tabular} &
        \begin{tabular}{@{}l@{}}
        \href{https://jira.lsstcorp.org/secure/Tests.jspa\#/testCase/LVV-T190}{LVV-T190} \\
        \vcdDocRef{LDM-639}
        \end{tabular} &
          & \notexec{} \\
  \midrule
  \begin{tabular}{@{}l@{}}
  DMS-REQ-0180\\\vcdDocRef{LSE-61}~{\tiny
 (p. 1b)   }
  \end{tabular} &
    \begin{tabular}{@{}l@{}}
    \hypertarget{dms-req-0180-v-01}{DMS-REQ-0180-V-01}
    \\\vcdJiraRef{LVV-81}~{\tiny
 (p. 1b)     }
    \end{tabular} &
        \begin{tabular}{@{}l@{}}
        \href{https://jira.lsstcorp.org/secure/Tests.jspa\#/testCase/LVV-T193}{LVV-T193} \\
        \vcdDocRef{LDM-639}
        \end{tabular} &
          & \notexec{} \\
  \midrule
  \begin{tabular}{@{}l@{}}
  DMS-REQ-0181\\\vcdDocRef{LSE-61}~{\tiny
 (p. 1b)   }
  \end{tabular} &
    \begin{tabular}{@{}l@{}}
    \hypertarget{dms-req-0181-v-01}{DMS-REQ-0181-V-01}
    \\\vcdJiraRef{LVV-82}~{\tiny
 (p. 1b)     }
    \end{tabular} &
        \begin{tabular}{@{}l@{}}
        \href{https://jira.lsstcorp.org/secure/Tests.jspa\#/testCase/LVV-T194}{LVV-T194} \\
        \vcdDocRef{LDM-639}
        \end{tabular} &
          & \notexec{} \\
  \midrule
  \begin{tabular}{@{}l@{}}
  DMS-REQ-0182\\\vcdDocRef{LSE-61}~{\tiny
 (p. 1b)   }
  \end{tabular} &
    \begin{tabular}{@{}l@{}}
    \hypertarget{dms-req-0182-v-01}{DMS-REQ-0182-V-01}
    \\\vcdJiraRef{LVV-83}~{\tiny
 (p. 1b)     }
    \end{tabular} &
        \begin{tabular}{@{}l@{}}
        \href{https://jira.lsstcorp.org/secure/Tests.jspa\#/testCase/LVV-T195}{LVV-T195} \\
        \vcdDocRef{LDM-639}
        \end{tabular} &
          & \notexec{} \\
  \midrule
  \begin{tabular}{@{}l@{}}
  DMS-REQ-0183\\\vcdDocRef{LSE-61}~{\tiny
 (p. 1b)   }
  \end{tabular} &
    \begin{tabular}{@{}l@{}}
    \hypertarget{dms-req-0183-v-01}{DMS-REQ-0183-V-01}
    \\\vcdJiraRef{LVV-84}~{\tiny
 (p. 1b)     }
    \end{tabular} &
        \begin{tabular}{@{}l@{}}
        \href{https://jira.lsstcorp.org/secure/Tests.jspa\#/testCase/LVV-T196}{LVV-T196} \\
        \vcdDocRef{LDM-639}
        \end{tabular} &
          & \notexec{} \\
  \midrule
  \begin{tabular}{@{}l@{}}
  DMS-REQ-0185\\\vcdDocRef{LSE-61}~{\tiny
 (p. 1a)   }
  \end{tabular} &
    \begin{tabular}{@{}l@{}}
    \hypertarget{dms-req-0185-v-01}{DMS-REQ-0185-V-01}
    \\\vcdJiraRef{LVV-85}~{\tiny
 (p. 1a)     }
    \end{tabular} &
        \begin{tabular}{@{}l@{}}
        \href{https://jira.lsstcorp.org/secure/Tests.jspa\#/testCase/LVV-T197}{LVV-T197} \\
        \vcdDocRef{LDM-639}
        \end{tabular} &
          & \notexec{} \\
  \midrule
  \begin{tabular}{@{}l@{}}
  DMS-REQ-0186\\\vcdDocRef{LSE-61}~{\tiny
 (p. 1a)   }
  \end{tabular} &
    \begin{tabular}{@{}l@{}}
    \hypertarget{dms-req-0186-v-01}{DMS-REQ-0186-V-01}
    \\\vcdJiraRef{LVV-86}~{\tiny
 (p. 1a)     }
    \end{tabular} &
        \begin{tabular}{@{}l@{}}
        \href{https://jira.lsstcorp.org/secure/Tests.jspa\#/testCase/LVV-T198}{LVV-T198} \\
        \vcdDocRef{LDM-639}
        \end{tabular} &
          & \notexec{} \\
  \midrule
  \begin{tabular}{@{}l@{}}
  DMS-REQ-0187\\\vcdDocRef{LSE-61}~{\tiny
 (p. 1b)   }
  \end{tabular} &
    \begin{tabular}{@{}l@{}}
    \hypertarget{dms-req-0187-v-01}{DMS-REQ-0187-V-01}
    \\\vcdJiraRef{LVV-87}~{\tiny
 (p. 1b)     }
    \end{tabular} &
        \begin{tabular}{@{}l@{}}
        \href{https://jira.lsstcorp.org/secure/Tests.jspa\#/testCase/LVV-T199}{LVV-T199} \\
        \vcdDocRef{LDM-639}
        \end{tabular} &
          & \notexec{} \\
  \midrule
  \begin{tabular}{@{}l@{}}
  DMS-REQ-0188\\\vcdDocRef{LSE-61}~{\tiny
 (p. 1b)   }
  \end{tabular} &
    \begin{tabular}{@{}l@{}}
    \hypertarget{dms-req-0188-v-01}{DMS-REQ-0188-V-01}
    \\\vcdJiraRef{LVV-88}~{\tiny
 (p. 1b)     }
    \end{tabular} &
        \begin{tabular}{@{}l@{}}
        \href{https://jira.lsstcorp.org/secure/Tests.jspa\#/testCase/LVV-T200}{LVV-T200} \\
        \vcdDocRef{LDM-639}
        \end{tabular} &
          & \notexec{} \\
  \midrule
  \begin{tabular}{@{}l@{}}
  DMS-REQ-0189\\\vcdDocRef{LSE-61}~{\tiny
 (p. 1b)   }
  \end{tabular} &
    \begin{tabular}{@{}l@{}}
    \hypertarget{dms-req-0189-v-01}{DMS-REQ-0189-V-01}
    \\\vcdJiraRef{LVV-89}~{\tiny
 (p. 1b)     }
    \end{tabular} &
        \begin{tabular}{@{}l@{}}
        \href{https://jira.lsstcorp.org/secure/Tests.jspa\#/testCase/LVV-T201}{LVV-T201} \\
        \vcdDocRef{LDM-639}
        \end{tabular} &
          & \notexec{} \\
  \midrule
  \begin{tabular}{@{}l@{}}
  DMS-REQ-0190\\\vcdDocRef{LSE-61}~{\tiny
 (p. 1b)   }
  \end{tabular} &
    \begin{tabular}{@{}l@{}}
    \hypertarget{dms-req-0190-v-01}{DMS-REQ-0190-V-01}
    \\\vcdJiraRef{LVV-90}~{\tiny
 (p. 1b)     }
    \end{tabular} &
        \begin{tabular}{@{}l@{}}
        \href{https://jira.lsstcorp.org/secure/Tests.jspa\#/testCase/LVV-T202}{LVV-T202} \\
        \vcdDocRef{LDM-639}
        \end{tabular} &
          & \notexec{} \\
  \midrule
  \begin{tabular}{@{}l@{}}
  DMS-REQ-0191\\\vcdDocRef{LSE-61}~{\tiny
 (p. 1b)   }
  \end{tabular} &
    \begin{tabular}{@{}l@{}}
    \hypertarget{dms-req-0191-v-01}{DMS-REQ-0191-V-01}
    \\\vcdJiraRef{LVV-91}~{\tiny
 (p. 1b)     }
    \end{tabular} &
        \begin{tabular}{@{}l@{}}
        \href{https://jira.lsstcorp.org/secure/Tests.jspa\#/testCase/LVV-T203}{LVV-T203} \\
        \vcdDocRef{LDM-639}
        \end{tabular} &
          & \notexec{} \\
  \midrule
  \begin{tabular}{@{}l@{}}
  DMS-REQ-0193\\\vcdDocRef{LSE-61}~{\tiny
 (p. 1b)   }
  \end{tabular} &
    \begin{tabular}{@{}l@{}}
    \hypertarget{dms-req-0193-v-01}{DMS-REQ-0193-V-01}
    \\\vcdJiraRef{LVV-92}~{\tiny
 (p. 1b)     }
    \end{tabular} &
        \begin{tabular}{@{}l@{}}
        \href{https://jira.lsstcorp.org/secure/Tests.jspa\#/testCase/LVV-T209}{LVV-T209} \\
        \vcdDocRef{LDM-639}
        \end{tabular} &
          & \notexec{} \\
  \midrule
  \begin{tabular}{@{}l@{}}
  DMS-REQ-0194\\\vcdDocRef{LSE-61}~{\tiny
 (p. 1b)   }
  \end{tabular} &
    \begin{tabular}{@{}l@{}}
    \hypertarget{dms-req-0194-v-01}{DMS-REQ-0194-V-01}
    \\\vcdJiraRef{LVV-93}~{\tiny
 (p. 1b)     }
    \end{tabular} &
        \begin{tabular}{@{}l@{}}
        \href{https://jira.lsstcorp.org/secure/Tests.jspa\#/testCase/LVV-T210}{LVV-T210} \\
        \vcdDocRef{LDM-639}
        \end{tabular} &
          & \notexec{} \\
  \midrule
  \begin{tabular}{@{}l@{}}
  DMS-REQ-0196\\\vcdDocRef{LSE-61}~{\tiny
 (p. 1b)   }
  \end{tabular} &
    \begin{tabular}{@{}l@{}}
    \hypertarget{dms-req-0196-v-01}{DMS-REQ-0196-V-01}
    \\\vcdJiraRef{LVV-94}~{\tiny
 (p. 1b)     }
    \end{tabular} &
        \begin{tabular}{@{}l@{}}
        \href{https://jira.lsstcorp.org/secure/Tests.jspa\#/testCase/LVV-T211}{LVV-T211} \\
        \vcdDocRef{LDM-639}
        \end{tabular} &
          & \notexec{} \\
  \midrule
  \begin{tabular}{@{}l@{}}
  DMS-REQ-0197\\\vcdDocRef{LSE-61}~{\tiny
 (p. 2)   }
  \end{tabular} &
    \begin{tabular}{@{}l@{}}
    \hypertarget{dms-req-0197-v-01}{DMS-REQ-0197-V-01}
    \\\vcdJiraRef{LVV-95}~{\tiny
 (p. 2)     }
    \end{tabular} &
        \begin{tabular}{@{}l@{}}
        \href{https://jira.lsstcorp.org/secure/Tests.jspa\#/testCase/LVV-T212}{LVV-T212} \\
        \vcdDocRef{LDM-639}
        \end{tabular} &
          & \notexec{} \\
  \midrule
  \begin{tabular}{@{}l@{}}
  DMS-REQ-0265\\\vcdDocRef{LSE-61}~{\tiny
 (p. 1a)   }
  \end{tabular} &
    \begin{tabular}{@{}l@{}}
    \hypertarget{dms-req-0265-v-01}{DMS-REQ-0265-V-01}
    \\\vcdJiraRef{LVV-96}~{\tiny
 (p. 1a)     }
    \end{tabular} &
        \begin{tabular}{@{}l@{}}
        \href{https://jira.lsstcorp.org/secure/Tests.jspa\#/testCase/LVV-T34}{LVV-T34} \\
        \vcdDocRef{LDM-639}
        \end{tabular} &
          & \notexec{} \\
          \cmidrule{3-5}
          & &
        \begin{tabular}{@{}l@{}}
        \href{https://jira.lsstcorp.org/secure/Tests.jspa\#/testCase/LVV-T283}{LVV-T283} \\
        \vcdDocRef{LDM-538}
        \end{tabular} &
          \begin{tabular}{@{}l@{}}
          2019-05-22 \\
            \vcdDocRef{DMTR-61}
            {\scriptsize \href{https://jira.lsstcorp.org/secure/Tests.jspa\#/testPlan/LVV-P45}{LVV-P45} }
          \end{tabular} &
          \passed \\
          \cmidrule{3-5}
          & &
        \begin{tabular}{@{}l@{}}
        \href{https://jira.lsstcorp.org/secure/Tests.jspa\#/testCase/LVV-T284}{LVV-T284} \\
        \vcdDocRef{LDM-538}
        \end{tabular} &
          \begin{tabular}{@{}l@{}}
          2019-06-24 \\
            \vcdDocRef{DMTR-102}
            {\scriptsize \href{https://jira.lsstcorp.org/secure/Tests.jspa\#/testPlan/LVV-P10}{LVV-P10} }
          \end{tabular} &
          \passed \\
  \midrule
  \begin{tabular}{@{}l@{}}
  DMS-REQ-0266\\\vcdDocRef{LSE-61}~{\tiny
 (p. 1a)   }
  \end{tabular} &
    \begin{tabular}{@{}l@{}}
    \hypertarget{dms-req-0266-v-01}{DMS-REQ-0266-V-01}
    \\\vcdJiraRef{LVV-97}~{\tiny
 (p. 1a)     }
    \end{tabular} &
        \begin{tabular}{@{}l@{}}
        \href{https://jira.lsstcorp.org/secure/Tests.jspa\#/testCase/LVV-T48}{LVV-T48} \\
        \vcdDocRef{LDM-639}
        \end{tabular} &
          & \notexec{} \\
  \midrule
  \begin{tabular}{@{}l@{}}
  DMS-REQ-0267\\\vcdDocRef{LSE-61}~{\tiny
 (p. 1b)   }
  \end{tabular} &
    \begin{tabular}{@{}l@{}}
    \hypertarget{dms-req-0267-v-01}{DMS-REQ-0267-V-01}
    \\\vcdJiraRef{LVV-98}~{\tiny
 (p. 1b)     }
    \end{tabular} &
        \begin{tabular}{@{}l@{}}
        \href{https://jira.lsstcorp.org/secure/Tests.jspa\#/testCase/LVV-T12}{LVV-T12} \\
        \vcdDocRef{LDM-534}
        \end{tabular} &
          \begin{tabular}{@{}l@{}}
          2019-05-22 \\
            \vcdDocRef{DMTR-51}
            {\scriptsize \href{https://jira.lsstcorp.org/secure/Tests.jspa\#/testPlan/LVV-P43}{LVV-P43} }
          \end{tabular} &
          \passed \\
          \cmidrule{3-5}
          & &
        \begin{tabular}{@{}l@{}}
        \href{https://jira.lsstcorp.org/secure/Tests.jspa\#/testCase/LVV-T13}{LVV-T13} \\
        \vcdDocRef{LDM-534}
        \end{tabular} &
          \begin{tabular}{@{}l@{}}
          2019-05-22 \\
            \vcdDocRef{DMTR-51}
            {\scriptsize \href{https://jira.lsstcorp.org/secure/Tests.jspa\#/testPlan/LVV-P43}{LVV-P43} }
          \end{tabular} &
          \passed \\
          \cmidrule{3-5}
          & &
        \begin{tabular}{@{}l@{}}
        \href{https://jira.lsstcorp.org/secure/Tests.jspa\#/testCase/LVV-T65}{LVV-T65} \\
        \vcdDocRef{LDM-639}
        \end{tabular} &
          & \notexec{} \\
          \cmidrule{3-5}
          & &
        \begin{tabular}{@{}l@{}}
        \href{https://jira.lsstcorp.org/secure/Tests.jspa\#/testCase/LVV-T362}{LVV-T362} \\
        \vcdDocRef{}
        \end{tabular} &
          \begin{tabular}{@{}l@{}}
          2019-03-31 \\
            \vcdDocRef{DMTR-111}
            {\scriptsize \href{https://jira.lsstcorp.org/secure/Tests.jspa\#/testPlan/LVV-P15}{LVV-P15} }
          \end{tabular} &
          \passed \\
  \midrule
  \begin{tabular}{@{}l@{}}
  DMS-REQ-0268\\\vcdDocRef{LSE-61}~{\tiny
 (p. 1b)   }
  \end{tabular} &
    \begin{tabular}{@{}l@{}}
    \hypertarget{dms-req-0268-v-01}{DMS-REQ-0268-V-01}
    \\\vcdJiraRef{LVV-99}~{\tiny
 (p. 1b)     }
    \end{tabular} &
        \begin{tabular}{@{}l@{}}
        \href{https://jira.lsstcorp.org/secure/Tests.jspa\#/testCase/LVV-T12}{LVV-T12} \\
        \vcdDocRef{LDM-534}
        \end{tabular} &
          \begin{tabular}{@{}l@{}}
          2019-05-22 \\
            \vcdDocRef{DMTR-51}
            {\scriptsize \href{https://jira.lsstcorp.org/secure/Tests.jspa\#/testPlan/LVV-P43}{LVV-P43} }
          \end{tabular} &
          \passed \\
          \cmidrule{3-5}
          & &
        \begin{tabular}{@{}l@{}}
        \href{https://jira.lsstcorp.org/secure/Tests.jspa\#/testCase/LVV-T66}{LVV-T66} \\
        \vcdDocRef{LDM-639}
        \end{tabular} &
          & \notexec{} \\
  \midrule
  \begin{tabular}{@{}l@{}}
  DMS-REQ-0269\\\vcdDocRef{LSE-61}~{\tiny
 (p. 1b)   }
  \end{tabular} &
    \begin{tabular}{@{}l@{}}
    \hypertarget{dms-req-0269-v-01}{DMS-REQ-0269-V-01}
    \\\vcdJiraRef{LVV-100}~{\tiny
 (p. 1b)     }
    \end{tabular} &
        \begin{tabular}{@{}l@{}}
        \href{https://jira.lsstcorp.org/secure/Tests.jspa\#/testCase/LVV-T18}{LVV-T18} \\
        \vcdDocRef{LDM-533}
        \end{tabular} &
          \begin{tabular}{@{}l@{}}
          2019-05-22 \\
            \vcdDocRef{DMTR-53}
            {\scriptsize \href{https://jira.lsstcorp.org/secure/Tests.jspa\#/testPlan/LVV-P44}{LVV-P44} }
          \end{tabular} &
          \passed \\
          \cmidrule{3-5}
          & &
        \begin{tabular}{@{}l@{}}
        \href{https://jira.lsstcorp.org/secure/Tests.jspa\#/testCase/LVV-T21}{LVV-T21} \\
        \vcdDocRef{LDM-533}
        \end{tabular} &
          \begin{tabular}{@{}l@{}}
          2019-05-22 \\
            \vcdDocRef{DMTR-53}
            {\scriptsize \href{https://jira.lsstcorp.org/secure/Tests.jspa\#/testPlan/LVV-P44}{LVV-P44} }
          \end{tabular} &
          \passed \\
          \cmidrule{3-5}
          & &
        \begin{tabular}{@{}l@{}}
        \href{https://jira.lsstcorp.org/secure/Tests.jspa\#/testCase/LVV-T49}{LVV-T49} \\
        \vcdDocRef{LDM-639}
        \end{tabular} &
          & \notexec{} \\
  \midrule
  \begin{tabular}{@{}l@{}}
  DMS-REQ-0270\\\vcdDocRef{LSE-61}~{\tiny
 (p. 2)   }
  \end{tabular} &
    \begin{tabular}{@{}l@{}}
    \hypertarget{dms-req-0270-v-01}{DMS-REQ-0270-V-01}
    \\\vcdJiraRef{LVV-101}~{\tiny
 (p. 2)     }
    \end{tabular} &
        \begin{tabular}{@{}l@{}}
        \href{https://jira.lsstcorp.org/secure/Tests.jspa\#/testCase/LVV-T21}{LVV-T21} \\
        \vcdDocRef{LDM-533}
        \end{tabular} &
          \begin{tabular}{@{}l@{}}
          2019-05-22 \\
            \vcdDocRef{DMTR-53}
            {\scriptsize \href{https://jira.lsstcorp.org/secure/Tests.jspa\#/testPlan/LVV-P44}{LVV-P44} }
          \end{tabular} &
          \passed \\
          \cmidrule{3-5}
          & &
        \begin{tabular}{@{}l@{}}
        \href{https://jira.lsstcorp.org/secure/Tests.jspa\#/testCase/LVV-T50}{LVV-T50} \\
        \vcdDocRef{LDM-639}
        \end{tabular} &
          & \notexec{} \\
  \midrule
  \begin{tabular}{@{}l@{}}
  DMS-REQ-0271\\\vcdDocRef{LSE-61}~{\tiny
 (p. 1b)   }
  \end{tabular} &
    \begin{tabular}{@{}l@{}}
    \hypertarget{dms-req-0271-v-01}{DMS-REQ-0271-V-01}
    \\\vcdJiraRef{LVV-102}~{\tiny
 (p. 1b)     }
    \end{tabular} &
        \begin{tabular}{@{}l@{}}
        \href{https://jira.lsstcorp.org/secure/Tests.jspa\#/testCase/LVV-T18}{LVV-T18} \\
        \vcdDocRef{LDM-533}
        \end{tabular} &
          \begin{tabular}{@{}l@{}}
          2019-05-22 \\
            \vcdDocRef{DMTR-53}
            {\scriptsize \href{https://jira.lsstcorp.org/secure/Tests.jspa\#/testPlan/LVV-P44}{LVV-P44} }
          \end{tabular} &
          \passed \\
          \cmidrule{3-5}
          & &
        \begin{tabular}{@{}l@{}}
        \href{https://jira.lsstcorp.org/secure/Tests.jspa\#/testCase/LVV-T22}{LVV-T22} \\
        \vcdDocRef{LDM-533}
        \end{tabular} &
          \begin{tabular}{@{}l@{}}
          2019-05-22 \\
            \vcdDocRef{DMTR-53}
            {\scriptsize \href{https://jira.lsstcorp.org/secure/Tests.jspa\#/testPlan/LVV-P44}{LVV-P44} }
          \end{tabular} &
          \passed \\
          \cmidrule{3-5}
          & &
        \begin{tabular}{@{}l@{}}
        \href{https://jira.lsstcorp.org/secure/Tests.jspa\#/testCase/LVV-T51}{LVV-T51} \\
        \vcdDocRef{LDM-639}
        \end{tabular} &
          & \notexec{} \\
      \cmidrule{2-5}
      &
    \begin{tabular}{@{}l@{}}
    \hypertarget{dms-req-0271-v-02}{DMS-REQ-0271-V-02}
    \\\vcdJiraRef{LVV-9742}~{\tiny
    }
    \end{tabular} &
        & & \\
      \cmidrule{2-5}
      &
    \begin{tabular}{@{}l@{}}
    \hypertarget{dms-req-0271-v-03}{DMS-REQ-0271-V-03}
    \\\vcdJiraRef{LVV-9743}~{\tiny
    }
    \end{tabular} &
        & & \\
  \midrule
  \begin{tabular}{@{}l@{}}
  DMS-REQ-0272\\\vcdDocRef{LSE-61}~{\tiny
 (p. 1b)   }
  \end{tabular} &
    \begin{tabular}{@{}l@{}}
    \hypertarget{dms-req-0272-v-01}{DMS-REQ-0272-V-01}
    \\\vcdJiraRef{LVV-103}~{\tiny
 (p. 1b)     }
    \end{tabular} &
        \begin{tabular}{@{}l@{}}
        \href{https://jira.lsstcorp.org/secure/Tests.jspa\#/testCase/LVV-T22}{LVV-T22} \\
        \vcdDocRef{LDM-533}
        \end{tabular} &
          \begin{tabular}{@{}l@{}}
          2019-05-22 \\
            \vcdDocRef{DMTR-53}
            {\scriptsize \href{https://jira.lsstcorp.org/secure/Tests.jspa\#/testPlan/LVV-P44}{LVV-P44} }
          \end{tabular} &
          \passed \\
          \cmidrule{3-5}
          & &
        \begin{tabular}{@{}l@{}}
        \href{https://jira.lsstcorp.org/secure/Tests.jspa\#/testCase/LVV-T52}{LVV-T52} \\
        \vcdDocRef{LDM-639}
        \end{tabular} &
          & \notexec{} \\
  \midrule
  \begin{tabular}{@{}l@{}}
  DMS-REQ-0273\\\vcdDocRef{LSE-61}~{\tiny
 (p. 2)   }
  \end{tabular} &
    \begin{tabular}{@{}l@{}}
    \hypertarget{dms-req-0273-v-01}{DMS-REQ-0273-V-01}
    \\\vcdJiraRef{LVV-104}~{\tiny
 (p. 2)     }
    \end{tabular} &
        \begin{tabular}{@{}l@{}}
        \href{https://jira.lsstcorp.org/secure/Tests.jspa\#/testCase/LVV-T53}{LVV-T53} \\
        \vcdDocRef{LDM-639}
        \end{tabular} &
          & \notexec{} \\
  \midrule
  \begin{tabular}{@{}l@{}}
  DMS-REQ-0274\\\vcdDocRef{LSE-61}~{\tiny
 (p. 1b)   }
  \end{tabular} &
    \begin{tabular}{@{}l@{}}
    \hypertarget{dms-req-0274-v-01}{DMS-REQ-0274-V-01}
    \\\vcdJiraRef{LVV-105}~{\tiny
 (p. 1b)     }
    \end{tabular} &
        \begin{tabular}{@{}l@{}}
        \href{https://jira.lsstcorp.org/secure/Tests.jspa\#/testCase/LVV-T54}{LVV-T54} \\
        \vcdDocRef{LDM-639}
        \end{tabular} &
          & \notexec{} \\
  \midrule
  \begin{tabular}{@{}l@{}}
  DMS-REQ-0275\\\vcdDocRef{LSE-61}~{\tiny
 (p. 1b)   }
  \end{tabular} &
    \begin{tabular}{@{}l@{}}
    \hypertarget{dms-req-0275-v-01}{DMS-REQ-0275-V-01}
    \\\vcdJiraRef{LVV-106}~{\tiny
 (p. 1b)     }
    \end{tabular} &
        \begin{tabular}{@{}l@{}}
        \href{https://jira.lsstcorp.org/secure/Tests.jspa\#/testCase/LVV-T12}{LVV-T12} \\
        \vcdDocRef{LDM-534}
        \end{tabular} &
          \begin{tabular}{@{}l@{}}
          2019-05-22 \\
            \vcdDocRef{DMTR-51}
            {\scriptsize \href{https://jira.lsstcorp.org/secure/Tests.jspa\#/testPlan/LVV-P43}{LVV-P43} }
          \end{tabular} &
          \passed \\
          \cmidrule{3-5}
          & &
        \begin{tabular}{@{}l@{}}
        \href{https://jira.lsstcorp.org/secure/Tests.jspa\#/testCase/LVV-T14}{LVV-T14} \\
        \vcdDocRef{LDM-534}
        \end{tabular} &
          \begin{tabular}{@{}l@{}}
          2019-05-22 \\
            \vcdDocRef{DMTR-51}
            {\scriptsize \href{https://jira.lsstcorp.org/secure/Tests.jspa\#/testPlan/LVV-P43}{LVV-P43} }
          \end{tabular} &
          \cndpass \\
          \cmidrule{3-5}
          & &
        \begin{tabular}{@{}l@{}}
        \href{https://jira.lsstcorp.org/secure/Tests.jspa\#/testCase/LVV-T67}{LVV-T67} \\
        \vcdDocRef{LDM-639}
        \end{tabular} &
          & \notexec{} \\
  \midrule
  \begin{tabular}{@{}l@{}}
  DMS-REQ-0276\\\vcdDocRef{LSE-61}~{\tiny
 (p. 1b)   }
  \end{tabular} &
    \begin{tabular}{@{}l@{}}
    \hypertarget{dms-req-0276-v-01}{DMS-REQ-0276-V-01}
    \\\vcdJiraRef{LVV-107}~{\tiny
 (p. 1b)     }
    \end{tabular} &
        \begin{tabular}{@{}l@{}}
        \href{https://jira.lsstcorp.org/secure/Tests.jspa\#/testCase/LVV-T69}{LVV-T69} \\
        \vcdDocRef{LDM-639}
        \end{tabular} &
          & \notexec{} \\
  \midrule
  \begin{tabular}{@{}l@{}}
  DMS-REQ-0277\\\vcdDocRef{LSE-61}~{\tiny
 (p. 1b)   }
  \end{tabular} &
    \begin{tabular}{@{}l@{}}
    \hypertarget{dms-req-0277-v-01}{DMS-REQ-0277-V-01}
    \\\vcdJiraRef{LVV-108}~{\tiny
 (p. 1b)     }
    \end{tabular} &
        & & \\
  \midrule
  \begin{tabular}{@{}l@{}}
  DMS-REQ-0278\\\vcdDocRef{LSE-61}~{\tiny
 (p. 1b)   }
  \end{tabular} &
    \begin{tabular}{@{}l@{}}
    \hypertarget{dms-req-0278-v-01}{DMS-REQ-0278-V-01}
    \\\vcdJiraRef{LVV-109}~{\tiny
 (p. 1b)     }
    \end{tabular} &
        \begin{tabular}{@{}l@{}}
        \href{https://jira.lsstcorp.org/secure/Tests.jspa\#/testCase/LVV-T16}{LVV-T16} \\
        \vcdDocRef{LDM-534}
        \end{tabular} &
          \begin{tabular}{@{}l@{}}
          2019-05-22 \\
            \vcdDocRef{DMTR-51}
            {\scriptsize \href{https://jira.lsstcorp.org/secure/Tests.jspa\#/testPlan/LVV-P43}{LVV-P43} }
          \end{tabular} &
          \passed \\
          \cmidrule{3-5}
          & &
        \begin{tabular}{@{}l@{}}
        \href{https://jira.lsstcorp.org/secure/Tests.jspa\#/testCase/LVV-T72}{LVV-T72} \\
        \vcdDocRef{LDM-639}
        \end{tabular} &
          & \notexec{} \\
  \midrule
  \begin{tabular}{@{}l@{}}
  DMS-REQ-0279\\\vcdDocRef{LSE-61}~{\tiny
 (p. 1b)   }
  \end{tabular} &
    \begin{tabular}{@{}l@{}}
    \hypertarget{dms-req-0279-v-01}{DMS-REQ-0279-V-01}
    \\\vcdJiraRef{LVV-110}~{\tiny
 (p. 1b)     }
    \end{tabular} &
        \begin{tabular}{@{}l@{}}
        \href{https://jira.lsstcorp.org/secure/Tests.jspa\#/testCase/LVV-T12}{LVV-T12} \\
        \vcdDocRef{LDM-534}
        \end{tabular} &
          \begin{tabular}{@{}l@{}}
          2019-05-22 \\
            \vcdDocRef{DMTR-51}
            {\scriptsize \href{https://jira.lsstcorp.org/secure/Tests.jspa\#/testPlan/LVV-P43}{LVV-P43} }
          \end{tabular} &
          \passed \\
          \cmidrule{3-5}
          & &
        \begin{tabular}{@{}l@{}}
        \href{https://jira.lsstcorp.org/secure/Tests.jspa\#/testCase/LVV-T16}{LVV-T16} \\
        \vcdDocRef{LDM-534}
        \end{tabular} &
          \begin{tabular}{@{}l@{}}
          2019-05-22 \\
            \vcdDocRef{DMTR-51}
            {\scriptsize \href{https://jira.lsstcorp.org/secure/Tests.jspa\#/testPlan/LVV-P43}{LVV-P43} }
          \end{tabular} &
          \passed \\
          \cmidrule{3-5}
          & &
        \begin{tabular}{@{}l@{}}
        \href{https://jira.lsstcorp.org/secure/Tests.jspa\#/testCase/LVV-T73}{LVV-T73} \\
        \vcdDocRef{LDM-639}
        \end{tabular} &
          & \notexec{} \\
  \midrule
  \begin{tabular}{@{}l@{}}
  DMS-REQ-0280\\\vcdDocRef{LSE-61}~{\tiny
 (p. 1b)   }
  \end{tabular} &
    \begin{tabular}{@{}l@{}}
    \hypertarget{dms-req-0280-v-01}{DMS-REQ-0280-V-01}
    \\\vcdJiraRef{LVV-111}~{\tiny
 (p. 1b)     }
    \end{tabular} &
        \begin{tabular}{@{}l@{}}
        \href{https://jira.lsstcorp.org/secure/Tests.jspa\#/testCase/LVV-T74}{LVV-T74} \\
        \vcdDocRef{LDM-639}
        \end{tabular} &
          & \notexec{} \\
  \midrule
  \begin{tabular}{@{}l@{}}
  DMS-REQ-0281\\\vcdDocRef{LSE-61}~{\tiny
 (p. 1b)   }
  \end{tabular} &
    \begin{tabular}{@{}l@{}}
    \hypertarget{dms-req-0281-v-01}{DMS-REQ-0281-V-01}
    \\\vcdJiraRef{LVV-112}~{\tiny
 (p. 1b)     }
    \end{tabular} &
        \begin{tabular}{@{}l@{}}
        \href{https://jira.lsstcorp.org/secure/Tests.jspa\#/testCase/LVV-T75}{LVV-T75} \\
        \vcdDocRef{LDM-639}
        \end{tabular} &
          & \notexec{} \\
  \midrule
  \begin{tabular}{@{}l@{}}
  DMS-REQ-0282\\\vcdDocRef{LSE-61}~{\tiny
 (p. 1a)   }
  \end{tabular} &
    \begin{tabular}{@{}l@{}}
    \hypertarget{dms-req-0282-v-01}{DMS-REQ-0282-V-01}
    \\\vcdJiraRef{LVV-113}~{\tiny
 (p. 1a)     }
    \end{tabular} &
        \begin{tabular}{@{}l@{}}
        \href{https://jira.lsstcorp.org/secure/Tests.jspa\#/testCase/LVV-T90}{LVV-T90} \\
        \vcdDocRef{LDM-639}
        \end{tabular} &
          & \notexec{} \\
      \cmidrule{2-5}
      &
    \begin{tabular}{@{}l@{}}
    \hypertarget{dms-req-0282-v-02}{DMS-REQ-0282-V-02}
    \\\vcdJiraRef{LVV-18881}~{\tiny
    }
    \end{tabular} &
        \begin{tabular}{@{}l@{}}
        \href{https://jira.lsstcorp.org/secure/Tests.jspa\#/testCase/LVV-T1862}{LVV-T1862} \\
        \vcdDocRef{LDM-639}
        \end{tabular} &
          & \notexec{} \\
  \midrule
  \begin{tabular}{@{}l@{}}
  DMS-REQ-0283\\\vcdDocRef{LSE-61}~{\tiny
 (p. 1b)   }
  \end{tabular} &
    \begin{tabular}{@{}l@{}}
    \hypertarget{dms-req-0283-v-01}{DMS-REQ-0283-V-01}
    \\\vcdJiraRef{LVV-114}~{\tiny
 (p. 1b)     }
    \end{tabular} &
        \begin{tabular}{@{}l@{}}
        \href{https://jira.lsstcorp.org/secure/Tests.jspa\#/testCase/LVV-T91}{LVV-T91} \\
        \vcdDocRef{LDM-639}
        \end{tabular} &
          & \notexec{} \\
  \midrule
  \begin{tabular}{@{}l@{}}
  DMS-REQ-0284\\\vcdDocRef{LSE-61}~{\tiny
 (p. 1b)   }
  \end{tabular} &
    \begin{tabular}{@{}l@{}}
    \hypertarget{dms-req-0284-v-01}{DMS-REQ-0284-V-01}
    \\\vcdJiraRef{LVV-115}~{\tiny
 (p. 1b)     }
    \end{tabular} &
        \begin{tabular}{@{}l@{}}
        \href{https://jira.lsstcorp.org/secure/Tests.jspa\#/testCase/LVV-T107}{LVV-T107} \\
        \vcdDocRef{LDM-639}
        \end{tabular} &
          & \notexec{} \\
          \cmidrule{3-5}
          & &
        \begin{tabular}{@{}l@{}}
        \href{https://jira.lsstcorp.org/secure/Tests.jspa\#/testCase/LVV-T283}{LVV-T283} \\
        \vcdDocRef{LDM-538}
        \end{tabular} &
          \begin{tabular}{@{}l@{}}
          2019-05-22 \\
            \vcdDocRef{DMTR-61}
            {\scriptsize \href{https://jira.lsstcorp.org/secure/Tests.jspa\#/testPlan/LVV-P45}{LVV-P45} }
          \end{tabular} &
          \passed \\
          \cmidrule{3-5}
          & &
        \begin{tabular}{@{}l@{}}
        \href{https://jira.lsstcorp.org/secure/Tests.jspa\#/testCase/LVV-T284}{LVV-T284} \\
        \vcdDocRef{LDM-538}
        \end{tabular} &
          \begin{tabular}{@{}l@{}}
          2019-06-24 \\
            \vcdDocRef{DMTR-102}
            {\scriptsize \href{https://jira.lsstcorp.org/secure/Tests.jspa\#/testPlan/LVV-P10}{LVV-P10} }
          \end{tabular} &
          \passed \\
          \cmidrule{3-5}
          & &
        \begin{tabular}{@{}l@{}}
        \href{https://jira.lsstcorp.org/secure/Tests.jspa\#/testCase/LVV-T286}{LVV-T286} \\
        \vcdDocRef{LDM-538}
        \end{tabular} &
          \begin{tabular}{@{}l@{}}
          2019-05-22 \\
            \vcdDocRef{DMTR-61}
            {\scriptsize \href{https://jira.lsstcorp.org/secure/Tests.jspa\#/testPlan/LVV-P45}{LVV-P45} }
          \end{tabular} &
          \passed \\
  \midrule
  \begin{tabular}{@{}l@{}}
  DMS-REQ-0285\\\vcdDocRef{LSE-61}~{\tiny
 (p. 1b)   }
  \end{tabular} &
    \begin{tabular}{@{}l@{}}
    \hypertarget{dms-req-0285-v-01}{DMS-REQ-0285-V-01}
    \\\vcdJiraRef{LVV-116}~{\tiny
 (p. 1b)     }
    \end{tabular} &
        \begin{tabular}{@{}l@{}}
        \href{https://jira.lsstcorp.org/secure/Tests.jspa\#/testCase/LVV-T22}{LVV-T22} \\
        \vcdDocRef{LDM-533}
        \end{tabular} &
          \begin{tabular}{@{}l@{}}
          2019-05-22 \\
            \vcdDocRef{DMTR-53}
            {\scriptsize \href{https://jira.lsstcorp.org/secure/Tests.jspa\#/testPlan/LVV-P44}{LVV-P44} }
          \end{tabular} &
          \passed \\
          \cmidrule{3-5}
          & &
        \begin{tabular}{@{}l@{}}
        \href{https://jira.lsstcorp.org/secure/Tests.jspa\#/testCase/LVV-T108}{LVV-T108} \\
        \vcdDocRef{LDM-639}
        \end{tabular} &
          & \notexec{} \\
          \cmidrule{3-5}
          & &
        \begin{tabular}{@{}l@{}}
        \href{https://jira.lsstcorp.org/secure/Tests.jspa\#/testCase/LVV-T550}{LVV-T550} \\
        \vcdDocRef{LSE-419}
        \end{tabular} &
          & \notexec{} \\
  \midrule
  \begin{tabular}{@{}l@{}}
  DMS-REQ-0286\\\vcdDocRef{LSE-61}~{\tiny
 (p. 2)   }
  \end{tabular} &
    \begin{tabular}{@{}l@{}}
    \hypertarget{dms-req-0286-v-01}{DMS-REQ-0286-V-01}
    \\\vcdJiraRef{LVV-117}~{\tiny
 (p. 2)     }
    \end{tabular} &
        \begin{tabular}{@{}l@{}}
        \href{https://jira.lsstcorp.org/secure/Tests.jspa\#/testCase/LVV-T109}{LVV-T109} \\
        \vcdDocRef{LDM-639}
        \end{tabular} &
          & \notexec{} \\
  \midrule
  \begin{tabular}{@{}l@{}}
  DMS-REQ-0287\\\vcdDocRef{LSE-61}~{\tiny
 (p. 1b)   }
  \end{tabular} &
    \begin{tabular}{@{}l@{}}
    \hypertarget{dms-req-0287-v-01}{DMS-REQ-0287-V-01}
    \\\vcdJiraRef{LVV-118}~{\tiny
    }
    \end{tabular} &
        \begin{tabular}{@{}l@{}}
        \href{https://jira.lsstcorp.org/secure/Tests.jspa\#/testCase/LVV-T110}{LVV-T110} \\
        \vcdDocRef{LDM-639}
        \end{tabular} &
          & \notexec{} \\
      \cmidrule{2-5}
      &
    \begin{tabular}{@{}l@{}}
    \hypertarget{dms-req-0287-v-02}{DMS-REQ-0287-V-02}
    \\\vcdJiraRef{LVV-9746}~{\tiny
    }
    \end{tabular} &
        & & \\
      \cmidrule{2-5}
      &
    \begin{tabular}{@{}l@{}}
    \hypertarget{dms-req-0287-v-03}{DMS-REQ-0287-V-03}
    \\\vcdJiraRef{LVV-9747}~{\tiny
    }
    \end{tabular} &
        & & \\
  \midrule
  \begin{tabular}{@{}l@{}}
  DMS-REQ-0288\\\vcdDocRef{LSE-61}~{\tiny
 (p. 2)   }
  \end{tabular} &
    \begin{tabular}{@{}l@{}}
    \hypertarget{dms-req-0288-v-01}{DMS-REQ-0288-V-01}
    \\\vcdJiraRef{LVV-119}~{\tiny
 (p. 2)     }
    \end{tabular} &
        \begin{tabular}{@{}l@{}}
        \href{https://jira.lsstcorp.org/secure/Tests.jspa\#/testCase/LVV-T111}{LVV-T111} \\
        \vcdDocRef{LDM-639}
        \end{tabular} &
          & \notexec{} \\
  \midrule
  \begin{tabular}{@{}l@{}}
  DMS-REQ-0289\\\vcdDocRef{LSE-61}~{\tiny
 (p. 1a)   }
  \end{tabular} &
    \begin{tabular}{@{}l@{}}
    \hypertarget{dms-req-0289-v-01}{DMS-REQ-0289-V-01}
    \\\vcdJiraRef{LVV-120}~{\tiny
 (p. 1a)     }
    \end{tabular} &
        \begin{tabular}{@{}l@{}}
        \href{https://jira.lsstcorp.org/secure/Tests.jspa\#/testCase/LVV-T115}{LVV-T115} \\
        \vcdDocRef{LDM-639}
        \end{tabular} &
          & \notexec{} \\
  \midrule
  \begin{tabular}{@{}l@{}}
  DMS-REQ-0290\\\vcdDocRef{LSE-61}~{\tiny
 (p. 2)   }
  \end{tabular} &
    \begin{tabular}{@{}l@{}}
    \hypertarget{dms-req-0290-v-01}{DMS-REQ-0290-V-01}
    \\\vcdJiraRef{LVV-121}~{\tiny
 (p. 2)     }
    \end{tabular} &
        \begin{tabular}{@{}l@{}}
        \href{https://jira.lsstcorp.org/secure/Tests.jspa\#/testCase/LVV-T122}{LVV-T122} \\
        \vcdDocRef{LDM-639}
        \end{tabular} &
          & \notexec{} \\
  \midrule
  \begin{tabular}{@{}l@{}}
  DMS-REQ-0291\\\vcdDocRef{LSE-61}~{\tiny
 (p. 1b)   }
  \end{tabular} &
    \begin{tabular}{@{}l@{}}
    \hypertarget{dms-req-0291-v-01}{DMS-REQ-0291-V-01}
    \\\vcdJiraRef{LVV-122}~{\tiny
 (p. 1b)     }
    \end{tabular} &
        \begin{tabular}{@{}l@{}}
        \href{https://jira.lsstcorp.org/secure/Tests.jspa\#/testCase/LVV-T96}{LVV-T96} \\
        \vcdDocRef{LDM-639}
        \end{tabular} &
          & \notexec{} \\
  \midrule
  \begin{tabular}{@{}l@{}}
  DMS-REQ-0292\\\vcdDocRef{LSE-61}~{\tiny
 (p. 1a)   }
  \end{tabular} &
    \begin{tabular}{@{}l@{}}
    \hypertarget{dms-req-0292-v-01}{DMS-REQ-0292-V-01}
    \\\vcdJiraRef{LVV-123}~{\tiny
 (p. 1a)     }
    \end{tabular} &
        \begin{tabular}{@{}l@{}}
        \href{https://jira.lsstcorp.org/secure/Tests.jspa\#/testCase/LVV-T97}{LVV-T97} \\
        \vcdDocRef{LDM-639}
        \end{tabular} &
          & \notexec{} \\
  \midrule
  \begin{tabular}{@{}l@{}}
  DMS-REQ-0293\\\vcdDocRef{LSE-61}~{\tiny
 (p. 1a)   }
  \end{tabular} &
    \begin{tabular}{@{}l@{}}
    \hypertarget{dms-req-0293-v-01}{DMS-REQ-0293-V-01}
    \\\vcdJiraRef{LVV-124}~{\tiny
 (p. 1a)     }
    \end{tabular} &
        \begin{tabular}{@{}l@{}}
        \href{https://jira.lsstcorp.org/secure/Tests.jspa\#/testCase/LVV-T11}{LVV-T11} \\
        \vcdDocRef{LDM-534}
        \end{tabular} &
          \begin{tabular}{@{}l@{}}
          2019-05-22 \\
            \vcdDocRef{DMTR-51}
            {\scriptsize \href{https://jira.lsstcorp.org/secure/Tests.jspa\#/testPlan/LVV-P43}{LVV-P43} }
          \end{tabular} &
          \passed \\
          \cmidrule{3-5}
          & &
        \begin{tabular}{@{}l@{}}
        \href{https://jira.lsstcorp.org/secure/Tests.jspa\#/testCase/LVV-T98}{LVV-T98} \\
        \vcdDocRef{LDM-639}
        \end{tabular} &
          & \notexec{} \\
  \midrule
  \begin{tabular}{@{}l@{}}
  DMS-REQ-0294\\\vcdDocRef{LSE-61}~{\tiny
 (p. 1b)   }
  \end{tabular} &
    \begin{tabular}{@{}l@{}}
    \hypertarget{dms-req-0294-v-01}{DMS-REQ-0294-V-01}
    \\\vcdJiraRef{LVV-125}~{\tiny
 (p. 1b)     }
    \end{tabular} &
        \begin{tabular}{@{}l@{}}
        \href{https://jira.lsstcorp.org/secure/Tests.jspa\#/testCase/LVV-T12}{LVV-T12} \\
        \vcdDocRef{LDM-534}
        \end{tabular} &
          \begin{tabular}{@{}l@{}}
          2019-05-22 \\
            \vcdDocRef{DMTR-51}
            {\scriptsize \href{https://jira.lsstcorp.org/secure/Tests.jspa\#/testPlan/LVV-P43}{LVV-P43} }
          \end{tabular} &
          \passed \\
          \cmidrule{3-5}
          & &
        \begin{tabular}{@{}l@{}}
        \href{https://jira.lsstcorp.org/secure/Tests.jspa\#/testCase/LVV-T99}{LVV-T99} \\
        \vcdDocRef{LDM-639}
        \end{tabular} &
          & \notexec{} \\
  \midrule
  \begin{tabular}{@{}l@{}}
  DMS-REQ-0295\\\vcdDocRef{LSE-61}~{\tiny
 (p. 2)   }
  \end{tabular} &
    \begin{tabular}{@{}l@{}}
    \hypertarget{dms-req-0295-v-01}{DMS-REQ-0295-V-01}
    \\\vcdJiraRef{LVV-126}~{\tiny
 (p. 2)     }
    \end{tabular} &
        \begin{tabular}{@{}l@{}}
        \href{https://jira.lsstcorp.org/secure/Tests.jspa\#/testCase/LVV-T100}{LVV-T100} \\
        \vcdDocRef{LDM-639}
        \end{tabular} &
          & \notexec{} \\
  \midrule
  \begin{tabular}{@{}l@{}}
  DMS-REQ-0296\\\vcdDocRef{LSE-61}~{\tiny
 (p. 1a)   }
  \end{tabular} &
    \begin{tabular}{@{}l@{}}
    \hypertarget{dms-req-0296-v-01}{DMS-REQ-0296-V-01}
    \\\vcdJiraRef{LVV-127}~{\tiny
 (p. 1a)     }
    \end{tabular} &
        \begin{tabular}{@{}l@{}}
        \href{https://jira.lsstcorp.org/secure/Tests.jspa\#/testCase/LVV-T132}{LVV-T132} \\
        \vcdDocRef{LDM-639}
        \end{tabular} &
          \begin{tabular}{@{}l@{}}
          2020-02-06 \\
            \vcdDocRef{DMTR-201}
            {\scriptsize \href{https://jira.lsstcorp.org/secure/Tests.jspa\#/testPlan/LVV-P65}{LVV-P65} }
          \end{tabular} &
          \passed \\
          \cmidrule{3-5}
          & &
        \begin{tabular}{@{}l@{}}
        \href{https://jira.lsstcorp.org/secure/Tests.jspa\#/testCase/LVV-T362}{LVV-T362} \\
        \vcdDocRef{}
        \end{tabular} &
          \begin{tabular}{@{}l@{}}
          2019-03-31 \\
            \vcdDocRef{DMTR-111}
            {\scriptsize \href{https://jira.lsstcorp.org/secure/Tests.jspa\#/testPlan/LVV-P15}{LVV-P15} }
          \end{tabular} &
          \passed \\
  \midrule
  \begin{tabular}{@{}l@{}}
  DMS-REQ-0297\\\vcdDocRef{LSE-61}~{\tiny
 (p. 1a)   }
  \end{tabular} &
    \begin{tabular}{@{}l@{}}
    \hypertarget{dms-req-0297-v-01}{DMS-REQ-0297-V-01}
    \\\vcdJiraRef{LVV-128}~{\tiny
 (p. 1a)     }
    \end{tabular} &
        \begin{tabular}{@{}l@{}}
        \href{https://jira.lsstcorp.org/secure/Tests.jspa\#/testCase/LVV-T146}{LVV-T146} \\
        \vcdDocRef{LDM-639}
        \end{tabular} &
          & \notexec{} \\
  \midrule
  \begin{tabular}{@{}l@{}}
  DMS-REQ-0298\\\vcdDocRef{LSE-61}~{\tiny
 (p. 1a)   }
  \end{tabular} &
    \begin{tabular}{@{}l@{}}
    \hypertarget{dms-req-0298-v-01}{DMS-REQ-0298-V-01}
    \\\vcdJiraRef{LVV-129}~{\tiny
 (p. 1a)     }
    \end{tabular} &
        \begin{tabular}{@{}l@{}}
        \href{https://jira.lsstcorp.org/secure/Tests.jspa\#/testCase/LVV-T136}{LVV-T136} \\
        \vcdDocRef{LDM-639}
        \end{tabular} &
          & \notexec{} \\
          \cmidrule{3-5}
          & &
        \begin{tabular}{@{}l@{}}
        \href{https://jira.lsstcorp.org/secure/Tests.jspa\#/testCase/LVV-T368}{LVV-T368} \\
        \vcdDocRef{}
        \end{tabular} &
          \begin{tabular}{@{}l@{}}
          2018-12-06 \\
            \vcdDocRef{DMTR-112}
            {\scriptsize \href{https://jira.lsstcorp.org/secure/Tests.jspa\#/testPlan/LVV-P16}{LVV-P16} }
          \end{tabular} &
          \passed \\
          \cmidrule{3-5}
          & &
        \begin{tabular}{@{}l@{}}
        \href{https://jira.lsstcorp.org/secure/Tests.jspa\#/testCase/LVV-T374}{LVV-T374} \\
        \vcdDocRef{}
        \end{tabular} &
          \begin{tabular}{@{}l@{}}
          2018-12-06 \\
            \vcdDocRef{DMTR-112}
            {\scriptsize \href{https://jira.lsstcorp.org/secure/Tests.jspa\#/testPlan/LVV-P16}{LVV-P16} }
          \end{tabular} &
          \passed \\
  \midrule
  \begin{tabular}{@{}l@{}}
  DMS-REQ-0299\\\vcdDocRef{LSE-61}~{\tiny
 (p. 1a)   }
  \end{tabular} &
    \begin{tabular}{@{}l@{}}
    \hypertarget{dms-req-0299-v-01}{DMS-REQ-0299-V-01}
    \\\vcdJiraRef{LVV-130}~{\tiny
 (p. 1a)     }
    \end{tabular} &
        \begin{tabular}{@{}l@{}}
        \href{https://jira.lsstcorp.org/secure/Tests.jspa\#/testCase/LVV-T137}{LVV-T137} \\
        \vcdDocRef{LDM-639}
        \end{tabular} &
          & \notexec{} \\
          \cmidrule{3-5}
          & &
        \begin{tabular}{@{}l@{}}
        \href{https://jira.lsstcorp.org/secure/Tests.jspa\#/testCase/LVV-T374}{LVV-T374} \\
        \vcdDocRef{}
        \end{tabular} &
          \begin{tabular}{@{}l@{}}
          2018-12-06 \\
            \vcdDocRef{DMTR-112}
            {\scriptsize \href{https://jira.lsstcorp.org/secure/Tests.jspa\#/testPlan/LVV-P16}{LVV-P16} }
          \end{tabular} &
          \passed \\
  \midrule
  \begin{tabular}{@{}l@{}}
  DMS-REQ-0300\\\vcdDocRef{LSE-61}~{\tiny
 (p. 1b)   }
  \end{tabular} &
    \begin{tabular}{@{}l@{}}
    \hypertarget{dms-req-0300-v-01}{DMS-REQ-0300-V-01}
    \\\vcdJiraRef{LVV-131}~{\tiny
 (p. 1b)     }
    \end{tabular} &
        \begin{tabular}{@{}l@{}}
        \href{https://jira.lsstcorp.org/secure/Tests.jspa\#/testCase/LVV-T138}{LVV-T138} \\
        \vcdDocRef{LDM-639}
        \end{tabular} &
          & \notexec{} \\
  \midrule
  \begin{tabular}{@{}l@{}}
  DMS-REQ-0301\\\vcdDocRef{LSE-61}~{\tiny
 (p. 1b)   }
  \end{tabular} &
    \begin{tabular}{@{}l@{}}
    \hypertarget{dms-req-0301-v-01}{DMS-REQ-0301-V-01}
    \\\vcdJiraRef{LVV-132}~{\tiny
 (p. 1b)     }
    \end{tabular} &
        \begin{tabular}{@{}l@{}}
        \href{https://jira.lsstcorp.org/secure/Tests.jspa\#/testCase/LVV-T147}{LVV-T147} \\
        \vcdDocRef{LDM-639}
        \end{tabular} &
          & \notexec{} \\
  \midrule
  \begin{tabular}{@{}l@{}}
  DMS-REQ-0302\\\vcdDocRef{LSE-61}~{\tiny
 (p. 1a)   }
  \end{tabular} &
    \begin{tabular}{@{}l@{}}
    \hypertarget{dms-req-0302-v-01}{DMS-REQ-0302-V-01}
    \\\vcdJiraRef{LVV-133}~{\tiny
 (p. 1a)     }
    \end{tabular} &
        \begin{tabular}{@{}l@{}}
        \href{https://jira.lsstcorp.org/secure/Tests.jspa\#/testCase/LVV-T11}{LVV-T11} \\
        \vcdDocRef{LDM-534}
        \end{tabular} &
          \begin{tabular}{@{}l@{}}
          2019-05-22 \\
            \vcdDocRef{DMTR-51}
            {\scriptsize \href{https://jira.lsstcorp.org/secure/Tests.jspa\#/testPlan/LVV-P43}{LVV-P43} }
          \end{tabular} &
          \passed \\
          \cmidrule{3-5}
          & &
        \begin{tabular}{@{}l@{}}
        \href{https://jira.lsstcorp.org/secure/Tests.jspa\#/testCase/LVV-T140}{LVV-T140} \\
        \vcdDocRef{LDM-639}
        \end{tabular} &
          & \notexec{} \\
  \midrule
  \begin{tabular}{@{}l@{}}
  DMS-REQ-0303\\\vcdDocRef{LSE-61}~{\tiny
 (p. 1a)   }
  \end{tabular} &
    \begin{tabular}{@{}l@{}}
    \hypertarget{dms-req-0303-v-01}{DMS-REQ-0303-V-01}
    \\\vcdJiraRef{LVV-134}~{\tiny
 (p. 1a)     }
    \end{tabular} &
        \begin{tabular}{@{}l@{}}
        \href{https://jira.lsstcorp.org/secure/Tests.jspa\#/testCase/LVV-T11}{LVV-T11} \\
        \vcdDocRef{LDM-534}
        \end{tabular} &
          \begin{tabular}{@{}l@{}}
          2019-05-22 \\
            \vcdDocRef{DMTR-51}
            {\scriptsize \href{https://jira.lsstcorp.org/secure/Tests.jspa\#/testPlan/LVV-P43}{LVV-P43} }
          \end{tabular} &
          \passed \\
          \cmidrule{3-5}
          & &
        \begin{tabular}{@{}l@{}}
        \href{https://jira.lsstcorp.org/secure/Tests.jspa\#/testCase/LVV-T141}{LVV-T141} \\
        \vcdDocRef{LDM-639}
        \end{tabular} &
          & \notexec{} \\
  \midrule
  \begin{tabular}{@{}l@{}}
  DMS-REQ-0304\\\vcdDocRef{LSE-61}~{\tiny
 (p. 1a)   }
  \end{tabular} &
    \begin{tabular}{@{}l@{}}
    \hypertarget{dms-req-0304-v-01}{DMS-REQ-0304-V-01}
    \\\vcdJiraRef{LVV-135}~{\tiny
 (p. 1a)     }
    \end{tabular} &
        \begin{tabular}{@{}l@{}}
        \href{https://jira.lsstcorp.org/secure/Tests.jspa\#/testCase/LVV-T142}{LVV-T142} \\
        \vcdDocRef{LDM-639}
        \end{tabular} &
          & \notexec{} \\
  \midrule
  \begin{tabular}{@{}l@{}}
  DMS-REQ-0305\\\vcdDocRef{LSE-61}~{\tiny
 (p. 1a)   }
  \end{tabular} &
    \begin{tabular}{@{}l@{}}
    \hypertarget{dms-req-0305-v-01}{DMS-REQ-0305-V-01}
    \\\vcdJiraRef{LVV-136}~{\tiny
 (p. 1a)     }
    \end{tabular} &
        \begin{tabular}{@{}l@{}}
        \href{https://jira.lsstcorp.org/secure/Tests.jspa\#/testCase/LVV-T11}{LVV-T11} \\
        \vcdDocRef{LDM-534}
        \end{tabular} &
          \begin{tabular}{@{}l@{}}
          2019-05-22 \\
            \vcdDocRef{DMTR-51}
            {\scriptsize \href{https://jira.lsstcorp.org/secure/Tests.jspa\#/testPlan/LVV-P43}{LVV-P43} }
          \end{tabular} &
          \passed \\
          \cmidrule{3-5}
          & &
        \begin{tabular}{@{}l@{}}
        \href{https://jira.lsstcorp.org/secure/Tests.jspa\#/testCase/LVV-T144}{LVV-T144} \\
        \vcdDocRef{LDM-639}
        \end{tabular} &
          & \notexec{} \\
  \midrule
  \begin{tabular}{@{}l@{}}
  DMS-REQ-0306\\\vcdDocRef{LSE-61}~{\tiny
 (p. 1a)   }
  \end{tabular} &
    \begin{tabular}{@{}l@{}}
    \hypertarget{dms-req-0306-v-01}{DMS-REQ-0306-V-01}
    \\\vcdJiraRef{LVV-137}~{\tiny
 (p. 1a)     }
    \end{tabular} &
        \begin{tabular}{@{}l@{}}
        \href{https://jira.lsstcorp.org/secure/Tests.jspa\#/testCase/LVV-T11}{LVV-T11} \\
        \vcdDocRef{LDM-534}
        \end{tabular} &
          \begin{tabular}{@{}l@{}}
          2019-05-22 \\
            \vcdDocRef{DMTR-51}
            {\scriptsize \href{https://jira.lsstcorp.org/secure/Tests.jspa\#/testPlan/LVV-P43}{LVV-P43} }
          \end{tabular} &
          \passed \\
          \cmidrule{3-5}
          & &
        \begin{tabular}{@{}l@{}}
        \href{https://jira.lsstcorp.org/secure/Tests.jspa\#/testCase/LVV-T145}{LVV-T145} \\
        \vcdDocRef{LDM-639}
        \end{tabular} &
          & \notexec{} \\
  \midrule
  \begin{tabular}{@{}l@{}}
  DMS-REQ-0307\\\vcdDocRef{LSE-61}~{\tiny
 (p. 2)   }
  \end{tabular} &
    \begin{tabular}{@{}l@{}}
    \hypertarget{dms-req-0307-v-01}{DMS-REQ-0307-V-01}
    \\\vcdJiraRef{LVV-138}~{\tiny
 (p. 2)     }
    \end{tabular} &
        \begin{tabular}{@{}l@{}}
        \href{https://jira.lsstcorp.org/secure/Tests.jspa\#/testCase/LVV-T148}{LVV-T148} \\
        \vcdDocRef{LDM-639}
        \end{tabular} &
          & \notexec{} \\
  \midrule
  \begin{tabular}{@{}l@{}}
  DMS-REQ-0308\\\vcdDocRef{LSE-61}~{\tiny
 (p. 1b)   }
  \end{tabular} &
    \begin{tabular}{@{}l@{}}
    \hypertarget{dms-req-0308-v-01}{DMS-REQ-0308-V-01}
    \\\vcdJiraRef{LVV-139}~{\tiny
 (p. 1b)     }
    \end{tabular} &
        \begin{tabular}{@{}l@{}}
        \href{https://jira.lsstcorp.org/secure/Tests.jspa\#/testCase/LVV-T10}{LVV-T10} \\
        \vcdDocRef{LDM-534}
        \end{tabular} &
          \begin{tabular}{@{}l@{}}
          2019-05-22 \\
            \vcdDocRef{DMTR-51}
            {\scriptsize \href{https://jira.lsstcorp.org/secure/Tests.jspa\#/testPlan/LVV-P43}{LVV-P43} }
          \end{tabular} &
          \passed \\
          \cmidrule{3-5}
          & &
        \begin{tabular}{@{}l@{}}
        \href{https://jira.lsstcorp.org/secure/Tests.jspa\#/testCase/LVV-T17}{LVV-T17} \\
        \vcdDocRef{LDM-533}
        \end{tabular} &
          \begin{tabular}{@{}l@{}}
          2019-05-22 \\
            \vcdDocRef{DMTR-53}
            {\scriptsize \href{https://jira.lsstcorp.org/secure/Tests.jspa\#/testPlan/LVV-P44}{LVV-P44} }
          \end{tabular} &
          \passed \\
          \cmidrule{3-5}
          & &
        \begin{tabular}{@{}l@{}}
        \href{https://jira.lsstcorp.org/secure/Tests.jspa\#/testCase/LVV-T124}{LVV-T124} \\
        \vcdDocRef{LDM-639}
        \end{tabular} &
          & \notexec{} \\
          \cmidrule{3-5}
          & &
        \begin{tabular}{@{}l@{}}
        \href{https://jira.lsstcorp.org/secure/Tests.jspa\#/testCase/LVV-T216}{LVV-T216} \\
        \vcdDocRef{LDM-533}
        \end{tabular} &
          \begin{tabular}{@{}l@{}}
          2018-07-17 \\
            \vcdDocRef{DMTR-91}
            {\scriptsize \href{https://jira.lsstcorp.org/secure/Tests.jspa\#/testPlan/LVV-P1}{LVV-P1} }
          \end{tabular} &
          \cndpass \\
          \cmidrule{3-5}
          & &
        \begin{tabular}{@{}l@{}}
        \href{https://jira.lsstcorp.org/secure/Tests.jspa\#/testCase/LVV-T362}{LVV-T362} \\
        \vcdDocRef{}
        \end{tabular} &
          \begin{tabular}{@{}l@{}}
          2019-03-31 \\
            \vcdDocRef{DMTR-111}
            {\scriptsize \href{https://jira.lsstcorp.org/secure/Tests.jspa\#/testPlan/LVV-P15}{LVV-P15} }
          \end{tabular} &
          \passed \\
          \cmidrule{3-5}
          & &
        \begin{tabular}{@{}l@{}}
        \href{https://jira.lsstcorp.org/secure/Tests.jspa\#/testCase/LVV-T363}{LVV-T363} \\
        \vcdDocRef{}
        \end{tabular} &
          \begin{tabular}{@{}l@{}}
          2019-03-31 \\
            \vcdDocRef{DMTR-111}
            {\scriptsize \href{https://jira.lsstcorp.org/secure/Tests.jspa\#/testPlan/LVV-P15}{LVV-P15} }
          \end{tabular} &
          \passed \\
  \midrule
  \begin{tabular}{@{}l@{}}
  DMS-REQ-0309\\\vcdDocRef{LSE-61}~{\tiny
 (p. 1a)   }
  \end{tabular} &
    \begin{tabular}{@{}l@{}}
    \hypertarget{dms-req-0309-v-01}{DMS-REQ-0309-V-01}
    \\\vcdJiraRef{LVV-140}~{\tiny
 (p. 1a)     }
    \end{tabular} &
        \begin{tabular}{@{}l@{}}
        \href{https://jira.lsstcorp.org/secure/Tests.jspa\#/testCase/LVV-T154}{LVV-T154} \\
        \vcdDocRef{LDM-639}
        \end{tabular} &
          & \notexec{} \\
          \cmidrule{3-5}
          & &
        \begin{tabular}{@{}l@{}}
        \href{https://jira.lsstcorp.org/secure/Tests.jspa\#/testCase/LVV-T287}{LVV-T287} \\
        \vcdDocRef{LDM-538}
        \end{tabular} &
          & \notexec{} \\
          \cmidrule{3-5}
          & &
        \begin{tabular}{@{}l@{}}
        \href{https://jira.lsstcorp.org/secure/Tests.jspa\#/testCase/LVV-T454}{LVV-T454} \\
        \vcdDocRef{}
        \end{tabular} &
          \begin{tabular}{@{}l@{}}
          2019-09-16 \\
            \vcdDocRef{DMTR-121}
            {\scriptsize \href{https://jira.lsstcorp.org/secure/Tests.jspa\#/testPlan/LVV-P32}{LVV-P32} }
          \end{tabular} &
          \passed \\
  \midrule
  \begin{tabular}{@{}l@{}}
  DMS-REQ-0310\\\vcdDocRef{LSE-61}~{\tiny
 (p. 1b)   }
  \end{tabular} &
    \begin{tabular}{@{}l@{}}
    \hypertarget{dms-req-0310-v-01}{DMS-REQ-0310-V-01}
    \\\vcdJiraRef{LVV-141}~{\tiny
 (p. 1b)     }
    \end{tabular} &
        \begin{tabular}{@{}l@{}}
        \href{https://jira.lsstcorp.org/secure/Tests.jspa\#/testCase/LVV-T155}{LVV-T155} \\
        \vcdDocRef{LDM-639}
        \end{tabular} &
          & \notexec{} \\
  \midrule
  \begin{tabular}{@{}l@{}}
  DMS-REQ-0311\\\vcdDocRef{LSE-61}~{\tiny
 (p. 1b)   }
  \end{tabular} &
    \begin{tabular}{@{}l@{}}
    \hypertarget{dms-req-0311-v-01}{DMS-REQ-0311-V-01}
    \\\vcdJiraRef{LVV-142}~{\tiny
 (p. 1b)     }
    \end{tabular} &
        \begin{tabular}{@{}l@{}}
        \href{https://jira.lsstcorp.org/secure/Tests.jspa\#/testCase/LVV-T156}{LVV-T156} \\
        \vcdDocRef{LDM-639}
        \end{tabular} &
          & \notexec{} \\
  \midrule
  \begin{tabular}{@{}l@{}}
  DMS-REQ-0312\\\vcdDocRef{LSE-61}~{\tiny
 (p. 1b)   }
  \end{tabular} &
    \begin{tabular}{@{}l@{}}
    \hypertarget{dms-req-0312-v-01}{DMS-REQ-0312-V-01}
    \\\vcdJiraRef{LVV-143}~{\tiny
 (p. 1b)     }
    \end{tabular} &
        \begin{tabular}{@{}l@{}}
        \href{https://jira.lsstcorp.org/secure/Tests.jspa\#/testCase/LVV-T157}{LVV-T157} \\
        \vcdDocRef{LDM-639}
        \end{tabular} &
          & \notexec{} \\
  \midrule
  \begin{tabular}{@{}l@{}}
  DMS-REQ-0313\\\vcdDocRef{LSE-61}~{\tiny
 (p. 1b)   }
  \end{tabular} &
    \begin{tabular}{@{}l@{}}
    \hypertarget{dms-req-0313-v-01}{DMS-REQ-0313-V-01}
    \\\vcdJiraRef{LVV-144}~{\tiny
 (p. 1b)     }
    \end{tabular} &
        \begin{tabular}{@{}l@{}}
        \href{https://jira.lsstcorp.org/secure/Tests.jspa\#/testCase/LVV-T158}{LVV-T158} \\
        \vcdDocRef{LDM-639}
        \end{tabular} &
          & \notexec{} \\
  \midrule
  \begin{tabular}{@{}l@{}}
  DMS-REQ-0314\\\vcdDocRef{LSE-61}~{\tiny
 (p. 1b)   }
  \end{tabular} &
    \begin{tabular}{@{}l@{}}
    \hypertarget{dms-req-0314-v-01}{DMS-REQ-0314-V-01}
    \\\vcdJiraRef{LVV-145}~{\tiny
 (p. 1b)     }
    \end{tabular} &
        \begin{tabular}{@{}l@{}}
        \href{https://jira.lsstcorp.org/secure/Tests.jspa\#/testCase/LVV-T179}{LVV-T179} \\
        \vcdDocRef{LDM-639}
        \end{tabular} &
          & \notexec{} \\
          \cmidrule{3-5}
          & &
        \begin{tabular}{@{}l@{}}
        \href{https://jira.lsstcorp.org/secure/Tests.jspa\#/testCase/LVV-T287}{LVV-T287} \\
        \vcdDocRef{LDM-538}
        \end{tabular} &
          & \notexec{} \\
  \midrule
  \begin{tabular}{@{}l@{}}
  DMS-REQ-0315\\\vcdDocRef{LSE-61}~{\tiny
 (p. 1a)   }
  \end{tabular} &
    \begin{tabular}{@{}l@{}}
    \hypertarget{dms-req-0315-v-01}{DMS-REQ-0315-V-01}
    \\\vcdJiraRef{LVV-146}~{\tiny
 (p. 1a)     }
    \end{tabular} &
        \begin{tabular}{@{}l@{}}
        \href{https://jira.lsstcorp.org/secure/Tests.jspa\#/testCase/LVV-T183}{LVV-T183} \\
        \vcdDocRef{LDM-639}
        \end{tabular} &
          & \notexec{} \\
          \cmidrule{3-5}
          & &
        \begin{tabular}{@{}l@{}}
        \href{https://jira.lsstcorp.org/secure/Tests.jspa\#/testCase/LVV-T283}{LVV-T283} \\
        \vcdDocRef{LDM-538}
        \end{tabular} &
          \begin{tabular}{@{}l@{}}
          2019-05-22 \\
            \vcdDocRef{DMTR-61}
            {\scriptsize \href{https://jira.lsstcorp.org/secure/Tests.jspa\#/testPlan/LVV-P45}{LVV-P45} }
          \end{tabular} &
          \passed \\
          \cmidrule{3-5}
          & &
        \begin{tabular}{@{}l@{}}
        \href{https://jira.lsstcorp.org/secure/Tests.jspa\#/testCase/LVV-T284}{LVV-T284} \\
        \vcdDocRef{LDM-538}
        \end{tabular} &
          \begin{tabular}{@{}l@{}}
          2019-06-24 \\
            \vcdDocRef{DMTR-102}
            {\scriptsize \href{https://jira.lsstcorp.org/secure/Tests.jspa\#/testPlan/LVV-P10}{LVV-P10} }
          \end{tabular} &
          \passed \\
          \cmidrule{3-5}
          & &
        \begin{tabular}{@{}l@{}}
        \href{https://jira.lsstcorp.org/secure/Tests.jspa\#/testCase/LVV-T1549}{LVV-T1549} \\
        \vcdDocRef{}
        \end{tabular} &
          & \notexec{} \\
          \cmidrule{3-5}
          & &
        \begin{tabular}{@{}l@{}}
        \href{https://jira.lsstcorp.org/secure/Tests.jspa\#/testCase/LVV-T1556}{LVV-T1556} \\
        \vcdDocRef{}
        \end{tabular} &
          & \notexec{} \\
  \midrule
  \begin{tabular}{@{}l@{}}
  DMS-REQ-0316\\\vcdDocRef{LSE-61}~{\tiny
 (p. 1a)   }
  \end{tabular} &
    \begin{tabular}{@{}l@{}}
    \hypertarget{dms-req-0316-v-01}{DMS-REQ-0316-V-01}
    \\\vcdJiraRef{LVV-147}~{\tiny
 (p. 1a)     }
    \end{tabular} &
        \begin{tabular}{@{}l@{}}
        \href{https://jira.lsstcorp.org/secure/Tests.jspa\#/testCase/LVV-T191}{LVV-T191} \\
        \vcdDocRef{LDM-639}
        \end{tabular} &
          & \notexec{} \\
  \midrule
  \begin{tabular}{@{}l@{}}
  DMS-REQ-0317\\\vcdDocRef{LSE-61}~{\tiny
 (p. 2)   }
  \end{tabular} &
    \begin{tabular}{@{}l@{}}
    \hypertarget{dms-req-0317-v-01}{DMS-REQ-0317-V-01}
    \\\vcdJiraRef{LVV-148}~{\tiny
 (p. 2)     }
    \end{tabular} &
        \begin{tabular}{@{}l@{}}
        \href{https://jira.lsstcorp.org/secure/Tests.jspa\#/testCase/LVV-T55}{LVV-T55} \\
        \vcdDocRef{LDM-639}
        \end{tabular} &
          & \notexec{} \\
  \midrule
  \begin{tabular}{@{}l@{}}
  DMS-REQ-0318\\\vcdDocRef{LSE-61}~{\tiny
 (p. 1b)   }
  \end{tabular} &
    \begin{tabular}{@{}l@{}}
    \hypertarget{dms-req-0318-v-01}{DMS-REQ-0318-V-01}
    \\\vcdJiraRef{LVV-149}~{\tiny
 (p. 1b)     }
    \end{tabular} &
        \begin{tabular}{@{}l@{}}
        \href{https://jira.lsstcorp.org/secure/Tests.jspa\#/testCase/LVV-T180}{LVV-T180} \\
        \vcdDocRef{LDM-639}
        \end{tabular} &
          & \notexec{} \\
          \cmidrule{3-5}
          & &
        \begin{tabular}{@{}l@{}}
        \href{https://jira.lsstcorp.org/secure/Tests.jspa\#/testCase/LVV-T287}{LVV-T287} \\
        \vcdDocRef{LDM-538}
        \end{tabular} &
          & \notexec{} \\
  \midrule
  \begin{tabular}{@{}l@{}}
  DMS-REQ-0319\\\vcdDocRef{LSE-61}~{\tiny
 (p. 1b)   }
  \end{tabular} &
    \begin{tabular}{@{}l@{}}
    \hypertarget{dms-req-0319-v-01}{DMS-REQ-0319-V-01}
    \\\vcdJiraRef{LVV-150}~{\tiny
 (p. 1b)     }
    \end{tabular} &
        \begin{tabular}{@{}l@{}}
        \href{https://jira.lsstcorp.org/secure/Tests.jspa\#/testCase/LVV-T56}{LVV-T56} \\
        \vcdDocRef{LDM-639}
        \end{tabular} &
          & \notexec{} \\
  \midrule
  \begin{tabular}{@{}l@{}}
  DMS-REQ-0320\\\vcdDocRef{LSE-61}~{\tiny
 (p. 2)   }
  \end{tabular} &
    \begin{tabular}{@{}l@{}}
    \hypertarget{dms-req-0320-v-01}{DMS-REQ-0320-V-01}
    \\\vcdJiraRef{LVV-151}~{\tiny
 (p. 2)     }
    \end{tabular} &
        \begin{tabular}{@{}l@{}}
        \href{https://jira.lsstcorp.org/secure/Tests.jspa\#/testCase/LVV-T92}{LVV-T92} \\
        \vcdDocRef{LDM-639}
        \end{tabular} &
          & \notexec{} \\
  \midrule
  \begin{tabular}{@{}l@{}}
  DMS-REQ-0321\\\vcdDocRef{LSE-61}~{\tiny
 (p. 2)   }
  \end{tabular} &
    \begin{tabular}{@{}l@{}}
    \hypertarget{dms-req-0321-v-01}{DMS-REQ-0321-V-01}
    \\\vcdJiraRef{LVV-152}~{\tiny
 (p. 2)     }
    \end{tabular} &
        \begin{tabular}{@{}l@{}}
        \href{https://jira.lsstcorp.org/secure/Tests.jspa\#/testCase/LVV-T93}{LVV-T93} \\
        \vcdDocRef{LDM-639}
        \end{tabular} &
          & \notexec{} \\
  \midrule
  \begin{tabular}{@{}l@{}}
  DMS-REQ-0322\\\vcdDocRef{LSE-61}~{\tiny
 (p. 1b)   }
  \end{tabular} &
    \begin{tabular}{@{}l@{}}
    \hypertarget{dms-req-0322-v-01}{DMS-REQ-0322-V-01}
    \\\vcdJiraRef{LVV-153}~{\tiny
 (p. 1b)     }
    \end{tabular} &
        \begin{tabular}{@{}l@{}}
        \href{https://jira.lsstcorp.org/secure/Tests.jspa\#/testCase/LVV-T94}{LVV-T94} \\
        \vcdDocRef{LDM-639}
        \end{tabular} &
          & \notexec{} \\
  \midrule
  \begin{tabular}{@{}l@{}}
  DMS-REQ-0323\\\vcdDocRef{LSE-61}~{\tiny
 (p. 3)   }
  \end{tabular} &
    \begin{tabular}{@{}l@{}}
    \hypertarget{dms-req-0323-v-01}{DMS-REQ-0323-V-01}
    \\\vcdJiraRef{LVV-154}~{\tiny
 (p. 3)     }
    \end{tabular} &
        \begin{tabular}{@{}l@{}}
        \href{https://jira.lsstcorp.org/secure/Tests.jspa\#/testCase/LVV-T57}{LVV-T57} \\
        \vcdDocRef{LDM-639}
        \end{tabular} &
          & \notexec{} \\
  \midrule
  \begin{tabular}{@{}l@{}}
  DMS-REQ-0324\\\vcdDocRef{LSE-61}~{\tiny
 (p. 1b)   }
  \end{tabular} &
    \begin{tabular}{@{}l@{}}
    \hypertarget{dms-req-0324-v-01}{DMS-REQ-0324-V-01}
    \\\vcdJiraRef{LVV-155}~{\tiny
 (p. 1b)     }
    \end{tabular} &
        \begin{tabular}{@{}l@{}}
        \href{https://jira.lsstcorp.org/secure/Tests.jspa\#/testCase/LVV-T58}{LVV-T58} \\
        \vcdDocRef{LDM-639}
        \end{tabular} &
          & \notexec{} \\
  \midrule
  \begin{tabular}{@{}l@{}}
  DMS-REQ-0325\\\vcdDocRef{LSE-61}~{\tiny
 (p. 2)   }
  \end{tabular} &
    \begin{tabular}{@{}l@{}}
    \hypertarget{dms-req-0325-v-01}{DMS-REQ-0325-V-01}
    \\\vcdJiraRef{LVV-156}~{\tiny
 (p. 2)     }
    \end{tabular} &
        \begin{tabular}{@{}l@{}}
        \href{https://jira.lsstcorp.org/secure/Tests.jspa\#/testCase/LVV-T59}{LVV-T59} \\
        \vcdDocRef{LDM-639}
        \end{tabular} &
          & \notexec{} \\
  \midrule
  \begin{tabular}{@{}l@{}}
  DMS-REQ-0326\\\vcdDocRef{LSE-61}~{\tiny
 (p. 2)   }
  \end{tabular} &
    \begin{tabular}{@{}l@{}}
    \hypertarget{dms-req-0326-v-01}{DMS-REQ-0326-V-01}
    \\\vcdJiraRef{LVV-157}~{\tiny
 (p. 2)     }
    \end{tabular} &
        \begin{tabular}{@{}l@{}}
        \href{https://jira.lsstcorp.org/secure/Tests.jspa\#/testCase/LVV-T23}{LVV-T23} \\
        \vcdDocRef{LDM-639}
        \end{tabular} &
          & \notexec{} \\
  \midrule
  \begin{tabular}{@{}l@{}}
  DMS-REQ-0327\\\vcdDocRef{LSE-61}~{\tiny
 (p. 1b)   }
  \end{tabular} &
    \begin{tabular}{@{}l@{}}
    \hypertarget{dms-req-0327-v-01}{DMS-REQ-0327-V-01}
    \\\vcdJiraRef{LVV-158}~{\tiny
 (p. 1b)     }
    \end{tabular} &
        \begin{tabular}{@{}l@{}}
        \href{https://jira.lsstcorp.org/secure/Tests.jspa\#/testCase/LVV-T15}{LVV-T15} \\
        \vcdDocRef{LDM-534}
        \end{tabular} &
          \begin{tabular}{@{}l@{}}
          2019-05-22 \\
            \vcdDocRef{DMTR-51}
            {\scriptsize \href{https://jira.lsstcorp.org/secure/Tests.jspa\#/testPlan/LVV-P43}{LVV-P43} }
          \end{tabular} &
          \passed \\
          \cmidrule{3-5}
          & &
        \begin{tabular}{@{}l@{}}
        \href{https://jira.lsstcorp.org/secure/Tests.jspa\#/testCase/LVV-T19}{LVV-T19} \\
        \vcdDocRef{LDM-533}
        \end{tabular} &
          \begin{tabular}{@{}l@{}}
          2019-05-22 \\
            \vcdDocRef{DMTR-53}
            {\scriptsize \href{https://jira.lsstcorp.org/secure/Tests.jspa\#/testPlan/LVV-P44}{LVV-P44} }
          \end{tabular} &
          \passed \\
          \cmidrule{3-5}
          & &
        \begin{tabular}{@{}l@{}}
        \href{https://jira.lsstcorp.org/secure/Tests.jspa\#/testCase/LVV-T43}{LVV-T43} \\
        \vcdDocRef{LDM-639}
        \end{tabular} &
          \begin{tabular}{@{}l@{}}
          2020-01-23 \\
            \vcdDocRef{DMTR-201}
            {\scriptsize \href{https://jira.lsstcorp.org/secure/Tests.jspa\#/testPlan/LVV-P65}{LVV-P65} }
          \end{tabular} &
          \passed \\
  \midrule
  \begin{tabular}{@{}l@{}}
  DMS-REQ-0328\\\vcdDocRef{LSE-61}~{\tiny
 (p. 1b)   }
  \end{tabular} &
    \begin{tabular}{@{}l@{}}
    \hypertarget{dms-req-0328-v-01}{DMS-REQ-0328-V-01}
    \\\vcdJiraRef{LVV-159}~{\tiny
 (p. 1b)     }
    \end{tabular} &
        \begin{tabular}{@{}l@{}}
        \href{https://jira.lsstcorp.org/secure/Tests.jspa\#/testCase/LVV-T44}{LVV-T44} \\
        \vcdDocRef{LDM-639}
        \end{tabular} &
          & \notexec{} \\
  \midrule
  \begin{tabular}{@{}l@{}}
  DMS-REQ-0329\\\vcdDocRef{LSE-61}~{\tiny
 (p. 2)   }
  \end{tabular} &
    \begin{tabular}{@{}l@{}}
    \hypertarget{dms-req-0329-v-01}{DMS-REQ-0329-V-01}
    \\\vcdJiraRef{LVV-160}~{\tiny
 (p. 2)     }
    \end{tabular} &
        \begin{tabular}{@{}l@{}}
        \href{https://jira.lsstcorp.org/secure/Tests.jspa\#/testCase/LVV-T76}{LVV-T76} \\
        \vcdDocRef{LDM-639}
        \end{tabular} &
          & \notexec{} \\
  \midrule
  \begin{tabular}{@{}l@{}}
  DMS-REQ-0330\\\vcdDocRef{LSE-61}~{\tiny
 (p. 2)   }
  \end{tabular} &
    \begin{tabular}{@{}l@{}}
    \hypertarget{dms-req-0330-v-01}{DMS-REQ-0330-V-01}
    \\\vcdJiraRef{LVV-161}~{\tiny
 (p. 2)     }
    \end{tabular} &
        \begin{tabular}{@{}l@{}}
        \href{https://jira.lsstcorp.org/secure/Tests.jspa\#/testCase/LVV-T77}{LVV-T77} \\
        \vcdDocRef{LDM-639}
        \end{tabular} &
          & \notexec{} \\
  \midrule
  \begin{tabular}{@{}l@{}}
  DMS-REQ-0331\\\vcdDocRef{LSE-61}~{\tiny
 (p. 1b)   }
  \end{tabular} &
    \begin{tabular}{@{}l@{}}
    \hypertarget{dms-req-0331-v-01}{DMS-REQ-0331-V-01}
    \\\vcdJiraRef{LVV-162}~{\tiny
 (p. 1b)     }
    \end{tabular} &
        \begin{tabular}{@{}l@{}}
        \href{https://jira.lsstcorp.org/secure/Tests.jspa\#/testCase/LVV-T13}{LVV-T13} \\
        \vcdDocRef{LDM-534}
        \end{tabular} &
          \begin{tabular}{@{}l@{}}
          2019-05-22 \\
            \vcdDocRef{DMTR-51}
            {\scriptsize \href{https://jira.lsstcorp.org/secure/Tests.jspa\#/testPlan/LVV-P43}{LVV-P43} }
          \end{tabular} &
          \passed \\
          \cmidrule{3-5}
          & &
        \begin{tabular}{@{}l@{}}
        \href{https://jira.lsstcorp.org/secure/Tests.jspa\#/testCase/LVV-T14}{LVV-T14} \\
        \vcdDocRef{LDM-534}
        \end{tabular} &
          \begin{tabular}{@{}l@{}}
          2019-05-22 \\
            \vcdDocRef{DMTR-51}
            {\scriptsize \href{https://jira.lsstcorp.org/secure/Tests.jspa\#/testPlan/LVV-P43}{LVV-P43} }
          \end{tabular} &
          \cndpass \\
          \cmidrule{3-5}
          & &
        \begin{tabular}{@{}l@{}}
        \href{https://jira.lsstcorp.org/secure/Tests.jspa\#/testCase/LVV-T21}{LVV-T21} \\
        \vcdDocRef{LDM-533}
        \end{tabular} &
          \begin{tabular}{@{}l@{}}
          2019-05-22 \\
            \vcdDocRef{DMTR-53}
            {\scriptsize \href{https://jira.lsstcorp.org/secure/Tests.jspa\#/testPlan/LVV-P44}{LVV-P44} }
          \end{tabular} &
          \passed \\
          \cmidrule{3-5}
          & &
        \begin{tabular}{@{}l@{}}
        \href{https://jira.lsstcorp.org/secure/Tests.jspa\#/testCase/LVV-T22}{LVV-T22} \\
        \vcdDocRef{LDM-533}
        \end{tabular} &
          \begin{tabular}{@{}l@{}}
          2019-05-22 \\
            \vcdDocRef{DMTR-53}
            {\scriptsize \href{https://jira.lsstcorp.org/secure/Tests.jspa\#/testPlan/LVV-P44}{LVV-P44} }
          \end{tabular} &
          \passed \\
          \cmidrule{3-5}
          & &
        \begin{tabular}{@{}l@{}}
        \href{https://jira.lsstcorp.org/secure/Tests.jspa\#/testCase/LVV-T24}{LVV-T24} \\
        \vcdDocRef{LDM-639}
        \end{tabular} &
          & \notexec{} \\
  \midrule
  \begin{tabular}{@{}l@{}}
  DMS-REQ-0332\\\vcdDocRef{LSE-61}~{\tiny
 (p. 2)   }
  \end{tabular} &
    \begin{tabular}{@{}l@{}}
    \hypertarget{dms-req-0332-v-01}{DMS-REQ-0332-V-01}
    \\\vcdJiraRef{LVV-163}~{\tiny
 (p. 2)     }
    \end{tabular} &
        \begin{tabular}{@{}l@{}}
        \href{https://jira.lsstcorp.org/secure/Tests.jspa\#/testCase/LVV-T25}{LVV-T25} \\
        \vcdDocRef{LDM-639}
        \end{tabular} &
          & \notexec{} \\
  \midrule
  \begin{tabular}{@{}l@{}}
  DMS-REQ-0333\\\vcdDocRef{LSE-61}~{\tiny
 (p. 1b)   }
  \end{tabular} &
    \begin{tabular}{@{}l@{}}
    \hypertarget{dms-req-0333-v-01}{DMS-REQ-0333-V-01}
    \\\vcdJiraRef{LVV-164}~{\tiny
 (p. 1b)     }
    \end{tabular} &
        \begin{tabular}{@{}l@{}}
        \href{https://jira.lsstcorp.org/secure/Tests.jspa\#/testCase/LVV-T26}{LVV-T26} \\
        \vcdDocRef{LDM-639}
        \end{tabular} &
          & \notexec{} \\
  \midrule
  \begin{tabular}{@{}l@{}}
  DMS-REQ-0334\\\vcdDocRef{LSE-61}~{\tiny
 (p. 1b)   }
  \end{tabular} &
    \begin{tabular}{@{}l@{}}
    \hypertarget{dms-req-0334-v-01}{DMS-REQ-0334-V-01}
    \\\vcdJiraRef{LVV-165}~{\tiny
 (p. 1b)     }
    \end{tabular} &
        \begin{tabular}{@{}l@{}}
        \href{https://jira.lsstcorp.org/secure/Tests.jspa\#/testCase/LVV-T12}{LVV-T12} \\
        \vcdDocRef{LDM-534}
        \end{tabular} &
          \begin{tabular}{@{}l@{}}
          2019-05-22 \\
            \vcdDocRef{DMTR-51}
            {\scriptsize \href{https://jira.lsstcorp.org/secure/Tests.jspa\#/testPlan/LVV-P43}{LVV-P43} }
          \end{tabular} &
          \passed \\
          \cmidrule{3-5}
          & &
        \begin{tabular}{@{}l@{}}
        \href{https://jira.lsstcorp.org/secure/Tests.jspa\#/testCase/LVV-T13}{LVV-T13} \\
        \vcdDocRef{LDM-534}
        \end{tabular} &
          \begin{tabular}{@{}l@{}}
          2019-05-22 \\
            \vcdDocRef{DMTR-51}
            {\scriptsize \href{https://jira.lsstcorp.org/secure/Tests.jspa\#/testPlan/LVV-P43}{LVV-P43} }
          \end{tabular} &
          \passed \\
          \cmidrule{3-5}
          & &
        \begin{tabular}{@{}l@{}}
        \href{https://jira.lsstcorp.org/secure/Tests.jspa\#/testCase/LVV-T14}{LVV-T14} \\
        \vcdDocRef{LDM-534}
        \end{tabular} &
          \begin{tabular}{@{}l@{}}
          2019-05-22 \\
            \vcdDocRef{DMTR-51}
            {\scriptsize \href{https://jira.lsstcorp.org/secure/Tests.jspa\#/testPlan/LVV-P43}{LVV-P43} }
          \end{tabular} &
          \cndpass \\
          \cmidrule{3-5}
          & &
        \begin{tabular}{@{}l@{}}
        \href{https://jira.lsstcorp.org/secure/Tests.jspa\#/testCase/LVV-T15}{LVV-T15} \\
        \vcdDocRef{LDM-534}
        \end{tabular} &
          \begin{tabular}{@{}l@{}}
          2019-05-22 \\
            \vcdDocRef{DMTR-51}
            {\scriptsize \href{https://jira.lsstcorp.org/secure/Tests.jspa\#/testPlan/LVV-P43}{LVV-P43} }
          \end{tabular} &
          \passed \\
          \cmidrule{3-5}
          & &
        \begin{tabular}{@{}l@{}}
        \href{https://jira.lsstcorp.org/secure/Tests.jspa\#/testCase/LVV-T16}{LVV-T16} \\
        \vcdDocRef{LDM-534}
        \end{tabular} &
          \begin{tabular}{@{}l@{}}
          2019-05-22 \\
            \vcdDocRef{DMTR-51}
            {\scriptsize \href{https://jira.lsstcorp.org/secure/Tests.jspa\#/testPlan/LVV-P43}{LVV-P43} }
          \end{tabular} &
          \passed \\
          \cmidrule{3-5}
          & &
        \begin{tabular}{@{}l@{}}
        \href{https://jira.lsstcorp.org/secure/Tests.jspa\#/testCase/LVV-T78}{LVV-T78} \\
        \vcdDocRef{LDM-639}
        \end{tabular} &
          & \notexec{} \\
  \midrule
  \begin{tabular}{@{}l@{}}
  DMS-REQ-0335\\\vcdDocRef{LSE-61}~{\tiny
 (p. 1b)   }
  \end{tabular} &
    \begin{tabular}{@{}l@{}}
    \hypertarget{dms-req-0335-v-01}{DMS-REQ-0335-V-01}
    \\\vcdJiraRef{LVV-166}~{\tiny
 (p. 1b)     }
    \end{tabular} &
        \begin{tabular}{@{}l@{}}
        \href{https://jira.lsstcorp.org/secure/Tests.jspa\#/testCase/LVV-T79}{LVV-T79} \\
        \vcdDocRef{LDM-639}
        \end{tabular} &
          & \notexec{} \\
  \midrule
  \begin{tabular}{@{}l@{}}
  DMS-REQ-0336\\\vcdDocRef{LSE-61}~{\tiny
 (p. 1b)   }
  \end{tabular} &
    \begin{tabular}{@{}l@{}}
    \hypertarget{dms-req-0336-v-01}{DMS-REQ-0336-V-01}
    \\\vcdJiraRef{LVV-167}~{\tiny
 (p. 1b)     }
    \end{tabular} &
        \begin{tabular}{@{}l@{}}
        \href{https://jira.lsstcorp.org/secure/Tests.jspa\#/testCase/LVV-T159}{LVV-T159} \\
        \vcdDocRef{LDM-639}
        \end{tabular} &
          & \notexec{} \\
  \midrule
  \begin{tabular}{@{}l@{}}
  DMS-REQ-0337\\\vcdDocRef{LSE-61}~{\tiny
 (p. 2)   }
  \end{tabular} &
    \begin{tabular}{@{}l@{}}
    \hypertarget{dms-req-0337-v-01}{DMS-REQ-0337-V-01}
    \\\vcdJiraRef{LVV-168}~{\tiny
 (p. 2)     }
    \end{tabular} &
        \begin{tabular}{@{}l@{}}
        \href{https://jira.lsstcorp.org/secure/Tests.jspa\#/testCase/LVV-T80}{LVV-T80} \\
        \vcdDocRef{LDM-639}
        \end{tabular} &
          & \notexec{} \\
  \midrule
  \begin{tabular}{@{}l@{}}
  DMS-REQ-0338\\\vcdDocRef{LSE-61}~{\tiny
 (p. 2)   }
  \end{tabular} &
    \begin{tabular}{@{}l@{}}
    \hypertarget{dms-req-0338-v-01}{DMS-REQ-0338-V-01}
    \\\vcdJiraRef{LVV-169}~{\tiny
 (p. 2)     }
    \end{tabular} &
        \begin{tabular}{@{}l@{}}
        \href{https://jira.lsstcorp.org/secure/Tests.jspa\#/testCase/LVV-T81}{LVV-T81} \\
        \vcdDocRef{LDM-639}
        \end{tabular} &
          & \notexec{} \\
  \midrule
  \begin{tabular}{@{}l@{}}
  DMS-REQ-0339\\\vcdDocRef{LSE-61}~{\tiny
 (p. 1a)   }
  \end{tabular} &
    \begin{tabular}{@{}l@{}}
    \hypertarget{dms-req-0339-v-01}{DMS-REQ-0339-V-01}
    \\\vcdJiraRef{LVV-170}~{\tiny
 (p. 1a)     }
    \end{tabular} &
        \begin{tabular}{@{}l@{}}
        \href{https://jira.lsstcorp.org/secure/Tests.jspa\#/testCase/LVV-T82}{LVV-T82} \\
        \vcdDocRef{LDM-639}
        \end{tabular} &
          & \notexec{} \\
  \midrule
  \begin{tabular}{@{}l@{}}
  DMS-REQ-0340\\\vcdDocRef{LSE-61}~{\tiny
 (p. 2)   }
  \end{tabular} &
    \begin{tabular}{@{}l@{}}
    \hypertarget{dms-req-0340-v-01}{DMS-REQ-0340-V-01}
    \\\vcdJiraRef{LVV-171}~{\tiny
 (p. 2)     }
    \end{tabular} &
        \begin{tabular}{@{}l@{}}
        \href{https://jira.lsstcorp.org/secure/Tests.jspa\#/testCase/LVV-T123}{LVV-T123} \\
        \vcdDocRef{LDM-639}
        \end{tabular} &
          & \notexec{} \\
  \midrule
  \begin{tabular}{@{}l@{}}
  DMS-REQ-0341\\\vcdDocRef{LSE-61}~{\tiny
 (p. 1b)   }
  \end{tabular} &
    \begin{tabular}{@{}l@{}}
    \hypertarget{dms-req-0341-v-01}{DMS-REQ-0341-V-01}
    \\\vcdJiraRef{LVV-172}~{\tiny
    }
    \end{tabular} &
        \begin{tabular}{@{}l@{}}
        \href{https://jira.lsstcorp.org/secure/Tests.jspa\#/testCase/LVV-T160}{LVV-T160} \\
        \vcdDocRef{LDM-639}
        \end{tabular} &
          & \notexec{} \\
      \cmidrule{2-5}
      &
    \begin{tabular}{@{}l@{}}
    \hypertarget{dms-req-0341-v-02}{DMS-REQ-0341-V-02}
    \\\vcdJiraRef{LVV-9749}~{\tiny
    }
    \end{tabular} &
        & & \\
  \midrule
  \begin{tabular}{@{}l@{}}
  DMS-REQ-0342\\\vcdDocRef{LSE-61}~{\tiny
 (p. 2)   }
  \end{tabular} &
    \begin{tabular}{@{}l@{}}
    \hypertarget{dms-req-0342-v-01}{DMS-REQ-0342-V-01}
    \\\vcdJiraRef{LVV-173}~{\tiny
 (p. 2)     }
    \end{tabular} &
        \begin{tabular}{@{}l@{}}
        \href{https://jira.lsstcorp.org/secure/Tests.jspa\#/testCase/LVV-T112}{LVV-T112} \\
        \vcdDocRef{LDM-639}
        \end{tabular} &
          & \notexec{} \\
          \cmidrule{3-5}
          & &
        \begin{tabular}{@{}l@{}}
        \href{https://jira.lsstcorp.org/secure/Tests.jspa\#/testCase/LVV-T218}{LVV-T218} \\
        \vcdDocRef{LDM-533}
        \end{tabular} &
          \begin{tabular}{@{}l@{}}
          2018-07-04 \\
            \vcdDocRef{DMTR-91}
            {\scriptsize \href{https://jira.lsstcorp.org/secure/Tests.jspa\#/testPlan/LVV-P1}{LVV-P1} }
          \end{tabular} &
          \cndpass \\
  \midrule
  \begin{tabular}{@{}l@{}}
  DMS-REQ-0343\\\vcdDocRef{LSE-61}~{\tiny
 (p. 2)   }
  \end{tabular} &
    \begin{tabular}{@{}l@{}}
    \hypertarget{dms-req-0343-v-01}{DMS-REQ-0343-V-01}
    \\\vcdJiraRef{LVV-174}~{\tiny
    }
    \end{tabular} &
        \begin{tabular}{@{}l@{}}
        \href{https://jira.lsstcorp.org/secure/Tests.jspa\#/testCase/LVV-T113}{LVV-T113} \\
        \vcdDocRef{LDM-639}
        \end{tabular} &
          & \notexec{} \\
          \cmidrule{3-5}
          & &
        \begin{tabular}{@{}l@{}}
        \href{https://jira.lsstcorp.org/secure/Tests.jspa\#/testCase/LVV-T218}{LVV-T218} \\
        \vcdDocRef{LDM-533}
        \end{tabular} &
          \begin{tabular}{@{}l@{}}
          2018-07-04 \\
            \vcdDocRef{DMTR-91}
            {\scriptsize \href{https://jira.lsstcorp.org/secure/Tests.jspa\#/testPlan/LVV-P1}{LVV-P1} }
          \end{tabular} &
          \cndpass \\
      \cmidrule{2-5}
      &
    \begin{tabular}{@{}l@{}}
    \hypertarget{dms-req-0343-v-02}{DMS-REQ-0343-V-02}
    \\\vcdJiraRef{LVV-9748}~{\tiny
    }
    \end{tabular} &
        \begin{tabular}{@{}l@{}}
        \href{https://jira.lsstcorp.org/secure/Tests.jspa\#/testCase/LVV-T1252}{LVV-T1252} \\
        \vcdDocRef{LDM-639}
        \end{tabular} &
          & \notexec{} \\
  \midrule
  \begin{tabular}{@{}l@{}}
  DMS-REQ-0004\\\vcdDocRef{LSE-61}~{\tiny
 (p. 1b)   }
  \end{tabular} &
    \begin{tabular}{@{}l@{}}
    \hypertarget{dms-req-0004-v-01}{DMS-REQ-0004-V-01}
    \\\vcdJiraRef{LVV-175}~{\tiny
    }
    \end{tabular} &
        \begin{tabular}{@{}l@{}}
        \href{https://jira.lsstcorp.org/secure/Tests.jspa\#/testCase/LVV-T35}{LVV-T35} \\
        \vcdDocRef{LDM-639}
        \end{tabular} &
          & \notexec{} \\
          \cmidrule{3-5}
          & &
        \begin{tabular}{@{}l@{}}
        \href{https://jira.lsstcorp.org/secure/Tests.jspa\#/testCase/LVV-T95}{LVV-T95} \\
        \vcdDocRef{LDM-639}
        \end{tabular} &
          & \notexec{} \\
      \cmidrule{2-5}
      &
    \begin{tabular}{@{}l@{}}
    \hypertarget{dms-req-0004-v-02}{DMS-REQ-0004-V-02}
    \\\vcdJiraRef{LVV-9740}~{\tiny
    }
    \end{tabular} &
        \begin{tabular}{@{}l@{}}
        \href{https://jira.lsstcorp.org/secure/Tests.jspa\#/testCase/LVV-T1276}{LVV-T1276} \\
        \vcdDocRef{LDM-639}
        \end{tabular} &
          & \notexec{} \\
      \cmidrule{2-5}
      &
    \begin{tabular}{@{}l@{}}
    \hypertarget{dms-req-0004-v-03}{DMS-REQ-0004-V-03}
    \\\vcdJiraRef{LVV-9803}~{\tiny
    }
    \end{tabular} &
        \begin{tabular}{@{}l@{}}
        \href{https://jira.lsstcorp.org/secure/Tests.jspa\#/testCase/LVV-T102}{LVV-T102} \\
        \vcdDocRef{LDM-639}
        \end{tabular} &
          & \notexec{} \\
  \midrule
  \begin{tabular}{@{}l@{}}
  DMS-REQ-0345\\\vcdDocRef{LSE-61}~{\tiny
 (p. 2)   }
  \end{tabular} &
    \begin{tabular}{@{}l@{}}
    \hypertarget{dms-req-0345-v-01}{DMS-REQ-0345-V-01}
    \\\vcdJiraRef{LVV-176}~{\tiny
 (p. 2)     }
    \end{tabular} &
        \begin{tabular}{@{}l@{}}
        \href{https://jira.lsstcorp.org/secure/Tests.jspa\#/testCase/LVV-T161}{LVV-T161} \\
        \vcdDocRef{LDM-639}
        \end{tabular} &
          & \notexec{} \\
  \midrule
  \begin{tabular}{@{}l@{}}
  DMS-REQ-0346\\\vcdDocRef{LSE-61}~{\tiny
 (p. 1b)   }
  \end{tabular} &
    \begin{tabular}{@{}l@{}}
    \hypertarget{dms-req-0346-v-01}{DMS-REQ-0346-V-01}
    \\\vcdJiraRef{LVV-177}~{\tiny
 (p. 1b)     }
    \end{tabular} &
        \begin{tabular}{@{}l@{}}
        \href{https://jira.lsstcorp.org/secure/Tests.jspa\#/testCase/LVV-T27}{LVV-T27} \\
        \vcdDocRef{LDM-639}
        \end{tabular} &
          & \notexec{} \\
          \cmidrule{3-5}
          & &
        \begin{tabular}{@{}l@{}}
        \href{https://jira.lsstcorp.org/secure/Tests.jspa\#/testCase/LVV-T286}{LVV-T286} \\
        \vcdDocRef{LDM-538}
        \end{tabular} &
          \begin{tabular}{@{}l@{}}
          2019-05-22 \\
            \vcdDocRef{DMTR-61}
            {\scriptsize \href{https://jira.lsstcorp.org/secure/Tests.jspa\#/testPlan/LVV-P45}{LVV-P45} }
          \end{tabular} &
          \passed \\
  \midrule
  \begin{tabular}{@{}l@{}}
  DMS-REQ-0347\\\vcdDocRef{LSE-61}~{\tiny
 (p. 1b)   }
  \end{tabular} &
    \begin{tabular}{@{}l@{}}
    \hypertarget{dms-req-0347-v-01}{DMS-REQ-0347-V-01}
    \\\vcdJiraRef{LVV-178}~{\tiny
 (p. 1b)     }
    \end{tabular} &
        \begin{tabular}{@{}l@{}}
        \href{https://jira.lsstcorp.org/secure/Tests.jspa\#/testCase/LVV-T13}{LVV-T13} \\
        \vcdDocRef{LDM-534}
        \end{tabular} &
          \begin{tabular}{@{}l@{}}
          2019-05-22 \\
            \vcdDocRef{DMTR-51}
            {\scriptsize \href{https://jira.lsstcorp.org/secure/Tests.jspa\#/testPlan/LVV-P43}{LVV-P43} }
          \end{tabular} &
          \passed \\
          \cmidrule{3-5}
          & &
        \begin{tabular}{@{}l@{}}
        \href{https://jira.lsstcorp.org/secure/Tests.jspa\#/testCase/LVV-T14}{LVV-T14} \\
        \vcdDocRef{LDM-534}
        \end{tabular} &
          \begin{tabular}{@{}l@{}}
          2019-05-22 \\
            \vcdDocRef{DMTR-51}
            {\scriptsize \href{https://jira.lsstcorp.org/secure/Tests.jspa\#/testPlan/LVV-P43}{LVV-P43} }
          \end{tabular} &
          \cndpass \\
          \cmidrule{3-5}
          & &
        \begin{tabular}{@{}l@{}}
        \href{https://jira.lsstcorp.org/secure/Tests.jspa\#/testCase/LVV-T21}{LVV-T21} \\
        \vcdDocRef{LDM-533}
        \end{tabular} &
          \begin{tabular}{@{}l@{}}
          2019-05-22 \\
            \vcdDocRef{DMTR-53}
            {\scriptsize \href{https://jira.lsstcorp.org/secure/Tests.jspa\#/testPlan/LVV-P44}{LVV-P44} }
          \end{tabular} &
          \passed \\
          \cmidrule{3-5}
          & &
        \begin{tabular}{@{}l@{}}
        \href{https://jira.lsstcorp.org/secure/Tests.jspa\#/testCase/LVV-T22}{LVV-T22} \\
        \vcdDocRef{LDM-533}
        \end{tabular} &
          \begin{tabular}{@{}l@{}}
          2019-05-22 \\
            \vcdDocRef{DMTR-53}
            {\scriptsize \href{https://jira.lsstcorp.org/secure/Tests.jspa\#/testPlan/LVV-P44}{LVV-P44} }
          \end{tabular} &
          \passed \\
          \cmidrule{3-5}
          & &
        \begin{tabular}{@{}l@{}}
        \href{https://jira.lsstcorp.org/secure/Tests.jspa\#/testCase/LVV-T28}{LVV-T28} \\
        \vcdDocRef{LDM-639}
        \end{tabular} &
          \begin{tabular}{@{}l@{}}
          2020-01-29 \\
            \vcdDocRef{DMTR-201}
            {\scriptsize \href{https://jira.lsstcorp.org/secure/Tests.jspa\#/testPlan/LVV-P65}{LVV-P65} }
          \end{tabular} &
          \cndpass \\
  \midrule
  \begin{tabular}{@{}l@{}}
  DMS-REQ-0348\\\vcdDocRef{LSE-61}~{\tiny
 (p. 2)   }
  \end{tabular} &
    \begin{tabular}{@{}l@{}}
    \hypertarget{dms-req-0348-v-01}{DMS-REQ-0348-V-01}
    \\\vcdJiraRef{LVV-179}~{\tiny
 (p. 2)     }
    \end{tabular} &
        \begin{tabular}{@{}l@{}}
        \href{https://jira.lsstcorp.org/secure/Tests.jspa\#/testCase/LVV-T114}{LVV-T114} \\
        \vcdDocRef{LDM-639}
        \end{tabular} &
          & \notexec{} \\
          \cmidrule{3-5}
          & &
        \begin{tabular}{@{}l@{}}
        \href{https://jira.lsstcorp.org/secure/Tests.jspa\#/testCase/LVV-T218}{LVV-T218} \\
        \vcdDocRef{LDM-533}
        \end{tabular} &
          \begin{tabular}{@{}l@{}}
          2018-07-04 \\
            \vcdDocRef{DMTR-91}
            {\scriptsize \href{https://jira.lsstcorp.org/secure/Tests.jspa\#/testPlan/LVV-P1}{LVV-P1} }
          \end{tabular} &
          \cndpass \\
  \midrule
  \begin{tabular}{@{}l@{}}
  DMS-REQ-0349\\\vcdDocRef{LSE-61}~{\tiny
 (p. 2)   }
  \end{tabular} &
    \begin{tabular}{@{}l@{}}
    \hypertarget{dms-req-0349-v-01}{DMS-REQ-0349-V-01}
    \\\vcdJiraRef{LVV-180}~{\tiny
 (p. 2)     }
    \end{tabular} &
        \begin{tabular}{@{}l@{}}
        \href{https://jira.lsstcorp.org/secure/Tests.jspa\#/testCase/LVV-T71}{LVV-T71} \\
        \vcdDocRef{LDM-639}
        \end{tabular} &
          & \notexec{} \\
  \midrule
  \begin{tabular}{@{}l@{}}
  DMS-REQ-0350\\\vcdDocRef{LSE-61}~{\tiny
 (p. 2)   }
  \end{tabular} &
    \begin{tabular}{@{}l@{}}
    \hypertarget{dms-req-0350-v-01}{DMS-REQ-0350-V-01}
    \\\vcdJiraRef{LVV-181}~{\tiny
 (p. 2)     }
    \end{tabular} &
        \begin{tabular}{@{}l@{}}
        \href{https://jira.lsstcorp.org/secure/Tests.jspa\#/testCase/LVV-T116}{LVV-T116} \\
        \vcdDocRef{LDM-639}
        \end{tabular} &
          & \notexec{} \\
  \midrule
  \begin{tabular}{@{}l@{}}
  DMS-REQ-0351\\\vcdDocRef{LSE-61}~{\tiny
 (p. 1a)   }
  \end{tabular} &
    \begin{tabular}{@{}l@{}}
    \hypertarget{dms-req-0351-v-01}{DMS-REQ-0351-V-01}
    \\\vcdJiraRef{LVV-182}~{\tiny
 (p. 1a)     }
    \end{tabular} &
        \begin{tabular}{@{}l@{}}
        \href{https://jira.lsstcorp.org/secure/Tests.jspa\#/testCase/LVV-T133}{LVV-T133} \\
        \vcdDocRef{LDM-639}
        \end{tabular} &
          & \notexec{} \\
  \midrule
  \begin{tabular}{@{}l@{}}
  DMS-REQ-0352\\\vcdDocRef{LSE-61}~{\tiny
 (p. 2)   }
  \end{tabular} &
    \begin{tabular}{@{}l@{}}
    \hypertarget{dms-req-0352-v-01}{DMS-REQ-0352-V-01}
    \\\vcdJiraRef{LVV-183}~{\tiny
 (p. 2)     }
    \end{tabular} &
        \begin{tabular}{@{}l@{}}
        \href{https://jira.lsstcorp.org/secure/Tests.jspa\#/testCase/LVV-T192}{LVV-T192} \\
        \vcdDocRef{LDM-639}
        \end{tabular} &
          & \notexec{} \\
      \cmidrule{2-5}
      &
    \begin{tabular}{@{}l@{}}
    \hypertarget{dms-req-0352-v-02}{DMS-REQ-0352-V-02}
    \\\vcdJiraRef{LVV-18491}~{\tiny
 (p. 2)     }
    \end{tabular} &
        \begin{tabular}{@{}l@{}}
        \href{https://jira.lsstcorp.org/secure/Tests.jspa\#/testCase/LVV-T181}{LVV-T181} \\
        \vcdDocRef{}
        \end{tabular} &
          & \notexec{} \\
  \midrule
  \begin{tabular}{@{}l@{}}
  DMS-REQ-0353\\\vcdDocRef{LSE-61}~{\tiny
 (p. 1b)   }
  \end{tabular} &
    \begin{tabular}{@{}l@{}}
    \hypertarget{dms-req-0353-v-01}{DMS-REQ-0353-V-01}
    \\\vcdJiraRef{LVV-184}~{\tiny
 (p. 1b)     }
    \end{tabular} &
        \begin{tabular}{@{}l@{}}
        \href{https://jira.lsstcorp.org/secure/Tests.jspa\#/testCase/LVV-T60}{LVV-T60} \\
        \vcdDocRef{LDM-639}
        \end{tabular} &
          & \notexec{} \\
  \midrule
  \begin{tabular}{@{}l@{}}
  DMS-REQ-0354\\\vcdDocRef{LSE-61}~{\tiny
 (p. 1b)   }
  \end{tabular} &
    \begin{tabular}{@{}l@{}}
    \hypertarget{dms-req-0354-v-01}{DMS-REQ-0354-V-01}
    \\\vcdJiraRef{LVV-185}~{\tiny
 (p. 1b)     }
    \end{tabular} &
        \begin{tabular}{@{}l@{}}
        \href{https://jira.lsstcorp.org/secure/Tests.jspa\#/testCase/LVV-T1086}{LVV-T1086} \\
        \vcdDocRef{LDM-552}
        \end{tabular} &
          \begin{tabular}{@{}l@{}}
          2019-07-08 \\
            \vcdDocRef{DMTR-71}
            {\scriptsize \href{https://jira.lsstcorp.org/secure/Tests.jspa\#/testPlan/LVV-P46}{LVV-P46} }
          \end{tabular} &
          \passed \\
          \cmidrule{3-5}
          & &
        \begin{tabular}{@{}l@{}}
        \href{https://jira.lsstcorp.org/secure/Tests.jspa\#/testCase/LVV-T1087}{LVV-T1087} \\
        \vcdDocRef{LDM-552}
        \end{tabular} &
          \begin{tabular}{@{}l@{}}
          2019-07-08 \\
            \vcdDocRef{DMTR-71}
            {\scriptsize \href{https://jira.lsstcorp.org/secure/Tests.jspa\#/testPlan/LVV-P46}{LVV-P46} }
          \end{tabular} &
          \passed \\
          \cmidrule{3-5}
          & &
        \begin{tabular}{@{}l@{}}
        \href{https://jira.lsstcorp.org/secure/Tests.jspa\#/testCase/LVV-T1088}{LVV-T1088} \\
        \vcdDocRef{LDM-552}
        \end{tabular} &
          \begin{tabular}{@{}l@{}}
          2019-07-09 \\
            \vcdDocRef{DMTR-71}
            {\scriptsize \href{https://jira.lsstcorp.org/secure/Tests.jspa\#/testPlan/LVV-P46}{LVV-P46} }
          \end{tabular} &
          \passed \\
          \cmidrule{3-5}
          & &
        \begin{tabular}{@{}l@{}}
        \href{https://jira.lsstcorp.org/secure/Tests.jspa\#/testCase/LVV-T1089}{LVV-T1089} \\
        \vcdDocRef{LDM-552}
        \end{tabular} &
          \begin{tabular}{@{}l@{}}
          2019-07-09 \\
            \vcdDocRef{DMTR-71}
            {\scriptsize \href{https://jira.lsstcorp.org/secure/Tests.jspa\#/testPlan/LVV-P46}{LVV-P46} }
          \end{tabular} &
          \cndpass \\
          \cmidrule{3-5}
          & &
        \begin{tabular}{@{}l@{}}
        \href{https://jira.lsstcorp.org/secure/Tests.jspa\#/testCase/LVV-T1090}{LVV-T1090} \\
        \vcdDocRef{LDM-552}
        \end{tabular} &
          \begin{tabular}{@{}l@{}}
          2019-07-09 \\
            \vcdDocRef{DMTR-71}
            {\scriptsize \href{https://jira.lsstcorp.org/secure/Tests.jspa\#/testPlan/LVV-P46}{LVV-P46} }
          \end{tabular} &
          \cndpass \\
  \midrule
  \begin{tabular}{@{}l@{}}
  DMS-REQ-0355\\\vcdDocRef{LSE-61}~{\tiny
 (p. 1b)   }
  \end{tabular} &
    \begin{tabular}{@{}l@{}}
    \hypertarget{dms-req-0355-v-01}{DMS-REQ-0355-V-01}
    \\\vcdJiraRef{LVV-186}~{\tiny
    }
    \end{tabular} &
        & & \\
      \cmidrule{2-5}
      &
    \begin{tabular}{@{}l@{}}
    \hypertarget{dms-req-0355-v-02}{DMS-REQ-0355-V-02}
    \\\vcdJiraRef{LVV-9784}~{\tiny
    }
    \end{tabular} &
        & & \\
  \midrule
  \begin{tabular}{@{}l@{}}
  DMS-REQ-0356\\\vcdDocRef{LSE-61}~{\tiny
 (p. 1b)   }
  \end{tabular} &
    \begin{tabular}{@{}l@{}}
    \hypertarget{dms-req-0356-v-01}{DMS-REQ-0356-V-01}
    \\\vcdJiraRef{LVV-187}~{\tiny
    }
    \end{tabular} &
        & & \\
      \cmidrule{2-5}
      &
    \begin{tabular}{@{}l@{}}
    \hypertarget{dms-req-0356-v-02}{DMS-REQ-0356-V-02}
    \\\vcdJiraRef{LVV-9785}~{\tiny
    }
    \end{tabular} &
        & & \\
      \cmidrule{2-5}
      &
    \begin{tabular}{@{}l@{}}
    \hypertarget{dms-req-0356-v-03}{DMS-REQ-0356-V-03}
    \\\vcdJiraRef{LVV-9786}~{\tiny
    }
    \end{tabular} &
        \begin{tabular}{@{}l@{}}
        \href{https://jira.lsstcorp.org/secure/Tests.jspa\#/testCase/LVV-T1089}{LVV-T1089} \\
        \vcdDocRef{LDM-552}
        \end{tabular} &
          \begin{tabular}{@{}l@{}}
          2019-07-09 \\
            \vcdDocRef{DMTR-71}
            {\scriptsize \href{https://jira.lsstcorp.org/secure/Tests.jspa\#/testPlan/LVV-P46}{LVV-P46} }
          \end{tabular} &
          \cndpass \\
          \cmidrule{3-5}
          & &
        \begin{tabular}{@{}l@{}}
        \href{https://jira.lsstcorp.org/secure/Tests.jspa\#/testCase/LVV-T1090}{LVV-T1090} \\
        \vcdDocRef{LDM-552}
        \end{tabular} &
          \begin{tabular}{@{}l@{}}
          2019-07-09 \\
            \vcdDocRef{DMTR-71}
            {\scriptsize \href{https://jira.lsstcorp.org/secure/Tests.jspa\#/testPlan/LVV-P46}{LVV-P46} }
          \end{tabular} &
          \cndpass \\
      \cmidrule{2-5}
      &
    \begin{tabular}{@{}l@{}}
    \hypertarget{dms-req-0356-v-04}{DMS-REQ-0356-V-04}
    \\\vcdJiraRef{LVV-9787}~{\tiny
    }
    \end{tabular} &
        \begin{tabular}{@{}l@{}}
        \href{https://jira.lsstcorp.org/secure/Tests.jspa\#/testCase/LVV-T1085}{LVV-T1085} \\
        \vcdDocRef{LDM-552}
        \end{tabular} &
          \begin{tabular}{@{}l@{}}
          2019-07-08 \\
            \vcdDocRef{DMTR-71}
            {\scriptsize \href{https://jira.lsstcorp.org/secure/Tests.jspa\#/testPlan/LVV-P46}{LVV-P46} }
          \end{tabular} &
          \passed \\
          \cmidrule{3-5}
          & &
        \begin{tabular}{@{}l@{}}
        \href{https://jira.lsstcorp.org/secure/Tests.jspa\#/testCase/LVV-T1089}{LVV-T1089} \\
        \vcdDocRef{LDM-552}
        \end{tabular} &
          \begin{tabular}{@{}l@{}}
          2019-07-09 \\
            \vcdDocRef{DMTR-71}
            {\scriptsize \href{https://jira.lsstcorp.org/secure/Tests.jspa\#/testPlan/LVV-P46}{LVV-P46} }
          \end{tabular} &
          \cndpass \\
          \cmidrule{3-5}
          & &
        \begin{tabular}{@{}l@{}}
        \href{https://jira.lsstcorp.org/secure/Tests.jspa\#/testCase/LVV-T1090}{LVV-T1090} \\
        \vcdDocRef{LDM-552}
        \end{tabular} &
          \begin{tabular}{@{}l@{}}
          2019-07-09 \\
            \vcdDocRef{DMTR-71}
            {\scriptsize \href{https://jira.lsstcorp.org/secure/Tests.jspa\#/testPlan/LVV-P46}{LVV-P46} }
          \end{tabular} &
          \cndpass \\
  \midrule
  \begin{tabular}{@{}l@{}}
  DMS-REQ-0357\\\vcdDocRef{LSE-61}~{\tiny
 (p. 1b)   }
  \end{tabular} &
    \begin{tabular}{@{}l@{}}
    \hypertarget{dms-req-0357-v-01}{DMS-REQ-0357-V-01}
    \\\vcdJiraRef{LVV-188}~{\tiny
 (p. 1b)     }
    \end{tabular} &
        \begin{tabular}{@{}l@{}}
        \href{https://jira.lsstcorp.org/secure/Tests.jspa\#/testCase/LVV-T1086}{LVV-T1086} \\
        \vcdDocRef{LDM-552}
        \end{tabular} &
          \begin{tabular}{@{}l@{}}
          2019-07-08 \\
            \vcdDocRef{DMTR-71}
            {\scriptsize \href{https://jira.lsstcorp.org/secure/Tests.jspa\#/testPlan/LVV-P46}{LVV-P46} }
          \end{tabular} &
          \passed \\
          \cmidrule{3-5}
          & &
        \begin{tabular}{@{}l@{}}
        \href{https://jira.lsstcorp.org/secure/Tests.jspa\#/testCase/LVV-T1088}{LVV-T1088} \\
        \vcdDocRef{LDM-552}
        \end{tabular} &
          \begin{tabular}{@{}l@{}}
          2019-07-09 \\
            \vcdDocRef{DMTR-71}
            {\scriptsize \href{https://jira.lsstcorp.org/secure/Tests.jspa\#/testPlan/LVV-P46}{LVV-P46} }
          \end{tabular} &
          \passed \\
          \cmidrule{3-5}
          & &
        \begin{tabular}{@{}l@{}}
        \href{https://jira.lsstcorp.org/secure/Tests.jspa\#/testCase/LVV-T1089}{LVV-T1089} \\
        \vcdDocRef{LDM-552}
        \end{tabular} &
          \begin{tabular}{@{}l@{}}
          2019-07-09 \\
            \vcdDocRef{DMTR-71}
            {\scriptsize \href{https://jira.lsstcorp.org/secure/Tests.jspa\#/testPlan/LVV-P46}{LVV-P46} }
          \end{tabular} &
          \cndpass \\
          \cmidrule{3-5}
          & &
        \begin{tabular}{@{}l@{}}
        \href{https://jira.lsstcorp.org/secure/Tests.jspa\#/testCase/LVV-T1090}{LVV-T1090} \\
        \vcdDocRef{LDM-552}
        \end{tabular} &
          \begin{tabular}{@{}l@{}}
          2019-07-09 \\
            \vcdDocRef{DMTR-71}
            {\scriptsize \href{https://jira.lsstcorp.org/secure/Tests.jspa\#/testPlan/LVV-P46}{LVV-P46} }
          \end{tabular} &
          \cndpass \\
  \midrule
  \begin{tabular}{@{}l@{}}
  DMS-REQ-0363\\\vcdDocRef{LSE-61}~{\tiny
 (p. 3)   }
  \end{tabular} &
    \begin{tabular}{@{}l@{}}
    \hypertarget{dms-req-0363-v-01}{DMS-REQ-0363-V-01}
    \\\vcdJiraRef{LVV-189}~{\tiny
 (p. 3)     }
    \end{tabular} &
        \begin{tabular}{@{}l@{}}
        \href{https://jira.lsstcorp.org/secure/Tests.jspa\#/testCase/LVV-T162}{LVV-T162} \\
        \vcdDocRef{LDM-639}
        \end{tabular} &
          & \notexec{} \\
  \midrule
  \begin{tabular}{@{}l@{}}
  DMS-REQ-0364\\\vcdDocRef{LSE-61}~{\tiny
 (p. 3)   }
  \end{tabular} &
    \begin{tabular}{@{}l@{}}
    \hypertarget{dms-req-0364-v-01}{DMS-REQ-0364-V-01}
    \\\vcdJiraRef{LVV-190}~{\tiny
    }
    \end{tabular} &
        \begin{tabular}{@{}l@{}}
        \href{https://jira.lsstcorp.org/secure/Tests.jspa\#/testCase/LVV-T163}{LVV-T163} \\
        \vcdDocRef{LDM-639}
        \end{tabular} &
          & \notexec{} \\
      \cmidrule{2-5}
      &
    \begin{tabular}{@{}l@{}}
    \hypertarget{dms-req-0364-v-02}{DMS-REQ-0364-V-02}
    \\\vcdJiraRef{LVV-9750}~{\tiny
    }
    \end{tabular} &
        & & \\
  \midrule
  \begin{tabular}{@{}l@{}}
  DMS-REQ-0365\\\vcdDocRef{LSE-61}~{\tiny
 (p. 2)   }
  \end{tabular} &
    \begin{tabular}{@{}l@{}}
    \hypertarget{dms-req-0365-v-01}{DMS-REQ-0365-V-01}
    \\\vcdJiraRef{LVV-191}~{\tiny
 (p. 2)     }
    \end{tabular} &
        \begin{tabular}{@{}l@{}}
        \href{https://jira.lsstcorp.org/secure/Tests.jspa\#/testCase/LVV-T164}{LVV-T164} \\
        \vcdDocRef{LDM-639}
        \end{tabular} &
          & \notexec{} \\
  \midrule
  \begin{tabular}{@{}l@{}}
  DMS-REQ-0366\\\vcdDocRef{LSE-61}~{\tiny
 (p. 2)   }
  \end{tabular} &
    \begin{tabular}{@{}l@{}}
    \hypertarget{dms-req-0366-v-01}{DMS-REQ-0366-V-01}
    \\\vcdJiraRef{LVV-192}~{\tiny
 (p. 2)     }
    \end{tabular} &
        \begin{tabular}{@{}l@{}}
        \href{https://jira.lsstcorp.org/secure/Tests.jspa\#/testCase/LVV-T165}{LVV-T165} \\
        \vcdDocRef{LDM-639}
        \end{tabular} &
          & \notexec{} \\
  \midrule
  \begin{tabular}{@{}l@{}}
  DMS-REQ-0367\\\vcdDocRef{LSE-61}~{\tiny
 (p. 2)   }
  \end{tabular} &
    \begin{tabular}{@{}l@{}}
    \hypertarget{dms-req-0367-v-01}{DMS-REQ-0367-V-01}
    \\\vcdJiraRef{LVV-193}~{\tiny
 (p. 2)     }
    \end{tabular} &
        \begin{tabular}{@{}l@{}}
        \href{https://jira.lsstcorp.org/secure/Tests.jspa\#/testCase/LVV-T166}{LVV-T166} \\
        \vcdDocRef{LDM-639}
        \end{tabular} &
          & \notexec{} \\
  \midrule
  \begin{tabular}{@{}l@{}}
  DMS-REQ-0368\\\vcdDocRef{LSE-61}~{\tiny
 (p. 3)   }
  \end{tabular} &
    \begin{tabular}{@{}l@{}}
    \hypertarget{dms-req-0368-v-01}{DMS-REQ-0368-V-01}
    \\\vcdJiraRef{LVV-194}~{\tiny
 (p. 3)     }
    \end{tabular} &
        \begin{tabular}{@{}l@{}}
        \href{https://jira.lsstcorp.org/secure/Tests.jspa\#/testCase/LVV-T167}{LVV-T167} \\
        \vcdDocRef{LDM-639}
        \end{tabular} &
          & \notexec{} \\
  \midrule
  \begin{tabular}{@{}l@{}}
  DMS-REQ-0369\\\vcdDocRef{LSE-61}~{\tiny
 (p. 1b)   }
  \end{tabular} &
    \begin{tabular}{@{}l@{}}
    \hypertarget{dms-req-0369-v-01}{DMS-REQ-0369-V-01}
    \\\vcdJiraRef{LVV-195}~{\tiny
 (p. 1b)     }
    \end{tabular} &
        \begin{tabular}{@{}l@{}}
        \href{https://jira.lsstcorp.org/secure/Tests.jspa\#/testCase/LVV-T168}{LVV-T168} \\
        \vcdDocRef{LDM-639}
        \end{tabular} &
          & \notexec{} \\
  \midrule
  \begin{tabular}{@{}l@{}}
  DMS-REQ-0370\\\vcdDocRef{LSE-61}~{\tiny
 (p. 3)   }
  \end{tabular} &
    \begin{tabular}{@{}l@{}}
    \hypertarget{dms-req-0370-v-01}{DMS-REQ-0370-V-01}
    \\\vcdJiraRef{LVV-196}~{\tiny
 (p. 3)     }
    \end{tabular} &
        \begin{tabular}{@{}l@{}}
        \href{https://jira.lsstcorp.org/secure/Tests.jspa\#/testCase/LVV-T169}{LVV-T169} \\
        \vcdDocRef{LDM-639}
        \end{tabular} &
          & \notexec{} \\
  \midrule
  \begin{tabular}{@{}l@{}}
  DMS-REQ-0371\\\vcdDocRef{LSE-61}~{\tiny
 (p. 3)   }
  \end{tabular} &
    \begin{tabular}{@{}l@{}}
    \hypertarget{dms-req-0371-v-01}{DMS-REQ-0371-V-01}
    \\\vcdJiraRef{LVV-197}~{\tiny
 (p. 3)     }
    \end{tabular} &
        \begin{tabular}{@{}l@{}}
        \href{https://jira.lsstcorp.org/secure/Tests.jspa\#/testCase/LVV-T170}{LVV-T170} \\
        \vcdDocRef{LDM-639}
        \end{tabular} &
          & \notexec{} \\
  \midrule
  \begin{tabular}{@{}l@{}}
  DMS-REQ-0377\\\vcdDocRef{LSE-61}~{\tiny
 (p. 1b)   }
  \end{tabular} &
    \begin{tabular}{@{}l@{}}
    \hypertarget{dms-req-0377-v-01}{DMS-REQ-0377-V-01}
    \\\vcdJiraRef{LVV-3394}~{\tiny
    }
    \end{tabular} &
        \begin{tabular}{@{}l@{}}
        \href{https://jira.lsstcorp.org/secure/Tests.jspa\#/testCase/LVV-T385}{LVV-T385} \\
        \vcdDocRef{LDM-639}
        \end{tabular} &
          & \notexec{} \\
      \cmidrule{2-5}
      &
    \begin{tabular}{@{}l@{}}
    \hypertarget{dms-req-0377-v-02}{DMS-REQ-0377-V-02}
    \\\vcdJiraRef{LVV-9797}~{\tiny
    }
    \end{tabular} &
        \begin{tabular}{@{}l@{}}
        \href{https://jira.lsstcorp.org/secure/Tests.jspa\#/testCase/LVV-T1332}{LVV-T1332} \\
        \vcdDocRef{LDM-639}
        \end{tabular} &
          & \notexec{} \\
  \midrule
  \begin{tabular}{@{}l@{}}
  DMS-REQ-0374\\\vcdDocRef{LSE-61}~{\tiny
 (p. 1b)   }
  \end{tabular} &
    \begin{tabular}{@{}l@{}}
    \hypertarget{dms-req-0374-v-01}{DMS-REQ-0374-V-01}
    \\\vcdJiraRef{LVV-3395}~{\tiny
    }
    \end{tabular} &
        & & \\
      \cmidrule{2-5}
      &
    \begin{tabular}{@{}l@{}}
    \hypertarget{dms-req-0374-v-02}{DMS-REQ-0374-V-02}
    \\\vcdJiraRef{LVV-9790}~{\tiny
    }
    \end{tabular} &
        & & \\
      \cmidrule{2-5}
      &
    \begin{tabular}{@{}l@{}}
    \hypertarget{dms-req-0374-v-03}{DMS-REQ-0374-V-03}
    \\\vcdJiraRef{LVV-9791}~{\tiny
    }
    \end{tabular} &
        & & \\
  \midrule
  \begin{tabular}{@{}l@{}}
  DMS-REQ-0376\\\vcdDocRef{LSE-61}~{\tiny
 (p. 1b)   }
  \end{tabular} &
    \begin{tabular}{@{}l@{}}
    \hypertarget{dms-req-0376-v-01}{DMS-REQ-0376-V-01}
    \\\vcdJiraRef{LVV-3396}~{\tiny
    }
    \end{tabular} &
        & & \\
      \cmidrule{2-5}
      &
    \begin{tabular}{@{}l@{}}
    \hypertarget{dms-req-0376-v-02}{DMS-REQ-0376-V-02}
    \\\vcdJiraRef{LVV-9795}~{\tiny
    }
    \end{tabular} &
        & & \\
      \cmidrule{2-5}
      &
    \begin{tabular}{@{}l@{}}
    \hypertarget{dms-req-0376-v-03}{DMS-REQ-0376-V-03}
    \\\vcdJiraRef{LVV-9796}~{\tiny
    }
    \end{tabular} &
        & & \\
  \midrule
  \begin{tabular}{@{}l@{}}
  DMS-REQ-0373\\\vcdDocRef{LSE-61}~{\tiny
 (p. 2)   }
  \end{tabular} &
    \begin{tabular}{@{}l@{}}
    \hypertarget{dms-req-0373-v-01}{DMS-REQ-0373-V-01}
    \\\vcdJiraRef{LVV-3397}~{\tiny
    }
    \end{tabular} &
        & & \\
      \cmidrule{2-5}
      &
    \begin{tabular}{@{}l@{}}
    \hypertarget{dms-req-0373-v-02}{DMS-REQ-0373-V-02}
    \\\vcdJiraRef{LVV-9789}~{\tiny
    }
    \end{tabular} &
        & & \\
  \midrule
  \begin{tabular}{@{}l@{}}
  DMS-REQ-0375\\\vcdDocRef{LSE-61}~{\tiny
 (p. 2)   }
  \end{tabular} &
    \begin{tabular}{@{}l@{}}
    \hypertarget{dms-req-0375-v-01}{DMS-REQ-0375-V-01}
    \\\vcdJiraRef{LVV-3398}~{\tiny
    }
    \end{tabular} &
        & & \\
      \cmidrule{2-5}
      &
    \begin{tabular}{@{}l@{}}
    \hypertarget{dms-req-0375-v-02}{DMS-REQ-0375-V-02}
    \\\vcdJiraRef{LVV-9792}~{\tiny
    }
    \end{tabular} &
        & & \\
      \cmidrule{2-5}
      &
    \begin{tabular}{@{}l@{}}
    \hypertarget{dms-req-0375-v-03}{DMS-REQ-0375-V-03}
    \\\vcdJiraRef{LVV-9793}~{\tiny
    }
    \end{tabular} &
        & & \\
      \cmidrule{2-5}
      &
    \begin{tabular}{@{}l@{}}
    \hypertarget{dms-req-0375-v-04}{DMS-REQ-0375-V-04}
    \\\vcdJiraRef{LVV-9794}~{\tiny
    }
    \end{tabular} &
        & & \\
  \midrule
  \begin{tabular}{@{}l@{}}
  DMS-REQ-0378\\\vcdDocRef{LSE-61}~{\tiny
 (p. 2)   }
  \end{tabular} &
    \begin{tabular}{@{}l@{}}
    \hypertarget{dms-req-0378-v-01}{DMS-REQ-0378-V-01}
    \\\vcdJiraRef{LVV-3399}~{\tiny
    }
    \end{tabular} &
        & & \\
  \midrule
  \begin{tabular}{@{}l@{}}
  DMS-REQ-0358\\\vcdDocRef{LSE-61}~{\tiny
 (p. 1a)   }
  \end{tabular} &
    \begin{tabular}{@{}l@{}}
    \hypertarget{dms-req-0358-v-01}{DMS-REQ-0358-V-01}
    \\\vcdJiraRef{LVV-3400}~{\tiny
    }
    \end{tabular} &
        \begin{tabular}{@{}l@{}}
        \href{https://jira.lsstcorp.org/secure/Tests.jspa\#/testCase/LVV-T1250}{LVV-T1250} \\
        \vcdDocRef{LDM-639}
        \end{tabular} &
          & \notexec{} \\
      \cmidrule{2-5}
      &
    \begin{tabular}{@{}l@{}}
    \hypertarget{dms-req-0358-v-02}{DMS-REQ-0358-V-02}
    \\\vcdJiraRef{LVV-9788}~{\tiny
    }
    \end{tabular} &
        \begin{tabular}{@{}l@{}}
        \href{https://jira.lsstcorp.org/secure/Tests.jspa\#/testCase/LVV-T1251}{LVV-T1251} \\
        \vcdDocRef{LDM-639}
        \end{tabular} &
          & \notexec{} \\
  \midrule
  \begin{tabular}{@{}l@{}}
  DMS-REQ-0359\\\vcdDocRef{LSE-61}~{\tiny
 (p. 1a)   }
  \end{tabular} &
    \begin{tabular}{@{}l@{}}
    \hypertarget{dms-req-0359-v-01}{DMS-REQ-0359-V-01}
    \\\vcdJiraRef{LVV-3401}~{\tiny
    }
    \end{tabular} &
        \begin{tabular}{@{}l@{}}
        \href{https://jira.lsstcorp.org/secure/Tests.jspa\#/testCase/LVV-T1756}{LVV-T1756} \\
        \vcdDocRef{LDM-639}
        \end{tabular} &
          & \notexec{} \\
      \cmidrule{2-5}
      &
    \begin{tabular}{@{}l@{}}
    \hypertarget{dms-req-0359-v-02}{DMS-REQ-0359-V-02}
    \\\vcdJiraRef{LVV-9751}~{\tiny
    }
    \end{tabular} &
        \begin{tabular}{@{}l@{}}
        \href{https://jira.lsstcorp.org/secure/Tests.jspa\#/testCase/LVV-T377}{LVV-T377} \\
        \vcdDocRef{LDM-639}
        \end{tabular} &
          & \notexec{} \\
          \cmidrule{3-5}
          & &
        \begin{tabular}{@{}l@{}}
        \href{https://jira.lsstcorp.org/secure/Tests.jspa\#/testCase/LVV-T1847}{LVV-T1847} \\
        \vcdDocRef{LDM-639}
        \end{tabular} &
          & \notexec{} \\
      \cmidrule{2-5}
      &
    \begin{tabular}{@{}l@{}}
    \hypertarget{dms-req-0359-v-03}{DMS-REQ-0359-V-03}
    \\\vcdJiraRef{LVV-9752}~{\tiny
    }
    \end{tabular} &
        \begin{tabular}{@{}l@{}}
        \href{https://jira.lsstcorp.org/secure/Tests.jspa\#/testCase/LVV-T1758}{LVV-T1758} \\
        \vcdDocRef{LDM-639}
        \end{tabular} &
          \begin{tabular}{@{}l@{}}
          2020-02-07 \\
            \vcdDocRef{DMTR-201}
            {\scriptsize \href{https://jira.lsstcorp.org/secure/Tests.jspa\#/testPlan/LVV-P65}{LVV-P65} }
          \end{tabular} &
          \failed \\
          \cmidrule{3-5}
          & &
        \begin{tabular}{@{}l@{}}
        \href{https://jira.lsstcorp.org/secure/Tests.jspa\#/testCase/LVV-T1759}{LVV-T1759} \\
        \vcdDocRef{LDM-639}
        \end{tabular} &
          \begin{tabular}{@{}l@{}}
          2020-02-07 \\
            \vcdDocRef{DMTR-201}
            {\scriptsize \href{https://jira.lsstcorp.org/secure/Tests.jspa\#/testPlan/LVV-P65}{LVV-P65} }
          \end{tabular} &
          \passed \\
      \cmidrule{2-5}
      &
    \begin{tabular}{@{}l@{}}
    \hypertarget{dms-req-0359-v-04}{DMS-REQ-0359-V-04}
    \\\vcdJiraRef{LVV-9753}~{\tiny
    }
    \end{tabular} &
        \begin{tabular}{@{}l@{}}
        \href{https://jira.lsstcorp.org/secure/Tests.jspa\#/testCase/LVV-T377}{LVV-T377} \\
        \vcdDocRef{LDM-639}
        \end{tabular} &
          & \notexec{} \\
          \cmidrule{3-5}
          & &
        \begin{tabular}{@{}l@{}}
        \href{https://jira.lsstcorp.org/secure/Tests.jspa\#/testCase/LVV-T1846}{LVV-T1846} \\
        \vcdDocRef{LDM-639}
        \end{tabular} &
          & \notexec{} \\
      \cmidrule{2-5}
      &
    \begin{tabular}{@{}l@{}}
    \hypertarget{dms-req-0359-v-05}{DMS-REQ-0359-V-05}
    \\\vcdJiraRef{LVV-9754}~{\tiny
    }
    \end{tabular} &
        \begin{tabular}{@{}l@{}}
        \href{https://jira.lsstcorp.org/secure/Tests.jspa\#/testCase/LVV-T1759}{LVV-T1759} \\
        \vcdDocRef{LDM-639}
        \end{tabular} &
          \begin{tabular}{@{}l@{}}
          2020-02-07 \\
            \vcdDocRef{DMTR-201}
            {\scriptsize \href{https://jira.lsstcorp.org/secure/Tests.jspa\#/testPlan/LVV-P65}{LVV-P65} }
          \end{tabular} &
          \passed \\
      \cmidrule{2-5}
      &
    \begin{tabular}{@{}l@{}}
    \hypertarget{dms-req-0359-v-06}{DMS-REQ-0359-V-06}
    \\\vcdJiraRef{LVV-9755}~{\tiny
    }
    \end{tabular} &
        \begin{tabular}{@{}l@{}}
        \href{https://jira.lsstcorp.org/secure/Tests.jspa\#/testCase/LVV-T377}{LVV-T377} \\
        \vcdDocRef{LDM-639}
        \end{tabular} &
          & \notexec{} \\
          \cmidrule{3-5}
          & &
        \begin{tabular}{@{}l@{}}
        \href{https://jira.lsstcorp.org/secure/Tests.jspa\#/testCase/LVV-T1845}{LVV-T1845} \\
        \vcdDocRef{LDM-639}
        \end{tabular} &
          & \notexec{} \\
      \cmidrule{2-5}
      &
    \begin{tabular}{@{}l@{}}
    \hypertarget{dms-req-0359-v-07}{DMS-REQ-0359-V-07}
    \\\vcdJiraRef{LVV-9756}~{\tiny
    }
    \end{tabular} &
        \begin{tabular}{@{}l@{}}
        \href{https://jira.lsstcorp.org/secure/Tests.jspa\#/testCase/LVV-T377}{LVV-T377} \\
        \vcdDocRef{LDM-639}
        \end{tabular} &
          & \notexec{} \\
          \cmidrule{3-5}
          & &
        \begin{tabular}{@{}l@{}}
        \href{https://jira.lsstcorp.org/secure/Tests.jspa\#/testCase/LVV-T1844}{LVV-T1844} \\
        \vcdDocRef{LDM-639}
        \end{tabular} &
          & \notexec{} \\
      \cmidrule{2-5}
      &
    \begin{tabular}{@{}l@{}}
    \hypertarget{dms-req-0359-v-08}{DMS-REQ-0359-V-08}
    \\\vcdJiraRef{LVV-9757}~{\tiny
    }
    \end{tabular} &
        \begin{tabular}{@{}l@{}}
        \href{https://jira.lsstcorp.org/secure/Tests.jspa\#/testCase/LVV-T377}{LVV-T377} \\
        \vcdDocRef{LDM-639}
        \end{tabular} &
          & \notexec{} \\
          \cmidrule{3-5}
          & &
        \begin{tabular}{@{}l@{}}
        \href{https://jira.lsstcorp.org/secure/Tests.jspa\#/testCase/LVV-T1843}{LVV-T1843} \\
        \vcdDocRef{LDM-639}
        \end{tabular} &
          & \notexec{} \\
      \cmidrule{2-5}
      &
    \begin{tabular}{@{}l@{}}
    \hypertarget{dms-req-0359-v-09}{DMS-REQ-0359-V-09}
    \\\vcdJiraRef{LVV-9758}~{\tiny
    }
    \end{tabular} &
        \begin{tabular}{@{}l@{}}
        \href{https://jira.lsstcorp.org/secure/Tests.jspa\#/testCase/LVV-T1758}{LVV-T1758} \\
        \vcdDocRef{LDM-639}
        \end{tabular} &
          \begin{tabular}{@{}l@{}}
          2020-02-07 \\
            \vcdDocRef{DMTR-201}
            {\scriptsize \href{https://jira.lsstcorp.org/secure/Tests.jspa\#/testPlan/LVV-P65}{LVV-P65} }
          \end{tabular} &
          \failed \\
      \cmidrule{2-5}
      &
    \begin{tabular}{@{}l@{}}
    \hypertarget{dms-req-0359-v-10}{DMS-REQ-0359-V-10}
    \\\vcdJiraRef{LVV-9759}~{\tiny
    }
    \end{tabular} &
        \begin{tabular}{@{}l@{}}
        \href{https://jira.lsstcorp.org/secure/Tests.jspa\#/testCase/LVV-T1757}{LVV-T1757} \\
        \vcdDocRef{LDM-639}
        \end{tabular} &
          & \notexec{} \\
      \cmidrule{2-5}
      &
    \begin{tabular}{@{}l@{}}
    \hypertarget{dms-req-0359-v-11}{DMS-REQ-0359-V-11}
    \\\vcdJiraRef{LVV-9760}~{\tiny
    }
    \end{tabular} &
        \begin{tabular}{@{}l@{}}
        \href{https://jira.lsstcorp.org/secure/Tests.jspa\#/testCase/LVV-T377}{LVV-T377} \\
        \vcdDocRef{LDM-639}
        \end{tabular} &
          & \notexec{} \\
          \cmidrule{3-5}
          & &
        \begin{tabular}{@{}l@{}}
        \href{https://jira.lsstcorp.org/secure/Tests.jspa\#/testCase/LVV-T1842}{LVV-T1842} \\
        \vcdDocRef{LDM-639}
        \end{tabular} &
          & \notexec{} \\
      \cmidrule{2-5}
      &
    \begin{tabular}{@{}l@{}}
    \hypertarget{dms-req-0359-v-12}{DMS-REQ-0359-V-12}
    \\\vcdJiraRef{LVV-9761}~{\tiny
    }
    \end{tabular} &
        \begin{tabular}{@{}l@{}}
        \href{https://jira.lsstcorp.org/secure/Tests.jspa\#/testCase/LVV-T377}{LVV-T377} \\
        \vcdDocRef{LDM-639}
        \end{tabular} &
          & \notexec{} \\
          \cmidrule{3-5}
          & &
        \begin{tabular}{@{}l@{}}
        \href{https://jira.lsstcorp.org/secure/Tests.jspa\#/testCase/LVV-T1841}{LVV-T1841} \\
        \vcdDocRef{LDM-639}
        \end{tabular} &
          & \notexec{} \\
      \cmidrule{2-5}
      &
    \begin{tabular}{@{}l@{}}
    \hypertarget{dms-req-0359-v-13}{DMS-REQ-0359-V-13}
    \\\vcdJiraRef{LVV-9762}~{\tiny
    }
    \end{tabular} &
        \begin{tabular}{@{}l@{}}
        \href{https://jira.lsstcorp.org/secure/Tests.jspa\#/testCase/LVV-T377}{LVV-T377} \\
        \vcdDocRef{LDM-639}
        \end{tabular} &
          & \notexec{} \\
          \cmidrule{3-5}
          & &
        \begin{tabular}{@{}l@{}}
        \href{https://jira.lsstcorp.org/secure/Tests.jspa\#/testCase/LVV-T1840}{LVV-T1840} \\
        \vcdDocRef{LDM-639}
        \end{tabular} &
          & \notexec{} \\
      \cmidrule{2-5}
      &
    \begin{tabular}{@{}l@{}}
    \hypertarget{dms-req-0359-v-14}{DMS-REQ-0359-V-14}
    \\\vcdJiraRef{LVV-9763}~{\tiny
    }
    \end{tabular} &
        \begin{tabular}{@{}l@{}}
        \href{https://jira.lsstcorp.org/secure/Tests.jspa\#/testCase/LVV-T377}{LVV-T377} \\
        \vcdDocRef{LDM-639}
        \end{tabular} &
          & \notexec{} \\
          \cmidrule{3-5}
          & &
        \begin{tabular}{@{}l@{}}
        \href{https://jira.lsstcorp.org/secure/Tests.jspa\#/testCase/LVV-T1839}{LVV-T1839} \\
        \vcdDocRef{LDM-639}
        \end{tabular} &
          & \notexec{} \\
      \cmidrule{2-5}
      &
    \begin{tabular}{@{}l@{}}
    \hypertarget{dms-req-0359-v-15}{DMS-REQ-0359-V-15}
    \\\vcdJiraRef{LVV-9764}~{\tiny
    }
    \end{tabular} &
        \begin{tabular}{@{}l@{}}
        \href{https://jira.lsstcorp.org/secure/Tests.jspa\#/testCase/LVV-T377}{LVV-T377} \\
        \vcdDocRef{LDM-639}
        \end{tabular} &
          & \notexec{} \\
          \cmidrule{3-5}
          & &
        \begin{tabular}{@{}l@{}}
        \href{https://jira.lsstcorp.org/secure/Tests.jspa\#/testCase/LVV-T1838}{LVV-T1838} \\
        \vcdDocRef{LDM-639}
        \end{tabular} &
          & \notexec{} \\
      \cmidrule{2-5}
      &
    \begin{tabular}{@{}l@{}}
    \hypertarget{dms-req-0359-v-16}{DMS-REQ-0359-V-16}
    \\\vcdJiraRef{LVV-9765}~{\tiny
    }
    \end{tabular} &
        \begin{tabular}{@{}l@{}}
        \href{https://jira.lsstcorp.org/secure/Tests.jspa\#/testCase/LVV-T377}{LVV-T377} \\
        \vcdDocRef{LDM-639}
        \end{tabular} &
          & \notexec{} \\
          \cmidrule{3-5}
          & &
        \begin{tabular}{@{}l@{}}
        \href{https://jira.lsstcorp.org/secure/Tests.jspa\#/testCase/LVV-T1837}{LVV-T1837} \\
        \vcdDocRef{LDM-639}
        \end{tabular} &
          & \notexec{} \\
      \cmidrule{2-5}
      &
    \begin{tabular}{@{}l@{}}
    \hypertarget{dms-req-0359-v-17}{DMS-REQ-0359-V-17}
    \\\vcdJiraRef{LVV-9766}~{\tiny
    }
    \end{tabular} &
        \begin{tabular}{@{}l@{}}
        \href{https://jira.lsstcorp.org/secure/Tests.jspa\#/testCase/LVV-T377}{LVV-T377} \\
        \vcdDocRef{LDM-639}
        \end{tabular} &
          & \notexec{} \\
          \cmidrule{3-5}
          & &
        \begin{tabular}{@{}l@{}}
        \href{https://jira.lsstcorp.org/secure/Tests.jspa\#/testCase/LVV-T1836}{LVV-T1836} \\
        \vcdDocRef{LDM-639}
        \end{tabular} &
          & \notexec{} \\
      \cmidrule{2-5}
      &
    \begin{tabular}{@{}l@{}}
    \hypertarget{dms-req-0359-v-18}{DMS-REQ-0359-V-18}
    \\\vcdJiraRef{LVV-18339}~{\tiny
    }
    \end{tabular} &
        & & \\
  \midrule
  \begin{tabular}{@{}l@{}}
  DMS-REQ-0360\\\vcdDocRef{LSE-61}~{\tiny
 (p. 1a)   }
  \end{tabular} &
    \begin{tabular}{@{}l@{}}
    \hypertarget{dms-req-0360-v-01}{DMS-REQ-0360-V-01}
    \\\vcdJiraRef{LVV-3402}~{\tiny
    }
    \end{tabular} &
        \begin{tabular}{@{}l@{}}
        \href{https://jira.lsstcorp.org/secure/Tests.jspa\#/testCase/LVV-T363}{LVV-T363} \\
        \vcdDocRef{}
        \end{tabular} &
          \begin{tabular}{@{}l@{}}
          2019-03-31 \\
            \vcdDocRef{DMTR-111}
            {\scriptsize \href{https://jira.lsstcorp.org/secure/Tests.jspa\#/testPlan/LVV-P15}{LVV-P15} }
          \end{tabular} &
          \passed \\
          \cmidrule{3-5}
          & &
        \begin{tabular}{@{}l@{}}
        \href{https://jira.lsstcorp.org/secure/Tests.jspa\#/testCase/LVV-T1745}{LVV-T1745} \\
        \vcdDocRef{LDM-639}
        \end{tabular} &
          \begin{tabular}{@{}l@{}}
          2020-02-06 \\
            \vcdDocRef{DMTR-201}
            {\scriptsize \href{https://jira.lsstcorp.org/secure/Tests.jspa\#/testPlan/LVV-P65}{LVV-P65} }
          \end{tabular} &
          \passed \\
      \cmidrule{2-5}
      &
    \begin{tabular}{@{}l@{}}
    \hypertarget{dms-req-0360-v-02}{DMS-REQ-0360-V-02}
    \\\vcdJiraRef{LVV-9767}~{\tiny
    }
    \end{tabular} &
        \begin{tabular}{@{}l@{}}
        \href{https://jira.lsstcorp.org/secure/Tests.jspa\#/testCase/LVV-T378}{LVV-T378} \\
        \vcdDocRef{LDM-639}
        \end{tabular} &
          & \notexec{} \\
          \cmidrule{3-5}
          & &
        \begin{tabular}{@{}l@{}}
        \href{https://jira.lsstcorp.org/secure/Tests.jspa\#/testCase/LVV-T1746}{LVV-T1746} \\
        \vcdDocRef{LDM-639}
        \end{tabular} &
          \begin{tabular}{@{}l@{}}
          2020-02-06 \\
            \vcdDocRef{DMTR-201}
            {\scriptsize \href{https://jira.lsstcorp.org/secure/Tests.jspa\#/testPlan/LVV-P65}{LVV-P65} }
          \end{tabular} &
          \passed \\
      \cmidrule{2-5}
      &
    \begin{tabular}{@{}l@{}}
    \hypertarget{dms-req-0360-v-03}{DMS-REQ-0360-V-03}
    \\\vcdJiraRef{LVV-9768}~{\tiny
    }
    \end{tabular} &
        \begin{tabular}{@{}l@{}}
        \href{https://jira.lsstcorp.org/secure/Tests.jspa\#/testCase/LVV-T378}{LVV-T378} \\
        \vcdDocRef{LDM-639}
        \end{tabular} &
          & \notexec{} \\
          \cmidrule{3-5}
          & &
        \begin{tabular}{@{}l@{}}
        \href{https://jira.lsstcorp.org/secure/Tests.jspa\#/testCase/LVV-T1747}{LVV-T1747} \\
        \vcdDocRef{LDM-639}
        \end{tabular} &
          \begin{tabular}{@{}l@{}}
          2020-02-06 \\
            \vcdDocRef{DMTR-201}
            {\scriptsize \href{https://jira.lsstcorp.org/secure/Tests.jspa\#/testPlan/LVV-P65}{LVV-P65} }
          \end{tabular} &
          \passed \\
      \cmidrule{2-5}
      &
    \begin{tabular}{@{}l@{}}
    \hypertarget{dms-req-0360-v-04}{DMS-REQ-0360-V-04}
    \\\vcdJiraRef{LVV-9769}~{\tiny
    }
    \end{tabular} &
        \begin{tabular}{@{}l@{}}
        \href{https://jira.lsstcorp.org/secure/Tests.jspa\#/testCase/LVV-T378}{LVV-T378} \\
        \vcdDocRef{LDM-639}
        \end{tabular} &
          & \notexec{} \\
          \cmidrule{3-5}
          & &
        \begin{tabular}{@{}l@{}}
        \href{https://jira.lsstcorp.org/secure/Tests.jspa\#/testCase/LVV-T1748}{LVV-T1748} \\
        \vcdDocRef{LDM-639}
        \end{tabular} &
          \begin{tabular}{@{}l@{}}
          2020-02-06 \\
            \vcdDocRef{DMTR-201}
            {\scriptsize \href{https://jira.lsstcorp.org/secure/Tests.jspa\#/testPlan/LVV-P65}{LVV-P65} }
          \end{tabular} &
          \passed \\
      \cmidrule{2-5}
      &
    \begin{tabular}{@{}l@{}}
    \hypertarget{dms-req-0360-v-05}{DMS-REQ-0360-V-05}
    \\\vcdJiraRef{LVV-9770}~{\tiny
    }
    \end{tabular} &
        \begin{tabular}{@{}l@{}}
        \href{https://jira.lsstcorp.org/secure/Tests.jspa\#/testCase/LVV-T378}{LVV-T378} \\
        \vcdDocRef{LDM-639}
        \end{tabular} &
          & \notexec{} \\
          \cmidrule{3-5}
          & &
        \begin{tabular}{@{}l@{}}
        \href{https://jira.lsstcorp.org/secure/Tests.jspa\#/testCase/LVV-T1749}{LVV-T1749} \\
        \vcdDocRef{LDM-639}
        \end{tabular} &
          \begin{tabular}{@{}l@{}}
          2020-02-06 \\
            \vcdDocRef{DMTR-201}
            {\scriptsize \href{https://jira.lsstcorp.org/secure/Tests.jspa\#/testPlan/LVV-P65}{LVV-P65} }
          \end{tabular} &
          \passed \\
      \cmidrule{2-5}
      &
    \begin{tabular}{@{}l@{}}
    \hypertarget{dms-req-0360-v-06}{DMS-REQ-0360-V-06}
    \\\vcdJiraRef{LVV-9771}~{\tiny
    }
    \end{tabular} &
        \begin{tabular}{@{}l@{}}
        \href{https://jira.lsstcorp.org/secure/Tests.jspa\#/testCase/LVV-T378}{LVV-T378} \\
        \vcdDocRef{LDM-639}
        \end{tabular} &
          & \notexec{} \\
          \cmidrule{3-5}
          & &
        \begin{tabular}{@{}l@{}}
        \href{https://jira.lsstcorp.org/secure/Tests.jspa\#/testCase/LVV-T1750}{LVV-T1750} \\
        \vcdDocRef{LDM-639}
        \end{tabular} &
          \begin{tabular}{@{}l@{}}
          2020-02-06 \\
            \vcdDocRef{DMTR-201}
            {\scriptsize \href{https://jira.lsstcorp.org/secure/Tests.jspa\#/testPlan/LVV-P65}{LVV-P65} }
          \end{tabular} &
          \passed \\
      \cmidrule{2-5}
      &
    \begin{tabular}{@{}l@{}}
    \hypertarget{dms-req-0360-v-07}{DMS-REQ-0360-V-07}
    \\\vcdJiraRef{LVV-9773}~{\tiny
    }
    \end{tabular} &
        \begin{tabular}{@{}l@{}}
        \href{https://jira.lsstcorp.org/secure/Tests.jspa\#/testCase/LVV-T378}{LVV-T378} \\
        \vcdDocRef{LDM-639}
        \end{tabular} &
          & \notexec{} \\
          \cmidrule{3-5}
          & &
        \begin{tabular}{@{}l@{}}
        \href{https://jira.lsstcorp.org/secure/Tests.jspa\#/testCase/LVV-T1746}{LVV-T1746} \\
        \vcdDocRef{LDM-639}
        \end{tabular} &
          \begin{tabular}{@{}l@{}}
          2020-02-06 \\
            \vcdDocRef{DMTR-201}
            {\scriptsize \href{https://jira.lsstcorp.org/secure/Tests.jspa\#/testPlan/LVV-P65}{LVV-P65} }
          \end{tabular} &
          \passed \\
      \cmidrule{2-5}
      &
    \begin{tabular}{@{}l@{}}
    \hypertarget{dms-req-0360-v-08}{DMS-REQ-0360-V-08}
    \\\vcdJiraRef{LVV-9774}~{\tiny
    }
    \end{tabular} &
        \begin{tabular}{@{}l@{}}
        \href{https://jira.lsstcorp.org/secure/Tests.jspa\#/testCase/LVV-T378}{LVV-T378} \\
        \vcdDocRef{LDM-639}
        \end{tabular} &
          & \notexec{} \\
          \cmidrule{3-5}
          & &
        \begin{tabular}{@{}l@{}}
        \href{https://jira.lsstcorp.org/secure/Tests.jspa\#/testCase/LVV-T1751}{LVV-T1751} \\
        \vcdDocRef{LDM-639}
        \end{tabular} &
          \begin{tabular}{@{}l@{}}
          2020-02-06 \\
            \vcdDocRef{DMTR-201}
            {\scriptsize \href{https://jira.lsstcorp.org/secure/Tests.jspa\#/testPlan/LVV-P65}{LVV-P65} }
          \end{tabular} &
          \passed \\
      \cmidrule{2-5}
      &
    \begin{tabular}{@{}l@{}}
    \hypertarget{dms-req-0360-v-09}{DMS-REQ-0360-V-09}
    \\\vcdJiraRef{LVV-9775}~{\tiny
    }
    \end{tabular} &
        \begin{tabular}{@{}l@{}}
        \href{https://jira.lsstcorp.org/secure/Tests.jspa\#/testCase/LVV-T378}{LVV-T378} \\
        \vcdDocRef{LDM-639}
        \end{tabular} &
          & \notexec{} \\
      \cmidrule{2-5}
      &
    \begin{tabular}{@{}l@{}}
    \hypertarget{dms-req-0360-v-10}{DMS-REQ-0360-V-10}
    \\\vcdJiraRef{LVV-9776}~{\tiny
    }
    \end{tabular} &
        \begin{tabular}{@{}l@{}}
        \href{https://jira.lsstcorp.org/secure/Tests.jspa\#/testCase/LVV-T378}{LVV-T378} \\
        \vcdDocRef{LDM-639}
        \end{tabular} &
          & \notexec{} \\
          \cmidrule{3-5}
          & &
        \begin{tabular}{@{}l@{}}
        \href{https://jira.lsstcorp.org/secure/Tests.jspa\#/testCase/LVV-T1749}{LVV-T1749} \\
        \vcdDocRef{LDM-639}
        \end{tabular} &
          \begin{tabular}{@{}l@{}}
          2020-02-06 \\
            \vcdDocRef{DMTR-201}
            {\scriptsize \href{https://jira.lsstcorp.org/secure/Tests.jspa\#/testPlan/LVV-P65}{LVV-P65} }
          \end{tabular} &
          \passed \\
      \cmidrule{2-5}
      &
    \begin{tabular}{@{}l@{}}
    \hypertarget{dms-req-0360-v-11}{DMS-REQ-0360-V-11}
    \\\vcdJiraRef{LVV-9777}~{\tiny
    }
    \end{tabular} &
        \begin{tabular}{@{}l@{}}
        \href{https://jira.lsstcorp.org/secure/Tests.jspa\#/testCase/LVV-T378}{LVV-T378} \\
        \vcdDocRef{LDM-639}
        \end{tabular} &
          & \notexec{} \\
          \cmidrule{3-5}
          & &
        \begin{tabular}{@{}l@{}}
        \href{https://jira.lsstcorp.org/secure/Tests.jspa\#/testCase/LVV-T1750}{LVV-T1750} \\
        \vcdDocRef{LDM-639}
        \end{tabular} &
          \begin{tabular}{@{}l@{}}
          2020-02-06 \\
            \vcdDocRef{DMTR-201}
            {\scriptsize \href{https://jira.lsstcorp.org/secure/Tests.jspa\#/testPlan/LVV-P65}{LVV-P65} }
          \end{tabular} &
          \passed \\
      \cmidrule{2-5}
      &
    \begin{tabular}{@{}l@{}}
    \hypertarget{dms-req-0360-v-12}{DMS-REQ-0360-V-12}
    \\\vcdJiraRef{LVV-9778}~{\tiny
    }
    \end{tabular} &
        \begin{tabular}{@{}l@{}}
        \href{https://jira.lsstcorp.org/secure/Tests.jspa\#/testCase/LVV-T378}{LVV-T378} \\
        \vcdDocRef{LDM-639}
        \end{tabular} &
          & \notexec{} \\
          \cmidrule{3-5}
          & &
        \begin{tabular}{@{}l@{}}
        \href{https://jira.lsstcorp.org/secure/Tests.jspa\#/testCase/LVV-T1753}{LVV-T1753} \\
        \vcdDocRef{LDM-639}
        \end{tabular} &
          \begin{tabular}{@{}l@{}}
          2020-02-06 \\
            \vcdDocRef{DMTR-201}
            {\scriptsize \href{https://jira.lsstcorp.org/secure/Tests.jspa\#/testPlan/LVV-P65}{LVV-P65} }
          \end{tabular} &
          \passed \\
      \cmidrule{2-5}
      &
    \begin{tabular}{@{}l@{}}
    \hypertarget{dms-req-0360-v-13}{DMS-REQ-0360-V-13}
    \\\vcdJiraRef{LVV-9779}~{\tiny
    }
    \end{tabular} &
        \begin{tabular}{@{}l@{}}
        \href{https://jira.lsstcorp.org/secure/Tests.jspa\#/testCase/LVV-T378}{LVV-T378} \\
        \vcdDocRef{LDM-639}
        \end{tabular} &
          & \notexec{} \\
          \cmidrule{3-5}
          & &
        \begin{tabular}{@{}l@{}}
        \href{https://jira.lsstcorp.org/secure/Tests.jspa\#/testCase/LVV-T1752}{LVV-T1752} \\
        \vcdDocRef{LDM-639}
        \end{tabular} &
          \begin{tabular}{@{}l@{}}
          2020-02-06 \\
            \vcdDocRef{DMTR-201}
            {\scriptsize \href{https://jira.lsstcorp.org/secure/Tests.jspa\#/testPlan/LVV-P65}{LVV-P65} }
          \end{tabular} &
          \passed \\
  \midrule
  \begin{tabular}{@{}l@{}}
  DMS-REQ-0361\\\vcdDocRef{LSE-61}~{\tiny
 (p. 1b)   }
  \end{tabular} &
    \begin{tabular}{@{}l@{}}
    \hypertarget{dms-req-0361-v-01}{DMS-REQ-0361-V-01}
    \\\vcdJiraRef{LVV-3403}~{\tiny
    }
    \end{tabular} &
        \begin{tabular}{@{}l@{}}
        \href{https://jira.lsstcorp.org/secure/Tests.jspa\#/testCase/LVV-T1088}{LVV-T1088} \\
        \vcdDocRef{LDM-552}
        \end{tabular} &
          \begin{tabular}{@{}l@{}}
          2019-07-09 \\
            \vcdDocRef{DMTR-71}
            {\scriptsize \href{https://jira.lsstcorp.org/secure/Tests.jspa\#/testPlan/LVV-P46}{LVV-P46} }
          \end{tabular} &
          \passed \\
          \cmidrule{3-5}
          & &
        \begin{tabular}{@{}l@{}}
        \href{https://jira.lsstcorp.org/secure/Tests.jspa\#/testCase/LVV-T1089}{LVV-T1089} \\
        \vcdDocRef{LDM-552}
        \end{tabular} &
          \begin{tabular}{@{}l@{}}
          2019-07-09 \\
            \vcdDocRef{DMTR-71}
            {\scriptsize \href{https://jira.lsstcorp.org/secure/Tests.jspa\#/testPlan/LVV-P46}{LVV-P46} }
          \end{tabular} &
          \cndpass \\
          \cmidrule{3-5}
          & &
        \begin{tabular}{@{}l@{}}
        \href{https://jira.lsstcorp.org/secure/Tests.jspa\#/testCase/LVV-T1090}{LVV-T1090} \\
        \vcdDocRef{LDM-552}
        \end{tabular} &
          \begin{tabular}{@{}l@{}}
          2019-07-09 \\
            \vcdDocRef{DMTR-71}
            {\scriptsize \href{https://jira.lsstcorp.org/secure/Tests.jspa\#/testPlan/LVV-P46}{LVV-P46} }
          \end{tabular} &
          \cndpass \\
  \midrule
  \begin{tabular}{@{}l@{}}
  DMS-REQ-0362\\\vcdDocRef{LSE-61}~{\tiny
 (p. 1b)   }
  \end{tabular} &
    \begin{tabular}{@{}l@{}}
    \hypertarget{dms-req-0362-v-01}{DMS-REQ-0362-V-01}
    \\\vcdJiraRef{LVV-3404}~{\tiny
    }
    \end{tabular} &
        \begin{tabular}{@{}l@{}}
        \href{https://jira.lsstcorp.org/secure/Tests.jspa\#/testCase/LVV-T376}{LVV-T376} \\
        \vcdDocRef{LDM-639}
        \end{tabular} &
          & \notexec{} \\
          \cmidrule{3-5}
          & &
        \begin{tabular}{@{}l@{}}
        \href{https://jira.lsstcorp.org/secure/Tests.jspa\#/testCase/LVV-T1754}{LVV-T1754} \\
        \vcdDocRef{LDM-639}
        \end{tabular} &
          \begin{tabular}{@{}l@{}}
          2020-02-06 \\
            \vcdDocRef{DMTR-201}
            {\scriptsize \href{https://jira.lsstcorp.org/secure/Tests.jspa\#/testPlan/LVV-P65}{LVV-P65} }
          \end{tabular} &
          \passed \\
      \cmidrule{2-5}
      &
    \begin{tabular}{@{}l@{}}
    \hypertarget{dms-req-0362-v-02}{DMS-REQ-0362-V-02}
    \\\vcdJiraRef{LVV-9780}~{\tiny
    }
    \end{tabular} &
        \begin{tabular}{@{}l@{}}
        \href{https://jira.lsstcorp.org/secure/Tests.jspa\#/testCase/LVV-T376}{LVV-T376} \\
        \vcdDocRef{LDM-639}
        \end{tabular} &
          & \notexec{} \\
      \cmidrule{2-5}
      &
    \begin{tabular}{@{}l@{}}
    \hypertarget{dms-req-0362-v-03}{DMS-REQ-0362-V-03}
    \\\vcdJiraRef{LVV-9781}~{\tiny
    }
    \end{tabular} &
        & & \\
      \cmidrule{2-5}
      &
    \begin{tabular}{@{}l@{}}
    \hypertarget{dms-req-0362-v-04}{DMS-REQ-0362-V-04}
    \\\vcdJiraRef{LVV-9782}~{\tiny
    }
    \end{tabular} &
        \begin{tabular}{@{}l@{}}
        \href{https://jira.lsstcorp.org/secure/Tests.jspa\#/testCase/LVV-T1755}{LVV-T1755} \\
        \vcdDocRef{LDM-639}
        \end{tabular} &
          \begin{tabular}{@{}l@{}}
          2020-02-07 \\
            \vcdDocRef{DMTR-201}
            {\scriptsize \href{https://jira.lsstcorp.org/secure/Tests.jspa\#/testPlan/LVV-P65}{LVV-P65} }
          \end{tabular} &
          \passed \\
      \cmidrule{2-5}
      &
    \begin{tabular}{@{}l@{}}
    \hypertarget{dms-req-0362-v-05}{DMS-REQ-0362-V-05}
    \\\vcdJiraRef{LVV-9783}~{\tiny
    }
    \end{tabular} &
        & & \\
  \midrule
  \begin{tabular}{@{}l@{}}
  DMS-REQ-0372\\\vcdDocRef{LSE-61}~{\tiny
 (p. 1a)   }
  \end{tabular} &
    \begin{tabular}{@{}l@{}}
    \hypertarget{dms-req-0372-v-01}{DMS-REQ-0372-V-01}
    \\\vcdJiraRef{LVV-9637}~{\tiny
    }
    \end{tabular} &
        \begin{tabular}{@{}l@{}}
        \href{https://jira.lsstcorp.org/secure/Tests.jspa\#/testCase/LVV-T1264}{LVV-T1264} \\
        \vcdDocRef{LDM-639}
        \end{tabular} &
          & \notexec{} \\
  \midrule
  \begin{tabular}{@{}l@{}}
  DMS-REQ-0344\\\vcdDocRef{LSE-61}~{\tiny
 (p. 2)   }
  \end{tabular} &
    \begin{tabular}{@{}l@{}}
    \hypertarget{dms-req-0344-v-02}{DMS-REQ-0344-V-02}
    \\\vcdJiraRef{LVV-9744}~{\tiny
    }
    \end{tabular} &
        \begin{tabular}{@{}l@{}}
        \href{https://jira.lsstcorp.org/secure/Tests.jspa\#/testCase/LVV-T1866}{LVV-T1866} \\
        \vcdDocRef{LDM-639}
        \end{tabular} &
          & \notexec{} \\
      \cmidrule{2-5}
      &
    \begin{tabular}{@{}l@{}}
    \hypertarget{dms-req-0344-v-01}{DMS-REQ-0344-V-01}
    \\\vcdJiraRef{LVV-18229}~{\tiny
    }
    \end{tabular} &
        \begin{tabular}{@{}l@{}}
        \href{https://jira.lsstcorp.org/secure/Tests.jspa\#/testCase/LVV-T1865}{LVV-T1865} \\
        \vcdDocRef{LDM-639}
        \end{tabular} &
          & \notexec{} \\
  \midrule
  \begin{tabular}{@{}l@{}}
  DMS-REQ-0384\\\vcdDocRef{LSE-61}~{\tiny
 (p. 1b)   }
  \end{tabular} &
    \begin{tabular}{@{}l@{}}
    \hypertarget{dms-req-0384-v-01}{DMS-REQ-0384-V-01}
    \\\vcdJiraRef{LVV-18222}~{\tiny
 (p. 1b)     }
    \end{tabular} &
        \begin{tabular}{@{}l@{}}
        \href{https://jira.lsstcorp.org/secure/Tests.jspa\#/testCase/LVV-T1524}{LVV-T1524} \\
        \vcdDocRef{LDM-639}
        \end{tabular} &
          & \notexec{} \\
  \midrule
  \begin{tabular}{@{}l@{}}
  DMS-REQ-0381\\\vcdDocRef{LSE-61}~{\tiny
 (p. 2)   }
  \end{tabular} &
    \begin{tabular}{@{}l@{}}
    \hypertarget{dms-req-0381-v-01}{DMS-REQ-0381-V-01}
    \\\vcdJiraRef{LVV-18223}~{\tiny
 (p. 2)     }
    \end{tabular} &
        \begin{tabular}{@{}l@{}}
        \href{https://jira.lsstcorp.org/secure/Tests.jspa\#/testCase/LVV-T1525}{LVV-T1525} \\
        \vcdDocRef{LDM-639}
        \end{tabular} &
          & \notexec{} \\
  \midrule
  \begin{tabular}{@{}l@{}}
  DMS-REQ-0380\\\vcdDocRef{LSE-61}~{\tiny
 (p. 1b)   }
  \end{tabular} &
    \begin{tabular}{@{}l@{}}
    \hypertarget{dms-req-0380-v-01}{DMS-REQ-0380-V-01}
    \\\vcdJiraRef{LVV-18224}~{\tiny
 (p. 1b)     }
    \end{tabular} &
        \begin{tabular}{@{}l@{}}
        \href{https://jira.lsstcorp.org/secure/Tests.jspa\#/testCase/LVV-T1526}{LVV-T1526} \\
        \vcdDocRef{LDM-639}
        \end{tabular} &
          & \notexec{} \\
  \midrule
  \begin{tabular}{@{}l@{}}
  DMS-REQ-0382\\\vcdDocRef{LSE-61}~{\tiny
 (p. 1b)   }
  \end{tabular} &
    \begin{tabular}{@{}l@{}}
    \hypertarget{dms-req-0382-v-01}{DMS-REQ-0382-V-01}
    \\\vcdJiraRef{LVV-18225}~{\tiny
 (p. 1b)     }
    \end{tabular} &
        \begin{tabular}{@{}l@{}}
        \href{https://jira.lsstcorp.org/secure/Tests.jspa\#/testCase/LVV-T1527}{LVV-T1527} \\
        \vcdDocRef{LDM-639}
        \end{tabular} &
          & \notexec{} \\
  \midrule
  \begin{tabular}{@{}l@{}}
  DMS-REQ-0385\\\vcdDocRef{LSE-61}~{\tiny
 (p. 1b)   }
  \end{tabular} &
    \begin{tabular}{@{}l@{}}
    \hypertarget{dms-req-0385-v-01}{DMS-REQ-0385-V-01}
    \\\vcdJiraRef{LVV-18226}~{\tiny
 (p. 1b)     }
    \end{tabular} &
        \begin{tabular}{@{}l@{}}
        \href{https://jira.lsstcorp.org/secure/Tests.jspa\#/testCase/LVV-T1528}{LVV-T1528} \\
        \vcdDocRef{LDM-639}
        \end{tabular} &
          & \notexec{} \\
  \midrule
  \begin{tabular}{@{}l@{}}
  DMS-REQ-0379\\\vcdDocRef{LSE-61}~{\tiny
 (p. 1b)   }
  \end{tabular} &
    \begin{tabular}{@{}l@{}}
    \hypertarget{dms-req-0379-v-01}{DMS-REQ-0379-V-01}
    \\\vcdJiraRef{LVV-18227}~{\tiny
 (p. 1b)     }
    \end{tabular} &
        \begin{tabular}{@{}l@{}}
        \href{https://jira.lsstcorp.org/secure/Tests.jspa\#/testCase/LVV-T1529}{LVV-T1529} \\
        \vcdDocRef{LDM-639}
        \end{tabular} &
          & \notexec{} \\
  \midrule
  \begin{tabular}{@{}l@{}}
  DMS-REQ-0383\\\vcdDocRef{LSE-61}~{\tiny
 (p. 1b)   }
  \end{tabular} &
    \begin{tabular}{@{}l@{}}
    \hypertarget{dms-req-0383-v-01}{DMS-REQ-0383-V-01}
    \\\vcdJiraRef{LVV-18228}~{\tiny
 (p. 1b)     }
    \end{tabular} &
        \begin{tabular}{@{}l@{}}
        \href{https://jira.lsstcorp.org/secure/Tests.jspa\#/testCase/LVV-T1530}{LVV-T1530} \\
        \vcdDocRef{LDM-639}
        \end{tabular} &
          & \notexec{} \\
  \midrule
  \begin{tabular}{@{}l@{}}
  DMS-REQ-0386\\\vcdDocRef{LSE-61}~{\tiny
 (p. 1b)   }
  \end{tabular} &
    \begin{tabular}{@{}l@{}}
    \hypertarget{dms-req-0386-v-01}{DMS-REQ-0386-V-01}
    \\\vcdJiraRef{LVV-18230}~{\tiny
    }
    \end{tabular} &
        \begin{tabular}{@{}l@{}}
        \href{https://jira.lsstcorp.org/secure/Tests.jspa\#/testCase/LVV-T1560}{LVV-T1560} \\
        \vcdDocRef{LDM-639}
        \end{tabular} &
          & \notexec{} \\
  \midrule
  \begin{tabular}{@{}l@{}}
  DMS-REQ-0387\\\vcdDocRef{LSE-61}~{\tiny
 (p. 1b)   }
  \end{tabular} &
    \begin{tabular}{@{}l@{}}
    \hypertarget{dms-req-0387-v-01}{DMS-REQ-0387-V-01}
    \\\vcdJiraRef{LVV-18231}~{\tiny
    }
    \end{tabular} &
        \begin{tabular}{@{}l@{}}
        \href{https://jira.lsstcorp.org/secure/Tests.jspa\#/testCase/LVV-T1561}{LVV-T1561} \\
        \vcdDocRef{LDM-639}
        \end{tabular} &
          & \notexec{} \\
  \midrule
  \begin{tabular}{@{}l@{}}
  DMS-REQ-0388\\\vcdDocRef{LSE-61}~{\tiny
 (p. 1b)   }
  \end{tabular} &
    \begin{tabular}{@{}l@{}}
    \hypertarget{dms-req-0388-v-01}{DMS-REQ-0388-V-01}
    \\\vcdJiraRef{LVV-18232}~{\tiny
    }
    \end{tabular} &
        \begin{tabular}{@{}l@{}}
        \href{https://jira.lsstcorp.org/secure/Tests.jspa\#/testCase/LVV-T1562}{LVV-T1562} \\
        \vcdDocRef{LDM-639}
        \end{tabular} &
          & \notexec{} \\
  \midrule
  \begin{tabular}{@{}l@{}}
  DMS-REQ-0390\\\vcdDocRef{LSE-61}~{\tiny
 (p. 1b)   }
  \end{tabular} &
    \begin{tabular}{@{}l@{}}
    \hypertarget{dms-req-0390-v-01}{DMS-REQ-0390-V-01}
    \\\vcdJiraRef{LVV-18233}~{\tiny
    }
    \end{tabular} &
        \begin{tabular}{@{}l@{}}
        \href{https://jira.lsstcorp.org/secure/Tests.jspa\#/testCase/LVV-T1563}{LVV-T1563} \\
        \vcdDocRef{LDM-639}
        \end{tabular} &
          & \notexec{} \\
  \midrule
  \begin{tabular}{@{}l@{}}
  DMS-REQ-0389\\\vcdDocRef{LSE-61}~{\tiny
 (p. 1b)   }
  \end{tabular} &
    \begin{tabular}{@{}l@{}}
    \hypertarget{dms-req-0389-v-01}{DMS-REQ-0389-V-01}
    \\\vcdJiraRef{LVV-18234}~{\tiny
    }
    \end{tabular} &
        \begin{tabular}{@{}l@{}}
        \href{https://jira.lsstcorp.org/secure/Tests.jspa\#/testCase/LVV-T1564}{LVV-T1564} \\
        \vcdDocRef{LDM-639}
        \end{tabular} &
          & \notexec{} \\
  \midrule
  \begin{tabular}{@{}l@{}}
  DMS-REQ-0394\\\vcdDocRef{LSE-61}~{\tiny
 (p. 1a)   }
  \end{tabular} &
    \begin{tabular}{@{}l@{}}
    \hypertarget{dms-req-0394-v-01}{DMS-REQ-0394-V-01}
    \\\vcdJiraRef{LVV-18295}~{\tiny
    }
    \end{tabular} &
        \begin{tabular}{@{}l@{}}
        \href{https://jira.lsstcorp.org/secure/Tests.jspa\#/testCase/LVV-T1831}{LVV-T1831} \\
        \vcdDocRef{LDM-639}
        \end{tabular} &
          & \notexec{} \\
  \midrule
  \begin{tabular}{@{}l@{}}
  DMS-REQ-0391\\\vcdDocRef{LSE-61}~{\tiny
 (p. 2)   }
  \end{tabular} &
    \begin{tabular}{@{}l@{}}
    \hypertarget{dms-req-0391-v-01}{DMS-REQ-0391-V-01}
    \\\vcdJiraRef{LVV-18297}~{\tiny
    }
    \end{tabular} &
        \begin{tabular}{@{}l@{}}
        \href{https://jira.lsstcorp.org/secure/Tests.jspa\#/testCase/LVV-T1867}{LVV-T1867} \\
        \vcdDocRef{LDM-639}
        \end{tabular} &
          & \notexec{} \\
      \cmidrule{2-5}
      &
    \begin{tabular}{@{}l@{}}
    \hypertarget{dms-req-0391-v-02}{DMS-REQ-0391-V-02}
    \\\vcdJiraRef{LVV-18911}~{\tiny
    }
    \end{tabular} &
        \begin{tabular}{@{}l@{}}
        \href{https://jira.lsstcorp.org/secure/Tests.jspa\#/testCase/LVV-T1868}{LVV-T1868} \\
        \vcdDocRef{LDM-639}
        \end{tabular} &
          & \notexec{} \\
  \midrule
  \begin{tabular}{@{}l@{}}
  DMS-REQ-0392\\\vcdDocRef{LSE-61}~{\tiny
 (p. 2)   }
  \end{tabular} &
    \begin{tabular}{@{}l@{}}
    \hypertarget{dms-req-0392-v-01}{DMS-REQ-0392-V-01}
    \\\vcdJiraRef{LVV-18298}~{\tiny
    }
    \end{tabular} &
        & & \\
  \midrule
  \begin{tabular}{@{}l@{}}
  DMS-REQ-0393\\\vcdDocRef{LSE-61}~{\tiny
 (p. 2)   }
  \end{tabular} &
    \begin{tabular}{@{}l@{}}
    \hypertarget{dms-req-0393-v-01}{DMS-REQ-0393-V-01}
    \\\vcdJiraRef{LVV-18299}~{\tiny
    }
    \end{tabular} &
        & & \\
  \midrule
  \begin{tabular}{@{}l@{}}
  DMS-REQ-0395\\\vcdDocRef{LSE-61}~{\tiny
 (p. 1a)   }
  \end{tabular} &
    \begin{tabular}{@{}l@{}}
    \hypertarget{dms-req-0395-v-01}{DMS-REQ-0395-V-01}
    \\\vcdJiraRef{LVV-18465}~{\tiny
    }
    \end{tabular} &
        \begin{tabular}{@{}l@{}}
        \href{https://jira.lsstcorp.org/secure/Tests.jspa\#/testCase/LVV-T1830}{LVV-T1830} \\
        \vcdDocRef{LDM-639}
        \end{tabular} &
          & \notexec{} \\
  \midrule
  \begin{tabular}{@{}l@{}}
  DMS-REQ-0396\\\vcdDocRef{LSE-61}~{\tiny
 (p. 2)   }
  \end{tabular} &
    \begin{tabular}{@{}l@{}}
    \hypertarget{dms-req-0396-v-01}{DMS-REQ-0396-V-01}
    \\\vcdJiraRef{LVV-18841}~{\tiny
    }
    \end{tabular} &
        & & \\
  \midrule
  \begin{tabular}{@{}l@{}}
  DMS-REQ-0397\\\vcdDocRef{LSE-61}~{\tiny
 (p. 2)   }
  \end{tabular} &
    \begin{tabular}{@{}l@{}}
    \hypertarget{dms-req-0397-v-01}{DMS-REQ-0397-V-01}
    \\\vcdJiraRef{LVV-18847}~{\tiny
    }
    \end{tabular} &
        \begin{tabular}{@{}l@{}}
        \href{https://jira.lsstcorp.org/secure/Tests.jspa\#/testCase/LVV-T1863}{LVV-T1863} \\
        \vcdDocRef{LDM-639}
        \end{tabular} &
          & \notexec{} \\
  \midrule
\label{tab:dmvcd}
\end{longtable}
}

\subsection{LSE-68 Requirements Coverage}

\setlength\LTleft{-0.25in}
\setlength\LTright{-0.5in}
{\small
\begin{longtable}{lllll}
\caption{ DM LSE-68 Requirements.} \\
\toprule
\textbf{Requirement} & \textbf{Verification Element} & \textbf{Test Case} & \textbf{Last Run} & \textbf{Test Status} \\
\toprule
\endhead
  \begin{tabular}{@{}l@{}}
  CA-DM-DAQ-ICD-0094\\\vcdDocRef{LSE-68}~{\tiny
  }
  \end{tabular} &
    \begin{tabular}{@{}l@{}}
    \hypertarget{ca-dm-daq-icd-0094-v-03}{CA-DM-DAQ-ICD-0094-V-03}
    \\\vcdJiraRef{LVV-4669}~{\tiny
    }
    \end{tabular} &
        & & \\
      \cmidrule{2-5}
      &
    \begin{tabular}{@{}l@{}}
    \hypertarget{ca-dm-daq-icd-0094-v-04}{CA-DM-DAQ-ICD-0094-V-04}
    \\\vcdJiraRef{LVV-4670}~{\tiny
    }
    \end{tabular} &
        & & \\
  \midrule
  \begin{tabular}{@{}l@{}}
  CA-DM-DAQ-ICD-0082\\\vcdDocRef{LSE-68}~{\tiny
  }
  \end{tabular} &
    \begin{tabular}{@{}l@{}}
    \hypertarget{ca-dm-daq-icd-0082-v-03}{CA-DM-DAQ-ICD-0082-V-03}
    \\\vcdJiraRef{LVV-4675}~{\tiny
    }
    \end{tabular} &
        & & \\
      \cmidrule{2-5}
      &
    \begin{tabular}{@{}l@{}}
    \hypertarget{ca-dm-daq-icd-0082-v-04}{CA-DM-DAQ-ICD-0082-V-04}
    \\\vcdJiraRef{LVV-4676}~{\tiny
    }
    \end{tabular} &
        & & \\
  \midrule
  \begin{tabular}{@{}l@{}}
  CA-DM-DAQ-ICD-0093\\\vcdDocRef{LSE-68}~{\tiny
  }
  \end{tabular} &
    \begin{tabular}{@{}l@{}}
    \hypertarget{ca-dm-daq-icd-0093-v-03}{CA-DM-DAQ-ICD-0093-V-03}
    \\\vcdJiraRef{LVV-4729}~{\tiny
    }
    \end{tabular} &
        & & \\
      \cmidrule{2-5}
      &
    \begin{tabular}{@{}l@{}}
    \hypertarget{ca-dm-daq-icd-0093-v-04}{CA-DM-DAQ-ICD-0093-V-04}
    \\\vcdJiraRef{LVV-4730}~{\tiny
    }
    \end{tabular} &
        & & \\
  \midrule
  \begin{tabular}{@{}l@{}}
  CA-DM-DAQ-ICD-0097\\\vcdDocRef{LSE-68}~{\tiny
  }
  \end{tabular} &
    \begin{tabular}{@{}l@{}}
    \hypertarget{ca-dm-daq-icd-0097-v-03}{CA-DM-DAQ-ICD-0097-V-03}
    \\\vcdJiraRef{LVV-4735}~{\tiny
    }
    \end{tabular} &
        & & \\
      \cmidrule{2-5}
      &
    \begin{tabular}{@{}l@{}}
    \hypertarget{ca-dm-daq-icd-0097-v-04}{CA-DM-DAQ-ICD-0097-V-04}
    \\\vcdJiraRef{LVV-4736}~{\tiny
    }
    \end{tabular} &
        & & \\
  \midrule
  \begin{tabular}{@{}l@{}}
  CA-DM-DAQ-ICD-0058\\\vcdDocRef{LSE-68}~{\tiny
  }
  \end{tabular} &
    \begin{tabular}{@{}l@{}}
    \hypertarget{ca-dm-daq-icd-0058-v-03}{CA-DM-DAQ-ICD-0058-V-03}
    \\\vcdJiraRef{LVV-4741}~{\tiny
    }
    \end{tabular} &
        & & \\
      \cmidrule{2-5}
      &
    \begin{tabular}{@{}l@{}}
    \hypertarget{ca-dm-daq-icd-0058-v-04}{CA-DM-DAQ-ICD-0058-V-04}
    \\\vcdJiraRef{LVV-4742}~{\tiny
    }
    \end{tabular} &
        & & \\
  \midrule
  \begin{tabular}{@{}l@{}}
  CA-DM-DAQ-ICD-0059\\\vcdDocRef{LSE-68}~{\tiny
  }
  \end{tabular} &
    \begin{tabular}{@{}l@{}}
    \hypertarget{ca-dm-daq-icd-0059-v-03}{CA-DM-DAQ-ICD-0059-V-03}
    \\\vcdJiraRef{LVV-4747}~{\tiny
    }
    \end{tabular} &
        & & \\
      \cmidrule{2-5}
      &
    \begin{tabular}{@{}l@{}}
    \hypertarget{ca-dm-daq-icd-0059-v-04}{CA-DM-DAQ-ICD-0059-V-04}
    \\\vcdJiraRef{LVV-4748}~{\tiny
    }
    \end{tabular} &
        & & \\
  \midrule
  \begin{tabular}{@{}l@{}}
  CA-DM-DAQ-ICD-0060\\\vcdDocRef{LSE-68}~{\tiny
  }
  \end{tabular} &
    \begin{tabular}{@{}l@{}}
    \hypertarget{ca-dm-daq-icd-0060-v-03}{CA-DM-DAQ-ICD-0060-V-03}
    \\\vcdJiraRef{LVV-4753}~{\tiny
    }
    \end{tabular} &
        & & \\
      \cmidrule{2-5}
      &
    \begin{tabular}{@{}l@{}}
    \hypertarget{ca-dm-daq-icd-0060-v-04}{CA-DM-DAQ-ICD-0060-V-04}
    \\\vcdJiraRef{LVV-4754}~{\tiny
    }
    \end{tabular} &
        & & \\
  \midrule
  \begin{tabular}{@{}l@{}}
  CA-DM-DAQ-ICD-0081\\\vcdDocRef{LSE-68}~{\tiny
  }
  \end{tabular} &
    \begin{tabular}{@{}l@{}}
    \hypertarget{ca-dm-daq-icd-0081-v-03}{CA-DM-DAQ-ICD-0081-V-03}
    \\\vcdJiraRef{LVV-4759}~{\tiny
    }
    \end{tabular} &
        & & \\
      \cmidrule{2-5}
      &
    \begin{tabular}{@{}l@{}}
    \hypertarget{ca-dm-daq-icd-0081-v-04}{CA-DM-DAQ-ICD-0081-V-04}
    \\\vcdJiraRef{LVV-4760}~{\tiny
    }
    \end{tabular} &
        & & \\
  \midrule
  \begin{tabular}{@{}l@{}}
  CA-DM-DAQ-ICD-0047\\\vcdDocRef{LSE-68}~{\tiny
  }
  \end{tabular} &
    \begin{tabular}{@{}l@{}}
    \hypertarget{ca-dm-daq-icd-0047-v-03}{CA-DM-DAQ-ICD-0047-V-03}
    \\\vcdJiraRef{LVV-4765}~{\tiny
    }
    \end{tabular} &
        & & \\
      \cmidrule{2-5}
      &
    \begin{tabular}{@{}l@{}}
    \hypertarget{ca-dm-daq-icd-0047-v-04}{CA-DM-DAQ-ICD-0047-V-04}
    \\\vcdJiraRef{LVV-4766}~{\tiny
    }
    \end{tabular} &
        & & \\
  \midrule
  \begin{tabular}{@{}l@{}}
  CA-DM-DAQ-ICD-0098\\\vcdDocRef{LSE-68}~{\tiny
  }
  \end{tabular} &
    \begin{tabular}{@{}l@{}}
    \hypertarget{ca-dm-daq-icd-0098-v-03}{CA-DM-DAQ-ICD-0098-V-03}
    \\\vcdJiraRef{LVV-4771}~{\tiny
    }
    \end{tabular} &
        & & \\
      \cmidrule{2-5}
      &
    \begin{tabular}{@{}l@{}}
    \hypertarget{ca-dm-daq-icd-0098-v-04}{CA-DM-DAQ-ICD-0098-V-04}
    \\\vcdJiraRef{LVV-4772}~{\tiny
    }
    \end{tabular} &
        & & \\
  \midrule
  \begin{tabular}{@{}l@{}}
  CA-DM-DAQ-ICD-0100\\\vcdDocRef{LSE-68}~{\tiny
  }
  \end{tabular} &
    \begin{tabular}{@{}l@{}}
    \hypertarget{ca-dm-daq-icd-0100-v-03}{CA-DM-DAQ-ICD-0100-V-03}
    \\\vcdJiraRef{LVV-4777}~{\tiny
    }
    \end{tabular} &
        & & \\
      \cmidrule{2-5}
      &
    \begin{tabular}{@{}l@{}}
    \hypertarget{ca-dm-daq-icd-0100-v-04}{CA-DM-DAQ-ICD-0100-V-04}
    \\\vcdJiraRef{LVV-4778}~{\tiny
    }
    \end{tabular} &
        & & \\
  \midrule
  \begin{tabular}{@{}l@{}}
  CA-DM-DAQ-ICD-0092\\\vcdDocRef{LSE-68}~{\tiny
  }
  \end{tabular} &
    \begin{tabular}{@{}l@{}}
    \hypertarget{ca-dm-daq-icd-0092-v-03}{CA-DM-DAQ-ICD-0092-V-03}
    \\\vcdJiraRef{LVV-4783}~{\tiny
    }
    \end{tabular} &
        & & \\
      \cmidrule{2-5}
      &
    \begin{tabular}{@{}l@{}}
    \hypertarget{ca-dm-daq-icd-0092-v-04}{CA-DM-DAQ-ICD-0092-V-04}
    \\\vcdJiraRef{LVV-4784}~{\tiny
    }
    \end{tabular} &
        & & \\
  \midrule
  \begin{tabular}{@{}l@{}}
  CA-DM-DAQ-ICD-0084\\\vcdDocRef{LSE-68}~{\tiny
  }
  \end{tabular} &
    \begin{tabular}{@{}l@{}}
    \hypertarget{ca-dm-daq-icd-0084-v-03}{CA-DM-DAQ-ICD-0084-V-03}
    \\\vcdJiraRef{LVV-4789}~{\tiny
    }
    \end{tabular} &
        & & \\
      \cmidrule{2-5}
      &
    \begin{tabular}{@{}l@{}}
    \hypertarget{ca-dm-daq-icd-0084-v-04}{CA-DM-DAQ-ICD-0084-V-04}
    \\\vcdJiraRef{LVV-4790}~{\tiny
    }
    \end{tabular} &
        & & \\
  \midrule
  \begin{tabular}{@{}l@{}}
  CA-DM-DAQ-ICD-0099\\\vcdDocRef{LSE-68}~{\tiny
  }
  \end{tabular} &
    \begin{tabular}{@{}l@{}}
    \hypertarget{ca-dm-daq-icd-0099-v-03}{CA-DM-DAQ-ICD-0099-V-03}
    \\\vcdJiraRef{LVV-4795}~{\tiny
    }
    \end{tabular} &
        & & \\
      \cmidrule{2-5}
      &
    \begin{tabular}{@{}l@{}}
    \hypertarget{ca-dm-daq-icd-0099-v-04}{CA-DM-DAQ-ICD-0099-V-04}
    \\\vcdJiraRef{LVV-4796}~{\tiny
    }
    \end{tabular} &
        & & \\
  \midrule
  \begin{tabular}{@{}l@{}}
  CA-DM-DAQ-ICD-0085\\\vcdDocRef{LSE-68}~{\tiny
  }
  \end{tabular} &
    \begin{tabular}{@{}l@{}}
    \hypertarget{ca-dm-daq-icd-0085-v-03}{CA-DM-DAQ-ICD-0085-V-03}
    \\\vcdJiraRef{LVV-4801}~{\tiny
    }
    \end{tabular} &
        & & \\
      \cmidrule{2-5}
      &
    \begin{tabular}{@{}l@{}}
    \hypertarget{ca-dm-daq-icd-0085-v-04}{CA-DM-DAQ-ICD-0085-V-04}
    \\\vcdJiraRef{LVV-4802}~{\tiny
    }
    \end{tabular} &
        & & \\
  \midrule
  \begin{tabular}{@{}l@{}}
  CA-DM-DAQ-ICD-0086\\\vcdDocRef{LSE-68}~{\tiny
  }
  \end{tabular} &
    \begin{tabular}{@{}l@{}}
    \hypertarget{ca-dm-daq-icd-0086-v-03}{CA-DM-DAQ-ICD-0086-V-03}
    \\\vcdJiraRef{LVV-4807}~{\tiny
    }
    \end{tabular} &
        & & \\
      \cmidrule{2-5}
      &
    \begin{tabular}{@{}l@{}}
    \hypertarget{ca-dm-daq-icd-0086-v-04}{CA-DM-DAQ-ICD-0086-V-04}
    \\\vcdJiraRef{LVV-4808}~{\tiny
    }
    \end{tabular} &
        & & \\
  \midrule
  \begin{tabular}{@{}l@{}}
  CA-DM-DAQ-ICD-0091\\\vcdDocRef{LSE-68}~{\tiny
  }
  \end{tabular} &
    \begin{tabular}{@{}l@{}}
    \hypertarget{ca-dm-daq-icd-0091-v-03}{CA-DM-DAQ-ICD-0091-V-03}
    \\\vcdJiraRef{LVV-4819}~{\tiny
    }
    \end{tabular} &
        & & \\
      \cmidrule{2-5}
      &
    \begin{tabular}{@{}l@{}}
    \hypertarget{ca-dm-daq-icd-0091-v-04}{CA-DM-DAQ-ICD-0091-V-04}
    \\\vcdJiraRef{LVV-4820}~{\tiny
    }
    \end{tabular} &
        & & \\
  \midrule
  \begin{tabular}{@{}l@{}}
  CA-DM-DAQ-ICD-0075\\\vcdDocRef{LSE-68}~{\tiny
  }
  \end{tabular} &
    \begin{tabular}{@{}l@{}}
    \hypertarget{ca-dm-daq-icd-0075-v-03}{CA-DM-DAQ-ICD-0075-V-03}
    \\\vcdJiraRef{LVV-4825}~{\tiny
    }
    \end{tabular} &
        & & \\
      \cmidrule{2-5}
      &
    \begin{tabular}{@{}l@{}}
    \hypertarget{ca-dm-daq-icd-0075-v-04}{CA-DM-DAQ-ICD-0075-V-04}
    \\\vcdJiraRef{LVV-4826}~{\tiny
    }
    \end{tabular} &
        & & \\
  \midrule
  \begin{tabular}{@{}l@{}}
  CA-DM-DAQ-ICD-0080\\\vcdDocRef{LSE-68}~{\tiny
  }
  \end{tabular} &
    \begin{tabular}{@{}l@{}}
    \hypertarget{ca-dm-daq-icd-0080-v-03}{CA-DM-DAQ-ICD-0080-V-03}
    \\\vcdJiraRef{LVV-4831}~{\tiny
    }
    \end{tabular} &
        & & \\
      \cmidrule{2-5}
      &
    \begin{tabular}{@{}l@{}}
    \hypertarget{ca-dm-daq-icd-0080-v-04}{CA-DM-DAQ-ICD-0080-V-04}
    \\\vcdJiraRef{LVV-4832}~{\tiny
    }
    \end{tabular} &
        & & \\
  \midrule
\label{tab:dmvcd}
\end{longtable}
}

\subsection{LSE-69 Requirements Coverage}

\setlength\LTleft{-0.25in}
\setlength\LTright{-0.5in}
{\small
\begin{longtable}{lllll}
\caption{ DM LSE-69 Requirements.} \\
\toprule
\textbf{Requirement} & \textbf{Verification Element} & \textbf{Test Case} & \textbf{Last Run} & \textbf{Test Status} \\
\toprule
\endhead
  \begin{tabular}{@{}l@{}}
  CA-DM-CON-ICD-0003\\\vcdDocRef{LSE-69}~{\tiny
  }
  \end{tabular} &
    \begin{tabular}{@{}l@{}}
    \hypertarget{ca-dm-con-icd-0003-v-03}{CA-DM-CON-ICD-0003-V-03}
    \\\vcdJiraRef{LVV-4843}~{\tiny
    }
    \end{tabular} &
        & & \\
      \cmidrule{2-5}
      &
    \begin{tabular}{@{}l@{}}
    \hypertarget{ca-dm-con-icd-0003-v-04}{CA-DM-CON-ICD-0003-V-04}
    \\\vcdJiraRef{LVV-4844}~{\tiny
    }
    \end{tabular} &
        & & \\
  \midrule
  \begin{tabular}{@{}l@{}}
  CA-DM-CON-ICD-0004\\\vcdDocRef{LSE-69}~{\tiny
  }
  \end{tabular} &
    \begin{tabular}{@{}l@{}}
    \hypertarget{ca-dm-con-icd-0004-v-03}{CA-DM-CON-ICD-0004-V-03}
    \\\vcdJiraRef{LVV-4849}~{\tiny
    }
    \end{tabular} &
        & & \\
      \cmidrule{2-5}
      &
    \begin{tabular}{@{}l@{}}
    \hypertarget{ca-dm-con-icd-0004-v-04}{CA-DM-CON-ICD-0004-V-04}
    \\\vcdJiraRef{LVV-4850}~{\tiny
    }
    \end{tabular} &
        & & \\
  \midrule
  \begin{tabular}{@{}l@{}}
  CA-DM-CON-ICD-0019\\\vcdDocRef{LSE-69}~{\tiny
  }
  \end{tabular} &
    \begin{tabular}{@{}l@{}}
    \hypertarget{ca-dm-con-icd-0019-v-03}{CA-DM-CON-ICD-0019-V-03}
    \\\vcdJiraRef{LVV-4855}~{\tiny
    }
    \end{tabular} &
        & & \\
      \cmidrule{2-5}
      &
    \begin{tabular}{@{}l@{}}
    \hypertarget{ca-dm-con-icd-0019-v-04}{CA-DM-CON-ICD-0019-V-04}
    \\\vcdJiraRef{LVV-4856}~{\tiny
    }
    \end{tabular} &
        & & \\
  \midrule
  \begin{tabular}{@{}l@{}}
  CA-DM-CON-ICD-0008\\\vcdDocRef{LSE-69}~{\tiny
  }
  \end{tabular} &
    \begin{tabular}{@{}l@{}}
    \hypertarget{ca-dm-con-icd-0008-v-03}{CA-DM-CON-ICD-0008-V-03}
    \\\vcdJiraRef{LVV-4861}~{\tiny
    }
    \end{tabular} &
        & & \\
      \cmidrule{2-5}
      &
    \begin{tabular}{@{}l@{}}
    \hypertarget{ca-dm-con-icd-0008-v-04}{CA-DM-CON-ICD-0008-V-04}
    \\\vcdJiraRef{LVV-4862}~{\tiny
    }
    \end{tabular} &
        & & \\
  \midrule
  \begin{tabular}{@{}l@{}}
  CA-DM-CON-ICD-0002\\\vcdDocRef{LSE-69}~{\tiny
  }
  \end{tabular} &
    \begin{tabular}{@{}l@{}}
    \hypertarget{ca-dm-con-icd-0002-v-03}{CA-DM-CON-ICD-0002-V-03}
    \\\vcdJiraRef{LVV-4873}~{\tiny
    }
    \end{tabular} &
        & & \\
      \cmidrule{2-5}
      &
    \begin{tabular}{@{}l@{}}
    \hypertarget{ca-dm-con-icd-0002-v-04}{CA-DM-CON-ICD-0002-V-04}
    \\\vcdJiraRef{LVV-4874}~{\tiny
    }
    \end{tabular} &
        & & \\
  \midrule
  \begin{tabular}{@{}l@{}}
  CA-DM-CON-ICD-0005\\\vcdDocRef{LSE-69}~{\tiny
  }
  \end{tabular} &
    \begin{tabular}{@{}l@{}}
    \hypertarget{ca-dm-con-icd-0005-v-03}{CA-DM-CON-ICD-0005-V-03}
    \\\vcdJiraRef{LVV-4879}~{\tiny
    }
    \end{tabular} &
        & & \\
      \cmidrule{2-5}
      &
    \begin{tabular}{@{}l@{}}
    \hypertarget{ca-dm-con-icd-0005-v-04}{CA-DM-CON-ICD-0005-V-04}
    \\\vcdJiraRef{LVV-4880}~{\tiny
    }
    \end{tabular} &
        & & \\
  \midrule
  \begin{tabular}{@{}l@{}}
  CA-DM-CON-ICD-0001\\\vcdDocRef{LSE-69}~{\tiny
  }
  \end{tabular} &
    \begin{tabular}{@{}l@{}}
    \hypertarget{ca-dm-con-icd-0001-v-03}{CA-DM-CON-ICD-0001-V-03}
    \\\vcdJiraRef{LVV-4885}~{\tiny
    }
    \end{tabular} &
        & & \\
      \cmidrule{2-5}
      &
    \begin{tabular}{@{}l@{}}
    \hypertarget{ca-dm-con-icd-0001-v-04}{CA-DM-CON-ICD-0001-V-04}
    \\\vcdJiraRef{LVV-4886}~{\tiny
    }
    \end{tabular} &
        & & \\
  \midrule
  \begin{tabular}{@{}l@{}}
  CA-DM-CON-ICD-0018\\\vcdDocRef{LSE-69}~{\tiny
  }
  \end{tabular} &
    \begin{tabular}{@{}l@{}}
    \hypertarget{ca-dm-con-icd-0018-v-03}{CA-DM-CON-ICD-0018-V-03}
    \\\vcdJiraRef{LVV-4897}~{\tiny
    }
    \end{tabular} &
        & & \\
      \cmidrule{2-5}
      &
    \begin{tabular}{@{}l@{}}
    \hypertarget{ca-dm-con-icd-0018-v-04}{CA-DM-CON-ICD-0018-V-04}
    \\\vcdJiraRef{LVV-4898}~{\tiny
    }
    \end{tabular} &
        & & \\
  \midrule
  \begin{tabular}{@{}l@{}}
  CA-DM-CON-ICD-0007\\\vcdDocRef{LSE-69}~{\tiny
  }
  \end{tabular} &
    \begin{tabular}{@{}l@{}}
    \hypertarget{ca-dm-con-icd-0007-v-03}{CA-DM-CON-ICD-0007-V-03}
    \\\vcdJiraRef{LVV-4903}~{\tiny
    }
    \end{tabular} &
        & & \\
      \cmidrule{2-5}
      &
    \begin{tabular}{@{}l@{}}
    \hypertarget{ca-dm-con-icd-0007-v-04}{CA-DM-CON-ICD-0007-V-04}
    \\\vcdJiraRef{LVV-4904}~{\tiny
    }
    \end{tabular} &
        & & \\
  \midrule
  \begin{tabular}{@{}l@{}}
  CA-DM-CON-ICD-0016\\\vcdDocRef{LSE-69}~{\tiny
  }
  \end{tabular} &
    \begin{tabular}{@{}l@{}}
    \hypertarget{ca-dm-con-icd-0016-v-03}{CA-DM-CON-ICD-0016-V-03}
    \\\vcdJiraRef{LVV-4909}~{\tiny
    }
    \end{tabular} &
        & & \\
      \cmidrule{2-5}
      &
    \begin{tabular}{@{}l@{}}
    \hypertarget{ca-dm-con-icd-0016-v-04}{CA-DM-CON-ICD-0016-V-04}
    \\\vcdJiraRef{LVV-4910}~{\tiny
    }
    \end{tabular} &
        & & \\
  \midrule
  \begin{tabular}{@{}l@{}}
  CA-DM-CON-ICD-0014\\\vcdDocRef{LSE-69}~{\tiny
  }
  \end{tabular} &
    \begin{tabular}{@{}l@{}}
    \hypertarget{ca-dm-con-icd-0014-v-03}{CA-DM-CON-ICD-0014-V-03}
    \\\vcdJiraRef{LVV-4915}~{\tiny
    }
    \end{tabular} &
        & & \\
      \cmidrule{2-5}
      &
    \begin{tabular}{@{}l@{}}
    \hypertarget{ca-dm-con-icd-0014-v-04}{CA-DM-CON-ICD-0014-V-04}
    \\\vcdJiraRef{LVV-4916}~{\tiny
    }
    \end{tabular} &
        & & \\
  \midrule
  \begin{tabular}{@{}l@{}}
  CA-DM-CON-ICD-0015\\\vcdDocRef{LSE-69}~{\tiny
  }
  \end{tabular} &
    \begin{tabular}{@{}l@{}}
    \hypertarget{ca-dm-con-icd-0015-v-03}{CA-DM-CON-ICD-0015-V-03}
    \\\vcdJiraRef{LVV-4921}~{\tiny
    }
    \end{tabular} &
        & & \\
      \cmidrule{2-5}
      &
    \begin{tabular}{@{}l@{}}
    \hypertarget{ca-dm-con-icd-0015-v-04}{CA-DM-CON-ICD-0015-V-04}
    \\\vcdJiraRef{LVV-4922}~{\tiny
    }
    \end{tabular} &
        & & \\
  \midrule
  \begin{tabular}{@{}l@{}}
  CA-DM-CON-ICD-0020\\\vcdDocRef{LSE-69}~{\tiny
  }
  \end{tabular} &
    \begin{tabular}{@{}l@{}}
    \hypertarget{ca-dm-con-icd-0020-v-02}{CA-DM-CON-ICD-0020-V-02}
    \\\vcdJiraRef{LVV-18849}~{\tiny
    }
    \end{tabular} &
        & & \\
  \midrule
  \begin{tabular}{@{}l@{}}
  CA-DM-CON-ICD-0022\\\vcdDocRef{LSE-69}~{\tiny
  }
  \end{tabular} &
    \begin{tabular}{@{}l@{}}
    \hypertarget{ca-dm-con-icd-0022-v-02}{CA-DM-CON-ICD-0022-V-02}
    \\\vcdJiraRef{LVV-18852}~{\tiny
    }
    \end{tabular} &
        & & \\
  \midrule
  \begin{tabular}{@{}l@{}}
  CA-DM-CON-ICD-0023\\\vcdDocRef{LSE-69}~{\tiny
  }
  \end{tabular} &
    \begin{tabular}{@{}l@{}}
    \hypertarget{ca-dm-con-icd-0023-v-02}{CA-DM-CON-ICD-0023-V-02}
    \\\vcdJiraRef{LVV-18855}~{\tiny
    }
    \end{tabular} &
        & & \\
  \midrule
  \begin{tabular}{@{}l@{}}
  CA-DM-CON-ICD-0021\\\vcdDocRef{LSE-69}~{\tiny
  }
  \end{tabular} &
    \begin{tabular}{@{}l@{}}
    \hypertarget{ca-dm-con-icd-0021-v-02}{CA-DM-CON-ICD-0021-V-02}
    \\\vcdJiraRef{LVV-18858}~{\tiny
    }
    \end{tabular} &
        & & \\
  \midrule
\label{tab:dmvcd}
\end{longtable}
}

\subsection{LSE-72 Requirements Coverage}

\setlength\LTleft{-0.25in}
\setlength\LTright{-0.5in}
{\small
\begin{longtable}{lllll}
\caption{ DM LSE-72 Requirements.} \\
\toprule
\textbf{Requirement} & \textbf{Verification Element} & \textbf{Test Case} & \textbf{Last Run} & \textbf{Test Status} \\
\toprule
\endhead
  \begin{tabular}{@{}l@{}}
  OCS-DM-COM-ICD-0040\\\vcdDocRef{LSE-72}~{\tiny
  }
  \end{tabular} &
    \begin{tabular}{@{}l@{}}
    \hypertarget{ocs-dm-com-icd-0040-v-01}{OCS-DM-COM-ICD-0040-V-01}
    \\\vcdJiraRef{LVV-5237}~{\tiny
    }
    \end{tabular} &
        & & \\
      \cmidrule{2-5}
      &
    \begin{tabular}{@{}l@{}}
    \hypertarget{ocs-dm-com-icd-0040-v-02}{OCS-DM-COM-ICD-0040-V-02}
    \\\vcdJiraRef{LVV-5238}~{\tiny
    }
    \end{tabular} &
        & & \\
  \midrule
  \begin{tabular}{@{}l@{}}
  OCS-DM-COM-ICD-0009\\\vcdDocRef{LSE-72}~{\tiny
  }
  \end{tabular} &
    \begin{tabular}{@{}l@{}}
    \hypertarget{ocs-dm-com-icd-0009-v-01}{OCS-DM-COM-ICD-0009-V-01}
    \\\vcdJiraRef{LVV-5243}~{\tiny
    }
    \end{tabular} &
        & & \\
      \cmidrule{2-5}
      &
    \begin{tabular}{@{}l@{}}
    \hypertarget{ocs-dm-com-icd-0009-v-02}{OCS-DM-COM-ICD-0009-V-02}
    \\\vcdJiraRef{LVV-5244}~{\tiny
    }
    \end{tabular} &
        & & \\
  \midrule
  \begin{tabular}{@{}l@{}}
  OCS-DM-COM-ICD-0013\\\vcdDocRef{LSE-72}~{\tiny
  }
  \end{tabular} &
    \begin{tabular}{@{}l@{}}
    \hypertarget{ocs-dm-com-icd-0013-v-01}{OCS-DM-COM-ICD-0013-V-01}
    \\\vcdJiraRef{LVV-5249}~{\tiny
    }
    \end{tabular} &
        & & \\
      \cmidrule{2-5}
      &
    \begin{tabular}{@{}l@{}}
    \hypertarget{ocs-dm-com-icd-0013-v-02}{OCS-DM-COM-ICD-0013-V-02}
    \\\vcdJiraRef{LVV-5250}~{\tiny
    }
    \end{tabular} &
        & & \\
  \midrule
  \begin{tabular}{@{}l@{}}
  OCS-DM-COM-ICD-0015\\\vcdDocRef{LSE-72}~{\tiny
  }
  \end{tabular} &
    \begin{tabular}{@{}l@{}}
    \hypertarget{ocs-dm-com-icd-0015-v-01}{OCS-DM-COM-ICD-0015-V-01}
    \\\vcdJiraRef{LVV-5255}~{\tiny
    }
    \end{tabular} &
        & & \\
      \cmidrule{2-5}
      &
    \begin{tabular}{@{}l@{}}
    \hypertarget{ocs-dm-com-icd-0015-v-02}{OCS-DM-COM-ICD-0015-V-02}
    \\\vcdJiraRef{LVV-5256}~{\tiny
    }
    \end{tabular} &
        & & \\
  \midrule
  \begin{tabular}{@{}l@{}}
  OCS-DM-COM-ICD-0014\\\vcdDocRef{LSE-72}~{\tiny
  }
  \end{tabular} &
    \begin{tabular}{@{}l@{}}
    \hypertarget{ocs-dm-com-icd-0014-v-01}{OCS-DM-COM-ICD-0014-V-01}
    \\\vcdJiraRef{LVV-5261}~{\tiny
    }
    \end{tabular} &
        & & \\
      \cmidrule{2-5}
      &
    \begin{tabular}{@{}l@{}}
    \hypertarget{ocs-dm-com-icd-0014-v-02}{OCS-DM-COM-ICD-0014-V-02}
    \\\vcdJiraRef{LVV-5262}~{\tiny
    }
    \end{tabular} &
        & & \\
  \midrule
  \begin{tabular}{@{}l@{}}
  OCS-DM-COM-ICD-0038\\\vcdDocRef{LSE-72}~{\tiny
  }
  \end{tabular} &
    \begin{tabular}{@{}l@{}}
    \hypertarget{ocs-dm-com-icd-0038-v-01}{OCS-DM-COM-ICD-0038-V-01}
    \\\vcdJiraRef{LVV-5267}~{\tiny
    }
    \end{tabular} &
        & & \\
      \cmidrule{2-5}
      &
    \begin{tabular}{@{}l@{}}
    \hypertarget{ocs-dm-com-icd-0038-v-02}{OCS-DM-COM-ICD-0038-V-02}
    \\\vcdJiraRef{LVV-5268}~{\tiny
    }
    \end{tabular} &
        & & \\
  \midrule
  \begin{tabular}{@{}l@{}}
  OCS-DM-COM-ICD-0039\\\vcdDocRef{LSE-72}~{\tiny
  }
  \end{tabular} &
    \begin{tabular}{@{}l@{}}
    \hypertarget{ocs-dm-com-icd-0039-v-01}{OCS-DM-COM-ICD-0039-V-01}
    \\\vcdJiraRef{LVV-5273}~{\tiny
    }
    \end{tabular} &
        & & \\
      \cmidrule{2-5}
      &
    \begin{tabular}{@{}l@{}}
    \hypertarget{ocs-dm-com-icd-0039-v-02}{OCS-DM-COM-ICD-0039-V-02}
    \\\vcdJiraRef{LVV-5274}~{\tiny
    }
    \end{tabular} &
        & & \\
  \midrule
  \begin{tabular}{@{}l@{}}
  OCS-DM-COM-ICD-0037\\\vcdDocRef{LSE-72}~{\tiny
  }
  \end{tabular} &
    \begin{tabular}{@{}l@{}}
    \hypertarget{ocs-dm-com-icd-0037-v-01}{OCS-DM-COM-ICD-0037-V-01}
    \\\vcdJiraRef{LVV-5279}~{\tiny
    }
    \end{tabular} &
        & & \\
      \cmidrule{2-5}
      &
    \begin{tabular}{@{}l@{}}
    \hypertarget{ocs-dm-com-icd-0037-v-02}{OCS-DM-COM-ICD-0037-V-02}
    \\\vcdJiraRef{LVV-5280}~{\tiny
    }
    \end{tabular} &
        & & \\
  \midrule
  \begin{tabular}{@{}l@{}}
  OCS-DM-COM-ICD-0036\\\vcdDocRef{LSE-72}~{\tiny
  }
  \end{tabular} &
    \begin{tabular}{@{}l@{}}
    \hypertarget{ocs-dm-com-icd-0036-v-01}{OCS-DM-COM-ICD-0036-V-01}
    \\\vcdJiraRef{LVV-5285}~{\tiny
    }
    \end{tabular} &
        & & \\
      \cmidrule{2-5}
      &
    \begin{tabular}{@{}l@{}}
    \hypertarget{ocs-dm-com-icd-0036-v-02}{OCS-DM-COM-ICD-0036-V-02}
    \\\vcdJiraRef{LVV-5286}~{\tiny
    }
    \end{tabular} &
        & & \\
  \midrule
  \begin{tabular}{@{}l@{}}
  OCS-DM-COM-ICD-0012\\\vcdDocRef{LSE-72}~{\tiny
  }
  \end{tabular} &
    \begin{tabular}{@{}l@{}}
    \hypertarget{ocs-dm-com-icd-0012-v-01}{OCS-DM-COM-ICD-0012-V-01}
    \\\vcdJiraRef{LVV-5291}~{\tiny
    }
    \end{tabular} &
        & & \\
      \cmidrule{2-5}
      &
    \begin{tabular}{@{}l@{}}
    \hypertarget{ocs-dm-com-icd-0012-v-02}{OCS-DM-COM-ICD-0012-V-02}
    \\\vcdJiraRef{LVV-5292}~{\tiny
    }
    \end{tabular} &
        & & \\
  \midrule
  \begin{tabular}{@{}l@{}}
  OCS-DM-COM-ICD-0003\\\vcdDocRef{LSE-72}~{\tiny
  }
  \end{tabular} &
    \begin{tabular}{@{}l@{}}
    \hypertarget{ocs-dm-com-icd-0003-v-01}{OCS-DM-COM-ICD-0003-V-01}
    \\\vcdJiraRef{LVV-5297}~{\tiny
    }
    \end{tabular} &
        & & \\
      \cmidrule{2-5}
      &
    \begin{tabular}{@{}l@{}}
    \hypertarget{ocs-dm-com-icd-0003-v-02}{OCS-DM-COM-ICD-0003-V-02}
    \\\vcdJiraRef{LVV-5298}~{\tiny
    }
    \end{tabular} &
        & & \\
  \midrule
  \begin{tabular}{@{}l@{}}
  OCS-DM-COM-ICD-0034\\\vcdDocRef{LSE-72}~{\tiny
  }
  \end{tabular} &
    \begin{tabular}{@{}l@{}}
    \hypertarget{ocs-dm-com-icd-0034-v-01}{OCS-DM-COM-ICD-0034-V-01}
    \\\vcdJiraRef{LVV-5303}~{\tiny
    }
    \end{tabular} &
        & & \\
      \cmidrule{2-5}
      &
    \begin{tabular}{@{}l@{}}
    \hypertarget{ocs-dm-com-icd-0034-v-02}{OCS-DM-COM-ICD-0034-V-02}
    \\\vcdJiraRef{LVV-5304}~{\tiny
    }
    \end{tabular} &
        & & \\
  \midrule
  \begin{tabular}{@{}l@{}}
  OCS-DM-COM-ICD-0032\\\vcdDocRef{LSE-72}~{\tiny
  }
  \end{tabular} &
    \begin{tabular}{@{}l@{}}
    \hypertarget{ocs-dm-com-icd-0032-v-01}{OCS-DM-COM-ICD-0032-V-01}
    \\\vcdJiraRef{LVV-5309}~{\tiny
    }
    \end{tabular} &
        & & \\
      \cmidrule{2-5}
      &
    \begin{tabular}{@{}l@{}}
    \hypertarget{ocs-dm-com-icd-0032-v-02}{OCS-DM-COM-ICD-0032-V-02}
    \\\vcdJiraRef{LVV-5310}~{\tiny
    }
    \end{tabular} &
        & & \\
  \midrule
  \begin{tabular}{@{}l@{}}
  OCS-DM-COM-ICD-0006\\\vcdDocRef{LSE-72}~{\tiny
  }
  \end{tabular} &
    \begin{tabular}{@{}l@{}}
    \hypertarget{ocs-dm-com-icd-0006-v-01}{OCS-DM-COM-ICD-0006-V-01}
    \\\vcdJiraRef{LVV-5315}~{\tiny
    }
    \end{tabular} &
        & & \\
      \cmidrule{2-5}
      &
    \begin{tabular}{@{}l@{}}
    \hypertarget{ocs-dm-com-icd-0006-v-02}{OCS-DM-COM-ICD-0006-V-02}
    \\\vcdJiraRef{LVV-5316}~{\tiny
    }
    \end{tabular} &
        & & \\
  \midrule
  \begin{tabular}{@{}l@{}}
  OCS-DM-COM-ICD-0004\\\vcdDocRef{LSE-72}~{\tiny
  }
  \end{tabular} &
    \begin{tabular}{@{}l@{}}
    \hypertarget{ocs-dm-com-icd-0004-v-01}{OCS-DM-COM-ICD-0004-V-01}
    \\\vcdJiraRef{LVV-5321}~{\tiny
    }
    \end{tabular} &
        & & \\
      \cmidrule{2-5}
      &
    \begin{tabular}{@{}l@{}}
    \hypertarget{ocs-dm-com-icd-0004-v-02}{OCS-DM-COM-ICD-0004-V-02}
    \\\vcdJiraRef{LVV-5322}~{\tiny
    }
    \end{tabular} &
        & & \\
  \midrule
  \begin{tabular}{@{}l@{}}
  OCS-DM-COM-ICD-0008\\\vcdDocRef{LSE-72}~{\tiny
  }
  \end{tabular} &
    \begin{tabular}{@{}l@{}}
    \hypertarget{ocs-dm-com-icd-0008-v-01}{OCS-DM-COM-ICD-0008-V-01}
    \\\vcdJiraRef{LVV-5327}~{\tiny
    }
    \end{tabular} &
        & & \\
      \cmidrule{2-5}
      &
    \begin{tabular}{@{}l@{}}
    \hypertarget{ocs-dm-com-icd-0008-v-02}{OCS-DM-COM-ICD-0008-V-02}
    \\\vcdJiraRef{LVV-5328}~{\tiny
    }
    \end{tabular} &
        & & \\
  \midrule
  \begin{tabular}{@{}l@{}}
  OCS-DM-COM-ICD-0033\\\vcdDocRef{LSE-72}~{\tiny
  }
  \end{tabular} &
    \begin{tabular}{@{}l@{}}
    \hypertarget{ocs-dm-com-icd-0033-v-01}{OCS-DM-COM-ICD-0033-V-01}
    \\\vcdJiraRef{LVV-5333}~{\tiny
    }
    \end{tabular} &
        & & \\
      \cmidrule{2-5}
      &
    \begin{tabular}{@{}l@{}}
    \hypertarget{ocs-dm-com-icd-0033-v-02}{OCS-DM-COM-ICD-0033-V-02}
    \\\vcdJiraRef{LVV-5334}~{\tiny
    }
    \end{tabular} &
        & & \\
  \midrule
  \begin{tabular}{@{}l@{}}
  OCS-DM-COM-ICD-0005\\\vcdDocRef{LSE-72}~{\tiny
  }
  \end{tabular} &
    \begin{tabular}{@{}l@{}}
    \hypertarget{ocs-dm-com-icd-0005-v-01}{OCS-DM-COM-ICD-0005-V-01}
    \\\vcdJiraRef{LVV-5339}~{\tiny
    }
    \end{tabular} &
        & & \\
      \cmidrule{2-5}
      &
    \begin{tabular}{@{}l@{}}
    \hypertarget{ocs-dm-com-icd-0005-v-02}{OCS-DM-COM-ICD-0005-V-02}
    \\\vcdJiraRef{LVV-5340}~{\tiny
    }
    \end{tabular} &
        & & \\
  \midrule
  \begin{tabular}{@{}l@{}}
  OCS-DM-COM-ICD-0035\\\vcdDocRef{LSE-72}~{\tiny
  }
  \end{tabular} &
    \begin{tabular}{@{}l@{}}
    \hypertarget{ocs-dm-com-icd-0035-v-01}{OCS-DM-COM-ICD-0035-V-01}
    \\\vcdJiraRef{LVV-5345}~{\tiny
    }
    \end{tabular} &
        & & \\
      \cmidrule{2-5}
      &
    \begin{tabular}{@{}l@{}}
    \hypertarget{ocs-dm-com-icd-0035-v-02}{OCS-DM-COM-ICD-0035-V-02}
    \\\vcdJiraRef{LVV-5346}~{\tiny
    }
    \end{tabular} &
        & & \\
  \midrule
  \begin{tabular}{@{}l@{}}
  OCS-DM-COM-ICD-0007\\\vcdDocRef{LSE-72}~{\tiny
  }
  \end{tabular} &
    \begin{tabular}{@{}l@{}}
    \hypertarget{ocs-dm-com-icd-0007-v-01}{OCS-DM-COM-ICD-0007-V-01}
    \\\vcdJiraRef{LVV-5351}~{\tiny
    }
    \end{tabular} &
        & & \\
      \cmidrule{2-5}
      &
    \begin{tabular}{@{}l@{}}
    \hypertarget{ocs-dm-com-icd-0007-v-02}{OCS-DM-COM-ICD-0007-V-02}
    \\\vcdJiraRef{LVV-5352}~{\tiny
    }
    \end{tabular} &
        & & \\
  \midrule
  \begin{tabular}{@{}l@{}}
  OCS-DM-COM-ICD-0048\\\vcdDocRef{LSE-72}~{\tiny
  }
  \end{tabular} &
    \begin{tabular}{@{}l@{}}
    \hypertarget{ocs-dm-com-icd-0048-v-01}{OCS-DM-COM-ICD-0048-V-01}
    \\\vcdJiraRef{LVV-5357}~{\tiny
    }
    \end{tabular} &
        & & \\
      \cmidrule{2-5}
      &
    \begin{tabular}{@{}l@{}}
    \hypertarget{ocs-dm-com-icd-0048-v-02}{OCS-DM-COM-ICD-0048-V-02}
    \\\vcdJiraRef{LVV-5358}~{\tiny
    }
    \end{tabular} &
        & & \\
  \midrule
  \begin{tabular}{@{}l@{}}
  OCS-DM-COM-ICD-0055\\\vcdDocRef{LSE-72}~{\tiny
  }
  \end{tabular} &
    \begin{tabular}{@{}l@{}}
    \hypertarget{ocs-dm-com-icd-0055-v-01}{OCS-DM-COM-ICD-0055-V-01}
    \\\vcdJiraRef{LVV-5363}~{\tiny
    }
    \end{tabular} &
        & & \\
      \cmidrule{2-5}
      &
    \begin{tabular}{@{}l@{}}
    \hypertarget{ocs-dm-com-icd-0055-v-02}{OCS-DM-COM-ICD-0055-V-02}
    \\\vcdJiraRef{LVV-5364}~{\tiny
    }
    \end{tabular} &
        & & \\
  \midrule
  \begin{tabular}{@{}l@{}}
  OCS-DM-COM-ICD-0054\\\vcdDocRef{LSE-72}~{\tiny
  }
  \end{tabular} &
    \begin{tabular}{@{}l@{}}
    \hypertarget{ocs-dm-com-icd-0054-v-01}{OCS-DM-COM-ICD-0054-V-01}
    \\\vcdJiraRef{LVV-5369}~{\tiny
    }
    \end{tabular} &
        & & \\
      \cmidrule{2-5}
      &
    \begin{tabular}{@{}l@{}}
    \hypertarget{ocs-dm-com-icd-0054-v-02}{OCS-DM-COM-ICD-0054-V-02}
    \\\vcdJiraRef{LVV-5370}~{\tiny
    }
    \end{tabular} &
        & & \\
  \midrule
  \begin{tabular}{@{}l@{}}
  OCS-DM-COM-ICD-0019\\\vcdDocRef{LSE-72}~{\tiny
  }
  \end{tabular} &
    \begin{tabular}{@{}l@{}}
    \hypertarget{ocs-dm-com-icd-0019-v-01}{OCS-DM-COM-ICD-0019-V-01}
    \\\vcdJiraRef{LVV-5375}~{\tiny
    }
    \end{tabular} &
        & & \\
      \cmidrule{2-5}
      &
    \begin{tabular}{@{}l@{}}
    \hypertarget{ocs-dm-com-icd-0019-v-02}{OCS-DM-COM-ICD-0019-V-02}
    \\\vcdJiraRef{LVV-5376}~{\tiny
    }
    \end{tabular} &
        & & \\
  \midrule
  \begin{tabular}{@{}l@{}}
  OCS-DM-COM-ICD-0017\\\vcdDocRef{LSE-72}~{\tiny
  }
  \end{tabular} &
    \begin{tabular}{@{}l@{}}
    \hypertarget{ocs-dm-com-icd-0017-v-01}{OCS-DM-COM-ICD-0017-V-01}
    \\\vcdJiraRef{LVV-5381}~{\tiny
    }
    \end{tabular} &
        & & \\
      \cmidrule{2-5}
      &
    \begin{tabular}{@{}l@{}}
    \hypertarget{ocs-dm-com-icd-0017-v-02}{OCS-DM-COM-ICD-0017-V-02}
    \\\vcdJiraRef{LVV-5382}~{\tiny
    }
    \end{tabular} &
        & & \\
  \midrule
  \begin{tabular}{@{}l@{}}
  OCS-DM-COM-ICD-0018\\\vcdDocRef{LSE-72}~{\tiny
  }
  \end{tabular} &
    \begin{tabular}{@{}l@{}}
    \hypertarget{ocs-dm-com-icd-0018-v-01}{OCS-DM-COM-ICD-0018-V-01}
    \\\vcdJiraRef{LVV-5387}~{\tiny
    }
    \end{tabular} &
        & & \\
      \cmidrule{2-5}
      &
    \begin{tabular}{@{}l@{}}
    \hypertarget{ocs-dm-com-icd-0018-v-02}{OCS-DM-COM-ICD-0018-V-02}
    \\\vcdJiraRef{LVV-5388}~{\tiny
    }
    \end{tabular} &
        & & \\
  \midrule
  \begin{tabular}{@{}l@{}}
  OCS-DM-COM-ICD-0021\\\vcdDocRef{LSE-72}~{\tiny
  }
  \end{tabular} &
    \begin{tabular}{@{}l@{}}
    \hypertarget{ocs-dm-com-icd-0021-v-01}{OCS-DM-COM-ICD-0021-V-01}
    \\\vcdJiraRef{LVV-5393}~{\tiny
    }
    \end{tabular} &
        & & \\
      \cmidrule{2-5}
      &
    \begin{tabular}{@{}l@{}}
    \hypertarget{ocs-dm-com-icd-0021-v-02}{OCS-DM-COM-ICD-0021-V-02}
    \\\vcdJiraRef{LVV-5394}~{\tiny
    }
    \end{tabular} &
        & & \\
  \midrule
  \begin{tabular}{@{}l@{}}
  OCS-DM-COM-ICD-0020\\\vcdDocRef{LSE-72}~{\tiny
  }
  \end{tabular} &
    \begin{tabular}{@{}l@{}}
    \hypertarget{ocs-dm-com-icd-0020-v-01}{OCS-DM-COM-ICD-0020-V-01}
    \\\vcdJiraRef{LVV-5399}~{\tiny
    }
    \end{tabular} &
        & & \\
      \cmidrule{2-5}
      &
    \begin{tabular}{@{}l@{}}
    \hypertarget{ocs-dm-com-icd-0020-v-02}{OCS-DM-COM-ICD-0020-V-02}
    \\\vcdJiraRef{LVV-5400}~{\tiny
    }
    \end{tabular} &
        & & \\
  \midrule
  \begin{tabular}{@{}l@{}}
  OCS-DM-COM-ICD-0047\\\vcdDocRef{LSE-72}~{\tiny
  }
  \end{tabular} &
    \begin{tabular}{@{}l@{}}
    \hypertarget{ocs-dm-com-icd-0047-v-01}{OCS-DM-COM-ICD-0047-V-01}
    \\\vcdJiraRef{LVV-5405}~{\tiny
    }
    \end{tabular} &
        & & \\
      \cmidrule{2-5}
      &
    \begin{tabular}{@{}l@{}}
    \hypertarget{ocs-dm-com-icd-0047-v-02}{OCS-DM-COM-ICD-0047-V-02}
    \\\vcdJiraRef{LVV-5406}~{\tiny
    }
    \end{tabular} &
        & & \\
  \midrule
  \begin{tabular}{@{}l@{}}
  OCS-DM-COM-ICD-0046\\\vcdDocRef{LSE-72}~{\tiny
  }
  \end{tabular} &
    \begin{tabular}{@{}l@{}}
    \hypertarget{ocs-dm-com-icd-0046-v-01}{OCS-DM-COM-ICD-0046-V-01}
    \\\vcdJiraRef{LVV-5411}~{\tiny
    }
    \end{tabular} &
        & & \\
      \cmidrule{2-5}
      &
    \begin{tabular}{@{}l@{}}
    \hypertarget{ocs-dm-com-icd-0046-v-02}{OCS-DM-COM-ICD-0046-V-02}
    \\\vcdJiraRef{LVV-5412}~{\tiny
    }
    \end{tabular} &
        & & \\
  \midrule
  \begin{tabular}{@{}l@{}}
  OCS-DM-COM-ICD-0045\\\vcdDocRef{LSE-72}~{\tiny
  }
  \end{tabular} &
    \begin{tabular}{@{}l@{}}
    \hypertarget{ocs-dm-com-icd-0045-v-01}{OCS-DM-COM-ICD-0045-V-01}
    \\\vcdJiraRef{LVV-5417}~{\tiny
    }
    \end{tabular} &
        & & \\
      \cmidrule{2-5}
      &
    \begin{tabular}{@{}l@{}}
    \hypertarget{ocs-dm-com-icd-0045-v-02}{OCS-DM-COM-ICD-0045-V-02}
    \\\vcdJiraRef{LVV-5418}~{\tiny
    }
    \end{tabular} &
        & & \\
  \midrule
  \begin{tabular}{@{}l@{}}
  OCS-DM-COM-ICD-0043\\\vcdDocRef{LSE-72}~{\tiny
  }
  \end{tabular} &
    \begin{tabular}{@{}l@{}}
    \hypertarget{ocs-dm-com-icd-0043-v-01}{OCS-DM-COM-ICD-0043-V-01}
    \\\vcdJiraRef{LVV-5423}~{\tiny
    }
    \end{tabular} &
        & & \\
      \cmidrule{2-5}
      &
    \begin{tabular}{@{}l@{}}
    \hypertarget{ocs-dm-com-icd-0043-v-02}{OCS-DM-COM-ICD-0043-V-02}
    \\\vcdJiraRef{LVV-5424}~{\tiny
    }
    \end{tabular} &
        & & \\
  \midrule
  \begin{tabular}{@{}l@{}}
  OCS-DM-COM-ICD-0044\\\vcdDocRef{LSE-72}~{\tiny
  }
  \end{tabular} &
    \begin{tabular}{@{}l@{}}
    \hypertarget{ocs-dm-com-icd-0044-v-01}{OCS-DM-COM-ICD-0044-V-01}
    \\\vcdJiraRef{LVV-5429}~{\tiny
    }
    \end{tabular} &
        & & \\
      \cmidrule{2-5}
      &
    \begin{tabular}{@{}l@{}}
    \hypertarget{ocs-dm-com-icd-0044-v-02}{OCS-DM-COM-ICD-0044-V-02}
    \\\vcdJiraRef{LVV-5430}~{\tiny
    }
    \end{tabular} &
        & & \\
  \midrule
  \begin{tabular}{@{}l@{}}
  OCS-DM-COM-ICD-0052\\\vcdDocRef{LSE-72}~{\tiny
  }
  \end{tabular} &
    \begin{tabular}{@{}l@{}}
    \hypertarget{ocs-dm-com-icd-0052-v-01}{OCS-DM-COM-ICD-0052-V-01}
    \\\vcdJiraRef{LVV-5435}~{\tiny
    }
    \end{tabular} &
        & & \\
      \cmidrule{2-5}
      &
    \begin{tabular}{@{}l@{}}
    \hypertarget{ocs-dm-com-icd-0052-v-02}{OCS-DM-COM-ICD-0052-V-02}
    \\\vcdJiraRef{LVV-5436}~{\tiny
    }
    \end{tabular} &
        & & \\
  \midrule
  \begin{tabular}{@{}l@{}}
  OCS-DM-COM-ICD-0051\\\vcdDocRef{LSE-72}~{\tiny
  }
  \end{tabular} &
    \begin{tabular}{@{}l@{}}
    \hypertarget{ocs-dm-com-icd-0051-v-01}{OCS-DM-COM-ICD-0051-V-01}
    \\\vcdJiraRef{LVV-5441}~{\tiny
    }
    \end{tabular} &
        & & \\
      \cmidrule{2-5}
      &
    \begin{tabular}{@{}l@{}}
    \hypertarget{ocs-dm-com-icd-0051-v-02}{OCS-DM-COM-ICD-0051-V-02}
    \\\vcdJiraRef{LVV-5442}~{\tiny
    }
    \end{tabular} &
        & & \\
  \midrule
  \begin{tabular}{@{}l@{}}
  OCS-DM-COM-ICD-0056\\\vcdDocRef{LSE-72}~{\tiny
  }
  \end{tabular} &
    \begin{tabular}{@{}l@{}}
    \hypertarget{ocs-dm-com-icd-0056-v-01}{OCS-DM-COM-ICD-0056-V-01}
    \\\vcdJiraRef{LVV-5447}~{\tiny
    }
    \end{tabular} &
        & & \\
      \cmidrule{2-5}
      &
    \begin{tabular}{@{}l@{}}
    \hypertarget{ocs-dm-com-icd-0056-v-02}{OCS-DM-COM-ICD-0056-V-02}
    \\\vcdJiraRef{LVV-5448}~{\tiny
    }
    \end{tabular} &
        & & \\
  \midrule
  \begin{tabular}{@{}l@{}}
  OCS-DM-COM-ICD-0050\\\vcdDocRef{LSE-72}~{\tiny
  }
  \end{tabular} &
    \begin{tabular}{@{}l@{}}
    \hypertarget{ocs-dm-com-icd-0050-v-01}{OCS-DM-COM-ICD-0050-V-01}
    \\\vcdJiraRef{LVV-5453}~{\tiny
    }
    \end{tabular} &
        & & \\
      \cmidrule{2-5}
      &
    \begin{tabular}{@{}l@{}}
    \hypertarget{ocs-dm-com-icd-0050-v-02}{OCS-DM-COM-ICD-0050-V-02}
    \\\vcdJiraRef{LVV-5454}~{\tiny
    }
    \end{tabular} &
        & & \\
  \midrule
  \begin{tabular}{@{}l@{}}
  OCS-DM-COM-ICD-0053\\\vcdDocRef{LSE-72}~{\tiny
  }
  \end{tabular} &
    \begin{tabular}{@{}l@{}}
    \hypertarget{ocs-dm-com-icd-0053-v-01}{OCS-DM-COM-ICD-0053-V-01}
    \\\vcdJiraRef{LVV-5459}~{\tiny
    }
    \end{tabular} &
        & & \\
      \cmidrule{2-5}
      &
    \begin{tabular}{@{}l@{}}
    \hypertarget{ocs-dm-com-icd-0053-v-02}{OCS-DM-COM-ICD-0053-V-02}
    \\\vcdJiraRef{LVV-5460}~{\tiny
    }
    \end{tabular} &
        & & \\
  \midrule
  \begin{tabular}{@{}l@{}}
  OCS-DM-COM-ICD-0022\\\vcdDocRef{LSE-72}~{\tiny
  }
  \end{tabular} &
    \begin{tabular}{@{}l@{}}
    \hypertarget{ocs-dm-com-icd-0022-v-01}{OCS-DM-COM-ICD-0022-V-01}
    \\\vcdJiraRef{LVV-5465}~{\tiny
    }
    \end{tabular} &
        & & \\
      \cmidrule{2-5}
      &
    \begin{tabular}{@{}l@{}}
    \hypertarget{ocs-dm-com-icd-0022-v-02}{OCS-DM-COM-ICD-0022-V-02}
    \\\vcdJiraRef{LVV-5466}~{\tiny
    }
    \end{tabular} &
        & & \\
  \midrule
  \begin{tabular}{@{}l@{}}
  OCS-DM-COM-ICD-0049\\\vcdDocRef{LSE-72}~{\tiny
  }
  \end{tabular} &
    \begin{tabular}{@{}l@{}}
    \hypertarget{ocs-dm-com-icd-0049-v-01}{OCS-DM-COM-ICD-0049-V-01}
    \\\vcdJiraRef{LVV-5471}~{\tiny
    }
    \end{tabular} &
        & & \\
      \cmidrule{2-5}
      &
    \begin{tabular}{@{}l@{}}
    \hypertarget{ocs-dm-com-icd-0049-v-02}{OCS-DM-COM-ICD-0049-V-02}
    \\\vcdJiraRef{LVV-5472}~{\tiny
    }
    \end{tabular} &
        & & \\
  \midrule
  \begin{tabular}{@{}l@{}}
  OCS-DM-COM-ICD-0023\\\vcdDocRef{LSE-72}~{\tiny
  }
  \end{tabular} &
    \begin{tabular}{@{}l@{}}
    \hypertarget{ocs-dm-com-icd-0023-v-01}{OCS-DM-COM-ICD-0023-V-01}
    \\\vcdJiraRef{LVV-5477}~{\tiny
    }
    \end{tabular} &
        & & \\
      \cmidrule{2-5}
      &
    \begin{tabular}{@{}l@{}}
    \hypertarget{ocs-dm-com-icd-0023-v-02}{OCS-DM-COM-ICD-0023-V-02}
    \\\vcdJiraRef{LVV-5478}~{\tiny
    }
    \end{tabular} &
        & & \\
  \midrule
  \begin{tabular}{@{}l@{}}
  OCS-DM-COM-ICD-0025\\\vcdDocRef{LSE-72}~{\tiny
  }
  \end{tabular} &
    \begin{tabular}{@{}l@{}}
    \hypertarget{ocs-dm-com-icd-0025-v-01}{OCS-DM-COM-ICD-0025-V-01}
    \\\vcdJiraRef{LVV-5483}~{\tiny
    }
    \end{tabular} &
        & & \\
      \cmidrule{2-5}
      &
    \begin{tabular}{@{}l@{}}
    \hypertarget{ocs-dm-com-icd-0025-v-02}{OCS-DM-COM-ICD-0025-V-02}
    \\\vcdJiraRef{LVV-5484}~{\tiny
    }
    \end{tabular} &
        & & \\
  \midrule
  \begin{tabular}{@{}l@{}}
  OCS-DM-COM-ICD-0029\\\vcdDocRef{LSE-72}~{\tiny
  }
  \end{tabular} &
    \begin{tabular}{@{}l@{}}
    \hypertarget{ocs-dm-com-icd-0029-v-01}{OCS-DM-COM-ICD-0029-V-01}
    \\\vcdJiraRef{LVV-5489}~{\tiny
    }
    \end{tabular} &
        & & \\
      \cmidrule{2-5}
      &
    \begin{tabular}{@{}l@{}}
    \hypertarget{ocs-dm-com-icd-0029-v-02}{OCS-DM-COM-ICD-0029-V-02}
    \\\vcdJiraRef{LVV-5490}~{\tiny
    }
    \end{tabular} &
        & & \\
  \midrule
  \begin{tabular}{@{}l@{}}
  OCS-DM-COM-ICD-0042\\\vcdDocRef{LSE-72}~{\tiny
  }
  \end{tabular} &
    \begin{tabular}{@{}l@{}}
    \hypertarget{ocs-dm-com-icd-0042-v-01}{OCS-DM-COM-ICD-0042-V-01}
    \\\vcdJiraRef{LVV-5495}~{\tiny
    }
    \end{tabular} &
        & & \\
      \cmidrule{2-5}
      &
    \begin{tabular}{@{}l@{}}
    \hypertarget{ocs-dm-com-icd-0042-v-02}{OCS-DM-COM-ICD-0042-V-02}
    \\\vcdJiraRef{LVV-5496}~{\tiny
    }
    \end{tabular} &
        & & \\
  \midrule
  \begin{tabular}{@{}l@{}}
  OCS-DM-COM-ICD-0030\\\vcdDocRef{LSE-72}~{\tiny
  }
  \end{tabular} &
    \begin{tabular}{@{}l@{}}
    \hypertarget{ocs-dm-com-icd-0030-v-01}{OCS-DM-COM-ICD-0030-V-01}
    \\\vcdJiraRef{LVV-5501}~{\tiny
    }
    \end{tabular} &
        & & \\
      \cmidrule{2-5}
      &
    \begin{tabular}{@{}l@{}}
    \hypertarget{ocs-dm-com-icd-0030-v-02}{OCS-DM-COM-ICD-0030-V-02}
    \\\vcdJiraRef{LVV-5502}~{\tiny
    }
    \end{tabular} &
        & & \\
  \midrule
  \begin{tabular}{@{}l@{}}
  OCS-DM-COM-ICD-0026\\\vcdDocRef{LSE-72}~{\tiny
  }
  \end{tabular} &
    \begin{tabular}{@{}l@{}}
    \hypertarget{ocs-dm-com-icd-0026-v-01}{OCS-DM-COM-ICD-0026-V-01}
    \\\vcdJiraRef{LVV-5507}~{\tiny
    }
    \end{tabular} &
        & & \\
      \cmidrule{2-5}
      &
    \begin{tabular}{@{}l@{}}
    \hypertarget{ocs-dm-com-icd-0026-v-02}{OCS-DM-COM-ICD-0026-V-02}
    \\\vcdJiraRef{LVV-5508}~{\tiny
    }
    \end{tabular} &
        & & \\
  \midrule
  \begin{tabular}{@{}l@{}}
  OCS-DM-COM-ICD-0028\\\vcdDocRef{LSE-72}~{\tiny
  }
  \end{tabular} &
    \begin{tabular}{@{}l@{}}
    \hypertarget{ocs-dm-com-icd-0028-v-01}{OCS-DM-COM-ICD-0028-V-01}
    \\\vcdJiraRef{LVV-5513}~{\tiny
    }
    \end{tabular} &
        & & \\
      \cmidrule{2-5}
      &
    \begin{tabular}{@{}l@{}}
    \hypertarget{ocs-dm-com-icd-0028-v-02}{OCS-DM-COM-ICD-0028-V-02}
    \\\vcdJiraRef{LVV-5514}~{\tiny
    }
    \end{tabular} &
        & & \\
  \midrule
  \begin{tabular}{@{}l@{}}
  OCS-DM-COM-ICD-0041\\\vcdDocRef{LSE-72}~{\tiny
  }
  \end{tabular} &
    \begin{tabular}{@{}l@{}}
    \hypertarget{ocs-dm-com-icd-0041-v-01}{OCS-DM-COM-ICD-0041-V-01}
    \\\vcdJiraRef{LVV-5519}~{\tiny
    }
    \end{tabular} &
        & & \\
      \cmidrule{2-5}
      &
    \begin{tabular}{@{}l@{}}
    \hypertarget{ocs-dm-com-icd-0041-v-02}{OCS-DM-COM-ICD-0041-V-02}
    \\\vcdJiraRef{LVV-5520}~{\tiny
    }
    \end{tabular} &
        & & \\
  \midrule
  \begin{tabular}{@{}l@{}}
  OCS-DM-COM-ICD-0027\\\vcdDocRef{LSE-72}~{\tiny
  }
  \end{tabular} &
    \begin{tabular}{@{}l@{}}
    \hypertarget{ocs-dm-com-icd-0027-v-01}{OCS-DM-COM-ICD-0027-V-01}
    \\\vcdJiraRef{LVV-5525}~{\tiny
    }
    \end{tabular} &
        & & \\
      \cmidrule{2-5}
      &
    \begin{tabular}{@{}l@{}}
    \hypertarget{ocs-dm-com-icd-0027-v-02}{OCS-DM-COM-ICD-0027-V-02}
    \\\vcdJiraRef{LVV-5526}~{\tiny
    }
    \end{tabular} &
        & & \\
  \midrule
  \begin{tabular}{@{}l@{}}
  OCS-DM-COM-ICD-0031\\\vcdDocRef{LSE-72}~{\tiny
  }
  \end{tabular} &
    \begin{tabular}{@{}l@{}}
    \hypertarget{ocs-dm-com-icd-0031-v-01}{OCS-DM-COM-ICD-0031-V-01}
    \\\vcdJiraRef{LVV-5531}~{\tiny
    }
    \end{tabular} &
        & & \\
      \cmidrule{2-5}
      &
    \begin{tabular}{@{}l@{}}
    \hypertarget{ocs-dm-com-icd-0031-v-02}{OCS-DM-COM-ICD-0031-V-02}
    \\\vcdJiraRef{LVV-5532}~{\tiny
    }
    \end{tabular} &
        & & \\
  \midrule
  \begin{tabular}{@{}l@{}}
  OCS-DM-COM-ICD-0002\\\vcdDocRef{LSE-72}~{\tiny
  }
  \end{tabular} &
    \begin{tabular}{@{}l@{}}
    \hypertarget{ocs-dm-com-icd-0002-v-01}{OCS-DM-COM-ICD-0002-V-01}
    \\\vcdJiraRef{LVV-5537}~{\tiny
    }
    \end{tabular} &
        & & \\
      \cmidrule{2-5}
      &
    \begin{tabular}{@{}l@{}}
    \hypertarget{ocs-dm-com-icd-0002-v-02}{OCS-DM-COM-ICD-0002-V-02}
    \\\vcdJiraRef{LVV-5538}~{\tiny
    }
    \end{tabular} &
        & & \\
  \midrule
  \begin{tabular}{@{}l@{}}
  OCS-DM-COM-ICD-0001\\\vcdDocRef{LSE-72}~{\tiny
  }
  \end{tabular} &
    \begin{tabular}{@{}l@{}}
    \hypertarget{ocs-dm-com-icd-0001-v-01}{OCS-DM-COM-ICD-0001-V-01}
    \\\vcdJiraRef{LVV-5543}~{\tiny
    }
    \end{tabular} &
        & & \\
      \cmidrule{2-5}
      &
    \begin{tabular}{@{}l@{}}
    \hypertarget{ocs-dm-com-icd-0001-v-02}{OCS-DM-COM-ICD-0001-V-02}
    \\\vcdJiraRef{LVV-5544}~{\tiny
    }
    \end{tabular} &
        & & \\
  \midrule
\label{tab:dmvcd}
\end{longtable}
}

\subsection{LSE-75 Requirements Coverage}

\setlength\LTleft{-0.25in}
\setlength\LTright{-0.5in}
{\small
\begin{longtable}{lllll}
\caption{ DM LSE-75 Requirements.} \\
\toprule
\textbf{Requirement} & \textbf{Verification Element} & \textbf{Test Case} & \textbf{Last Run} & \textbf{Test Status} \\
\toprule
\endhead
  \begin{tabular}{@{}l@{}}
  DM-TS-CON-ICD-0003\\\vcdDocRef{LSE-75}~{\tiny
  }
  \end{tabular} &
    \begin{tabular}{@{}l@{}}
    \hypertarget{dm-ts-con-icd-0003-v-01}{DM-TS-CON-ICD-0003-V-01}
    \\\vcdJiraRef{LVV-5628}~{\tiny
    }
    \end{tabular} &
        & & \\
      \cmidrule{2-5}
      &
    \begin{tabular}{@{}l@{}}
    \hypertarget{dm-ts-con-icd-0003-v-02}{DM-TS-CON-ICD-0003-V-02}
    \\\vcdJiraRef{LVV-5629}~{\tiny
    }
    \end{tabular} &
        & & \\
  \midrule
  \begin{tabular}{@{}l@{}}
  DM-TS-CON-ICD-0010\\\vcdDocRef{LSE-75}~{\tiny
  }
  \end{tabular} &
    \begin{tabular}{@{}l@{}}
    \hypertarget{dm-ts-con-icd-0010-v-01}{DM-TS-CON-ICD-0010-V-01}
    \\\vcdJiraRef{LVV-5634}~{\tiny
    }
    \end{tabular} &
        & & \\
      \cmidrule{2-5}
      &
    \begin{tabular}{@{}l@{}}
    \hypertarget{dm-ts-con-icd-0010-v-02}{DM-TS-CON-ICD-0010-V-02}
    \\\vcdJiraRef{LVV-5635}~{\tiny
    }
    \end{tabular} &
        & & \\
  \midrule
  \begin{tabular}{@{}l@{}}
  DM-TS-CON-ICD-0011\\\vcdDocRef{LSE-75}~{\tiny
  }
  \end{tabular} &
    \begin{tabular}{@{}l@{}}
    \hypertarget{dm-ts-con-icd-0011-v-01}{DM-TS-CON-ICD-0011-V-01}
    \\\vcdJiraRef{LVV-5640}~{\tiny
    }
    \end{tabular} &
        & & \\
      \cmidrule{2-5}
      &
    \begin{tabular}{@{}l@{}}
    \hypertarget{dm-ts-con-icd-0011-v-02}{DM-TS-CON-ICD-0011-V-02}
    \\\vcdJiraRef{LVV-5641}~{\tiny
    }
    \end{tabular} &
        & & \\
  \midrule
  \begin{tabular}{@{}l@{}}
  DM-TS-CON-ICD-0002\\\vcdDocRef{LSE-75}~{\tiny
  }
  \end{tabular} &
    \begin{tabular}{@{}l@{}}
    \hypertarget{dm-ts-con-icd-0002-v-01}{DM-TS-CON-ICD-0002-V-01}
    \\\vcdJiraRef{LVV-5646}~{\tiny
    }
    \end{tabular} &
        & & \\
      \cmidrule{2-5}
      &
    \begin{tabular}{@{}l@{}}
    \hypertarget{dm-ts-con-icd-0002-v-02}{DM-TS-CON-ICD-0002-V-02}
    \\\vcdJiraRef{LVV-5647}~{\tiny
    }
    \end{tabular} &
        & & \\
  \midrule
  \begin{tabular}{@{}l@{}}
  DM-TS-CON-ICD-0006\\\vcdDocRef{LSE-75}~{\tiny
  }
  \end{tabular} &
    \begin{tabular}{@{}l@{}}
    \hypertarget{dm-ts-con-icd-0006-v-01}{DM-TS-CON-ICD-0006-V-01}
    \\\vcdJiraRef{LVV-5652}~{\tiny
    }
    \end{tabular} &
        & & \\
      \cmidrule{2-5}
      &
    \begin{tabular}{@{}l@{}}
    \hypertarget{dm-ts-con-icd-0006-v-02}{DM-TS-CON-ICD-0006-V-02}
    \\\vcdJiraRef{LVV-5653}~{\tiny
    }
    \end{tabular} &
        & & \\
  \midrule
  \begin{tabular}{@{}l@{}}
  DM-TS-CON-ICD-0007\\\vcdDocRef{LSE-75}~{\tiny
  }
  \end{tabular} &
    \begin{tabular}{@{}l@{}}
    \hypertarget{dm-ts-con-icd-0007-v-01}{DM-TS-CON-ICD-0007-V-01}
    \\\vcdJiraRef{LVV-5658}~{\tiny
    }
    \end{tabular} &
        & & \\
      \cmidrule{2-5}
      &
    \begin{tabular}{@{}l@{}}
    \hypertarget{dm-ts-con-icd-0007-v-02}{DM-TS-CON-ICD-0007-V-02}
    \\\vcdJiraRef{LVV-5659}~{\tiny
    }
    \end{tabular} &
        & & \\
  \midrule
  \begin{tabular}{@{}l@{}}
  DM-TS-CON-ICD-0009\\\vcdDocRef{LSE-75}~{\tiny
  }
  \end{tabular} &
    \begin{tabular}{@{}l@{}}
    \hypertarget{dm-ts-con-icd-0009-v-01}{DM-TS-CON-ICD-0009-V-01}
    \\\vcdJiraRef{LVV-5664}~{\tiny
    }
    \end{tabular} &
        & & \\
      \cmidrule{2-5}
      &
    \begin{tabular}{@{}l@{}}
    \hypertarget{dm-ts-con-icd-0009-v-02}{DM-TS-CON-ICD-0009-V-02}
    \\\vcdJiraRef{LVV-5665}~{\tiny
    }
    \end{tabular} &
        & & \\
  \midrule
  \begin{tabular}{@{}l@{}}
  DM-TS-CON-ICD-0008\\\vcdDocRef{LSE-75}~{\tiny
  }
  \end{tabular} &
    \begin{tabular}{@{}l@{}}
    \hypertarget{dm-ts-con-icd-0008-v-01}{DM-TS-CON-ICD-0008-V-01}
    \\\vcdJiraRef{LVV-5670}~{\tiny
    }
    \end{tabular} &
        & & \\
      \cmidrule{2-5}
      &
    \begin{tabular}{@{}l@{}}
    \hypertarget{dm-ts-con-icd-0008-v-02}{DM-TS-CON-ICD-0008-V-02}
    \\\vcdJiraRef{LVV-5671}~{\tiny
    }
    \end{tabular} &
        & & \\
  \midrule
  \begin{tabular}{@{}l@{}}
  DM-TS-CON-ICD-0004\\\vcdDocRef{LSE-75}~{\tiny
  }
  \end{tabular} &
    \begin{tabular}{@{}l@{}}
    \hypertarget{dm-ts-con-icd-0004-v-01}{DM-TS-CON-ICD-0004-V-01}
    \\\vcdJiraRef{LVV-5676}~{\tiny
    }
    \end{tabular} &
        & & \\
      \cmidrule{2-5}
      &
    \begin{tabular}{@{}l@{}}
    \hypertarget{dm-ts-con-icd-0004-v-02}{DM-TS-CON-ICD-0004-V-02}
    \\\vcdJiraRef{LVV-5677}~{\tiny
    }
    \end{tabular} &
        & & \\
  \midrule
\label{tab:dmvcd}
\end{longtable}
}

\subsection{LSE-130 Requirements Coverage}

\setlength\LTleft{-0.25in}
\setlength\LTright{-0.5in}
{\small
\begin{longtable}{lllll}
\caption{ DM LSE-130 Requirements.} \\
\toprule
\textbf{Requirement} & \textbf{Verification Element} & \textbf{Test Case} & \textbf{Last Run} & \textbf{Test Status} \\
\toprule
\endhead
  \begin{tabular}{@{}l@{}}
  CA-DM-SUP-ICD-0026\\\vcdDocRef{LSE-130}~{\tiny
  }
  \end{tabular} &
    \begin{tabular}{@{}l@{}}
    \hypertarget{ca-dm-sup-icd-0026-v-03}{CA-DM-SUP-ICD-0026-V-03}
    \\\vcdJiraRef{LVV-6140}~{\tiny
    }
    \end{tabular} &
        & & \\
      \cmidrule{2-5}
      &
    \begin{tabular}{@{}l@{}}
    \hypertarget{ca-dm-sup-icd-0026-v-04}{CA-DM-SUP-ICD-0026-V-04}
    \\\vcdJiraRef{LVV-6141}~{\tiny
    }
    \end{tabular} &
        & & \\
  \midrule
  \begin{tabular}{@{}l@{}}
  CA-DM-SUP-ICD-0027\\\vcdDocRef{LSE-130}~{\tiny
  }
  \end{tabular} &
    \begin{tabular}{@{}l@{}}
    \hypertarget{ca-dm-sup-icd-0027-v-03}{CA-DM-SUP-ICD-0027-V-03}
    \\\vcdJiraRef{LVV-6146}~{\tiny
    }
    \end{tabular} &
        & & \\
      \cmidrule{2-5}
      &
    \begin{tabular}{@{}l@{}}
    \hypertarget{ca-dm-sup-icd-0027-v-04}{CA-DM-SUP-ICD-0027-V-04}
    \\\vcdJiraRef{LVV-6147}~{\tiny
    }
    \end{tabular} &
        & & \\
  \midrule
  \begin{tabular}{@{}l@{}}
  CA-DM-SUP-ICD-0024\\\vcdDocRef{LSE-130}~{\tiny
  }
  \end{tabular} &
    \begin{tabular}{@{}l@{}}
    \hypertarget{ca-dm-sup-icd-0024-v-03}{CA-DM-SUP-ICD-0024-V-03}
    \\\vcdJiraRef{LVV-6152}~{\tiny
    }
    \end{tabular} &
        & & \\
      \cmidrule{2-5}
      &
    \begin{tabular}{@{}l@{}}
    \hypertarget{ca-dm-sup-icd-0024-v-04}{CA-DM-SUP-ICD-0024-V-04}
    \\\vcdJiraRef{LVV-6153}~{\tiny
    }
    \end{tabular} &
        & & \\
  \midrule
  \begin{tabular}{@{}l@{}}
  CA-DM-SUP-ICD-0023\\\vcdDocRef{LSE-130}~{\tiny
  }
  \end{tabular} &
    \begin{tabular}{@{}l@{}}
    \hypertarget{ca-dm-sup-icd-0023-v-03}{CA-DM-SUP-ICD-0023-V-03}
    \\\vcdJiraRef{LVV-6158}~{\tiny
    }
    \end{tabular} &
        & & \\
      \cmidrule{2-5}
      &
    \begin{tabular}{@{}l@{}}
    \hypertarget{ca-dm-sup-icd-0023-v-04}{CA-DM-SUP-ICD-0023-V-04}
    \\\vcdJiraRef{LVV-6159}~{\tiny
    }
    \end{tabular} &
        & & \\
  \midrule
  \begin{tabular}{@{}l@{}}
  CA-DM-SUP-ICD-0025\\\vcdDocRef{LSE-130}~{\tiny
  }
  \end{tabular} &
    \begin{tabular}{@{}l@{}}
    \hypertarget{ca-dm-sup-icd-0025-v-03}{CA-DM-SUP-ICD-0025-V-03}
    \\\vcdJiraRef{LVV-6164}~{\tiny
    }
    \end{tabular} &
        & & \\
      \cmidrule{2-5}
      &
    \begin{tabular}{@{}l@{}}
    \hypertarget{ca-dm-sup-icd-0025-v-04}{CA-DM-SUP-ICD-0025-V-04}
    \\\vcdJiraRef{LVV-6165}~{\tiny
    }
    \end{tabular} &
        & & \\
  \midrule
  \begin{tabular}{@{}l@{}}
  CA-DM-SUP-ICD-0022\\\vcdDocRef{LSE-130}~{\tiny
  }
  \end{tabular} &
    \begin{tabular}{@{}l@{}}
    \hypertarget{ca-dm-sup-icd-0022-v-03}{CA-DM-SUP-ICD-0022-V-03}
    \\\vcdJiraRef{LVV-6170}~{\tiny
    }
    \end{tabular} &
        & & \\
      \cmidrule{2-5}
      &
    \begin{tabular}{@{}l@{}}
    \hypertarget{ca-dm-sup-icd-0022-v-04}{CA-DM-SUP-ICD-0022-V-04}
    \\\vcdJiraRef{LVV-6171}~{\tiny
    }
    \end{tabular} &
        & & \\
  \midrule
  \begin{tabular}{@{}l@{}}
  CA-DM-SUP-ICD-0021\\\vcdDocRef{LSE-130}~{\tiny
  }
  \end{tabular} &
    \begin{tabular}{@{}l@{}}
    \hypertarget{ca-dm-sup-icd-0021-v-03}{CA-DM-SUP-ICD-0021-V-03}
    \\\vcdJiraRef{LVV-6176}~{\tiny
    }
    \end{tabular} &
        & & \\
      \cmidrule{2-5}
      &
    \begin{tabular}{@{}l@{}}
    \hypertarget{ca-dm-sup-icd-0021-v-04}{CA-DM-SUP-ICD-0021-V-04}
    \\\vcdJiraRef{LVV-6177}~{\tiny
    }
    \end{tabular} &
        & & \\
  \midrule
  \begin{tabular}{@{}l@{}}
  CA-DM-SUP-ICD-0028\\\vcdDocRef{LSE-130}~{\tiny
  }
  \end{tabular} &
    \begin{tabular}{@{}l@{}}
    \hypertarget{ca-dm-sup-icd-0028-v-03}{CA-DM-SUP-ICD-0028-V-03}
    \\\vcdJiraRef{LVV-6182}~{\tiny
    }
    \end{tabular} &
        & & \\
      \cmidrule{2-5}
      &
    \begin{tabular}{@{}l@{}}
    \hypertarget{ca-dm-sup-icd-0028-v-04}{CA-DM-SUP-ICD-0028-V-04}
    \\\vcdJiraRef{LVV-6183}~{\tiny
    }
    \end{tabular} &
        & & \\
  \midrule
  \begin{tabular}{@{}l@{}}
  CA-DM-SUP-ICD-0029\\\vcdDocRef{LSE-130}~{\tiny
  }
  \end{tabular} &
    \begin{tabular}{@{}l@{}}
    \hypertarget{ca-dm-sup-icd-0029-v-03}{CA-DM-SUP-ICD-0029-V-03}
    \\\vcdJiraRef{LVV-6188}~{\tiny
    }
    \end{tabular} &
        & & \\
      \cmidrule{2-5}
      &
    \begin{tabular}{@{}l@{}}
    \hypertarget{ca-dm-sup-icd-0029-v-04}{CA-DM-SUP-ICD-0029-V-04}
    \\\vcdJiraRef{LVV-6189}~{\tiny
    }
    \end{tabular} &
        & & \\
  \midrule
  \begin{tabular}{@{}l@{}}
  CA-DM-SUP-ICD-0031\\\vcdDocRef{LSE-130}~{\tiny
  }
  \end{tabular} &
    \begin{tabular}{@{}l@{}}
    \hypertarget{ca-dm-sup-icd-0031-v-03}{CA-DM-SUP-ICD-0031-V-03}
    \\\vcdJiraRef{LVV-6194}~{\tiny
    }
    \end{tabular} &
        & & \\
      \cmidrule{2-5}
      &
    \begin{tabular}{@{}l@{}}
    \hypertarget{ca-dm-sup-icd-0031-v-04}{CA-DM-SUP-ICD-0031-V-04}
    \\\vcdJiraRef{LVV-6195}~{\tiny
    }
    \end{tabular} &
        & & \\
  \midrule
  \begin{tabular}{@{}l@{}}
  CA-DM-SUP-ICD-0030\\\vcdDocRef{LSE-130}~{\tiny
  }
  \end{tabular} &
    \begin{tabular}{@{}l@{}}
    \hypertarget{ca-dm-sup-icd-0030-v-03}{CA-DM-SUP-ICD-0030-V-03}
    \\\vcdJiraRef{LVV-6200}~{\tiny
    }
    \end{tabular} &
        & & \\
      \cmidrule{2-5}
      &
    \begin{tabular}{@{}l@{}}
    \hypertarget{ca-dm-sup-icd-0030-v-04}{CA-DM-SUP-ICD-0030-V-04}
    \\\vcdJiraRef{LVV-6201}~{\tiny
    }
    \end{tabular} &
        & & \\
  \midrule
  \begin{tabular}{@{}l@{}}
  CA-DM-SUP-ICD-0008\\\vcdDocRef{LSE-130}~{\tiny
  }
  \end{tabular} &
    \begin{tabular}{@{}l@{}}
    \hypertarget{ca-dm-sup-icd-0008-v-03}{CA-DM-SUP-ICD-0008-V-03}
    \\\vcdJiraRef{LVV-6206}~{\tiny
    }
    \end{tabular} &
        & & \\
      \cmidrule{2-5}
      &
    \begin{tabular}{@{}l@{}}
    \hypertarget{ca-dm-sup-icd-0008-v-04}{CA-DM-SUP-ICD-0008-V-04}
    \\\vcdJiraRef{LVV-6207}~{\tiny
    }
    \end{tabular} &
        & & \\
  \midrule
  \begin{tabular}{@{}l@{}}
  CA-DM-SUP-ICD-0007\\\vcdDocRef{LSE-130}~{\tiny
  }
  \end{tabular} &
    \begin{tabular}{@{}l@{}}
    \hypertarget{ca-dm-sup-icd-0007-v-03}{CA-DM-SUP-ICD-0007-V-03}
    \\\vcdJiraRef{LVV-6212}~{\tiny
    }
    \end{tabular} &
        & & \\
      \cmidrule{2-5}
      &
    \begin{tabular}{@{}l@{}}
    \hypertarget{ca-dm-sup-icd-0007-v-04}{CA-DM-SUP-ICD-0007-V-04}
    \\\vcdJiraRef{LVV-6213}~{\tiny
    }
    \end{tabular} &
        & & \\
  \midrule
  \begin{tabular}{@{}l@{}}
  CA-DM-SUP-ICD-0009\\\vcdDocRef{LSE-130}~{\tiny
  }
  \end{tabular} &
    \begin{tabular}{@{}l@{}}
    \hypertarget{ca-dm-sup-icd-0009-v-03}{CA-DM-SUP-ICD-0009-V-03}
    \\\vcdJiraRef{LVV-6218}~{\tiny
    }
    \end{tabular} &
        & & \\
      \cmidrule{2-5}
      &
    \begin{tabular}{@{}l@{}}
    \hypertarget{ca-dm-sup-icd-0009-v-04}{CA-DM-SUP-ICD-0009-V-04}
    \\\vcdJiraRef{LVV-6219}~{\tiny
    }
    \end{tabular} &
        & & \\
  \midrule
  \begin{tabular}{@{}l@{}}
  CA-DM-SUP-ICD-0010\\\vcdDocRef{LSE-130}~{\tiny
  }
  \end{tabular} &
    \begin{tabular}{@{}l@{}}
    \hypertarget{ca-dm-sup-icd-0010-v-03}{CA-DM-SUP-ICD-0010-V-03}
    \\\vcdJiraRef{LVV-6224}~{\tiny
    }
    \end{tabular} &
        & & \\
      \cmidrule{2-5}
      &
    \begin{tabular}{@{}l@{}}
    \hypertarget{ca-dm-sup-icd-0010-v-04}{CA-DM-SUP-ICD-0010-V-04}
    \\\vcdJiraRef{LVV-6225}~{\tiny
    }
    \end{tabular} &
        & & \\
  \midrule
  \begin{tabular}{@{}l@{}}
  CA-DM-SUP-ICD-0020\\\vcdDocRef{LSE-130}~{\tiny
  }
  \end{tabular} &
    \begin{tabular}{@{}l@{}}
    \hypertarget{ca-dm-sup-icd-0020-v-03}{CA-DM-SUP-ICD-0020-V-03}
    \\\vcdJiraRef{LVV-6230}~{\tiny
    }
    \end{tabular} &
        & & \\
      \cmidrule{2-5}
      &
    \begin{tabular}{@{}l@{}}
    \hypertarget{ca-dm-sup-icd-0020-v-04}{CA-DM-SUP-ICD-0020-V-04}
    \\\vcdJiraRef{LVV-6231}~{\tiny
    }
    \end{tabular} &
        & & \\
  \midrule
  \begin{tabular}{@{}l@{}}
  CA-DM-SUP-ICD-0019\\\vcdDocRef{LSE-130}~{\tiny
  }
  \end{tabular} &
    \begin{tabular}{@{}l@{}}
    \hypertarget{ca-dm-sup-icd-0019-v-03}{CA-DM-SUP-ICD-0019-V-03}
    \\\vcdJiraRef{LVV-6236}~{\tiny
    }
    \end{tabular} &
        & & \\
      \cmidrule{2-5}
      &
    \begin{tabular}{@{}l@{}}
    \hypertarget{ca-dm-sup-icd-0019-v-04}{CA-DM-SUP-ICD-0019-V-04}
    \\\vcdJiraRef{LVV-6237}~{\tiny
    }
    \end{tabular} &
        & & \\
  \midrule
  \begin{tabular}{@{}l@{}}
  CA-DM-SUP-ICD-0005\\\vcdDocRef{LSE-130}~{\tiny
  }
  \end{tabular} &
    \begin{tabular}{@{}l@{}}
    \hypertarget{ca-dm-sup-icd-0005-v-03}{CA-DM-SUP-ICD-0005-V-03}
    \\\vcdJiraRef{LVV-6242}~{\tiny
    }
    \end{tabular} &
        & & \\
      \cmidrule{2-5}
      &
    \begin{tabular}{@{}l@{}}
    \hypertarget{ca-dm-sup-icd-0005-v-04}{CA-DM-SUP-ICD-0005-V-04}
    \\\vcdJiraRef{LVV-6243}~{\tiny
    }
    \end{tabular} &
        & & \\
  \midrule
  \begin{tabular}{@{}l@{}}
  CA-DM-SUP-ICD-0006\\\vcdDocRef{LSE-130}~{\tiny
  }
  \end{tabular} &
    \begin{tabular}{@{}l@{}}
    \hypertarget{ca-dm-sup-icd-0006-v-03}{CA-DM-SUP-ICD-0006-V-03}
    \\\vcdJiraRef{LVV-6248}~{\tiny
    }
    \end{tabular} &
        & & \\
      \cmidrule{2-5}
      &
    \begin{tabular}{@{}l@{}}
    \hypertarget{ca-dm-sup-icd-0006-v-04}{CA-DM-SUP-ICD-0006-V-04}
    \\\vcdJiraRef{LVV-6249}~{\tiny
    }
    \end{tabular} &
        & & \\
  \midrule
  \begin{tabular}{@{}l@{}}
  CA-DM-SUP-ICD-0002\\\vcdDocRef{LSE-130}~{\tiny
  }
  \end{tabular} &
    \begin{tabular}{@{}l@{}}
    \hypertarget{ca-dm-sup-icd-0002-v-03}{CA-DM-SUP-ICD-0002-V-03}
    \\\vcdJiraRef{LVV-6254}~{\tiny
    }
    \end{tabular} &
        & & \\
      \cmidrule{2-5}
      &
    \begin{tabular}{@{}l@{}}
    \hypertarget{ca-dm-sup-icd-0002-v-04}{CA-DM-SUP-ICD-0002-V-04}
    \\\vcdJiraRef{LVV-6255}~{\tiny
    }
    \end{tabular} &
        & & \\
  \midrule
  \begin{tabular}{@{}l@{}}
  CA-DM-SUP-ICD-0003\\\vcdDocRef{LSE-130}~{\tiny
  }
  \end{tabular} &
    \begin{tabular}{@{}l@{}}
    \hypertarget{ca-dm-sup-icd-0003-v-03}{CA-DM-SUP-ICD-0003-V-03}
    \\\vcdJiraRef{LVV-6260}~{\tiny
    }
    \end{tabular} &
        & & \\
      \cmidrule{2-5}
      &
    \begin{tabular}{@{}l@{}}
    \hypertarget{ca-dm-sup-icd-0003-v-04}{CA-DM-SUP-ICD-0003-V-04}
    \\\vcdJiraRef{LVV-6261}~{\tiny
    }
    \end{tabular} &
        & & \\
  \midrule
  \begin{tabular}{@{}l@{}}
  CA-DM-SUP-ICD-0004\\\vcdDocRef{LSE-130}~{\tiny
  }
  \end{tabular} &
    \begin{tabular}{@{}l@{}}
    \hypertarget{ca-dm-sup-icd-0004-v-03}{CA-DM-SUP-ICD-0004-V-03}
    \\\vcdJiraRef{LVV-6266}~{\tiny
    }
    \end{tabular} &
        & & \\
      \cmidrule{2-5}
      &
    \begin{tabular}{@{}l@{}}
    \hypertarget{ca-dm-sup-icd-0004-v-04}{CA-DM-SUP-ICD-0004-V-04}
    \\\vcdJiraRef{LVV-6267}~{\tiny
    }
    \end{tabular} &
        & & \\
  \midrule
  \begin{tabular}{@{}l@{}}
  CA-DM-SUP-ICD-0016\\\vcdDocRef{LSE-130}~{\tiny
  }
  \end{tabular} &
    \begin{tabular}{@{}l@{}}
    \hypertarget{ca-dm-sup-icd-0016-v-03}{CA-DM-SUP-ICD-0016-V-03}
    \\\vcdJiraRef{LVV-6272}~{\tiny
    }
    \end{tabular} &
        & & \\
      \cmidrule{2-5}
      &
    \begin{tabular}{@{}l@{}}
    \hypertarget{ca-dm-sup-icd-0016-v-04}{CA-DM-SUP-ICD-0016-V-04}
    \\\vcdJiraRef{LVV-6273}~{\tiny
    }
    \end{tabular} &
        & & \\
  \midrule
  \begin{tabular}{@{}l@{}}
  CA-DM-SUP-ICD-0015\\\vcdDocRef{LSE-130}~{\tiny
  }
  \end{tabular} &
    \begin{tabular}{@{}l@{}}
    \hypertarget{ca-dm-sup-icd-0015-v-03}{CA-DM-SUP-ICD-0015-V-03}
    \\\vcdJiraRef{LVV-6278}~{\tiny
    }
    \end{tabular} &
        & & \\
      \cmidrule{2-5}
      &
    \begin{tabular}{@{}l@{}}
    \hypertarget{ca-dm-sup-icd-0015-v-04}{CA-DM-SUP-ICD-0015-V-04}
    \\\vcdJiraRef{LVV-6279}~{\tiny
    }
    \end{tabular} &
        & & \\
  \midrule
  \begin{tabular}{@{}l@{}}
  CA-DM-SUP-ICD-0017\\\vcdDocRef{LSE-130}~{\tiny
  }
  \end{tabular} &
    \begin{tabular}{@{}l@{}}
    \hypertarget{ca-dm-sup-icd-0017-v-03}{CA-DM-SUP-ICD-0017-V-03}
    \\\vcdJiraRef{LVV-6284}~{\tiny
    }
    \end{tabular} &
        & & \\
      \cmidrule{2-5}
      &
    \begin{tabular}{@{}l@{}}
    \hypertarget{ca-dm-sup-icd-0017-v-04}{CA-DM-SUP-ICD-0017-V-04}
    \\\vcdJiraRef{LVV-6285}~{\tiny
    }
    \end{tabular} &
        & & \\
  \midrule
  \begin{tabular}{@{}l@{}}
  CA-DM-SUP-ICD-0014\\\vcdDocRef{LSE-130}~{\tiny
  }
  \end{tabular} &
    \begin{tabular}{@{}l@{}}
    \hypertarget{ca-dm-sup-icd-0014-v-03}{CA-DM-SUP-ICD-0014-V-03}
    \\\vcdJiraRef{LVV-6290}~{\tiny
    }
    \end{tabular} &
        & & \\
      \cmidrule{2-5}
      &
    \begin{tabular}{@{}l@{}}
    \hypertarget{ca-dm-sup-icd-0014-v-04}{CA-DM-SUP-ICD-0014-V-04}
    \\\vcdJiraRef{LVV-6291}~{\tiny
    }
    \end{tabular} &
        & & \\
  \midrule
  \begin{tabular}{@{}l@{}}
  CA-DM-SUP-ICD-0013\\\vcdDocRef{LSE-130}~{\tiny
  }
  \end{tabular} &
    \begin{tabular}{@{}l@{}}
    \hypertarget{ca-dm-sup-icd-0013-v-03}{CA-DM-SUP-ICD-0013-V-03}
    \\\vcdJiraRef{LVV-6296}~{\tiny
    }
    \end{tabular} &
        & & \\
      \cmidrule{2-5}
      &
    \begin{tabular}{@{}l@{}}
    \hypertarget{ca-dm-sup-icd-0013-v-04}{CA-DM-SUP-ICD-0013-V-04}
    \\\vcdJiraRef{LVV-6297}~{\tiny
    }
    \end{tabular} &
        & & \\
  \midrule
  \begin{tabular}{@{}l@{}}
  CA-DM-SUP-ICD-0011\\\vcdDocRef{LSE-130}~{\tiny
  }
  \end{tabular} &
    \begin{tabular}{@{}l@{}}
    \hypertarget{ca-dm-sup-icd-0011-v-03}{CA-DM-SUP-ICD-0011-V-03}
    \\\vcdJiraRef{LVV-6302}~{\tiny
    }
    \end{tabular} &
        & & \\
      \cmidrule{2-5}
      &
    \begin{tabular}{@{}l@{}}
    \hypertarget{ca-dm-sup-icd-0011-v-04}{CA-DM-SUP-ICD-0011-V-04}
    \\\vcdJiraRef{LVV-6303}~{\tiny
    }
    \end{tabular} &
        & & \\
  \midrule
  \begin{tabular}{@{}l@{}}
  CA-DM-SUP-ICD-0012\\\vcdDocRef{LSE-130}~{\tiny
  }
  \end{tabular} &
    \begin{tabular}{@{}l@{}}
    \hypertarget{ca-dm-sup-icd-0012-v-03}{CA-DM-SUP-ICD-0012-V-03}
    \\\vcdJiraRef{LVV-6308}~{\tiny
    }
    \end{tabular} &
        & & \\
      \cmidrule{2-5}
      &
    \begin{tabular}{@{}l@{}}
    \hypertarget{ca-dm-sup-icd-0012-v-04}{CA-DM-SUP-ICD-0012-V-04}
    \\\vcdJiraRef{LVV-6309}~{\tiny
    }
    \end{tabular} &
        & & \\
  \midrule
  \begin{tabular}{@{}l@{}}
  CA-DM-SUP-ICD-0018\\\vcdDocRef{LSE-130}~{\tiny
  }
  \end{tabular} &
    \begin{tabular}{@{}l@{}}
    \hypertarget{ca-dm-sup-icd-0018-v-03}{CA-DM-SUP-ICD-0018-V-03}
    \\\vcdJiraRef{LVV-6314}~{\tiny
    }
    \end{tabular} &
        & & \\
      \cmidrule{2-5}
      &
    \begin{tabular}{@{}l@{}}
    \hypertarget{ca-dm-sup-icd-0018-v-04}{CA-DM-SUP-ICD-0018-V-04}
    \\\vcdJiraRef{LVV-6315}~{\tiny
    }
    \end{tabular} &
        & & \\
  \midrule
  \begin{tabular}{@{}l@{}}
  CA-DM-SUP-ICD-0001\\\vcdDocRef{LSE-130}~{\tiny
  }
  \end{tabular} &
    \begin{tabular}{@{}l@{}}
    \hypertarget{ca-dm-sup-icd-0001-v-03}{CA-DM-SUP-ICD-0001-V-03}
    \\\vcdJiraRef{LVV-6320}~{\tiny
    }
    \end{tabular} &
        & & \\
      \cmidrule{2-5}
      &
    \begin{tabular}{@{}l@{}}
    \hypertarget{ca-dm-sup-icd-0001-v-04}{CA-DM-SUP-ICD-0001-V-04}
    \\\vcdJiraRef{LVV-6321}~{\tiny
    }
    \end{tabular} &
        & & \\
  \midrule
\label{tab:dmvcd}
\end{longtable}
}

\subsection{LSE-131 Requirements Coverage}

\setlength\LTleft{-0.25in}
\setlength\LTright{-0.5in}
{\small
\begin{longtable}{lllll}
\caption{ DM LSE-131 Requirements.} \\
\toprule
\textbf{Requirement} & \textbf{Verification Element} & \textbf{Test Case} & \textbf{Last Run} & \textbf{Test Status} \\
\toprule
\endhead
  \begin{tabular}{@{}l@{}}
  EP-DM-CON-ICD-0004\\\vcdDocRef{LSE-131}~{\tiny
  }
  \end{tabular} &
    \begin{tabular}{@{}l@{}}
    \hypertarget{ep-dm-con-icd-0004-v-01}{EP-DM-CON-ICD-0004-V-01}
    \\\vcdJiraRef{LVV-6324}~{\tiny
    }
    \end{tabular} &
        & & \\
      \cmidrule{2-5}
      &
    \begin{tabular}{@{}l@{}}
    \hypertarget{ep-dm-con-icd-0004-v-02}{EP-DM-CON-ICD-0004-V-02}
    \\\vcdJiraRef{LVV-6325}~{\tiny
    }
    \end{tabular} &
        & & \\
  \midrule
  \begin{tabular}{@{}l@{}}
  EP-DM-CON-ICD-0021\\\vcdDocRef{LSE-131}~{\tiny
  }
  \end{tabular} &
    \begin{tabular}{@{}l@{}}
    \hypertarget{ep-dm-con-icd-0021-v-01}{EP-DM-CON-ICD-0021-V-01}
    \\\vcdJiraRef{LVV-6330}~{\tiny
    }
    \end{tabular} &
        & & \\
      \cmidrule{2-5}
      &
    \begin{tabular}{@{}l@{}}
    \hypertarget{ep-dm-con-icd-0021-v-02}{EP-DM-CON-ICD-0021-V-02}
    \\\vcdJiraRef{LVV-6331}~{\tiny
    }
    \end{tabular} &
        & & \\
  \midrule
  \begin{tabular}{@{}l@{}}
  EP-DM-CON-ICD-0009\\\vcdDocRef{LSE-131}~{\tiny
  }
  \end{tabular} &
    \begin{tabular}{@{}l@{}}
    \hypertarget{ep-dm-con-icd-0009-v-01}{EP-DM-CON-ICD-0009-V-01}
    \\\vcdJiraRef{LVV-6342}~{\tiny
    }
    \end{tabular} &
        & & \\
      \cmidrule{2-5}
      &
    \begin{tabular}{@{}l@{}}
    \hypertarget{ep-dm-con-icd-0009-v-02}{EP-DM-CON-ICD-0009-V-02}
    \\\vcdJiraRef{LVV-6343}~{\tiny
    }
    \end{tabular} &
        & & \\
  \midrule
  \begin{tabular}{@{}l@{}}
  EP-DM-CON-ICD-0034\\\vcdDocRef{LSE-131}~{\tiny
  }
  \end{tabular} &
    \begin{tabular}{@{}l@{}}
    \hypertarget{ep-dm-con-icd-0034-v-01}{EP-DM-CON-ICD-0034-V-01}
    \\\vcdJiraRef{LVV-6348}~{\tiny
    }
    \end{tabular} &
        & & \\
      \cmidrule{2-5}
      &
    \begin{tabular}{@{}l@{}}
    \hypertarget{ep-dm-con-icd-0034-v-02}{EP-DM-CON-ICD-0034-V-02}
    \\\vcdJiraRef{LVV-6349}~{\tiny
    }
    \end{tabular} &
        & & \\
  \midrule
  \begin{tabular}{@{}l@{}}
  EP-DM-CON-ICD-0031\\\vcdDocRef{LSE-131}~{\tiny
  }
  \end{tabular} &
    \begin{tabular}{@{}l@{}}
    \hypertarget{ep-dm-con-icd-0031-v-01}{EP-DM-CON-ICD-0031-V-01}
    \\\vcdJiraRef{LVV-6360}~{\tiny
    }
    \end{tabular} &
        & & \\
      \cmidrule{2-5}
      &
    \begin{tabular}{@{}l@{}}
    \hypertarget{ep-dm-con-icd-0031-v-02}{EP-DM-CON-ICD-0031-V-02}
    \\\vcdJiraRef{LVV-6361}~{\tiny
    }
    \end{tabular} &
        & & \\
  \midrule
  \begin{tabular}{@{}l@{}}
  EP-DM-CON-ICD-0019\\\vcdDocRef{LSE-131}~{\tiny
  }
  \end{tabular} &
    \begin{tabular}{@{}l@{}}
    \hypertarget{ep-dm-con-icd-0019-v-01}{EP-DM-CON-ICD-0019-V-01}
    \\\vcdJiraRef{LVV-6372}~{\tiny
    }
    \end{tabular} &
        & & \\
      \cmidrule{2-5}
      &
    \begin{tabular}{@{}l@{}}
    \hypertarget{ep-dm-con-icd-0019-v-02}{EP-DM-CON-ICD-0019-V-02}
    \\\vcdJiraRef{LVV-6373}~{\tiny
    }
    \end{tabular} &
        & & \\
  \midrule
  \begin{tabular}{@{}l@{}}
  EP-DM-CON-ICD-0002\\\vcdDocRef{LSE-131}~{\tiny
  }
  \end{tabular} &
    \begin{tabular}{@{}l@{}}
    \hypertarget{ep-dm-con-icd-0002-v-03}{EP-DM-CON-ICD-0002-V-03}
    \\\vcdJiraRef{LVV-6378}~{\tiny
    }
    \end{tabular} &
        & & \\
      \cmidrule{2-5}
      &
    \begin{tabular}{@{}l@{}}
    \hypertarget{ep-dm-con-icd-0002-v-04}{EP-DM-CON-ICD-0002-V-04}
    \\\vcdJiraRef{LVV-6379}~{\tiny
    }
    \end{tabular} &
        & & \\
  \midrule
  \begin{tabular}{@{}l@{}}
  EP-DM-CON-ICD-0033\\\vcdDocRef{LSE-131}~{\tiny
  }
  \end{tabular} &
    \begin{tabular}{@{}l@{}}
    \hypertarget{ep-dm-con-icd-0033-v-01}{EP-DM-CON-ICD-0033-V-01}
    \\\vcdJiraRef{LVV-6384}~{\tiny
    }
    \end{tabular} &
        & & \\
      \cmidrule{2-5}
      &
    \begin{tabular}{@{}l@{}}
    \hypertarget{ep-dm-con-icd-0033-v-02}{EP-DM-CON-ICD-0033-V-02}
    \\\vcdJiraRef{LVV-6385}~{\tiny
    }
    \end{tabular} &
        & & \\
  \midrule
  \begin{tabular}{@{}l@{}}
  EP-DM-CON-ICD-0032\\\vcdDocRef{LSE-131}~{\tiny
  }
  \end{tabular} &
    \begin{tabular}{@{}l@{}}
    \hypertarget{ep-dm-con-icd-0032-v-01}{EP-DM-CON-ICD-0032-V-01}
    \\\vcdJiraRef{LVV-6390}~{\tiny
    }
    \end{tabular} &
        & & \\
      \cmidrule{2-5}
      &
    \begin{tabular}{@{}l@{}}
    \hypertarget{ep-dm-con-icd-0032-v-02}{EP-DM-CON-ICD-0032-V-02}
    \\\vcdJiraRef{LVV-6391}~{\tiny
    }
    \end{tabular} &
        & & \\
  \midrule
  \begin{tabular}{@{}l@{}}
  EP-DM-CON-ICD-0020\\\vcdDocRef{LSE-131}~{\tiny
  }
  \end{tabular} &
    \begin{tabular}{@{}l@{}}
    \hypertarget{ep-dm-con-icd-0020-v-03}{EP-DM-CON-ICD-0020-V-03}
    \\\vcdJiraRef{LVV-6402}~{\tiny
    }
    \end{tabular} &
        & & \\
      \cmidrule{2-5}
      &
    \begin{tabular}{@{}l@{}}
    \hypertarget{ep-dm-con-icd-0020-v-04}{EP-DM-CON-ICD-0020-V-04}
    \\\vcdJiraRef{LVV-6403}~{\tiny
    }
    \end{tabular} &
        & & \\
  \midrule
  \begin{tabular}{@{}l@{}}
  EP-DM-CON-ICD-0036\\\vcdDocRef{LSE-131}~{\tiny
  }
  \end{tabular} &
    \begin{tabular}{@{}l@{}}
    \hypertarget{ep-dm-con-icd-0036-v-01}{EP-DM-CON-ICD-0036-V-01}
    \\\vcdJiraRef{LVV-6751}~{\tiny
    }
    \end{tabular} &
        & & \\
      \cmidrule{2-5}
      &
    \begin{tabular}{@{}l@{}}
    \hypertarget{ep-dm-con-icd-0036-v-02}{EP-DM-CON-ICD-0036-V-02}
    \\\vcdJiraRef{LVV-6752}~{\tiny
    }
    \end{tabular} &
        & & \\
  \midrule
  \begin{tabular}{@{}l@{}}
  EP-DM-CON-ICD-0035\\\vcdDocRef{LSE-131}~{\tiny
  }
  \end{tabular} &
    \begin{tabular}{@{}l@{}}
    \hypertarget{ep-dm-con-icd-0035-v-01}{EP-DM-CON-ICD-0035-V-01}
    \\\vcdJiraRef{LVV-6757}~{\tiny
    }
    \end{tabular} &
        & & \\
      \cmidrule{2-5}
      &
    \begin{tabular}{@{}l@{}}
    \hypertarget{ep-dm-con-icd-0035-v-02}{EP-DM-CON-ICD-0035-V-02}
    \\\vcdJiraRef{LVV-6758}~{\tiny
    }
    \end{tabular} &
        & & \\
  \midrule
  \begin{tabular}{@{}l@{}}
  EP-DM-CON-ICD-0037\\\vcdDocRef{LSE-131}~{\tiny
  }
  \end{tabular} &
    \begin{tabular}{@{}l@{}}
    \hypertarget{ep-dm-con-icd-0037-v-01}{EP-DM-CON-ICD-0037-V-01}
    \\\vcdJiraRef{LVV-6763}~{\tiny
    }
    \end{tabular} &
        & & \\
      \cmidrule{2-5}
      &
    \begin{tabular}{@{}l@{}}
    \hypertarget{ep-dm-con-icd-0037-v-02}{EP-DM-CON-ICD-0037-V-02}
    \\\vcdJiraRef{LVV-6764}~{\tiny
    }
    \end{tabular} &
        & & \\
  \midrule
\label{tab:dmvcd}
\end{longtable}
}

\subsection{LSE-140 Requirements Coverage}

\setlength\LTleft{-0.25in}
\setlength\LTright{-0.5in}
{\small
\begin{longtable}{lllll}
\caption{ DM LSE-140 Requirements.} \\
\toprule
\textbf{Requirement} & \textbf{Verification Element} & \textbf{Test Case} & \textbf{Last Run} & \textbf{Test Status} \\
\toprule
\endhead
  \begin{tabular}{@{}l@{}}
  DM-TS-AUX-ICD-0020\\\vcdDocRef{LSE-140}~{\tiny
  }
  \end{tabular} &
    \begin{tabular}{@{}l@{}}
    \hypertarget{dm-ts-aux-icd-0020-v-01}{DM-TS-AUX-ICD-0020-V-01}
    \\\vcdJiraRef{LVV-6420}~{\tiny
    }
    \end{tabular} &
        & & \\
      \cmidrule{2-5}
      &
    \begin{tabular}{@{}l@{}}
    \hypertarget{dm-ts-aux-icd-0020-v-02}{DM-TS-AUX-ICD-0020-V-02}
    \\\vcdJiraRef{LVV-6421}~{\tiny
    }
    \end{tabular} &
        & & \\
  \midrule
  \begin{tabular}{@{}l@{}}
  DM-TS-AUX-ICD-0029\\\vcdDocRef{LSE-140}~{\tiny
  }
  \end{tabular} &
    \begin{tabular}{@{}l@{}}
    \hypertarget{dm-ts-aux-icd-0029-v-01}{DM-TS-AUX-ICD-0029-V-01}
    \\\vcdJiraRef{LVV-6426}~{\tiny
    }
    \end{tabular} &
        & & \\
      \cmidrule{2-5}
      &
    \begin{tabular}{@{}l@{}}
    \hypertarget{dm-ts-aux-icd-0029-v-02}{DM-TS-AUX-ICD-0029-V-02}
    \\\vcdJiraRef{LVV-6427}~{\tiny
    }
    \end{tabular} &
        & & \\
  \midrule
  \begin{tabular}{@{}l@{}}
  DM-TS-AUX-ICD-0027\\\vcdDocRef{LSE-140}~{\tiny
  }
  \end{tabular} &
    \begin{tabular}{@{}l@{}}
    \hypertarget{dm-ts-aux-icd-0027-v-01}{DM-TS-AUX-ICD-0027-V-01}
    \\\vcdJiraRef{LVV-6432}~{\tiny
    }
    \end{tabular} &
        & & \\
      \cmidrule{2-5}
      &
    \begin{tabular}{@{}l@{}}
    \hypertarget{dm-ts-aux-icd-0027-v-02}{DM-TS-AUX-ICD-0027-V-02}
    \\\vcdJiraRef{LVV-6433}~{\tiny
    }
    \end{tabular} &
        & & \\
  \midrule
  \begin{tabular}{@{}l@{}}
  DM-TS-AUX-ICD-0025\\\vcdDocRef{LSE-140}~{\tiny
  }
  \end{tabular} &
    \begin{tabular}{@{}l@{}}
    \hypertarget{dm-ts-aux-icd-0025-v-01}{DM-TS-AUX-ICD-0025-V-01}
    \\\vcdJiraRef{LVV-6456}~{\tiny
    }
    \end{tabular} &
        & & \\
      \cmidrule{2-5}
      &
    \begin{tabular}{@{}l@{}}
    \hypertarget{dm-ts-aux-icd-0025-v-02}{DM-TS-AUX-ICD-0025-V-02}
    \\\vcdJiraRef{LVV-6457}~{\tiny
    }
    \end{tabular} &
        & & \\
  \midrule
  \begin{tabular}{@{}l@{}}
  DM-TS-AUX-ICD-0026\\\vcdDocRef{LSE-140}~{\tiny
  }
  \end{tabular} &
    \begin{tabular}{@{}l@{}}
    \hypertarget{dm-ts-aux-icd-0026-v-01}{DM-TS-AUX-ICD-0026-V-01}
    \\\vcdJiraRef{LVV-6462}~{\tiny
    }
    \end{tabular} &
        & & \\
      \cmidrule{2-5}
      &
    \begin{tabular}{@{}l@{}}
    \hypertarget{dm-ts-aux-icd-0026-v-02}{DM-TS-AUX-ICD-0026-V-02}
    \\\vcdJiraRef{LVV-6463}~{\tiny
    }
    \end{tabular} &
        & & \\
  \midrule
  \begin{tabular}{@{}l@{}}
  DM-TS-AUX-ICD-0024\\\vcdDocRef{LSE-140}~{\tiny
  }
  \end{tabular} &
    \begin{tabular}{@{}l@{}}
    \hypertarget{dm-ts-aux-icd-0024-v-01}{DM-TS-AUX-ICD-0024-V-01}
    \\\vcdJiraRef{LVV-6468}~{\tiny
    }
    \end{tabular} &
        & & \\
      \cmidrule{2-5}
      &
    \begin{tabular}{@{}l@{}}
    \hypertarget{dm-ts-aux-icd-0024-v-02}{DM-TS-AUX-ICD-0024-V-02}
    \\\vcdJiraRef{LVV-6469}~{\tiny
    }
    \end{tabular} &
        & & \\
  \midrule
  \begin{tabular}{@{}l@{}}
  DM-TS-AUX-ICD-0037\\\vcdDocRef{LSE-140}~{\tiny
  }
  \end{tabular} &
    \begin{tabular}{@{}l@{}}
    \hypertarget{dm-ts-aux-icd-0037-v-01}{DM-TS-AUX-ICD-0037-V-01}
    \\\vcdJiraRef{LVV-6474}~{\tiny
    }
    \end{tabular} &
        & & \\
      \cmidrule{2-5}
      &
    \begin{tabular}{@{}l@{}}
    \hypertarget{dm-ts-aux-icd-0037-v-02}{DM-TS-AUX-ICD-0037-V-02}
    \\\vcdJiraRef{LVV-6475}~{\tiny
    }
    \end{tabular} &
        & & \\
  \midrule
  \begin{tabular}{@{}l@{}}
  DM-TS-AUX-ICD-0002\\\vcdDocRef{LSE-140}~{\tiny
  }
  \end{tabular} &
    \begin{tabular}{@{}l@{}}
    \hypertarget{dm-ts-aux-icd-0002-v-01}{DM-TS-AUX-ICD-0002-V-01}
    \\\vcdJiraRef{LVV-6480}~{\tiny
    }
    \end{tabular} &
        & & \\
      \cmidrule{2-5}
      &
    \begin{tabular}{@{}l@{}}
    \hypertarget{dm-ts-aux-icd-0002-v-02}{DM-TS-AUX-ICD-0002-V-02}
    \\\vcdJiraRef{LVV-6481}~{\tiny
    }
    \end{tabular} &
        & & \\
  \midrule
  \begin{tabular}{@{}l@{}}
  DM-TS-AUX-ICD-0001\\\vcdDocRef{LSE-140}~{\tiny
  }
  \end{tabular} &
    \begin{tabular}{@{}l@{}}
    \hypertarget{dm-ts-aux-icd-0001-v-01}{DM-TS-AUX-ICD-0001-V-01}
    \\\vcdJiraRef{LVV-6486}~{\tiny
    }
    \end{tabular} &
        & & \\
      \cmidrule{2-5}
      &
    \begin{tabular}{@{}l@{}}
    \hypertarget{dm-ts-aux-icd-0001-v-02}{DM-TS-AUX-ICD-0001-V-02}
    \\\vcdJiraRef{LVV-6487}~{\tiny
    }
    \end{tabular} &
        & & \\
  \midrule
  \begin{tabular}{@{}l@{}}
  DM-TS-AUX-ICD-0007\\\vcdDocRef{LSE-140}~{\tiny
  }
  \end{tabular} &
    \begin{tabular}{@{}l@{}}
    \hypertarget{dm-ts-aux-icd-0007-v-01}{DM-TS-AUX-ICD-0007-V-01}
    \\\vcdJiraRef{LVV-6492}~{\tiny
    }
    \end{tabular} &
        & & \\
      \cmidrule{2-5}
      &
    \begin{tabular}{@{}l@{}}
    \hypertarget{dm-ts-aux-icd-0007-v-02}{DM-TS-AUX-ICD-0007-V-02}
    \\\vcdJiraRef{LVV-6493}~{\tiny
    }
    \end{tabular} &
        & & \\
  \midrule
  \begin{tabular}{@{}l@{}}
  DM-TS-AUX-ICD-0008\\\vcdDocRef{LSE-140}~{\tiny
  }
  \end{tabular} &
    \begin{tabular}{@{}l@{}}
    \hypertarget{dm-ts-aux-icd-0008-v-01}{DM-TS-AUX-ICD-0008-V-01}
    \\\vcdJiraRef{LVV-6498}~{\tiny
    }
    \end{tabular} &
        & & \\
      \cmidrule{2-5}
      &
    \begin{tabular}{@{}l@{}}
    \hypertarget{dm-ts-aux-icd-0008-v-02}{DM-TS-AUX-ICD-0008-V-02}
    \\\vcdJiraRef{LVV-6499}~{\tiny
    }
    \end{tabular} &
        & & \\
  \midrule
  \begin{tabular}{@{}l@{}}
  DM-TS-AUX-ICD-0004\\\vcdDocRef{LSE-140}~{\tiny
  }
  \end{tabular} &
    \begin{tabular}{@{}l@{}}
    \hypertarget{dm-ts-aux-icd-0004-v-01}{DM-TS-AUX-ICD-0004-V-01}
    \\\vcdJiraRef{LVV-6528}~{\tiny
    }
    \end{tabular} &
        & & \\
      \cmidrule{2-5}
      &
    \begin{tabular}{@{}l@{}}
    \hypertarget{dm-ts-aux-icd-0004-v-02}{DM-TS-AUX-ICD-0004-V-02}
    \\\vcdJiraRef{LVV-6529}~{\tiny
    }
    \end{tabular} &
        & & \\
  \midrule
  \begin{tabular}{@{}l@{}}
  DM-TS-AUX-ICD-0003\\\vcdDocRef{LSE-140}~{\tiny
  }
  \end{tabular} &
    \begin{tabular}{@{}l@{}}
    \hypertarget{dm-ts-aux-icd-0003-v-01}{DM-TS-AUX-ICD-0003-V-01}
    \\\vcdJiraRef{LVV-6534}~{\tiny
    }
    \end{tabular} &
        & & \\
      \cmidrule{2-5}
      &
    \begin{tabular}{@{}l@{}}
    \hypertarget{dm-ts-aux-icd-0003-v-02}{DM-TS-AUX-ICD-0003-V-02}
    \\\vcdJiraRef{LVV-6535}~{\tiny
    }
    \end{tabular} &
        & & \\
  \midrule
  \begin{tabular}{@{}l@{}}
  DM-TS-AUX-ICD-0034\\\vcdDocRef{LSE-140}~{\tiny
  }
  \end{tabular} &
    \begin{tabular}{@{}l@{}}
    \hypertarget{dm-ts-aux-icd-0034-v-01}{DM-TS-AUX-ICD-0034-V-01}
    \\\vcdJiraRef{LVV-6540}~{\tiny
    }
    \end{tabular} &
        & & \\
      \cmidrule{2-5}
      &
    \begin{tabular}{@{}l@{}}
    \hypertarget{dm-ts-aux-icd-0034-v-02}{DM-TS-AUX-ICD-0034-V-02}
    \\\vcdJiraRef{LVV-6541}~{\tiny
    }
    \end{tabular} &
        & & \\
  \midrule
  \begin{tabular}{@{}l@{}}
  DM-TS-AUX-ICD-0036\\\vcdDocRef{LSE-140}~{\tiny
  }
  \end{tabular} &
    \begin{tabular}{@{}l@{}}
    \hypertarget{dm-ts-aux-icd-0036-v-01}{DM-TS-AUX-ICD-0036-V-01}
    \\\vcdJiraRef{LVV-6546}~{\tiny
    }
    \end{tabular} &
        & & \\
      \cmidrule{2-5}
      &
    \begin{tabular}{@{}l@{}}
    \hypertarget{dm-ts-aux-icd-0036-v-02}{DM-TS-AUX-ICD-0036-V-02}
    \\\vcdJiraRef{LVV-6547}~{\tiny
    }
    \end{tabular} &
        & & \\
  \midrule
  \begin{tabular}{@{}l@{}}
  DM-TS-AUX-ICD-0019\\\vcdDocRef{LSE-140}~{\tiny
  }
  \end{tabular} &
    \begin{tabular}{@{}l@{}}
    \hypertarget{dm-ts-aux-icd-0019-v-01}{DM-TS-AUX-ICD-0019-V-01}
    \\\vcdJiraRef{LVV-6552}~{\tiny
    }
    \end{tabular} &
        & & \\
      \cmidrule{2-5}
      &
    \begin{tabular}{@{}l@{}}
    \hypertarget{dm-ts-aux-icd-0019-v-02}{DM-TS-AUX-ICD-0019-V-02}
    \\\vcdJiraRef{LVV-6553}~{\tiny
    }
    \end{tabular} &
        & & \\
  \midrule
  \begin{tabular}{@{}l@{}}
  DM-TS-AUX-ICD-0018\\\vcdDocRef{LSE-140}~{\tiny
  }
  \end{tabular} &
    \begin{tabular}{@{}l@{}}
    \hypertarget{dm-ts-aux-icd-0018-v-01}{DM-TS-AUX-ICD-0018-V-01}
    \\\vcdJiraRef{LVV-6558}~{\tiny
    }
    \end{tabular} &
        & & \\
      \cmidrule{2-5}
      &
    \begin{tabular}{@{}l@{}}
    \hypertarget{dm-ts-aux-icd-0018-v-02}{DM-TS-AUX-ICD-0018-V-02}
    \\\vcdJiraRef{LVV-6559}~{\tiny
    }
    \end{tabular} &
        & & \\
  \midrule
  \begin{tabular}{@{}l@{}}
  DM-TS-AUX-ICD-0014\\\vcdDocRef{LSE-140}~{\tiny
  }
  \end{tabular} &
    \begin{tabular}{@{}l@{}}
    \hypertarget{dm-ts-aux-icd-0014-v-01}{DM-TS-AUX-ICD-0014-V-01}
    \\\vcdJiraRef{LVV-6564}~{\tiny
    }
    \end{tabular} &
        & & \\
      \cmidrule{2-5}
      &
    \begin{tabular}{@{}l@{}}
    \hypertarget{dm-ts-aux-icd-0014-v-02}{DM-TS-AUX-ICD-0014-V-02}
    \\\vcdJiraRef{LVV-6565}~{\tiny
    }
    \end{tabular} &
        & & \\
  \midrule
  \begin{tabular}{@{}l@{}}
  DM-TS-AUX-ICD-0012\\\vcdDocRef{LSE-140}~{\tiny
  }
  \end{tabular} &
    \begin{tabular}{@{}l@{}}
    \hypertarget{dm-ts-aux-icd-0012-v-01}{DM-TS-AUX-ICD-0012-V-01}
    \\\vcdJiraRef{LVV-6570}~{\tiny
    }
    \end{tabular} &
        & & \\
      \cmidrule{2-5}
      &
    \begin{tabular}{@{}l@{}}
    \hypertarget{dm-ts-aux-icd-0012-v-02}{DM-TS-AUX-ICD-0012-V-02}
    \\\vcdJiraRef{LVV-6571}~{\tiny
    }
    \end{tabular} &
        & & \\
  \midrule
  \begin{tabular}{@{}l@{}}
  DM-TS-AUX-ICD-0028\\\vcdDocRef{LSE-140}~{\tiny
  }
  \end{tabular} &
    \begin{tabular}{@{}l@{}}
    \hypertarget{dm-ts-aux-icd-0028-v-01}{DM-TS-AUX-ICD-0028-V-01}
    \\\vcdJiraRef{LVV-6576}~{\tiny
    }
    \end{tabular} &
        & & \\
      \cmidrule{2-5}
      &
    \begin{tabular}{@{}l@{}}
    \hypertarget{dm-ts-aux-icd-0028-v-02}{DM-TS-AUX-ICD-0028-V-02}
    \\\vcdJiraRef{LVV-6577}~{\tiny
    }
    \end{tabular} &
        & & \\
  \midrule
  \begin{tabular}{@{}l@{}}
  DM-TS-AUX-ICD-0035\\\vcdDocRef{LSE-140}~{\tiny
  }
  \end{tabular} &
    \begin{tabular}{@{}l@{}}
    \hypertarget{dm-ts-aux-icd-0035-v-01}{DM-TS-AUX-ICD-0035-V-01}
    \\\vcdJiraRef{LVV-6594}~{\tiny
    }
    \end{tabular} &
        & & \\
      \cmidrule{2-5}
      &
    \begin{tabular}{@{}l@{}}
    \hypertarget{dm-ts-aux-icd-0035-v-02}{DM-TS-AUX-ICD-0035-V-02}
    \\\vcdJiraRef{LVV-6595}~{\tiny
    }
    \end{tabular} &
        & & \\
  \midrule
  \begin{tabular}{@{}l@{}}
  DM-TS-AUX-ICD-0033\\\vcdDocRef{LSE-140}~{\tiny
  }
  \end{tabular} &
    \begin{tabular}{@{}l@{}}
    \hypertarget{dm-ts-aux-icd-0033-v-01}{DM-TS-AUX-ICD-0033-V-01}
    \\\vcdJiraRef{LVV-6600}~{\tiny
    }
    \end{tabular} &
        & & \\
      \cmidrule{2-5}
      &
    \begin{tabular}{@{}l@{}}
    \hypertarget{dm-ts-aux-icd-0033-v-02}{DM-TS-AUX-ICD-0033-V-02}
    \\\vcdJiraRef{LVV-6601}~{\tiny
    }
    \end{tabular} &
        & & \\
  \midrule
  \begin{tabular}{@{}l@{}}
  DM-TS-AUX-ICD-0032\\\vcdDocRef{LSE-140}~{\tiny
  }
  \end{tabular} &
    \begin{tabular}{@{}l@{}}
    \hypertarget{dm-ts-aux-icd-0032-v-01}{DM-TS-AUX-ICD-0032-V-01}
    \\\vcdJiraRef{LVV-6606}~{\tiny
    }
    \end{tabular} &
        & & \\
      \cmidrule{2-5}
      &
    \begin{tabular}{@{}l@{}}
    \hypertarget{dm-ts-aux-icd-0032-v-02}{DM-TS-AUX-ICD-0032-V-02}
    \\\vcdJiraRef{LVV-6607}~{\tiny
    }
    \end{tabular} &
        & & \\
  \midrule
\label{tab:dmvcd}
\end{longtable}
}

\subsection{LSE-70 Requirements Coverage}

\setlength\LTleft{-0.25in}
\setlength\LTright{-0.5in}
{\small
\begin{longtable}{lllll}
\caption{ DM LSE-70 Requirements.} \\
\toprule
\textbf{Requirement} & \textbf{Verification Element} & \textbf{Test Case} & \textbf{Last Run} & \textbf{Test Status} \\
\toprule
\endhead
  \begin{tabular}{@{}l@{}}
  SYS-ALL-COM-ICD-0047\\\vcdDocRef{LSE-70}~{\tiny
  }
  \end{tabular} &
    \begin{tabular}{@{}l@{}}
    \hypertarget{sys-all-com-icd-0047-v-06}{SYS-ALL-COM-ICD-0047-V-06}
    \\\vcdJiraRef{LVV-6771}~{\tiny
    }
    \end{tabular} &
        & & \\
      \cmidrule{2-5}
      &
    \begin{tabular}{@{}l@{}}
    \hypertarget{sys-all-com-icd-0047-v-07}{SYS-ALL-COM-ICD-0047-V-07}
    \\\vcdJiraRef{LVV-6772}~{\tiny
    }
    \end{tabular} &
        & & \\
  \midrule
  \begin{tabular}{@{}l@{}}
  SYS-ALL-COM-ICD-0048\\\vcdDocRef{LSE-70}~{\tiny
  }
  \end{tabular} &
    \begin{tabular}{@{}l@{}}
    \hypertarget{sys-all-com-icd-0048-v-06}{SYS-ALL-COM-ICD-0048-V-06}
    \\\vcdJiraRef{LVV-6777}~{\tiny
    }
    \end{tabular} &
        & & \\
      \cmidrule{2-5}
      &
    \begin{tabular}{@{}l@{}}
    \hypertarget{sys-all-com-icd-0048-v-07}{SYS-ALL-COM-ICD-0048-V-07}
    \\\vcdJiraRef{LVV-6778}~{\tiny
    }
    \end{tabular} &
        & & \\
  \midrule
  \begin{tabular}{@{}l@{}}
  SYS-ALL-COM-ICD-0043\\\vcdDocRef{LSE-70}~{\tiny
  }
  \end{tabular} &
    \begin{tabular}{@{}l@{}}
    \hypertarget{sys-all-com-icd-0043-v-06}{SYS-ALL-COM-ICD-0043-V-06}
    \\\vcdJiraRef{LVV-6783}~{\tiny
    }
    \end{tabular} &
        & & \\
      \cmidrule{2-5}
      &
    \begin{tabular}{@{}l@{}}
    \hypertarget{sys-all-com-icd-0043-v-07}{SYS-ALL-COM-ICD-0043-V-07}
    \\\vcdJiraRef{LVV-6784}~{\tiny
    }
    \end{tabular} &
        & & \\
  \midrule
  \begin{tabular}{@{}l@{}}
  SYS-ALL-COM-ICD-0046\\\vcdDocRef{LSE-70}~{\tiny
  }
  \end{tabular} &
    \begin{tabular}{@{}l@{}}
    \hypertarget{sys-all-com-icd-0046-v-06}{SYS-ALL-COM-ICD-0046-V-06}
    \\\vcdJiraRef{LVV-6789}~{\tiny
    }
    \end{tabular} &
        & & \\
      \cmidrule{2-5}
      &
    \begin{tabular}{@{}l@{}}
    \hypertarget{sys-all-com-icd-0046-v-07}{SYS-ALL-COM-ICD-0046-V-07}
    \\\vcdJiraRef{LVV-6790}~{\tiny
    }
    \end{tabular} &
        & & \\
  \midrule
  \begin{tabular}{@{}l@{}}
  SYS-ALL-COM-ICD-0044\\\vcdDocRef{LSE-70}~{\tiny
  }
  \end{tabular} &
    \begin{tabular}{@{}l@{}}
    \hypertarget{sys-all-com-icd-0044-v-06}{SYS-ALL-COM-ICD-0044-V-06}
    \\\vcdJiraRef{LVV-6795}~{\tiny
    }
    \end{tabular} &
        & & \\
      \cmidrule{2-5}
      &
    \begin{tabular}{@{}l@{}}
    \hypertarget{sys-all-com-icd-0044-v-07}{SYS-ALL-COM-ICD-0044-V-07}
    \\\vcdJiraRef{LVV-6796}~{\tiny
    }
    \end{tabular} &
        & & \\
  \midrule
  \begin{tabular}{@{}l@{}}
  SYS-ALL-COM-ICD-0045\\\vcdDocRef{LSE-70}~{\tiny
  }
  \end{tabular} &
    \begin{tabular}{@{}l@{}}
    \hypertarget{sys-all-com-icd-0045-v-06}{SYS-ALL-COM-ICD-0045-V-06}
    \\\vcdJiraRef{LVV-6801}~{\tiny
    }
    \end{tabular} &
        & & \\
      \cmidrule{2-5}
      &
    \begin{tabular}{@{}l@{}}
    \hypertarget{sys-all-com-icd-0045-v-07}{SYS-ALL-COM-ICD-0045-V-07}
    \\\vcdJiraRef{LVV-6802}~{\tiny
    }
    \end{tabular} &
        & & \\
  \midrule
  \begin{tabular}{@{}l@{}}
  SYS-ALL-COM-ICD-0042\\\vcdDocRef{LSE-70}~{\tiny
  }
  \end{tabular} &
    \begin{tabular}{@{}l@{}}
    \hypertarget{sys-all-com-icd-0042-v-06}{SYS-ALL-COM-ICD-0042-V-06}
    \\\vcdJiraRef{LVV-6807}~{\tiny
    }
    \end{tabular} &
        & & \\
      \cmidrule{2-5}
      &
    \begin{tabular}{@{}l@{}}
    \hypertarget{sys-all-com-icd-0042-v-07}{SYS-ALL-COM-ICD-0042-V-07}
    \\\vcdJiraRef{LVV-6808}~{\tiny
    }
    \end{tabular} &
        & & \\
  \midrule
  \begin{tabular}{@{}l@{}}
  SYS-ALL-COM-ICD-0029\\\vcdDocRef{LSE-70}~{\tiny
  }
  \end{tabular} &
    \begin{tabular}{@{}l@{}}
    \hypertarget{sys-all-com-icd-0029-v-06}{SYS-ALL-COM-ICD-0029-V-06}
    \\\vcdJiraRef{LVV-6813}~{\tiny
    }
    \end{tabular} &
        & & \\
      \cmidrule{2-5}
      &
    \begin{tabular}{@{}l@{}}
    \hypertarget{sys-all-com-icd-0029-v-07}{SYS-ALL-COM-ICD-0029-V-07}
    \\\vcdJiraRef{LVV-6814}~{\tiny
    }
    \end{tabular} &
        & & \\
  \midrule
  \begin{tabular}{@{}l@{}}
  SYS-ALL-COM-ICD-0028\\\vcdDocRef{LSE-70}~{\tiny
  }
  \end{tabular} &
    \begin{tabular}{@{}l@{}}
    \hypertarget{sys-all-com-icd-0028-v-06}{SYS-ALL-COM-ICD-0028-V-06}
    \\\vcdJiraRef{LVV-6819}~{\tiny
    }
    \end{tabular} &
        & & \\
      \cmidrule{2-5}
      &
    \begin{tabular}{@{}l@{}}
    \hypertarget{sys-all-com-icd-0028-v-07}{SYS-ALL-COM-ICD-0028-V-07}
    \\\vcdJiraRef{LVV-6820}~{\tiny
    }
    \end{tabular} &
        & & \\
  \midrule
  \begin{tabular}{@{}l@{}}
  SYS-ALL-COM-ICD-0005\\\vcdDocRef{LSE-70}~{\tiny
  }
  \end{tabular} &
    \begin{tabular}{@{}l@{}}
    \hypertarget{sys-all-com-icd-0005-v-06}{SYS-ALL-COM-ICD-0005-V-06}
    \\\vcdJiraRef{LVV-6825}~{\tiny
    }
    \end{tabular} &
        & & \\
      \cmidrule{2-5}
      &
    \begin{tabular}{@{}l@{}}
    \hypertarget{sys-all-com-icd-0005-v-07}{SYS-ALL-COM-ICD-0005-V-07}
    \\\vcdJiraRef{LVV-6826}~{\tiny
    }
    \end{tabular} &
        & & \\
  \midrule
  \begin{tabular}{@{}l@{}}
  SYS-ALL-COM-ICD-0030\\\vcdDocRef{LSE-70}~{\tiny
  }
  \end{tabular} &
    \begin{tabular}{@{}l@{}}
    \hypertarget{sys-all-com-icd-0030-v-06}{SYS-ALL-COM-ICD-0030-V-06}
    \\\vcdJiraRef{LVV-6831}~{\tiny
    }
    \end{tabular} &
        & & \\
      \cmidrule{2-5}
      &
    \begin{tabular}{@{}l@{}}
    \hypertarget{sys-all-com-icd-0030-v-07}{SYS-ALL-COM-ICD-0030-V-07}
    \\\vcdJiraRef{LVV-6832}~{\tiny
    }
    \end{tabular} &
        & & \\
  \midrule
  \begin{tabular}{@{}l@{}}
  SYS-ALL-COM-ICD-0026\\\vcdDocRef{LSE-70}~{\tiny
  }
  \end{tabular} &
    \begin{tabular}{@{}l@{}}
    \hypertarget{sys-all-com-icd-0026-v-06}{SYS-ALL-COM-ICD-0026-V-06}
    \\\vcdJiraRef{LVV-6837}~{\tiny
    }
    \end{tabular} &
        & & \\
      \cmidrule{2-5}
      &
    \begin{tabular}{@{}l@{}}
    \hypertarget{sys-all-com-icd-0026-v-07}{SYS-ALL-COM-ICD-0026-V-07}
    \\\vcdJiraRef{LVV-6838}~{\tiny
    }
    \end{tabular} &
        & & \\
  \midrule
  \begin{tabular}{@{}l@{}}
  SYS-ALL-COM-ICD-0027\\\vcdDocRef{LSE-70}~{\tiny
  }
  \end{tabular} &
    \begin{tabular}{@{}l@{}}
    \hypertarget{sys-all-com-icd-0027-v-06}{SYS-ALL-COM-ICD-0027-V-06}
    \\\vcdJiraRef{LVV-6843}~{\tiny
    }
    \end{tabular} &
        & & \\
      \cmidrule{2-5}
      &
    \begin{tabular}{@{}l@{}}
    \hypertarget{sys-all-com-icd-0027-v-07}{SYS-ALL-COM-ICD-0027-V-07}
    \\\vcdJiraRef{LVV-6844}~{\tiny
    }
    \end{tabular} &
        & & \\
  \midrule
  \begin{tabular}{@{}l@{}}
  SYS-ALL-COM-ICD-0050\\\vcdDocRef{LSE-70}~{\tiny
  }
  \end{tabular} &
    \begin{tabular}{@{}l@{}}
    \hypertarget{sys-all-com-icd-0050-v-06}{SYS-ALL-COM-ICD-0050-V-06}
    \\\vcdJiraRef{LVV-6849}~{\tiny
    }
    \end{tabular} &
        & & \\
      \cmidrule{2-5}
      &
    \begin{tabular}{@{}l@{}}
    \hypertarget{sys-all-com-icd-0050-v-07}{SYS-ALL-COM-ICD-0050-V-07}
    \\\vcdJiraRef{LVV-6850}~{\tiny
    }
    \end{tabular} &
        & & \\
  \midrule
  \begin{tabular}{@{}l@{}}
  SYS-ALL-COM-ICD-0049\\\vcdDocRef{LSE-70}~{\tiny
  }
  \end{tabular} &
    \begin{tabular}{@{}l@{}}
    \hypertarget{sys-all-com-icd-0049-v-06}{SYS-ALL-COM-ICD-0049-V-06}
    \\\vcdJiraRef{LVV-6855}~{\tiny
    }
    \end{tabular} &
        & & \\
      \cmidrule{2-5}
      &
    \begin{tabular}{@{}l@{}}
    \hypertarget{sys-all-com-icd-0049-v-07}{SYS-ALL-COM-ICD-0049-V-07}
    \\\vcdJiraRef{LVV-6856}~{\tiny
    }
    \end{tabular} &
        & & \\
  \midrule
  \begin{tabular}{@{}l@{}}
  SYS-ALL-COM-ICD-0031\\\vcdDocRef{LSE-70}~{\tiny
  }
  \end{tabular} &
    \begin{tabular}{@{}l@{}}
    \hypertarget{sys-all-com-icd-0031-v-06}{SYS-ALL-COM-ICD-0031-V-06}
    \\\vcdJiraRef{LVV-6861}~{\tiny
    }
    \end{tabular} &
        & & \\
      \cmidrule{2-5}
      &
    \begin{tabular}{@{}l@{}}
    \hypertarget{sys-all-com-icd-0031-v-07}{SYS-ALL-COM-ICD-0031-V-07}
    \\\vcdJiraRef{LVV-6862}~{\tiny
    }
    \end{tabular} &
        & & \\
  \midrule
  \begin{tabular}{@{}l@{}}
  SYS-ALL-COM-ICD-0033\\\vcdDocRef{LSE-70}~{\tiny
  }
  \end{tabular} &
    \begin{tabular}{@{}l@{}}
    \hypertarget{sys-all-com-icd-0033-v-06}{SYS-ALL-COM-ICD-0033-V-06}
    \\\vcdJiraRef{LVV-6867}~{\tiny
    }
    \end{tabular} &
        & & \\
      \cmidrule{2-5}
      &
    \begin{tabular}{@{}l@{}}
    \hypertarget{sys-all-com-icd-0033-v-07}{SYS-ALL-COM-ICD-0033-V-07}
    \\\vcdJiraRef{LVV-6868}~{\tiny
    }
    \end{tabular} &
        & & \\
  \midrule
  \begin{tabular}{@{}l@{}}
  SYS-ALL-COM-ICD-0035\\\vcdDocRef{LSE-70}~{\tiny
  }
  \end{tabular} &
    \begin{tabular}{@{}l@{}}
    \hypertarget{sys-all-com-icd-0035-v-06}{SYS-ALL-COM-ICD-0035-V-06}
    \\\vcdJiraRef{LVV-6873}~{\tiny
    }
    \end{tabular} &
        & & \\
      \cmidrule{2-5}
      &
    \begin{tabular}{@{}l@{}}
    \hypertarget{sys-all-com-icd-0035-v-07}{SYS-ALL-COM-ICD-0035-V-07}
    \\\vcdJiraRef{LVV-6874}~{\tiny
    }
    \end{tabular} &
        & & \\
  \midrule
  \begin{tabular}{@{}l@{}}
  SYS-ALL-COM-ICD-0037\\\vcdDocRef{LSE-70}~{\tiny
  }
  \end{tabular} &
    \begin{tabular}{@{}l@{}}
    \hypertarget{sys-all-com-icd-0037-v-06}{SYS-ALL-COM-ICD-0037-V-06}
    \\\vcdJiraRef{LVV-6879}~{\tiny
    }
    \end{tabular} &
        & & \\
      \cmidrule{2-5}
      &
    \begin{tabular}{@{}l@{}}
    \hypertarget{sys-all-com-icd-0037-v-07}{SYS-ALL-COM-ICD-0037-V-07}
    \\\vcdJiraRef{LVV-6880}~{\tiny
    }
    \end{tabular} &
        & & \\
  \midrule
  \begin{tabular}{@{}l@{}}
  SYS-ALL-COM-ICD-0040\\\vcdDocRef{LSE-70}~{\tiny
  }
  \end{tabular} &
    \begin{tabular}{@{}l@{}}
    \hypertarget{sys-all-com-icd-0040-v-06}{SYS-ALL-COM-ICD-0040-V-06}
    \\\vcdJiraRef{LVV-6885}~{\tiny
    }
    \end{tabular} &
        & & \\
      \cmidrule{2-5}
      &
    \begin{tabular}{@{}l@{}}
    \hypertarget{sys-all-com-icd-0040-v-07}{SYS-ALL-COM-ICD-0040-V-07}
    \\\vcdJiraRef{LVV-6886}~{\tiny
    }
    \end{tabular} &
        & & \\
  \midrule
  \begin{tabular}{@{}l@{}}
  SYS-ALL-COM-ICD-0036\\\vcdDocRef{LSE-70}~{\tiny
  }
  \end{tabular} &
    \begin{tabular}{@{}l@{}}
    \hypertarget{sys-all-com-icd-0036-v-06}{SYS-ALL-COM-ICD-0036-V-06}
    \\\vcdJiraRef{LVV-6891}~{\tiny
    }
    \end{tabular} &
        & & \\
      \cmidrule{2-5}
      &
    \begin{tabular}{@{}l@{}}
    \hypertarget{sys-all-com-icd-0036-v-07}{SYS-ALL-COM-ICD-0036-V-07}
    \\\vcdJiraRef{LVV-6892}~{\tiny
    }
    \end{tabular} &
        & & \\
  \midrule
  \begin{tabular}{@{}l@{}}
  SYS-ALL-COM-ICD-0041\\\vcdDocRef{LSE-70}~{\tiny
  }
  \end{tabular} &
    \begin{tabular}{@{}l@{}}
    \hypertarget{sys-all-com-icd-0041-v-06}{SYS-ALL-COM-ICD-0041-V-06}
    \\\vcdJiraRef{LVV-6897}~{\tiny
    }
    \end{tabular} &
        & & \\
      \cmidrule{2-5}
      &
    \begin{tabular}{@{}l@{}}
    \hypertarget{sys-all-com-icd-0041-v-07}{SYS-ALL-COM-ICD-0041-V-07}
    \\\vcdJiraRef{LVV-6898}~{\tiny
    }
    \end{tabular} &
        & & \\
  \midrule
  \begin{tabular}{@{}l@{}}
  SYS-ALL-COM-ICD-0038\\\vcdDocRef{LSE-70}~{\tiny
  }
  \end{tabular} &
    \begin{tabular}{@{}l@{}}
    \hypertarget{sys-all-com-icd-0038-v-06}{SYS-ALL-COM-ICD-0038-V-06}
    \\\vcdJiraRef{LVV-6903}~{\tiny
    }
    \end{tabular} &
        & & \\
      \cmidrule{2-5}
      &
    \begin{tabular}{@{}l@{}}
    \hypertarget{sys-all-com-icd-0038-v-07}{SYS-ALL-COM-ICD-0038-V-07}
    \\\vcdJiraRef{LVV-6904}~{\tiny
    }
    \end{tabular} &
        & & \\
  \midrule
  \begin{tabular}{@{}l@{}}
  SYS-ALL-COM-ICD-0034\\\vcdDocRef{LSE-70}~{\tiny
  }
  \end{tabular} &
    \begin{tabular}{@{}l@{}}
    \hypertarget{sys-all-com-icd-0034-v-06}{SYS-ALL-COM-ICD-0034-V-06}
    \\\vcdJiraRef{LVV-6909}~{\tiny
    }
    \end{tabular} &
        & & \\
      \cmidrule{2-5}
      &
    \begin{tabular}{@{}l@{}}
    \hypertarget{sys-all-com-icd-0034-v-07}{SYS-ALL-COM-ICD-0034-V-07}
    \\\vcdJiraRef{LVV-6910}~{\tiny
    }
    \end{tabular} &
        & & \\
  \midrule
  \begin{tabular}{@{}l@{}}
  SYS-ALL-COM-ICD-0032\\\vcdDocRef{LSE-70}~{\tiny
  }
  \end{tabular} &
    \begin{tabular}{@{}l@{}}
    \hypertarget{sys-all-com-icd-0032-v-06}{SYS-ALL-COM-ICD-0032-V-06}
    \\\vcdJiraRef{LVV-6915}~{\tiny
    }
    \end{tabular} &
        & & \\
      \cmidrule{2-5}
      &
    \begin{tabular}{@{}l@{}}
    \hypertarget{sys-all-com-icd-0032-v-07}{SYS-ALL-COM-ICD-0032-V-07}
    \\\vcdJiraRef{LVV-6916}~{\tiny
    }
    \end{tabular} &
        & & \\
  \midrule
  \begin{tabular}{@{}l@{}}
  SYS-ALL-COM-ICD-0039\\\vcdDocRef{LSE-70}~{\tiny
  }
  \end{tabular} &
    \begin{tabular}{@{}l@{}}
    \hypertarget{sys-all-com-icd-0039-v-06}{SYS-ALL-COM-ICD-0039-V-06}
    \\\vcdJiraRef{LVV-6921}~{\tiny
    }
    \end{tabular} &
        & & \\
      \cmidrule{2-5}
      &
    \begin{tabular}{@{}l@{}}
    \hypertarget{sys-all-com-icd-0039-v-07}{SYS-ALL-COM-ICD-0039-V-07}
    \\\vcdJiraRef{LVV-6922}~{\tiny
    }
    \end{tabular} &
        & & \\
  \midrule
\label{tab:dmvcd}
\end{longtable}
}

\subsection{LSE-209 Requirements Coverage}

\setlength\LTleft{-0.25in}
\setlength\LTright{-0.5in}
{\small
\begin{longtable}{lllll}
\caption{ DM LSE-209 Requirements.} \\
\toprule
\textbf{Requirement} & \textbf{Verification Element} & \textbf{Test Case} & \textbf{Last Run} & \textbf{Test Status} \\
\toprule
\endhead
  \begin{tabular}{@{}l@{}}
  CPT-OCS-INT-ICD-0001\\\vcdDocRef{LSE-209}~{\tiny
  }
  \end{tabular} &
    \begin{tabular}{@{}l@{}}
    \hypertarget{cpt-ocs-int-icd-0001-v-06}{CPT-OCS-INT-ICD-0001-V-06}
    \\\vcdJiraRef{LVV-6927}~{\tiny
    }
    \end{tabular} &
        & & \\
      \cmidrule{2-5}
      &
    \begin{tabular}{@{}l@{}}
    \hypertarget{cpt-ocs-int-icd-0001-v-07}{CPT-OCS-INT-ICD-0001-V-07}
    \\\vcdJiraRef{LVV-6928}~{\tiny
    }
    \end{tabular} &
        & & \\
  \midrule
  \begin{tabular}{@{}l@{}}
  CPT-OCS-INT-ICD-0005\\\vcdDocRef{LSE-209}~{\tiny
  }
  \end{tabular} &
    \begin{tabular}{@{}l@{}}
    \hypertarget{cpt-ocs-int-icd-0005-v-06}{CPT-OCS-INT-ICD-0005-V-06}
    \\\vcdJiraRef{LVV-6933}~{\tiny
    }
    \end{tabular} &
        & & \\
      \cmidrule{2-5}
      &
    \begin{tabular}{@{}l@{}}
    \hypertarget{cpt-ocs-int-icd-0005-v-07}{CPT-OCS-INT-ICD-0005-V-07}
    \\\vcdJiraRef{LVV-6934}~{\tiny
    }
    \end{tabular} &
        & & \\
  \midrule
  \begin{tabular}{@{}l@{}}
  CPT-OCS-INT-ICD-0006\\\vcdDocRef{LSE-209}~{\tiny
  }
  \end{tabular} &
    \begin{tabular}{@{}l@{}}
    \hypertarget{cpt-ocs-int-icd-0006-v-06}{CPT-OCS-INT-ICD-0006-V-06}
    \\\vcdJiraRef{LVV-6939}~{\tiny
    }
    \end{tabular} &
        & & \\
      \cmidrule{2-5}
      &
    \begin{tabular}{@{}l@{}}
    \hypertarget{cpt-ocs-int-icd-0006-v-07}{CPT-OCS-INT-ICD-0006-V-07}
    \\\vcdJiraRef{LVV-6940}~{\tiny
    }
    \end{tabular} &
        & & \\
  \midrule
  \begin{tabular}{@{}l@{}}
  CPT-OCS-INT-ICD-0008\\\vcdDocRef{LSE-209}~{\tiny
  }
  \end{tabular} &
    \begin{tabular}{@{}l@{}}
    \hypertarget{cpt-ocs-int-icd-0008-v-06}{CPT-OCS-INT-ICD-0008-V-06}
    \\\vcdJiraRef{LVV-6945}~{\tiny
    }
    \end{tabular} &
        & & \\
      \cmidrule{2-5}
      &
    \begin{tabular}{@{}l@{}}
    \hypertarget{cpt-ocs-int-icd-0008-v-07}{CPT-OCS-INT-ICD-0008-V-07}
    \\\vcdJiraRef{LVV-6946}~{\tiny
    }
    \end{tabular} &
        & & \\
  \midrule
  \begin{tabular}{@{}l@{}}
  CPT-OCS-INT-ICD-0040\\\vcdDocRef{LSE-209}~{\tiny
  }
  \end{tabular} &
    \begin{tabular}{@{}l@{}}
    \hypertarget{cpt-ocs-int-icd-0040-v-06}{CPT-OCS-INT-ICD-0040-V-06}
    \\\vcdJiraRef{LVV-6951}~{\tiny
    }
    \end{tabular} &
        & & \\
      \cmidrule{2-5}
      &
    \begin{tabular}{@{}l@{}}
    \hypertarget{cpt-ocs-int-icd-0040-v-07}{CPT-OCS-INT-ICD-0040-V-07}
    \\\vcdJiraRef{LVV-6952}~{\tiny
    }
    \end{tabular} &
        & & \\
  \midrule
  \begin{tabular}{@{}l@{}}
  CPT-OCS-INT-ICD-0041\\\vcdDocRef{LSE-209}~{\tiny
  }
  \end{tabular} &
    \begin{tabular}{@{}l@{}}
    \hypertarget{cpt-ocs-int-icd-0041-v-06}{CPT-OCS-INT-ICD-0041-V-06}
    \\\vcdJiraRef{LVV-6957}~{\tiny
    }
    \end{tabular} &
        & & \\
      \cmidrule{2-5}
      &
    \begin{tabular}{@{}l@{}}
    \hypertarget{cpt-ocs-int-icd-0041-v-07}{CPT-OCS-INT-ICD-0041-V-07}
    \\\vcdJiraRef{LVV-6958}~{\tiny
    }
    \end{tabular} &
        & & \\
  \midrule
  \begin{tabular}{@{}l@{}}
  CPT-OCS-INT-ICD-0042\\\vcdDocRef{LSE-209}~{\tiny
  }
  \end{tabular} &
    \begin{tabular}{@{}l@{}}
    \hypertarget{cpt-ocs-int-icd-0042-v-06}{CPT-OCS-INT-ICD-0042-V-06}
    \\\vcdJiraRef{LVV-6963}~{\tiny
    }
    \end{tabular} &
        & & \\
      \cmidrule{2-5}
      &
    \begin{tabular}{@{}l@{}}
    \hypertarget{cpt-ocs-int-icd-0042-v-07}{CPT-OCS-INT-ICD-0042-V-07}
    \\\vcdJiraRef{LVV-6964}~{\tiny
    }
    \end{tabular} &
        & & \\
  \midrule
  \begin{tabular}{@{}l@{}}
  CPT-OCS-INT-ICD-0002\\\vcdDocRef{LSE-209}~{\tiny
  }
  \end{tabular} &
    \begin{tabular}{@{}l@{}}
    \hypertarget{cpt-ocs-int-icd-0002-v-06}{CPT-OCS-INT-ICD-0002-V-06}
    \\\vcdJiraRef{LVV-6969}~{\tiny
    }
    \end{tabular} &
        & & \\
      \cmidrule{2-5}
      &
    \begin{tabular}{@{}l@{}}
    \hypertarget{cpt-ocs-int-icd-0002-v-07}{CPT-OCS-INT-ICD-0002-V-07}
    \\\vcdJiraRef{LVV-6970}~{\tiny
    }
    \end{tabular} &
        & & \\
  \midrule
  \begin{tabular}{@{}l@{}}
  CPT-OCS-INT-ICD-0003\\\vcdDocRef{LSE-209}~{\tiny
  }
  \end{tabular} &
    \begin{tabular}{@{}l@{}}
    \hypertarget{cpt-ocs-int-icd-0003-v-06}{CPT-OCS-INT-ICD-0003-V-06}
    \\\vcdJiraRef{LVV-6975}~{\tiny
    }
    \end{tabular} &
        & & \\
      \cmidrule{2-5}
      &
    \begin{tabular}{@{}l@{}}
    \hypertarget{cpt-ocs-int-icd-0003-v-07}{CPT-OCS-INT-ICD-0003-V-07}
    \\\vcdJiraRef{LVV-6976}~{\tiny
    }
    \end{tabular} &
        & & \\
  \midrule
  \begin{tabular}{@{}l@{}}
  CPT-OCS-INT-ICD-0009\\\vcdDocRef{LSE-209}~{\tiny
  }
  \end{tabular} &
    \begin{tabular}{@{}l@{}}
    \hypertarget{cpt-ocs-int-icd-0009-v-06}{CPT-OCS-INT-ICD-0009-V-06}
    \\\vcdJiraRef{LVV-6981}~{\tiny
    }
    \end{tabular} &
        & & \\
      \cmidrule{2-5}
      &
    \begin{tabular}{@{}l@{}}
    \hypertarget{cpt-ocs-int-icd-0009-v-07}{CPT-OCS-INT-ICD-0009-V-07}
    \\\vcdJiraRef{LVV-6982}~{\tiny
    }
    \end{tabular} &
        & & \\
  \midrule
  \begin{tabular}{@{}l@{}}
  CPT-OCS-INT-ICD-0072\\\vcdDocRef{LSE-209}~{\tiny
  }
  \end{tabular} &
    \begin{tabular}{@{}l@{}}
    \hypertarget{cpt-ocs-int-icd-0072-v-06}{CPT-OCS-INT-ICD-0072-V-06}
    \\\vcdJiraRef{LVV-6987}~{\tiny
    }
    \end{tabular} &
        & & \\
      \cmidrule{2-5}
      &
    \begin{tabular}{@{}l@{}}
    \hypertarget{cpt-ocs-int-icd-0072-v-07}{CPT-OCS-INT-ICD-0072-V-07}
    \\\vcdJiraRef{LVV-6988}~{\tiny
    }
    \end{tabular} &
        & & \\
  \midrule
  \begin{tabular}{@{}l@{}}
  CPT-OCS-INT-ICD-0010\\\vcdDocRef{LSE-209}~{\tiny
  }
  \end{tabular} &
    \begin{tabular}{@{}l@{}}
    \hypertarget{cpt-ocs-int-icd-0010-v-06}{CPT-OCS-INT-ICD-0010-V-06}
    \\\vcdJiraRef{LVV-6993}~{\tiny
    }
    \end{tabular} &
        & & \\
      \cmidrule{2-5}
      &
    \begin{tabular}{@{}l@{}}
    \hypertarget{cpt-ocs-int-icd-0010-v-07}{CPT-OCS-INT-ICD-0010-V-07}
    \\\vcdJiraRef{LVV-6994}~{\tiny
    }
    \end{tabular} &
        & & \\
  \midrule
  \begin{tabular}{@{}l@{}}
  CPT-OCS-INT-ICD-0012\\\vcdDocRef{LSE-209}~{\tiny
  }
  \end{tabular} &
    \begin{tabular}{@{}l@{}}
    \hypertarget{cpt-ocs-int-icd-0012-v-06}{CPT-OCS-INT-ICD-0012-V-06}
    \\\vcdJiraRef{LVV-6999}~{\tiny
    }
    \end{tabular} &
        & & \\
      \cmidrule{2-5}
      &
    \begin{tabular}{@{}l@{}}
    \hypertarget{cpt-ocs-int-icd-0012-v-07}{CPT-OCS-INT-ICD-0012-V-07}
    \\\vcdJiraRef{LVV-7000}~{\tiny
    }
    \end{tabular} &
        & & \\
  \midrule
  \begin{tabular}{@{}l@{}}
  CPT-OCS-INT-ICD-0004\\\vcdDocRef{LSE-209}~{\tiny
  }
  \end{tabular} &
    \begin{tabular}{@{}l@{}}
    \hypertarget{cpt-ocs-int-icd-0004-v-06}{CPT-OCS-INT-ICD-0004-V-06}
    \\\vcdJiraRef{LVV-7005}~{\tiny
    }
    \end{tabular} &
        & & \\
      \cmidrule{2-5}
      &
    \begin{tabular}{@{}l@{}}
    \hypertarget{cpt-ocs-int-icd-0004-v-07}{CPT-OCS-INT-ICD-0004-V-07}
    \\\vcdJiraRef{LVV-7006}~{\tiny
    }
    \end{tabular} &
        & & \\
  \midrule
  \begin{tabular}{@{}l@{}}
  CPT-OCS-INT-ICD-0007\\\vcdDocRef{LSE-209}~{\tiny
  }
  \end{tabular} &
    \begin{tabular}{@{}l@{}}
    \hypertarget{cpt-ocs-int-icd-0007-v-06}{CPT-OCS-INT-ICD-0007-V-06}
    \\\vcdJiraRef{LVV-7011}~{\tiny
    }
    \end{tabular} &
        & & \\
      \cmidrule{2-5}
      &
    \begin{tabular}{@{}l@{}}
    \hypertarget{cpt-ocs-int-icd-0007-v-07}{CPT-OCS-INT-ICD-0007-V-07}
    \\\vcdJiraRef{LVV-7012}~{\tiny
    }
    \end{tabular} &
        & & \\
  \midrule
  \begin{tabular}{@{}l@{}}
  CPT-OCS-INT-ICD-0011\\\vcdDocRef{LSE-209}~{\tiny
  }
  \end{tabular} &
    \begin{tabular}{@{}l@{}}
    \hypertarget{cpt-ocs-int-icd-0011-v-06}{CPT-OCS-INT-ICD-0011-V-06}
    \\\vcdJiraRef{LVV-7017}~{\tiny
    }
    \end{tabular} &
        & & \\
      \cmidrule{2-5}
      &
    \begin{tabular}{@{}l@{}}
    \hypertarget{cpt-ocs-int-icd-0011-v-07}{CPT-OCS-INT-ICD-0011-V-07}
    \\\vcdJiraRef{LVV-7018}~{\tiny
    }
    \end{tabular} &
        & & \\
  \midrule
  \begin{tabular}{@{}l@{}}
  CPT-OCS-INT-ICD-0049\\\vcdDocRef{LSE-209}~{\tiny
  }
  \end{tabular} &
    \begin{tabular}{@{}l@{}}
    \hypertarget{cpt-ocs-int-icd-0049-v-06}{CPT-OCS-INT-ICD-0049-V-06}
    \\\vcdJiraRef{LVV-7023}~{\tiny
    }
    \end{tabular} &
        & & \\
      \cmidrule{2-5}
      &
    \begin{tabular}{@{}l@{}}
    \hypertarget{cpt-ocs-int-icd-0049-v-07}{CPT-OCS-INT-ICD-0049-V-07}
    \\\vcdJiraRef{LVV-7024}~{\tiny
    }
    \end{tabular} &
        & & \\
  \midrule
  \begin{tabular}{@{}l@{}}
  CPT-OCS-INT-ICD-0071\\\vcdDocRef{LSE-209}~{\tiny
  }
  \end{tabular} &
    \begin{tabular}{@{}l@{}}
    \hypertarget{cpt-ocs-int-icd-0071-v-06}{CPT-OCS-INT-ICD-0071-V-06}
    \\\vcdJiraRef{LVV-7029}~{\tiny
    }
    \end{tabular} &
        & & \\
      \cmidrule{2-5}
      &
    \begin{tabular}{@{}l@{}}
    \hypertarget{cpt-ocs-int-icd-0071-v-07}{CPT-OCS-INT-ICD-0071-V-07}
    \\\vcdJiraRef{LVV-7030}~{\tiny
    }
    \end{tabular} &
        & & \\
  \midrule
  \begin{tabular}{@{}l@{}}
  CPT-OCS-INT-ICD-0046\\\vcdDocRef{LSE-209}~{\tiny
  }
  \end{tabular} &
    \begin{tabular}{@{}l@{}}
    \hypertarget{cpt-ocs-int-icd-0046-v-06}{CPT-OCS-INT-ICD-0046-V-06}
    \\\vcdJiraRef{LVV-7035}~{\tiny
    }
    \end{tabular} &
        & & \\
      \cmidrule{2-5}
      &
    \begin{tabular}{@{}l@{}}
    \hypertarget{cpt-ocs-int-icd-0046-v-07}{CPT-OCS-INT-ICD-0046-V-07}
    \\\vcdJiraRef{LVV-7036}~{\tiny
    }
    \end{tabular} &
        & & \\
  \midrule
  \begin{tabular}{@{}l@{}}
  CPT-OCS-INT-ICD-0045\\\vcdDocRef{LSE-209}~{\tiny
  }
  \end{tabular} &
    \begin{tabular}{@{}l@{}}
    \hypertarget{cpt-ocs-int-icd-0045-v-06}{CPT-OCS-INT-ICD-0045-V-06}
    \\\vcdJiraRef{LVV-7041}~{\tiny
    }
    \end{tabular} &
        & & \\
      \cmidrule{2-5}
      &
    \begin{tabular}{@{}l@{}}
    \hypertarget{cpt-ocs-int-icd-0045-v-07}{CPT-OCS-INT-ICD-0045-V-07}
    \\\vcdJiraRef{LVV-7042}~{\tiny
    }
    \end{tabular} &
        & & \\
  \midrule
  \begin{tabular}{@{}l@{}}
  CPT-OCS-INT-ICD-0048\\\vcdDocRef{LSE-209}~{\tiny
  }
  \end{tabular} &
    \begin{tabular}{@{}l@{}}
    \hypertarget{cpt-ocs-int-icd-0048-v-06}{CPT-OCS-INT-ICD-0048-V-06}
    \\\vcdJiraRef{LVV-7047}~{\tiny
    }
    \end{tabular} &
        & & \\
      \cmidrule{2-5}
      &
    \begin{tabular}{@{}l@{}}
    \hypertarget{cpt-ocs-int-icd-0048-v-07}{CPT-OCS-INT-ICD-0048-V-07}
    \\\vcdJiraRef{LVV-7048}~{\tiny
    }
    \end{tabular} &
        & & \\
  \midrule
  \begin{tabular}{@{}l@{}}
  CPT-OCS-INT-ICD-0043\\\vcdDocRef{LSE-209}~{\tiny
  }
  \end{tabular} &
    \begin{tabular}{@{}l@{}}
    \hypertarget{cpt-ocs-int-icd-0043-v-06}{CPT-OCS-INT-ICD-0043-V-06}
    \\\vcdJiraRef{LVV-7053}~{\tiny
    }
    \end{tabular} &
        & & \\
      \cmidrule{2-5}
      &
    \begin{tabular}{@{}l@{}}
    \hypertarget{cpt-ocs-int-icd-0043-v-07}{CPT-OCS-INT-ICD-0043-V-07}
    \\\vcdJiraRef{LVV-7054}~{\tiny
    }
    \end{tabular} &
        & & \\
  \midrule
  \begin{tabular}{@{}l@{}}
  CPT-OCS-INT-ICD-0044\\\vcdDocRef{LSE-209}~{\tiny
  }
  \end{tabular} &
    \begin{tabular}{@{}l@{}}
    \hypertarget{cpt-ocs-int-icd-0044-v-06}{CPT-OCS-INT-ICD-0044-V-06}
    \\\vcdJiraRef{LVV-7059}~{\tiny
    }
    \end{tabular} &
        & & \\
      \cmidrule{2-5}
      &
    \begin{tabular}{@{}l@{}}
    \hypertarget{cpt-ocs-int-icd-0044-v-07}{CPT-OCS-INT-ICD-0044-V-07}
    \\\vcdJiraRef{LVV-7060}~{\tiny
    }
    \end{tabular} &
        & & \\
  \midrule
  \begin{tabular}{@{}l@{}}
  CPT-OCS-INT-ICD-0047\\\vcdDocRef{LSE-209}~{\tiny
  }
  \end{tabular} &
    \begin{tabular}{@{}l@{}}
    \hypertarget{cpt-ocs-int-icd-0047-v-06}{CPT-OCS-INT-ICD-0047-V-06}
    \\\vcdJiraRef{LVV-7065}~{\tiny
    }
    \end{tabular} &
        & & \\
      \cmidrule{2-5}
      &
    \begin{tabular}{@{}l@{}}
    \hypertarget{cpt-ocs-int-icd-0047-v-07}{CPT-OCS-INT-ICD-0047-V-07}
    \\\vcdJiraRef{LVV-7066}~{\tiny
    }
    \end{tabular} &
        & & \\
  \midrule
  \begin{tabular}{@{}l@{}}
  CPT-OCS-INT-ICD-0061\\\vcdDocRef{LSE-209}~{\tiny
  }
  \end{tabular} &
    \begin{tabular}{@{}l@{}}
    \hypertarget{cpt-ocs-int-icd-0061-v-06}{CPT-OCS-INT-ICD-0061-V-06}
    \\\vcdJiraRef{LVV-7071}~{\tiny
    }
    \end{tabular} &
        & & \\
      \cmidrule{2-5}
      &
    \begin{tabular}{@{}l@{}}
    \hypertarget{cpt-ocs-int-icd-0061-v-07}{CPT-OCS-INT-ICD-0061-V-07}
    \\\vcdJiraRef{LVV-7072}~{\tiny
    }
    \end{tabular} &
        & & \\
  \midrule
  \begin{tabular}{@{}l@{}}
  CPT-OCS-INT-ICD-0057\\\vcdDocRef{LSE-209}~{\tiny
  }
  \end{tabular} &
    \begin{tabular}{@{}l@{}}
    \hypertarget{cpt-ocs-int-icd-0057-v-06}{CPT-OCS-INT-ICD-0057-V-06}
    \\\vcdJiraRef{LVV-7077}~{\tiny
    }
    \end{tabular} &
        & & \\
      \cmidrule{2-5}
      &
    \begin{tabular}{@{}l@{}}
    \hypertarget{cpt-ocs-int-icd-0057-v-07}{CPT-OCS-INT-ICD-0057-V-07}
    \\\vcdJiraRef{LVV-7078}~{\tiny
    }
    \end{tabular} &
        & & \\
  \midrule
  \begin{tabular}{@{}l@{}}
  CPT-OCS-INT-ICD-0052\\\vcdDocRef{LSE-209}~{\tiny
  }
  \end{tabular} &
    \begin{tabular}{@{}l@{}}
    \hypertarget{cpt-ocs-int-icd-0052-v-06}{CPT-OCS-INT-ICD-0052-V-06}
    \\\vcdJiraRef{LVV-7083}~{\tiny
    }
    \end{tabular} &
        & & \\
      \cmidrule{2-5}
      &
    \begin{tabular}{@{}l@{}}
    \hypertarget{cpt-ocs-int-icd-0052-v-07}{CPT-OCS-INT-ICD-0052-V-07}
    \\\vcdJiraRef{LVV-7084}~{\tiny
    }
    \end{tabular} &
        & & \\
  \midrule
  \begin{tabular}{@{}l@{}}
  CPT-OCS-INT-ICD-0050\\\vcdDocRef{LSE-209}~{\tiny
  }
  \end{tabular} &
    \begin{tabular}{@{}l@{}}
    \hypertarget{cpt-ocs-int-icd-0050-v-06}{CPT-OCS-INT-ICD-0050-V-06}
    \\\vcdJiraRef{LVV-7089}~{\tiny
    }
    \end{tabular} &
        & & \\
      \cmidrule{2-5}
      &
    \begin{tabular}{@{}l@{}}
    \hypertarget{cpt-ocs-int-icd-0050-v-07}{CPT-OCS-INT-ICD-0050-V-07}
    \\\vcdJiraRef{LVV-7090}~{\tiny
    }
    \end{tabular} &
        & & \\
  \midrule
  \begin{tabular}{@{}l@{}}
  CPT-OCS-INT-ICD-0053\\\vcdDocRef{LSE-209}~{\tiny
  }
  \end{tabular} &
    \begin{tabular}{@{}l@{}}
    \hypertarget{cpt-ocs-int-icd-0053-v-06}{CPT-OCS-INT-ICD-0053-V-06}
    \\\vcdJiraRef{LVV-7095}~{\tiny
    }
    \end{tabular} &
        & & \\
      \cmidrule{2-5}
      &
    \begin{tabular}{@{}l@{}}
    \hypertarget{cpt-ocs-int-icd-0053-v-07}{CPT-OCS-INT-ICD-0053-V-07}
    \\\vcdJiraRef{LVV-7096}~{\tiny
    }
    \end{tabular} &
        & & \\
  \midrule
  \begin{tabular}{@{}l@{}}
  CPT-OCS-INT-ICD-0054\\\vcdDocRef{LSE-209}~{\tiny
  }
  \end{tabular} &
    \begin{tabular}{@{}l@{}}
    \hypertarget{cpt-ocs-int-icd-0054-v-06}{CPT-OCS-INT-ICD-0054-V-06}
    \\\vcdJiraRef{LVV-7101}~{\tiny
    }
    \end{tabular} &
        & & \\
      \cmidrule{2-5}
      &
    \begin{tabular}{@{}l@{}}
    \hypertarget{cpt-ocs-int-icd-0054-v-07}{CPT-OCS-INT-ICD-0054-V-07}
    \\\vcdJiraRef{LVV-7102}~{\tiny
    }
    \end{tabular} &
        & & \\
  \midrule
  \begin{tabular}{@{}l@{}}
  CPT-OCS-INT-ICD-0055\\\vcdDocRef{LSE-209}~{\tiny
  }
  \end{tabular} &
    \begin{tabular}{@{}l@{}}
    \hypertarget{cpt-ocs-int-icd-0055-v-06}{CPT-OCS-INT-ICD-0055-V-06}
    \\\vcdJiraRef{LVV-7107}~{\tiny
    }
    \end{tabular} &
        & & \\
      \cmidrule{2-5}
      &
    \begin{tabular}{@{}l@{}}
    \hypertarget{cpt-ocs-int-icd-0055-v-07}{CPT-OCS-INT-ICD-0055-V-07}
    \\\vcdJiraRef{LVV-7108}~{\tiny
    }
    \end{tabular} &
        & & \\
  \midrule
  \begin{tabular}{@{}l@{}}
  CPT-OCS-INT-ICD-0051\\\vcdDocRef{LSE-209}~{\tiny
  }
  \end{tabular} &
    \begin{tabular}{@{}l@{}}
    \hypertarget{cpt-ocs-int-icd-0051-v-06}{CPT-OCS-INT-ICD-0051-V-06}
    \\\vcdJiraRef{LVV-7113}~{\tiny
    }
    \end{tabular} &
        & & \\
      \cmidrule{2-5}
      &
    \begin{tabular}{@{}l@{}}
    \hypertarget{cpt-ocs-int-icd-0051-v-07}{CPT-OCS-INT-ICD-0051-V-07}
    \\\vcdJiraRef{LVV-7114}~{\tiny
    }
    \end{tabular} &
        & & \\
  \midrule
  \begin{tabular}{@{}l@{}}
  CPT-OCS-INT-ICD-0073\\\vcdDocRef{LSE-209}~{\tiny
  }
  \end{tabular} &
    \begin{tabular}{@{}l@{}}
    \hypertarget{cpt-ocs-int-icd-0073-v-06}{CPT-OCS-INT-ICD-0073-V-06}
    \\\vcdJiraRef{LVV-7119}~{\tiny
    }
    \end{tabular} &
        & & \\
      \cmidrule{2-5}
      &
    \begin{tabular}{@{}l@{}}
    \hypertarget{cpt-ocs-int-icd-0073-v-07}{CPT-OCS-INT-ICD-0073-V-07}
    \\\vcdJiraRef{LVV-7120}~{\tiny
    }
    \end{tabular} &
        & & \\
  \midrule
  \begin{tabular}{@{}l@{}}
  CPT-OCS-INT-ICD-0058\\\vcdDocRef{LSE-209}~{\tiny
  }
  \end{tabular} &
    \begin{tabular}{@{}l@{}}
    \hypertarget{cpt-ocs-int-icd-0058-v-06}{CPT-OCS-INT-ICD-0058-V-06}
    \\\vcdJiraRef{LVV-7125}~{\tiny
    }
    \end{tabular} &
        & & \\
      \cmidrule{2-5}
      &
    \begin{tabular}{@{}l@{}}
    \hypertarget{cpt-ocs-int-icd-0058-v-07}{CPT-OCS-INT-ICD-0058-V-07}
    \\\vcdJiraRef{LVV-7126}~{\tiny
    }
    \end{tabular} &
        & & \\
  \midrule
  \begin{tabular}{@{}l@{}}
  CPT-OCS-INT-ICD-0059\\\vcdDocRef{LSE-209}~{\tiny
  }
  \end{tabular} &
    \begin{tabular}{@{}l@{}}
    \hypertarget{cpt-ocs-int-icd-0059-v-06}{CPT-OCS-INT-ICD-0059-V-06}
    \\\vcdJiraRef{LVV-7131}~{\tiny
    }
    \end{tabular} &
        & & \\
      \cmidrule{2-5}
      &
    \begin{tabular}{@{}l@{}}
    \hypertarget{cpt-ocs-int-icd-0059-v-07}{CPT-OCS-INT-ICD-0059-V-07}
    \\\vcdJiraRef{LVV-7132}~{\tiny
    }
    \end{tabular} &
        & & \\
  \midrule
  \begin{tabular}{@{}l@{}}
  CPT-OCS-INT-ICD-0060\\\vcdDocRef{LSE-209}~{\tiny
  }
  \end{tabular} &
    \begin{tabular}{@{}l@{}}
    \hypertarget{cpt-ocs-int-icd-0060-v-06}{CPT-OCS-INT-ICD-0060-V-06}
    \\\vcdJiraRef{LVV-7137}~{\tiny
    }
    \end{tabular} &
        & & \\
      \cmidrule{2-5}
      &
    \begin{tabular}{@{}l@{}}
    \hypertarget{cpt-ocs-int-icd-0060-v-07}{CPT-OCS-INT-ICD-0060-V-07}
    \\\vcdJiraRef{LVV-7138}~{\tiny
    }
    \end{tabular} &
        & & \\
  \midrule
  \begin{tabular}{@{}l@{}}
  CPT-OCS-INT-ICD-0056\\\vcdDocRef{LSE-209}~{\tiny
  }
  \end{tabular} &
    \begin{tabular}{@{}l@{}}
    \hypertarget{cpt-ocs-int-icd-0056-v-06}{CPT-OCS-INT-ICD-0056-V-06}
    \\\vcdJiraRef{LVV-7143}~{\tiny
    }
    \end{tabular} &
        & & \\
      \cmidrule{2-5}
      &
    \begin{tabular}{@{}l@{}}
    \hypertarget{cpt-ocs-int-icd-0056-v-07}{CPT-OCS-INT-ICD-0056-V-07}
    \\\vcdJiraRef{LVV-7144}~{\tiny
    }
    \end{tabular} &
        & & \\
  \midrule
  \begin{tabular}{@{}l@{}}
  CPT-OCS-INT-ICD-0063\\\vcdDocRef{LSE-209}~{\tiny
  }
  \end{tabular} &
    \begin{tabular}{@{}l@{}}
    \hypertarget{cpt-ocs-int-icd-0063-v-06}{CPT-OCS-INT-ICD-0063-V-06}
    \\\vcdJiraRef{LVV-7149}~{\tiny
    }
    \end{tabular} &
        & & \\
      \cmidrule{2-5}
      &
    \begin{tabular}{@{}l@{}}
    \hypertarget{cpt-ocs-int-icd-0063-v-07}{CPT-OCS-INT-ICD-0063-V-07}
    \\\vcdJiraRef{LVV-7150}~{\tiny
    }
    \end{tabular} &
        & & \\
  \midrule
  \begin{tabular}{@{}l@{}}
  CPT-OCS-INT-ICD-0064\\\vcdDocRef{LSE-209}~{\tiny
  }
  \end{tabular} &
    \begin{tabular}{@{}l@{}}
    \hypertarget{cpt-ocs-int-icd-0064-v-06}{CPT-OCS-INT-ICD-0064-V-06}
    \\\vcdJiraRef{LVV-7155}~{\tiny
    }
    \end{tabular} &
        & & \\
      \cmidrule{2-5}
      &
    \begin{tabular}{@{}l@{}}
    \hypertarget{cpt-ocs-int-icd-0064-v-07}{CPT-OCS-INT-ICD-0064-V-07}
    \\\vcdJiraRef{LVV-7156}~{\tiny
    }
    \end{tabular} &
        & & \\
  \midrule
  \begin{tabular}{@{}l@{}}
  CPT-OCS-INT-ICD-0065\\\vcdDocRef{LSE-209}~{\tiny
  }
  \end{tabular} &
    \begin{tabular}{@{}l@{}}
    \hypertarget{cpt-ocs-int-icd-0065-v-06}{CPT-OCS-INT-ICD-0065-V-06}
    \\\vcdJiraRef{LVV-7161}~{\tiny
    }
    \end{tabular} &
        & & \\
      \cmidrule{2-5}
      &
    \begin{tabular}{@{}l@{}}
    \hypertarget{cpt-ocs-int-icd-0065-v-07}{CPT-OCS-INT-ICD-0065-V-07}
    \\\vcdJiraRef{LVV-7162}~{\tiny
    }
    \end{tabular} &
        & & \\
  \midrule
  \begin{tabular}{@{}l@{}}
  CPT-OCS-INT-ICD-0066\\\vcdDocRef{LSE-209}~{\tiny
  }
  \end{tabular} &
    \begin{tabular}{@{}l@{}}
    \hypertarget{cpt-ocs-int-icd-0066-v-06}{CPT-OCS-INT-ICD-0066-V-06}
    \\\vcdJiraRef{LVV-7167}~{\tiny
    }
    \end{tabular} &
        & & \\
      \cmidrule{2-5}
      &
    \begin{tabular}{@{}l@{}}
    \hypertarget{cpt-ocs-int-icd-0066-v-07}{CPT-OCS-INT-ICD-0066-V-07}
    \\\vcdJiraRef{LVV-7168}~{\tiny
    }
    \end{tabular} &
        & & \\
  \midrule
  \begin{tabular}{@{}l@{}}
  CPT-OCS-INT-ICD-0062\\\vcdDocRef{LSE-209}~{\tiny
  }
  \end{tabular} &
    \begin{tabular}{@{}l@{}}
    \hypertarget{cpt-ocs-int-icd-0062-v-06}{CPT-OCS-INT-ICD-0062-V-06}
    \\\vcdJiraRef{LVV-7173}~{\tiny
    }
    \end{tabular} &
        & & \\
      \cmidrule{2-5}
      &
    \begin{tabular}{@{}l@{}}
    \hypertarget{cpt-ocs-int-icd-0062-v-07}{CPT-OCS-INT-ICD-0062-V-07}
    \\\vcdJiraRef{LVV-7174}~{\tiny
    }
    \end{tabular} &
        & & \\
  \midrule
  \begin{tabular}{@{}l@{}}
  CPT-OCS-INT-ICD-0067\\\vcdDocRef{LSE-209}~{\tiny
  }
  \end{tabular} &
    \begin{tabular}{@{}l@{}}
    \hypertarget{cpt-ocs-int-icd-0067-v-06}{CPT-OCS-INT-ICD-0067-V-06}
    \\\vcdJiraRef{LVV-7179}~{\tiny
    }
    \end{tabular} &
        & & \\
      \cmidrule{2-5}
      &
    \begin{tabular}{@{}l@{}}
    \hypertarget{cpt-ocs-int-icd-0067-v-07}{CPT-OCS-INT-ICD-0067-V-07}
    \\\vcdJiraRef{LVV-7180}~{\tiny
    }
    \end{tabular} &
        & & \\
  \midrule
  \begin{tabular}{@{}l@{}}
  CPT-OCS-INT-ICD-0068\\\vcdDocRef{LSE-209}~{\tiny
  }
  \end{tabular} &
    \begin{tabular}{@{}l@{}}
    \hypertarget{cpt-ocs-int-icd-0068-v-06}{CPT-OCS-INT-ICD-0068-V-06}
    \\\vcdJiraRef{LVV-7185}~{\tiny
    }
    \end{tabular} &
        & & \\
      \cmidrule{2-5}
      &
    \begin{tabular}{@{}l@{}}
    \hypertarget{cpt-ocs-int-icd-0068-v-07}{CPT-OCS-INT-ICD-0068-V-07}
    \\\vcdJiraRef{LVV-7186}~{\tiny
    }
    \end{tabular} &
        & & \\
  \midrule
  \begin{tabular}{@{}l@{}}
  CPT-OCS-INT-ICD-0069\\\vcdDocRef{LSE-209}~{\tiny
  }
  \end{tabular} &
    \begin{tabular}{@{}l@{}}
    \hypertarget{cpt-ocs-int-icd-0069-v-06}{CPT-OCS-INT-ICD-0069-V-06}
    \\\vcdJiraRef{LVV-7191}~{\tiny
    }
    \end{tabular} &
        & & \\
      \cmidrule{2-5}
      &
    \begin{tabular}{@{}l@{}}
    \hypertarget{cpt-ocs-int-icd-0069-v-07}{CPT-OCS-INT-ICD-0069-V-07}
    \\\vcdJiraRef{LVV-7192}~{\tiny
    }
    \end{tabular} &
        & & \\
  \midrule
  \begin{tabular}{@{}l@{}}
  CPT-OCS-INT-ICD-0070\\\vcdDocRef{LSE-209}~{\tiny
  }
  \end{tabular} &
    \begin{tabular}{@{}l@{}}
    \hypertarget{cpt-ocs-int-icd-0070-v-06}{CPT-OCS-INT-ICD-0070-V-06}
    \\\vcdJiraRef{LVV-7197}~{\tiny
    }
    \end{tabular} &
        & & \\
      \cmidrule{2-5}
      &
    \begin{tabular}{@{}l@{}}
    \hypertarget{cpt-ocs-int-icd-0070-v-07}{CPT-OCS-INT-ICD-0070-V-07}
    \\\vcdJiraRef{LVV-7198}~{\tiny
    }
    \end{tabular} &
        & & \\
  \midrule
\label{tab:dmvcd}
\end{longtable}
}

\subsection{LDM-554 Requirements Coverage}

\setlength\LTleft{-0.25in}
\setlength\LTright{-0.5in}
{\small
\begin{longtable}{lllll}
\caption{ DM LDM-554 Requirements.} \\
\toprule
\textbf{Requirement} & \textbf{Verification Element} & \textbf{Test Case} & \textbf{Last Run} & \textbf{Test Status} \\
\toprule
\endhead
  \begin{tabular}{@{}l@{}}
  DMS-LSP-REQ-0007\\\vcdDocRef{LDM-554}~{\tiny
  }
  \end{tabular} &
    \begin{tabular}{@{}l@{}}
    \hypertarget{dms-lsp-req-0007-v-01}{DMS-LSP-REQ-0007-V-01}
    \\\vcdJiraRef{LVV-9806}~{\tiny
    }
    \end{tabular} &
        \begin{tabular}{@{}l@{}}
        \href{https://jira.lsstcorp.org/secure/Tests.jspa\#/testCase/LVV-T605}{LVV-T605} \\
        \vcdDocRef{LDM-540}
        \end{tabular} &
          & \notexec{} \\
  \midrule
  \begin{tabular}{@{}l@{}}
  DMS-LSP-REQ-0001\\\vcdDocRef{LDM-554}~{\tiny
  }
  \end{tabular} &
    \begin{tabular}{@{}l@{}}
    \hypertarget{dms-lsp-req-0001-v-01}{DMS-LSP-REQ-0001-V-01}
    \\\vcdJiraRef{LVV-9807}~{\tiny
    }
    \end{tabular} &
        \begin{tabular}{@{}l@{}}
        \href{https://jira.lsstcorp.org/secure/Tests.jspa\#/testCase/LVV-T2}{LVV-T2} \\
        \vcdDocRef{LDM-540}
        \end{tabular} &
          \begin{tabular}{@{}l@{}}
          2019-05-20 \\
            \vcdDocRef{DMTR-52}
            {\scriptsize \href{https://jira.lsstcorp.org/secure/Tests.jspa\#/testPlan/LVV-P42}{LVV-P42} }
          \end{tabular} &
          \passed \\
          \cmidrule{3-5}
          & &
        \begin{tabular}{@{}l@{}}
        \href{https://jira.lsstcorp.org/secure/Tests.jspa\#/testCase/LVV-T598}{LVV-T598} \\
        \vcdDocRef{LDM-540}
        \end{tabular} &
          & \notexec{} \\
  \midrule
  \begin{tabular}{@{}l@{}}
  DMS-LSP-REQ-0004\\\vcdDocRef{LDM-554}~{\tiny
  }
  \end{tabular} &
    \begin{tabular}{@{}l@{}}
    \hypertarget{dms-lsp-req-0004-v-01}{DMS-LSP-REQ-0004-V-01}
    \\\vcdJiraRef{LVV-9808}~{\tiny
    }
    \end{tabular} &
        \begin{tabular}{@{}l@{}}
        \href{https://jira.lsstcorp.org/secure/Tests.jspa\#/testCase/LVV-T3}{LVV-T3} \\
        \vcdDocRef{LDM-540}
        \end{tabular} &
          \begin{tabular}{@{}l@{}}
          2019-05-20 \\
            \vcdDocRef{DMTR-52}
            {\scriptsize \href{https://jira.lsstcorp.org/secure/Tests.jspa\#/testPlan/LVV-P42}{LVV-P42} }
          \end{tabular} &
          \passed \\
          \cmidrule{3-5}
          & &
        \begin{tabular}{@{}l@{}}
        \href{https://jira.lsstcorp.org/secure/Tests.jspa\#/testCase/LVV-T602}{LVV-T602} \\
        \vcdDocRef{LDM-540}
        \end{tabular} &
          & \notexec{} \\
          \cmidrule{3-5}
          & &
        \begin{tabular}{@{}l@{}}
        \href{https://jira.lsstcorp.org/secure/Tests.jspa\#/testCase/LVV-T1437}{LVV-T1437} \\
        \vcdDocRef{LDM-540}
        \end{tabular} &
          \begin{tabular}{@{}l@{}}
          2019-12-09 \\
            \vcdDocRef{DMTR-161}
            {\scriptsize \href{https://jira.lsstcorp.org/secure/Tests.jspa\#/testPlan/LVV-P48}{LVV-P48} }
          \end{tabular} &
          \cndpass \\
  \midrule
  \begin{tabular}{@{}l@{}}
  DMS-LSP-REQ-0005\\\vcdDocRef{LDM-554}~{\tiny
  }
  \end{tabular} &
    \begin{tabular}{@{}l@{}}
    \hypertarget{dms-lsp-req-0005-v-01}{DMS-LSP-REQ-0005-V-01}
    \\\vcdJiraRef{LVV-9809}~{\tiny
    }
    \end{tabular} &
        \begin{tabular}{@{}l@{}}
        \href{https://jira.lsstcorp.org/secure/Tests.jspa\#/testCase/LVV-T2}{LVV-T2} \\
        \vcdDocRef{LDM-540}
        \end{tabular} &
          \begin{tabular}{@{}l@{}}
          2019-05-20 \\
            \vcdDocRef{DMTR-52}
            {\scriptsize \href{https://jira.lsstcorp.org/secure/Tests.jspa\#/testPlan/LVV-P42}{LVV-P42} }
          \end{tabular} &
          \passed \\
          \cmidrule{3-5}
          & &
        \begin{tabular}{@{}l@{}}
        \href{https://jira.lsstcorp.org/secure/Tests.jspa\#/testCase/LVV-T603}{LVV-T603} \\
        \vcdDocRef{LDM-540}
        \end{tabular} &
          & \notexec{} \\
          \cmidrule{3-5}
          & &
        \begin{tabular}{@{}l@{}}
        \href{https://jira.lsstcorp.org/secure/Tests.jspa\#/testCase/LVV-T1334}{LVV-T1334} \\
        \vcdDocRef{LDM-540}
        \end{tabular} &
          \begin{tabular}{@{}l@{}}
          2019-12-02 \\
            \vcdDocRef{DMTR-161}
            {\scriptsize \href{https://jira.lsstcorp.org/secure/Tests.jspa\#/testPlan/LVV-P48}{LVV-P48} }
          \end{tabular} &
          \cndpass \\
          \cmidrule{3-5}
          & &
        \begin{tabular}{@{}l@{}}
        \href{https://jira.lsstcorp.org/secure/Tests.jspa\#/testCase/LVV-T1436}{LVV-T1436} \\
        \vcdDocRef{LDM-540}
        \end{tabular} &
          \begin{tabular}{@{}l@{}}
          2019-12-09 \\
            \vcdDocRef{DMTR-161}
            {\scriptsize \href{https://jira.lsstcorp.org/secure/Tests.jspa\#/testPlan/LVV-P48}{LVV-P48} }
          \end{tabular} &
          \cndpass \\
          \cmidrule{3-5}
          & &
        \begin{tabular}{@{}l@{}}
        \href{https://jira.lsstcorp.org/secure/Tests.jspa\#/testCase/LVV-T1437}{LVV-T1437} \\
        \vcdDocRef{LDM-540}
        \end{tabular} &
          \begin{tabular}{@{}l@{}}
          2019-12-09 \\
            \vcdDocRef{DMTR-161}
            {\scriptsize \href{https://jira.lsstcorp.org/secure/Tests.jspa\#/testPlan/LVV-P48}{LVV-P48} }
          \end{tabular} &
          \cndpass \\
  \midrule
  \begin{tabular}{@{}l@{}}
  DMS-LSP-REQ-0003\\\vcdDocRef{LDM-554}~{\tiny
  }
  \end{tabular} &
    \begin{tabular}{@{}l@{}}
    \hypertarget{dms-lsp-req-0003-v-01}{DMS-LSP-REQ-0003-V-01}
    \\\vcdJiraRef{LVV-9810}~{\tiny
    }
    \end{tabular} &
        \begin{tabular}{@{}l@{}}
        \href{https://jira.lsstcorp.org/secure/Tests.jspa\#/testCase/LVV-T601}{LVV-T601} \\
        \vcdDocRef{LDM-540}
        \end{tabular} &
          & \notexec{} \\
          \cmidrule{3-5}
          & &
        \begin{tabular}{@{}l@{}}
        \href{https://jira.lsstcorp.org/secure/Tests.jspa\#/testCase/LVV-T1436}{LVV-T1436} \\
        \vcdDocRef{LDM-540}
        \end{tabular} &
          \begin{tabular}{@{}l@{}}
          2019-12-09 \\
            \vcdDocRef{DMTR-161}
            {\scriptsize \href{https://jira.lsstcorp.org/secure/Tests.jspa\#/testPlan/LVV-P48}{LVV-P48} }
          \end{tabular} &
          \cndpass \\
  \midrule
  \begin{tabular}{@{}l@{}}
  DMS-LSP-REQ-0002\\\vcdDocRef{LDM-554}~{\tiny
  }
  \end{tabular} &
    \begin{tabular}{@{}l@{}}
    \hypertarget{dms-lsp-req-0002-v-01}{DMS-LSP-REQ-0002-V-01}
    \\\vcdJiraRef{LVV-9811}~{\tiny
    }
    \end{tabular} &
        \begin{tabular}{@{}l@{}}
        \href{https://jira.lsstcorp.org/secure/Tests.jspa\#/testCase/LVV-T5}{LVV-T5} \\
        \vcdDocRef{LDM-540}
        \end{tabular} &
          \begin{tabular}{@{}l@{}}
          2019-05-20 \\
            \vcdDocRef{DMTR-52}
            {\scriptsize \href{https://jira.lsstcorp.org/secure/Tests.jspa\#/testPlan/LVV-P42}{LVV-P42} }
          \end{tabular} &
          \passed \\
          \cmidrule{3-5}
          & &
        \begin{tabular}{@{}l@{}}
        \href{https://jira.lsstcorp.org/secure/Tests.jspa\#/testCase/LVV-T600}{LVV-T600} \\
        \vcdDocRef{LDM-540}
        \end{tabular} &
          & \notexec{} \\
          \cmidrule{3-5}
          & &
        \begin{tabular}{@{}l@{}}
        \href{https://jira.lsstcorp.org/secure/Tests.jspa\#/testCase/LVV-T1334}{LVV-T1334} \\
        \vcdDocRef{LDM-540}
        \end{tabular} &
          \begin{tabular}{@{}l@{}}
          2019-12-02 \\
            \vcdDocRef{DMTR-161}
            {\scriptsize \href{https://jira.lsstcorp.org/secure/Tests.jspa\#/testPlan/LVV-P48}{LVV-P48} }
          \end{tabular} &
          \cndpass \\
  \midrule
  \begin{tabular}{@{}l@{}}
  DMS-LSP-REQ-0006\\\vcdDocRef{LDM-554}~{\tiny
  }
  \end{tabular} &
    \begin{tabular}{@{}l@{}}
    \hypertarget{dms-lsp-req-0006-v-01}{DMS-LSP-REQ-0006-V-01}
    \\\vcdJiraRef{LVV-9812}~{\tiny
    }
    \end{tabular} &
        \begin{tabular}{@{}l@{}}
        \href{https://jira.lsstcorp.org/secure/Tests.jspa\#/testCase/LVV-T604}{LVV-T604} \\
        \vcdDocRef{LDM-540}
        \end{tabular} &
          & \notexec{} \\
          \cmidrule{3-5}
          & &
        \begin{tabular}{@{}l@{}}
        \href{https://jira.lsstcorp.org/secure/Tests.jspa\#/testCase/LVV-T1334}{LVV-T1334} \\
        \vcdDocRef{LDM-540}
        \end{tabular} &
          \begin{tabular}{@{}l@{}}
          2019-12-02 \\
            \vcdDocRef{DMTR-161}
            {\scriptsize \href{https://jira.lsstcorp.org/secure/Tests.jspa\#/testPlan/LVV-P48}{LVV-P48} }
          \end{tabular} &
          \cndpass \\
          \cmidrule{3-5}
          & &
        \begin{tabular}{@{}l@{}}
        \href{https://jira.lsstcorp.org/secure/Tests.jspa\#/testCase/LVV-T1436}{LVV-T1436} \\
        \vcdDocRef{LDM-540}
        \end{tabular} &
          \begin{tabular}{@{}l@{}}
          2019-12-09 \\
            \vcdDocRef{DMTR-161}
            {\scriptsize \href{https://jira.lsstcorp.org/secure/Tests.jspa\#/testPlan/LVV-P48}{LVV-P48} }
          \end{tabular} &
          \cndpass \\
          \cmidrule{3-5}
          & &
        \begin{tabular}{@{}l@{}}
        \href{https://jira.lsstcorp.org/secure/Tests.jspa\#/testCase/LVV-T1437}{LVV-T1437} \\
        \vcdDocRef{LDM-540}
        \end{tabular} &
          \begin{tabular}{@{}l@{}}
          2019-12-09 \\
            \vcdDocRef{DMTR-161}
            {\scriptsize \href{https://jira.lsstcorp.org/secure/Tests.jspa\#/testPlan/LVV-P48}{LVV-P48} }
          \end{tabular} &
          \cndpass \\
  \midrule
  \begin{tabular}{@{}l@{}}
  DMS-LSP-REQ-0009\\\vcdDocRef{LDM-554}~{\tiny
  }
  \end{tabular} &
    \begin{tabular}{@{}l@{}}
    \hypertarget{dms-lsp-req-0009-v-01}{DMS-LSP-REQ-0009-V-01}
    \\\vcdJiraRef{LVV-9813}~{\tiny
    }
    \end{tabular} &
        \begin{tabular}{@{}l@{}}
        \href{https://jira.lsstcorp.org/secure/Tests.jspa\#/testCase/LVV-T607}{LVV-T607} \\
        \vcdDocRef{LDM-540}
        \end{tabular} &
          & \notexec{} \\
  \midrule
  \begin{tabular}{@{}l@{}}
  DMS-LSP-REQ-0008\\\vcdDocRef{LDM-554}~{\tiny
  }
  \end{tabular} &
    \begin{tabular}{@{}l@{}}
    \hypertarget{dms-lsp-req-0008-v-01}{DMS-LSP-REQ-0008-V-01}
    \\\vcdJiraRef{LVV-9814}~{\tiny
    }
    \end{tabular} &
        \begin{tabular}{@{}l@{}}
        \href{https://jira.lsstcorp.org/secure/Tests.jspa\#/testCase/LVV-T8}{LVV-T8} \\
        \vcdDocRef{LDM-540}
        \end{tabular} &
          & \notexec{} \\
          \cmidrule{3-5}
          & &
        \begin{tabular}{@{}l@{}}
        \href{https://jira.lsstcorp.org/secure/Tests.jspa\#/testCase/LVV-T9}{LVV-T9} \\
        \vcdDocRef{LDM-540}
        \end{tabular} &
          \begin{tabular}{@{}l@{}}
          2019-05-20 \\
            \vcdDocRef{DMTR-52}
            {\scriptsize \href{https://jira.lsstcorp.org/secure/Tests.jspa\#/testPlan/LVV-P42}{LVV-P42} }
          \end{tabular} &
          \passed \\
          \cmidrule{3-5}
          & &
        \begin{tabular}{@{}l@{}}
        \href{https://jira.lsstcorp.org/secure/Tests.jspa\#/testCase/LVV-T606}{LVV-T606} \\
        \vcdDocRef{LDM-540}
        \end{tabular} &
          & \notexec{} \\
  \midrule
  \begin{tabular}{@{}l@{}}
  DMS-LSP-REQ-0010\\\vcdDocRef{LDM-554}~{\tiny
  }
  \end{tabular} &
    \begin{tabular}{@{}l@{}}
    \hypertarget{dms-lsp-req-0010-v-01}{DMS-LSP-REQ-0010-V-01}
    \\\vcdJiraRef{LVV-9815}~{\tiny
    }
    \end{tabular} &
        \begin{tabular}{@{}l@{}}
        \href{https://jira.lsstcorp.org/secure/Tests.jspa\#/testCase/LVV-T608}{LVV-T608} \\
        \vcdDocRef{LDM-540}
        \end{tabular} &
          & \notexec{} \\
  \midrule
  \begin{tabular}{@{}l@{}}
  DMS-LSP-REQ-0012\\\vcdDocRef{LDM-554}~{\tiny
  }
  \end{tabular} &
    \begin{tabular}{@{}l@{}}
    \hypertarget{dms-lsp-req-0012-v-01}{DMS-LSP-REQ-0012-V-01}
    \\\vcdJiraRef{LVV-9816}~{\tiny
    }
    \end{tabular} &
        \begin{tabular}{@{}l@{}}
        \href{https://jira.lsstcorp.org/secure/Tests.jspa\#/testCase/LVV-T610}{LVV-T610} \\
        \vcdDocRef{LDM-540}
        \end{tabular} &
          & \notexec{} \\
  \midrule
  \begin{tabular}{@{}l@{}}
  DMS-LSP-REQ-0011\\\vcdDocRef{LDM-554}~{\tiny
  }
  \end{tabular} &
    \begin{tabular}{@{}l@{}}
    \hypertarget{dms-lsp-req-0011-v-01}{DMS-LSP-REQ-0011-V-01}
    \\\vcdJiraRef{LVV-9817}~{\tiny
    }
    \end{tabular} &
        \begin{tabular}{@{}l@{}}
        \href{https://jira.lsstcorp.org/secure/Tests.jspa\#/testCase/LVV-T609}{LVV-T609} \\
        \vcdDocRef{LDM-540}
        \end{tabular} &
          & \notexec{} \\
  \midrule
  \begin{tabular}{@{}l@{}}
  DMS-LSP-REQ-0013\\\vcdDocRef{LDM-554}~{\tiny
  }
  \end{tabular} &
    \begin{tabular}{@{}l@{}}
    \hypertarget{dms-lsp-req-0013-v-01}{DMS-LSP-REQ-0013-V-01}
    \\\vcdJiraRef{LVV-9818}~{\tiny
    }
    \end{tabular} &
        \begin{tabular}{@{}l@{}}
        \href{https://jira.lsstcorp.org/secure/Tests.jspa\#/testCase/LVV-T611}{LVV-T611} \\
        \vcdDocRef{LDM-540}
        \end{tabular} &
          & \notexec{} \\
  \midrule
  \begin{tabular}{@{}l@{}}
  DMS-LSP-REQ-0014\\\vcdDocRef{LDM-554}~{\tiny
  }
  \end{tabular} &
    \begin{tabular}{@{}l@{}}
    \hypertarget{dms-lsp-req-0014-v-01}{DMS-LSP-REQ-0014-V-01}
    \\\vcdJiraRef{LVV-9819}~{\tiny
    }
    \end{tabular} &
        \begin{tabular}{@{}l@{}}
        \href{https://jira.lsstcorp.org/secure/Tests.jspa\#/testCase/LVV-T5}{LVV-T5} \\
        \vcdDocRef{LDM-540}
        \end{tabular} &
          \begin{tabular}{@{}l@{}}
          2019-05-20 \\
            \vcdDocRef{DMTR-52}
            {\scriptsize \href{https://jira.lsstcorp.org/secure/Tests.jspa\#/testPlan/LVV-P42}{LVV-P42} }
          \end{tabular} &
          \passed \\
          \cmidrule{3-5}
          & &
        \begin{tabular}{@{}l@{}}
        \href{https://jira.lsstcorp.org/secure/Tests.jspa\#/testCase/LVV-T6}{LVV-T6} \\
        \vcdDocRef{LDM-540}
        \end{tabular} &
          \begin{tabular}{@{}l@{}}
          2019-05-20 \\
            \vcdDocRef{DMTR-52}
            {\scriptsize \href{https://jira.lsstcorp.org/secure/Tests.jspa\#/testPlan/LVV-P42}{LVV-P42} }
          \end{tabular} &
          \cndpass \\
          \cmidrule{3-5}
          & &
        \begin{tabular}{@{}l@{}}
        \href{https://jira.lsstcorp.org/secure/Tests.jspa\#/testCase/LVV-T7}{LVV-T7} \\
        \vcdDocRef{LDM-540}
        \end{tabular} &
          & \notexec{} \\
          \cmidrule{3-5}
          & &
        \begin{tabular}{@{}l@{}}
        \href{https://jira.lsstcorp.org/secure/Tests.jspa\#/testCase/LVV-T612}{LVV-T612} \\
        \vcdDocRef{LDM-540}
        \end{tabular} &
          & \notexec{} \\
  \midrule
  \begin{tabular}{@{}l@{}}
  DMS-LSP-REQ-0018\\\vcdDocRef{LDM-554}~{\tiny
  }
  \end{tabular} &
    \begin{tabular}{@{}l@{}}
    \hypertarget{dms-lsp-req-0018-v-01}{DMS-LSP-REQ-0018-V-01}
    \\\vcdJiraRef{LVV-9820}~{\tiny
    }
    \end{tabular} &
        \begin{tabular}{@{}l@{}}
        \href{https://jira.lsstcorp.org/secure/Tests.jspa\#/testCase/LVV-T7}{LVV-T7} \\
        \vcdDocRef{LDM-540}
        \end{tabular} &
          & \notexec{} \\
          \cmidrule{3-5}
          & &
        \begin{tabular}{@{}l@{}}
        \href{https://jira.lsstcorp.org/secure/Tests.jspa\#/testCase/LVV-T616}{LVV-T616} \\
        \vcdDocRef{LDM-540}
        \end{tabular} &
          & \notexec{} \\
  \midrule
  \begin{tabular}{@{}l@{}}
  DMS-LSP-REQ-0017\\\vcdDocRef{LDM-554}~{\tiny
  }
  \end{tabular} &
    \begin{tabular}{@{}l@{}}
    \hypertarget{dms-lsp-req-0017-v-01}{DMS-LSP-REQ-0017-V-01}
    \\\vcdJiraRef{LVV-9821}~{\tiny
    }
    \end{tabular} &
        \begin{tabular}{@{}l@{}}
        \href{https://jira.lsstcorp.org/secure/Tests.jspa\#/testCase/LVV-T6}{LVV-T6} \\
        \vcdDocRef{LDM-540}
        \end{tabular} &
          \begin{tabular}{@{}l@{}}
          2019-05-20 \\
            \vcdDocRef{DMTR-52}
            {\scriptsize \href{https://jira.lsstcorp.org/secure/Tests.jspa\#/testPlan/LVV-P42}{LVV-P42} }
          \end{tabular} &
          \cndpass \\
          \cmidrule{3-5}
          & &
        \begin{tabular}{@{}l@{}}
        \href{https://jira.lsstcorp.org/secure/Tests.jspa\#/testCase/LVV-T615}{LVV-T615} \\
        \vcdDocRef{LDM-540}
        \end{tabular} &
          & \notexec{} \\
  \midrule
  \begin{tabular}{@{}l@{}}
  DMS-LSP-REQ-0016\\\vcdDocRef{LDM-554}~{\tiny
  }
  \end{tabular} &
    \begin{tabular}{@{}l@{}}
    \hypertarget{dms-lsp-req-0016-v-01}{DMS-LSP-REQ-0016-V-01}
    \\\vcdJiraRef{LVV-9822}~{\tiny
    }
    \end{tabular} &
        \begin{tabular}{@{}l@{}}
        \href{https://jira.lsstcorp.org/secure/Tests.jspa\#/testCase/LVV-T614}{LVV-T614} \\
        \vcdDocRef{LDM-540}
        \end{tabular} &
          & \notexec{} \\
  \midrule
  \begin{tabular}{@{}l@{}}
  DMS-LSP-REQ-0015\\\vcdDocRef{LDM-554}~{\tiny
  }
  \end{tabular} &
    \begin{tabular}{@{}l@{}}
    \hypertarget{dms-lsp-req-0015-v-01}{DMS-LSP-REQ-0015-V-01}
    \\\vcdJiraRef{LVV-9823}~{\tiny
    }
    \end{tabular} &
        \begin{tabular}{@{}l@{}}
        \href{https://jira.lsstcorp.org/secure/Tests.jspa\#/testCase/LVV-T613}{LVV-T613} \\
        \vcdDocRef{LDM-540}
        \end{tabular} &
          & \notexec{} \\
  \midrule
  \begin{tabular}{@{}l@{}}
  DMS-LSP-REQ-0028\\\vcdDocRef{LDM-554}~{\tiny
  }
  \end{tabular} &
    \begin{tabular}{@{}l@{}}
    \hypertarget{dms-lsp-req-0028-v-01}{DMS-LSP-REQ-0028-V-01}
    \\\vcdJiraRef{LVV-9824}~{\tiny
    }
    \end{tabular} &
        \begin{tabular}{@{}l@{}}
        \href{https://jira.lsstcorp.org/secure/Tests.jspa\#/testCase/LVV-T4}{LVV-T4} \\
        \vcdDocRef{LDM-540}
        \end{tabular} &
          \begin{tabular}{@{}l@{}}
          2019-05-20 \\
            \vcdDocRef{DMTR-52}
            {\scriptsize \href{https://jira.lsstcorp.org/secure/Tests.jspa\#/testPlan/LVV-P42}{LVV-P42} }
          \end{tabular} &
          \passed \\
          \cmidrule{3-5}
          & &
        \begin{tabular}{@{}l@{}}
        \href{https://jira.lsstcorp.org/secure/Tests.jspa\#/testCase/LVV-T617}{LVV-T617} \\
        \vcdDocRef{LDM-540}
        \end{tabular} &
          & \notexec{} \\
  \midrule
  \begin{tabular}{@{}l@{}}
  DMS-LSP-REQ-0029\\\vcdDocRef{LDM-554}~{\tiny
  }
  \end{tabular} &
    \begin{tabular}{@{}l@{}}
    \hypertarget{dms-lsp-req-0029-v-01}{DMS-LSP-REQ-0029-V-01}
    \\\vcdJiraRef{LVV-9825}~{\tiny
    }
    \end{tabular} &
        \begin{tabular}{@{}l@{}}
        \href{https://jira.lsstcorp.org/secure/Tests.jspa\#/testCase/LVV-T4}{LVV-T4} \\
        \vcdDocRef{LDM-540}
        \end{tabular} &
          \begin{tabular}{@{}l@{}}
          2019-05-20 \\
            \vcdDocRef{DMTR-52}
            {\scriptsize \href{https://jira.lsstcorp.org/secure/Tests.jspa\#/testPlan/LVV-P42}{LVV-P42} }
          \end{tabular} &
          \passed \\
          \cmidrule{3-5}
          & &
        \begin{tabular}{@{}l@{}}
        \href{https://jira.lsstcorp.org/secure/Tests.jspa\#/testCase/LVV-T618}{LVV-T618} \\
        \vcdDocRef{LDM-540}
        \end{tabular} &
          & \notexec{} \\
  \midrule
  \begin{tabular}{@{}l@{}}
  DMS-LSP-REQ-0030\\\vcdDocRef{LDM-554}~{\tiny
  }
  \end{tabular} &
    \begin{tabular}{@{}l@{}}
    \hypertarget{dms-lsp-req-0030-v-01}{DMS-LSP-REQ-0030-V-01}
    \\\vcdJiraRef{LVV-9826}~{\tiny
    }
    \end{tabular} &
        \begin{tabular}{@{}l@{}}
        \href{https://jira.lsstcorp.org/secure/Tests.jspa\#/testCase/LVV-T619}{LVV-T619} \\
        \vcdDocRef{LDM-540}
        \end{tabular} &
          & \notexec{} \\
  \midrule
  \begin{tabular}{@{}l@{}}
  DMS-LSP-REQ-0031\\\vcdDocRef{LDM-554}~{\tiny
  }
  \end{tabular} &
    \begin{tabular}{@{}l@{}}
    \hypertarget{dms-lsp-req-0031-v-01}{DMS-LSP-REQ-0031-V-01}
    \\\vcdJiraRef{LVV-9827}~{\tiny
    }
    \end{tabular} &
        \begin{tabular}{@{}l@{}}
        \href{https://jira.lsstcorp.org/secure/Tests.jspa\#/testCase/LVV-T620}{LVV-T620} \\
        \vcdDocRef{LDM-540}
        \end{tabular} &
          & \notexec{} \\
  \midrule
  \begin{tabular}{@{}l@{}}
  DMS-LSP-REQ-0019\\\vcdDocRef{LDM-554}~{\tiny
  }
  \end{tabular} &
    \begin{tabular}{@{}l@{}}
    \hypertarget{dms-lsp-req-0019-v-01}{DMS-LSP-REQ-0019-V-01}
    \\\vcdJiraRef{LVV-9828}~{\tiny
    }
    \end{tabular} &
        \begin{tabular}{@{}l@{}}
        \href{https://jira.lsstcorp.org/secure/Tests.jspa\#/testCase/LVV-T621}{LVV-T621} \\
        \vcdDocRef{LDM-540}
        \end{tabular} &
          & \notexec{} \\
  \midrule
  \begin{tabular}{@{}l@{}}
  DMS-LSP-REQ-0025\\\vcdDocRef{LDM-554}~{\tiny
  }
  \end{tabular} &
    \begin{tabular}{@{}l@{}}
    \hypertarget{dms-lsp-req-0025-v-01}{DMS-LSP-REQ-0025-V-01}
    \\\vcdJiraRef{LVV-9829}~{\tiny
    }
    \end{tabular} &
        \begin{tabular}{@{}l@{}}
        \href{https://jira.lsstcorp.org/secure/Tests.jspa\#/testCase/LVV-T627}{LVV-T627} \\
        \vcdDocRef{LDM-540}
        \end{tabular} &
          & \notexec{} \\
  \midrule
  \begin{tabular}{@{}l@{}}
  DMS-LSP-REQ-0020\\\vcdDocRef{LDM-554}~{\tiny
  }
  \end{tabular} &
    \begin{tabular}{@{}l@{}}
    \hypertarget{dms-lsp-req-0020-v-01}{DMS-LSP-REQ-0020-V-01}
    \\\vcdJiraRef{LVV-9830}~{\tiny
    }
    \end{tabular} &
        \begin{tabular}{@{}l@{}}
        \href{https://jira.lsstcorp.org/secure/Tests.jspa\#/testCase/LVV-T622}{LVV-T622} \\
        \vcdDocRef{LDM-540}
        \end{tabular} &
          \begin{tabular}{@{}l@{}}
          2019-11-25 \\
            \vcdDocRef{DMTR-161}
            {\scriptsize \href{https://jira.lsstcorp.org/secure/Tests.jspa\#/testPlan/LVV-P48}{LVV-P48} }
          \end{tabular} &
          \cndpass \\
          \cmidrule{3-5}
          & &
        \begin{tabular}{@{}l@{}}
        \href{https://jira.lsstcorp.org/secure/Tests.jspa\#/testCase/LVV-T1334}{LVV-T1334} \\
        \vcdDocRef{LDM-540}
        \end{tabular} &
          \begin{tabular}{@{}l@{}}
          2019-12-02 \\
            \vcdDocRef{DMTR-161}
            {\scriptsize \href{https://jira.lsstcorp.org/secure/Tests.jspa\#/testPlan/LVV-P48}{LVV-P48} }
          \end{tabular} &
          \cndpass \\
          \cmidrule{3-5}
          & &
        \begin{tabular}{@{}l@{}}
        \href{https://jira.lsstcorp.org/secure/Tests.jspa\#/testCase/LVV-T1436}{LVV-T1436} \\
        \vcdDocRef{LDM-540}
        \end{tabular} &
          \begin{tabular}{@{}l@{}}
          2019-12-09 \\
            \vcdDocRef{DMTR-161}
            {\scriptsize \href{https://jira.lsstcorp.org/secure/Tests.jspa\#/testPlan/LVV-P48}{LVV-P48} }
          \end{tabular} &
          \cndpass \\
          \cmidrule{3-5}
          & &
        \begin{tabular}{@{}l@{}}
        \href{https://jira.lsstcorp.org/secure/Tests.jspa\#/testCase/LVV-T1437}{LVV-T1437} \\
        \vcdDocRef{LDM-540}
        \end{tabular} &
          \begin{tabular}{@{}l@{}}
          2019-12-09 \\
            \vcdDocRef{DMTR-161}
            {\scriptsize \href{https://jira.lsstcorp.org/secure/Tests.jspa\#/testPlan/LVV-P48}{LVV-P48} }
          \end{tabular} &
          \cndpass \\
  \midrule
  \begin{tabular}{@{}l@{}}
  DMS-LSP-REQ-0022\\\vcdDocRef{LDM-554}~{\tiny
  }
  \end{tabular} &
    \begin{tabular}{@{}l@{}}
    \hypertarget{dms-lsp-req-0022-v-01}{DMS-LSP-REQ-0022-V-01}
    \\\vcdJiraRef{LVV-9831}~{\tiny
    }
    \end{tabular} &
        \begin{tabular}{@{}l@{}}
        \href{https://jira.lsstcorp.org/secure/Tests.jspa\#/testCase/LVV-T624}{LVV-T624} \\
        \vcdDocRef{LDM-540}
        \end{tabular} &
          & \notexec{} \\
          \cmidrule{3-5}
          & &
        \begin{tabular}{@{}l@{}}
        \href{https://jira.lsstcorp.org/secure/Tests.jspa\#/testCase/LVV-T1334}{LVV-T1334} \\
        \vcdDocRef{LDM-540}
        \end{tabular} &
          \begin{tabular}{@{}l@{}}
          2019-12-02 \\
            \vcdDocRef{DMTR-161}
            {\scriptsize \href{https://jira.lsstcorp.org/secure/Tests.jspa\#/testPlan/LVV-P48}{LVV-P48} }
          \end{tabular} &
          \cndpass \\
          \cmidrule{3-5}
          & &
        \begin{tabular}{@{}l@{}}
        \href{https://jira.lsstcorp.org/secure/Tests.jspa\#/testCase/LVV-T1436}{LVV-T1436} \\
        \vcdDocRef{LDM-540}
        \end{tabular} &
          \begin{tabular}{@{}l@{}}
          2019-12-09 \\
            \vcdDocRef{DMTR-161}
            {\scriptsize \href{https://jira.lsstcorp.org/secure/Tests.jspa\#/testPlan/LVV-P48}{LVV-P48} }
          \end{tabular} &
          \cndpass \\
          \cmidrule{3-5}
          & &
        \begin{tabular}{@{}l@{}}
        \href{https://jira.lsstcorp.org/secure/Tests.jspa\#/testCase/LVV-T1437}{LVV-T1437} \\
        \vcdDocRef{LDM-540}
        \end{tabular} &
          \begin{tabular}{@{}l@{}}
          2019-12-09 \\
            \vcdDocRef{DMTR-161}
            {\scriptsize \href{https://jira.lsstcorp.org/secure/Tests.jspa\#/testPlan/LVV-P48}{LVV-P48} }
          \end{tabular} &
          \cndpass \\
  \midrule
  \begin{tabular}{@{}l@{}}
  DMS-LSP-REQ-0021\\\vcdDocRef{LDM-554}~{\tiny
  }
  \end{tabular} &
    \begin{tabular}{@{}l@{}}
    \hypertarget{dms-lsp-req-0021-v-01}{DMS-LSP-REQ-0021-V-01}
    \\\vcdJiraRef{LVV-9832}~{\tiny
    }
    \end{tabular} &
        \begin{tabular}{@{}l@{}}
        \href{https://jira.lsstcorp.org/secure/Tests.jspa\#/testCase/LVV-T623}{LVV-T623} \\
        \vcdDocRef{LDM-540}
        \end{tabular} &
          & \notexec{} \\
  \midrule
  \begin{tabular}{@{}l@{}}
  DMS-LSP-REQ-0027\\\vcdDocRef{LDM-554}~{\tiny
  }
  \end{tabular} &
    \begin{tabular}{@{}l@{}}
    \hypertarget{dms-lsp-req-0027-v-01}{DMS-LSP-REQ-0027-V-01}
    \\\vcdJiraRef{LVV-9833}~{\tiny
    }
    \end{tabular} &
        \begin{tabular}{@{}l@{}}
        \href{https://jira.lsstcorp.org/secure/Tests.jspa\#/testCase/LVV-T629}{LVV-T629} \\
        \vcdDocRef{LDM-540}
        \end{tabular} &
          & \notexec{} \\
  \midrule
  \begin{tabular}{@{}l@{}}
  DMS-LSP-REQ-0023\\\vcdDocRef{LDM-554}~{\tiny
  }
  \end{tabular} &
    \begin{tabular}{@{}l@{}}
    \hypertarget{dms-lsp-req-0023-v-01}{DMS-LSP-REQ-0023-V-01}
    \\\vcdJiraRef{LVV-9834}~{\tiny
    }
    \end{tabular} &
        \begin{tabular}{@{}l@{}}
        \href{https://jira.lsstcorp.org/secure/Tests.jspa\#/testCase/LVV-T625}{LVV-T625} \\
        \vcdDocRef{LDM-540}
        \end{tabular} &
          & \notexec{} \\
          \cmidrule{3-5}
          & &
        \begin{tabular}{@{}l@{}}
        \href{https://jira.lsstcorp.org/secure/Tests.jspa\#/testCase/LVV-T1334}{LVV-T1334} \\
        \vcdDocRef{LDM-540}
        \end{tabular} &
          \begin{tabular}{@{}l@{}}
          2019-12-02 \\
            \vcdDocRef{DMTR-161}
            {\scriptsize \href{https://jira.lsstcorp.org/secure/Tests.jspa\#/testPlan/LVV-P48}{LVV-P48} }
          \end{tabular} &
          \cndpass \\
          \cmidrule{3-5}
          & &
        \begin{tabular}{@{}l@{}}
        \href{https://jira.lsstcorp.org/secure/Tests.jspa\#/testCase/LVV-T1436}{LVV-T1436} \\
        \vcdDocRef{LDM-540}
        \end{tabular} &
          \begin{tabular}{@{}l@{}}
          2019-12-09 \\
            \vcdDocRef{DMTR-161}
            {\scriptsize \href{https://jira.lsstcorp.org/secure/Tests.jspa\#/testPlan/LVV-P48}{LVV-P48} }
          \end{tabular} &
          \cndpass \\
          \cmidrule{3-5}
          & &
        \begin{tabular}{@{}l@{}}
        \href{https://jira.lsstcorp.org/secure/Tests.jspa\#/testCase/LVV-T1437}{LVV-T1437} \\
        \vcdDocRef{LDM-540}
        \end{tabular} &
          \begin{tabular}{@{}l@{}}
          2019-12-09 \\
            \vcdDocRef{DMTR-161}
            {\scriptsize \href{https://jira.lsstcorp.org/secure/Tests.jspa\#/testPlan/LVV-P48}{LVV-P48} }
          \end{tabular} &
          \cndpass \\
  \midrule
  \begin{tabular}{@{}l@{}}
  DMS-LSP-REQ-0024\\\vcdDocRef{LDM-554}~{\tiny
  }
  \end{tabular} &
    \begin{tabular}{@{}l@{}}
    \hypertarget{dms-lsp-req-0024-v-01}{DMS-LSP-REQ-0024-V-01}
    \\\vcdJiraRef{LVV-9835}~{\tiny
    }
    \end{tabular} &
        \begin{tabular}{@{}l@{}}
        \href{https://jira.lsstcorp.org/secure/Tests.jspa\#/testCase/LVV-T626}{LVV-T626} \\
        \vcdDocRef{LDM-540}
        \end{tabular} &
          & \notexec{} \\
          \cmidrule{3-5}
          & &
        \begin{tabular}{@{}l@{}}
        \href{https://jira.lsstcorp.org/secure/Tests.jspa\#/testCase/LVV-T1334}{LVV-T1334} \\
        \vcdDocRef{LDM-540}
        \end{tabular} &
          \begin{tabular}{@{}l@{}}
          2019-12-02 \\
            \vcdDocRef{DMTR-161}
            {\scriptsize \href{https://jira.lsstcorp.org/secure/Tests.jspa\#/testPlan/LVV-P48}{LVV-P48} }
          \end{tabular} &
          \cndpass \\
          \cmidrule{3-5}
          & &
        \begin{tabular}{@{}l@{}}
        \href{https://jira.lsstcorp.org/secure/Tests.jspa\#/testCase/LVV-T1436}{LVV-T1436} \\
        \vcdDocRef{LDM-540}
        \end{tabular} &
          \begin{tabular}{@{}l@{}}
          2019-12-09 \\
            \vcdDocRef{DMTR-161}
            {\scriptsize \href{https://jira.lsstcorp.org/secure/Tests.jspa\#/testPlan/LVV-P48}{LVV-P48} }
          \end{tabular} &
          \cndpass \\
          \cmidrule{3-5}
          & &
        \begin{tabular}{@{}l@{}}
        \href{https://jira.lsstcorp.org/secure/Tests.jspa\#/testCase/LVV-T1437}{LVV-T1437} \\
        \vcdDocRef{LDM-540}
        \end{tabular} &
          \begin{tabular}{@{}l@{}}
          2019-12-09 \\
            \vcdDocRef{DMTR-161}
            {\scriptsize \href{https://jira.lsstcorp.org/secure/Tests.jspa\#/testPlan/LVV-P48}{LVV-P48} }
          \end{tabular} &
          \cndpass \\
  \midrule
  \begin{tabular}{@{}l@{}}
  DMS-LSP-REQ-0026\\\vcdDocRef{LDM-554}~{\tiny
  }
  \end{tabular} &
    \begin{tabular}{@{}l@{}}
    \hypertarget{dms-lsp-req-0026-v-01}{DMS-LSP-REQ-0026-V-01}
    \\\vcdJiraRef{LVV-9836}~{\tiny
    }
    \end{tabular} &
        \begin{tabular}{@{}l@{}}
        \href{https://jira.lsstcorp.org/secure/Tests.jspa\#/testCase/LVV-T628}{LVV-T628} \\
        \vcdDocRef{LDM-540}
        \end{tabular} &
          & \notexec{} \\
          \cmidrule{3-5}
          & &
        \begin{tabular}{@{}l@{}}
        \href{https://jira.lsstcorp.org/secure/Tests.jspa\#/testCase/LVV-T1436}{LVV-T1436} \\
        \vcdDocRef{LDM-540}
        \end{tabular} &
          \begin{tabular}{@{}l@{}}
          2019-12-09 \\
            \vcdDocRef{DMTR-161}
            {\scriptsize \href{https://jira.lsstcorp.org/secure/Tests.jspa\#/testPlan/LVV-P48}{LVV-P48} }
          \end{tabular} &
          \cndpass \\
  \midrule
  \begin{tabular}{@{}l@{}}
  DMS-LSP-REQ-0033\\\vcdDocRef{LDM-554}~{\tiny
  }
  \end{tabular} &
    \begin{tabular}{@{}l@{}}
    \hypertarget{dms-lsp-req-0033-v-01}{DMS-LSP-REQ-0033-V-01}
    \\\vcdJiraRef{LVV-9837}~{\tiny
    }
    \end{tabular} &
        \begin{tabular}{@{}l@{}}
        \href{https://jira.lsstcorp.org/secure/Tests.jspa\#/testCase/LVV-T631}{LVV-T631} \\
        \vcdDocRef{LDM-540}
        \end{tabular} &
          & \notexec{} \\
  \midrule
  \begin{tabular}{@{}l@{}}
  DMS-LSP-REQ-0034\\\vcdDocRef{LDM-554}~{\tiny
  }
  \end{tabular} &
    \begin{tabular}{@{}l@{}}
    \hypertarget{dms-lsp-req-0034-v-01}{DMS-LSP-REQ-0034-V-01}
    \\\vcdJiraRef{LVV-9838}~{\tiny
    }
    \end{tabular} &
        \begin{tabular}{@{}l@{}}
        \href{https://jira.lsstcorp.org/secure/Tests.jspa\#/testCase/LVV-T632}{LVV-T632} \\
        \vcdDocRef{LDM-540}
        \end{tabular} &
          & \notexec{} \\
  \midrule
  \begin{tabular}{@{}l@{}}
  DMS-LSP-REQ-0032\\\vcdDocRef{LDM-554}~{\tiny
  }
  \end{tabular} &
    \begin{tabular}{@{}l@{}}
    \hypertarget{dms-lsp-req-0032-v-01}{DMS-LSP-REQ-0032-V-01}
    \\\vcdJiraRef{LVV-9839}~{\tiny
    }
    \end{tabular} &
        \begin{tabular}{@{}l@{}}
        \href{https://jira.lsstcorp.org/secure/Tests.jspa\#/testCase/LVV-T630}{LVV-T630} \\
        \vcdDocRef{LDM-540}
        \end{tabular} &
          & \notexec{} \\
  \midrule
  \begin{tabular}{@{}l@{}}
  DMS-LSP-REQ-0035\\\vcdDocRef{LDM-554}~{\tiny
  }
  \end{tabular} &
    \begin{tabular}{@{}l@{}}
    \hypertarget{dms-lsp-req-0035-v-01}{DMS-LSP-REQ-0035-V-01}
    \\\vcdJiraRef{LVV-9840}~{\tiny
    }
    \end{tabular} &
        \begin{tabular}{@{}l@{}}
        \href{https://jira.lsstcorp.org/secure/Tests.jspa\#/testCase/LVV-T633}{LVV-T633} \\
        \vcdDocRef{LDM-540}
        \end{tabular} &
          & \notexec{} \\
  \midrule
  \begin{tabular}{@{}l@{}}
  DMS-PRTL-REQ-0001\\\vcdDocRef{LDM-554}~{\tiny
  }
  \end{tabular} &
    \begin{tabular}{@{}l@{}}
    \hypertarget{dms-prtl-req-0001-v-01}{DMS-PRTL-REQ-0001-V-01}
    \\\vcdJiraRef{LVV-9841}~{\tiny
    }
    \end{tabular} &
        \begin{tabular}{@{}l@{}}
        \href{https://jira.lsstcorp.org/secure/Tests.jspa\#/testCase/LVV-T634}{LVV-T634} \\
        \vcdDocRef{LDM-540}
        \end{tabular} &
          & \notexec{} \\
          \cmidrule{3-5}
          & &
        \begin{tabular}{@{}l@{}}
        \href{https://jira.lsstcorp.org/secure/Tests.jspa\#/testCase/LVV-T1334}{LVV-T1334} \\
        \vcdDocRef{LDM-540}
        \end{tabular} &
          \begin{tabular}{@{}l@{}}
          2019-12-02 \\
            \vcdDocRef{DMTR-161}
            {\scriptsize \href{https://jira.lsstcorp.org/secure/Tests.jspa\#/testPlan/LVV-P48}{LVV-P48} }
          \end{tabular} &
          \cndpass \\
  \midrule
  \begin{tabular}{@{}l@{}}
  DMS-PRTL-REQ-0005\\\vcdDocRef{LDM-554}~{\tiny
  }
  \end{tabular} &
    \begin{tabular}{@{}l@{}}
    \hypertarget{dms-prtl-req-0005-v-01}{DMS-PRTL-REQ-0005-V-01}
    \\\vcdJiraRef{LVV-9842}~{\tiny
    }
    \end{tabular} &
        \begin{tabular}{@{}l@{}}
        \href{https://jira.lsstcorp.org/secure/Tests.jspa\#/testCase/LVV-T638}{LVV-T638} \\
        \vcdDocRef{LDM-540}
        \end{tabular} &
          & \notexec{} \\
  \midrule
  \begin{tabular}{@{}l@{}}
  DMS-PRTL-REQ-0007\\\vcdDocRef{LDM-554}~{\tiny
  }
  \end{tabular} &
    \begin{tabular}{@{}l@{}}
    \hypertarget{dms-prtl-req-0007-v-01}{DMS-PRTL-REQ-0007-V-01}
    \\\vcdJiraRef{LVV-9843}~{\tiny
    }
    \end{tabular} &
        \begin{tabular}{@{}l@{}}
        \href{https://jira.lsstcorp.org/secure/Tests.jspa\#/testCase/LVV-T640}{LVV-T640} \\
        \vcdDocRef{LDM-540}
        \end{tabular} &
          & \notexec{} \\
  \midrule
  \begin{tabular}{@{}l@{}}
  DMS-PRTL-REQ-0008\\\vcdDocRef{LDM-554}~{\tiny
  }
  \end{tabular} &
    \begin{tabular}{@{}l@{}}
    \hypertarget{dms-prtl-req-0008-v-01}{DMS-PRTL-REQ-0008-V-01}
    \\\vcdJiraRef{LVV-9844}~{\tiny
    }
    \end{tabular} &
        \begin{tabular}{@{}l@{}}
        \href{https://jira.lsstcorp.org/secure/Tests.jspa\#/testCase/LVV-T641}{LVV-T641} \\
        \vcdDocRef{LDM-540}
        \end{tabular} &
          & \notexec{} \\
  \midrule
  \begin{tabular}{@{}l@{}}
  DMS-PRTL-REQ-0006\\\vcdDocRef{LDM-554}~{\tiny
  }
  \end{tabular} &
    \begin{tabular}{@{}l@{}}
    \hypertarget{dms-prtl-req-0006-v-01}{DMS-PRTL-REQ-0006-V-01}
    \\\vcdJiraRef{LVV-9845}~{\tiny
    }
    \end{tabular} &
        \begin{tabular}{@{}l@{}}
        \href{https://jira.lsstcorp.org/secure/Tests.jspa\#/testCase/LVV-T639}{LVV-T639} \\
        \vcdDocRef{LDM-540}
        \end{tabular} &
          & \notexec{} \\
  \midrule
  \begin{tabular}{@{}l@{}}
  DMS-PRTL-REQ-0003\\\vcdDocRef{LDM-554}~{\tiny
  }
  \end{tabular} &
    \begin{tabular}{@{}l@{}}
    \hypertarget{dms-prtl-req-0003-v-01}{DMS-PRTL-REQ-0003-V-01}
    \\\vcdJiraRef{LVV-9846}~{\tiny
    }
    \end{tabular} &
        \begin{tabular}{@{}l@{}}
        \href{https://jira.lsstcorp.org/secure/Tests.jspa\#/testCase/LVV-T636}{LVV-T636} \\
        \vcdDocRef{LDM-540}
        \end{tabular} &
          & \notexec{} \\
          \cmidrule{3-5}
          & &
        \begin{tabular}{@{}l@{}}
        \href{https://jira.lsstcorp.org/secure/Tests.jspa\#/testCase/LVV-T1818}{LVV-T1818} \\
        \vcdDocRef{LDM-540}
        \end{tabular} &
          \begin{tabular}{@{}l@{}}
          2020-05-11 \\
            \vcdDocRef{DMTR-211}
            {\scriptsize \href{https://jira.lsstcorp.org/secure/Tests.jspa\#/testPlan/LVV-P69}{LVV-P69} }
          \end{tabular} &
          \cndpass \\
  \midrule
  \begin{tabular}{@{}l@{}}
  DMS-PRTL-REQ-0002\\\vcdDocRef{LDM-554}~{\tiny
  }
  \end{tabular} &
    \begin{tabular}{@{}l@{}}
    \hypertarget{dms-prtl-req-0002-v-01}{DMS-PRTL-REQ-0002-V-01}
    \\\vcdJiraRef{LVV-9847}~{\tiny
    }
    \end{tabular} &
        \begin{tabular}{@{}l@{}}
        \href{https://jira.lsstcorp.org/secure/Tests.jspa\#/testCase/LVV-T635}{LVV-T635} \\
        \vcdDocRef{LDM-540}
        \end{tabular} &
          & \notexec{} \\
  \midrule
  \begin{tabular}{@{}l@{}}
  DMS-PRTL-REQ-0004\\\vcdDocRef{LDM-554}~{\tiny
  }
  \end{tabular} &
    \begin{tabular}{@{}l@{}}
    \hypertarget{dms-prtl-req-0004-v-01}{DMS-PRTL-REQ-0004-V-01}
    \\\vcdJiraRef{LVV-9848}~{\tiny
    }
    \end{tabular} &
        \begin{tabular}{@{}l@{}}
        \href{https://jira.lsstcorp.org/secure/Tests.jspa\#/testCase/LVV-T8}{LVV-T8} \\
        \vcdDocRef{LDM-540}
        \end{tabular} &
          & \notexec{} \\
          \cmidrule{3-5}
          & &
        \begin{tabular}{@{}l@{}}
        \href{https://jira.lsstcorp.org/secure/Tests.jspa\#/testCase/LVV-T637}{LVV-T637} \\
        \vcdDocRef{LDM-540}
        \end{tabular} &
          & \notexec{} \\
  \midrule
  \begin{tabular}{@{}l@{}}
  DMS-PRTL-REQ-0010\\\vcdDocRef{LDM-554}~{\tiny
  }
  \end{tabular} &
    \begin{tabular}{@{}l@{}}
    \hypertarget{dms-prtl-req-0010-v-01}{DMS-PRTL-REQ-0010-V-01}
    \\\vcdJiraRef{LVV-9849}~{\tiny
    }
    \end{tabular} &
        \begin{tabular}{@{}l@{}}
        \href{https://jira.lsstcorp.org/secure/Tests.jspa\#/testCase/LVV-T643}{LVV-T643} \\
        \vcdDocRef{LDM-540}
        \end{tabular} &
          & \notexec{} \\
  \midrule
  \begin{tabular}{@{}l@{}}
  DMS-PRTL-REQ-0013\\\vcdDocRef{LDM-554}~{\tiny
  }
  \end{tabular} &
    \begin{tabular}{@{}l@{}}
    \hypertarget{dms-prtl-req-0013-v-01}{DMS-PRTL-REQ-0013-V-01}
    \\\vcdJiraRef{LVV-9850}~{\tiny
    }
    \end{tabular} &
        \begin{tabular}{@{}l@{}}
        \href{https://jira.lsstcorp.org/secure/Tests.jspa\#/testCase/LVV-T646}{LVV-T646} \\
        \vcdDocRef{LDM-540}
        \end{tabular} &
          & \notexec{} \\
  \midrule
  \begin{tabular}{@{}l@{}}
  DMS-PRTL-REQ-0012\\\vcdDocRef{LDM-554}~{\tiny
  }
  \end{tabular} &
    \begin{tabular}{@{}l@{}}
    \hypertarget{dms-prtl-req-0012-v-01}{DMS-PRTL-REQ-0012-V-01}
    \\\vcdJiraRef{LVV-9851}~{\tiny
    }
    \end{tabular} &
        \begin{tabular}{@{}l@{}}
        \href{https://jira.lsstcorp.org/secure/Tests.jspa\#/testCase/LVV-T645}{LVV-T645} \\
        \vcdDocRef{LDM-540}
        \end{tabular} &
          & \notexec{} \\
  \midrule
  \begin{tabular}{@{}l@{}}
  DMS-PRTL-REQ-0014\\\vcdDocRef{LDM-554}~{\tiny
  }
  \end{tabular} &
    \begin{tabular}{@{}l@{}}
    \hypertarget{dms-prtl-req-0014-v-01}{DMS-PRTL-REQ-0014-V-01}
    \\\vcdJiraRef{LVV-9852}~{\tiny
    }
    \end{tabular} &
        \begin{tabular}{@{}l@{}}
        \href{https://jira.lsstcorp.org/secure/Tests.jspa\#/testCase/LVV-T647}{LVV-T647} \\
        \vcdDocRef{LDM-540}
        \end{tabular} &
          & \notexec{} \\
  \midrule
  \begin{tabular}{@{}l@{}}
  DMS-PRTL-REQ-0011\\\vcdDocRef{LDM-554}~{\tiny
  }
  \end{tabular} &
    \begin{tabular}{@{}l@{}}
    \hypertarget{dms-prtl-req-0011-v-01}{DMS-PRTL-REQ-0011-V-01}
    \\\vcdJiraRef{LVV-9853}~{\tiny
    }
    \end{tabular} &
        \begin{tabular}{@{}l@{}}
        \href{https://jira.lsstcorp.org/secure/Tests.jspa\#/testCase/LVV-T644}{LVV-T644} \\
        \vcdDocRef{LDM-540}
        \end{tabular} &
          & \notexec{} \\
  \midrule
  \begin{tabular}{@{}l@{}}
  DMS-PRTL-REQ-0009\\\vcdDocRef{LDM-554}~{\tiny
  }
  \end{tabular} &
    \begin{tabular}{@{}l@{}}
    \hypertarget{dms-prtl-req-0009-v-01}{DMS-PRTL-REQ-0009-V-01}
    \\\vcdJiraRef{LVV-9854}~{\tiny
    }
    \end{tabular} &
        \begin{tabular}{@{}l@{}}
        \href{https://jira.lsstcorp.org/secure/Tests.jspa\#/testCase/LVV-T642}{LVV-T642} \\
        \vcdDocRef{LDM-540}
        \end{tabular} &
          & \notexec{} \\
  \midrule
  \begin{tabular}{@{}l@{}}
  DMS-PRTL-REQ-0017\\\vcdDocRef{LDM-554}~{\tiny
  }
  \end{tabular} &
    \begin{tabular}{@{}l@{}}
    \hypertarget{dms-prtl-req-0017-v-01}{DMS-PRTL-REQ-0017-V-01}
    \\\vcdJiraRef{LVV-9855}~{\tiny
    }
    \end{tabular} &
        \begin{tabular}{@{}l@{}}
        \href{https://jira.lsstcorp.org/secure/Tests.jspa\#/testCase/LVV-T650}{LVV-T650} \\
        \vcdDocRef{LDM-540}
        \end{tabular} &
          & \notexec{} \\
          \cmidrule{3-5}
          & &
        \begin{tabular}{@{}l@{}}
        \href{https://jira.lsstcorp.org/secure/Tests.jspa\#/testCase/LVV-T1334}{LVV-T1334} \\
        \vcdDocRef{LDM-540}
        \end{tabular} &
          \begin{tabular}{@{}l@{}}
          2019-12-02 \\
            \vcdDocRef{DMTR-161}
            {\scriptsize \href{https://jira.lsstcorp.org/secure/Tests.jspa\#/testPlan/LVV-P48}{LVV-P48} }
          \end{tabular} &
          \cndpass \\
  \midrule
  \begin{tabular}{@{}l@{}}
  DMS-PRTL-REQ-0016\\\vcdDocRef{LDM-554}~{\tiny
  }
  \end{tabular} &
    \begin{tabular}{@{}l@{}}
    \hypertarget{dms-prtl-req-0016-v-01}{DMS-PRTL-REQ-0016-V-01}
    \\\vcdJiraRef{LVV-9856}~{\tiny
    }
    \end{tabular} &
        \begin{tabular}{@{}l@{}}
        \href{https://jira.lsstcorp.org/secure/Tests.jspa\#/testCase/LVV-T5}{LVV-T5} \\
        \vcdDocRef{LDM-540}
        \end{tabular} &
          \begin{tabular}{@{}l@{}}
          2019-05-20 \\
            \vcdDocRef{DMTR-52}
            {\scriptsize \href{https://jira.lsstcorp.org/secure/Tests.jspa\#/testPlan/LVV-P42}{LVV-P42} }
          \end{tabular} &
          \passed \\
          \cmidrule{3-5}
          & &
        \begin{tabular}{@{}l@{}}
        \href{https://jira.lsstcorp.org/secure/Tests.jspa\#/testCase/LVV-T649}{LVV-T649} \\
        \vcdDocRef{LDM-540}
        \end{tabular} &
          & \notexec{} \\
          \cmidrule{3-5}
          & &
        \begin{tabular}{@{}l@{}}
        \href{https://jira.lsstcorp.org/secure/Tests.jspa\#/testCase/LVV-T1334}{LVV-T1334} \\
        \vcdDocRef{LDM-540}
        \end{tabular} &
          \begin{tabular}{@{}l@{}}
          2019-12-02 \\
            \vcdDocRef{DMTR-161}
            {\scriptsize \href{https://jira.lsstcorp.org/secure/Tests.jspa\#/testPlan/LVV-P48}{LVV-P48} }
          \end{tabular} &
          \cndpass \\
  \midrule
  \begin{tabular}{@{}l@{}}
  DMS-PRTL-REQ-0015\\\vcdDocRef{LDM-554}~{\tiny
  }
  \end{tabular} &
    \begin{tabular}{@{}l@{}}
    \hypertarget{dms-prtl-req-0015-v-01}{DMS-PRTL-REQ-0015-V-01}
    \\\vcdJiraRef{LVV-9857}~{\tiny
    }
    \end{tabular} &
        \begin{tabular}{@{}l@{}}
        \href{https://jira.lsstcorp.org/secure/Tests.jspa\#/testCase/LVV-T648}{LVV-T648} \\
        \vcdDocRef{LDM-540}
        \end{tabular} &
          & \notexec{} \\
          \cmidrule{3-5}
          & &
        \begin{tabular}{@{}l@{}}
        \href{https://jira.lsstcorp.org/secure/Tests.jspa\#/testCase/LVV-T1334}{LVV-T1334} \\
        \vcdDocRef{LDM-540}
        \end{tabular} &
          \begin{tabular}{@{}l@{}}
          2019-12-02 \\
            \vcdDocRef{DMTR-161}
            {\scriptsize \href{https://jira.lsstcorp.org/secure/Tests.jspa\#/testPlan/LVV-P48}{LVV-P48} }
          \end{tabular} &
          \cndpass \\
  \midrule
  \begin{tabular}{@{}l@{}}
  DMS-PRTL-REQ-0018\\\vcdDocRef{LDM-554}~{\tiny
  }
  \end{tabular} &
    \begin{tabular}{@{}l@{}}
    \hypertarget{dms-prtl-req-0018-v-01}{DMS-PRTL-REQ-0018-V-01}
    \\\vcdJiraRef{LVV-9858}~{\tiny
    }
    \end{tabular} &
        \begin{tabular}{@{}l@{}}
        \href{https://jira.lsstcorp.org/secure/Tests.jspa\#/testCase/LVV-T651}{LVV-T651} \\
        \vcdDocRef{LDM-540}
        \end{tabular} &
          & \notexec{} \\
  \midrule
  \begin{tabular}{@{}l@{}}
  DMS-PRTL-REQ-0028\\\vcdDocRef{LDM-554}~{\tiny
  }
  \end{tabular} &
    \begin{tabular}{@{}l@{}}
    \hypertarget{dms-prtl-req-0028-v-01}{DMS-PRTL-REQ-0028-V-01}
    \\\vcdJiraRef{LVV-9859}~{\tiny
    }
    \end{tabular} &
        \begin{tabular}{@{}l@{}}
        \href{https://jira.lsstcorp.org/secure/Tests.jspa\#/testCase/LVV-T5}{LVV-T5} \\
        \vcdDocRef{LDM-540}
        \end{tabular} &
          \begin{tabular}{@{}l@{}}
          2019-05-20 \\
            \vcdDocRef{DMTR-52}
            {\scriptsize \href{https://jira.lsstcorp.org/secure/Tests.jspa\#/testPlan/LVV-P42}{LVV-P42} }
          \end{tabular} &
          \passed \\
          \cmidrule{3-5}
          & &
        \begin{tabular}{@{}l@{}}
        \href{https://jira.lsstcorp.org/secure/Tests.jspa\#/testCase/LVV-T652}{LVV-T652} \\
        \vcdDocRef{LDM-540}
        \end{tabular} &
          & \notexec{} \\
  \midrule
  \begin{tabular}{@{}l@{}}
  DMS-PRTL-REQ-0029\\\vcdDocRef{LDM-554}~{\tiny
  }
  \end{tabular} &
    \begin{tabular}{@{}l@{}}
    \hypertarget{dms-prtl-req-0029-v-01}{DMS-PRTL-REQ-0029-V-01}
    \\\vcdJiraRef{LVV-9860}~{\tiny
    }
    \end{tabular} &
        \begin{tabular}{@{}l@{}}
        \href{https://jira.lsstcorp.org/secure/Tests.jspa\#/testCase/LVV-T653}{LVV-T653} \\
        \vcdDocRef{LDM-540}
        \end{tabular} &
          & \notexec{} \\
  \midrule
  \begin{tabular}{@{}l@{}}
  DMS-PRTL-REQ-0030\\\vcdDocRef{LDM-554}~{\tiny
  }
  \end{tabular} &
    \begin{tabular}{@{}l@{}}
    \hypertarget{dms-prtl-req-0030-v-01}{DMS-PRTL-REQ-0030-V-01}
    \\\vcdJiraRef{LVV-9861}~{\tiny
    }
    \end{tabular} &
        \begin{tabular}{@{}l@{}}
        \href{https://jira.lsstcorp.org/secure/Tests.jspa\#/testCase/LVV-T654}{LVV-T654} \\
        \vcdDocRef{LDM-540}
        \end{tabular} &
          & \notexec{} \\
  \midrule
  \begin{tabular}{@{}l@{}}
  DMS-PRTL-REQ-0022\\\vcdDocRef{LDM-554}~{\tiny
  }
  \end{tabular} &
    \begin{tabular}{@{}l@{}}
    \hypertarget{dms-prtl-req-0022-v-01}{DMS-PRTL-REQ-0022-V-01}
    \\\vcdJiraRef{LVV-9862}~{\tiny
    }
    \end{tabular} &
        \begin{tabular}{@{}l@{}}
        \href{https://jira.lsstcorp.org/secure/Tests.jspa\#/testCase/LVV-T5}{LVV-T5} \\
        \vcdDocRef{LDM-540}
        \end{tabular} &
          \begin{tabular}{@{}l@{}}
          2019-05-20 \\
            \vcdDocRef{DMTR-52}
            {\scriptsize \href{https://jira.lsstcorp.org/secure/Tests.jspa\#/testPlan/LVV-P42}{LVV-P42} }
          \end{tabular} &
          \passed \\
          \cmidrule{3-5}
          & &
        \begin{tabular}{@{}l@{}}
        \href{https://jira.lsstcorp.org/secure/Tests.jspa\#/testCase/LVV-T657}{LVV-T657} \\
        \vcdDocRef{LDM-540}
        \end{tabular} &
          & \notexec{} \\
  \midrule
  \begin{tabular}{@{}l@{}}
  DMS-PRTL-REQ-0023\\\vcdDocRef{LDM-554}~{\tiny
  }
  \end{tabular} &
    \begin{tabular}{@{}l@{}}
    \hypertarget{dms-prtl-req-0023-v-01}{DMS-PRTL-REQ-0023-V-01}
    \\\vcdJiraRef{LVV-9863}~{\tiny
    }
    \end{tabular} &
        \begin{tabular}{@{}l@{}}
        \href{https://jira.lsstcorp.org/secure/Tests.jspa\#/testCase/LVV-T658}{LVV-T658} \\
        \vcdDocRef{LDM-540}
        \end{tabular} &
          & \notexec{} \\
  \midrule
  \begin{tabular}{@{}l@{}}
  DMS-PRTL-REQ-0024\\\vcdDocRef{LDM-554}~{\tiny
  }
  \end{tabular} &
    \begin{tabular}{@{}l@{}}
    \hypertarget{dms-prtl-req-0024-v-01}{DMS-PRTL-REQ-0024-V-01}
    \\\vcdJiraRef{LVV-9864}~{\tiny
    }
    \end{tabular} &
        \begin{tabular}{@{}l@{}}
        \href{https://jira.lsstcorp.org/secure/Tests.jspa\#/testCase/LVV-T659}{LVV-T659} \\
        \vcdDocRef{LDM-540}
        \end{tabular} &
          & \notexec{} \\
  \midrule
  \begin{tabular}{@{}l@{}}
  DMS-PRTL-REQ-0021\\\vcdDocRef{LDM-554}~{\tiny
  }
  \end{tabular} &
    \begin{tabular}{@{}l@{}}
    \hypertarget{dms-prtl-req-0021-v-01}{DMS-PRTL-REQ-0021-V-01}
    \\\vcdJiraRef{LVV-9865}~{\tiny
    }
    \end{tabular} &
        \begin{tabular}{@{}l@{}}
        \href{https://jira.lsstcorp.org/secure/Tests.jspa\#/testCase/LVV-T5}{LVV-T5} \\
        \vcdDocRef{LDM-540}
        \end{tabular} &
          \begin{tabular}{@{}l@{}}
          2019-05-20 \\
            \vcdDocRef{DMTR-52}
            {\scriptsize \href{https://jira.lsstcorp.org/secure/Tests.jspa\#/testPlan/LVV-P42}{LVV-P42} }
          \end{tabular} &
          \passed \\
          \cmidrule{3-5}
          & &
        \begin{tabular}{@{}l@{}}
        \href{https://jira.lsstcorp.org/secure/Tests.jspa\#/testCase/LVV-T656}{LVV-T656} \\
        \vcdDocRef{LDM-540}
        \end{tabular} &
          & \notexec{} \\
  \midrule
  \begin{tabular}{@{}l@{}}
  DMS-PRTL-REQ-0020\\\vcdDocRef{LDM-554}~{\tiny
  }
  \end{tabular} &
    \begin{tabular}{@{}l@{}}
    \hypertarget{dms-prtl-req-0020-v-01}{DMS-PRTL-REQ-0020-V-01}
    \\\vcdJiraRef{LVV-9866}~{\tiny
    }
    \end{tabular} &
        \begin{tabular}{@{}l@{}}
        \href{https://jira.lsstcorp.org/secure/Tests.jspa\#/testCase/LVV-T655}{LVV-T655} \\
        \vcdDocRef{LDM-540}
        \end{tabular} &
          & \notexec{} \\
          \cmidrule{3-5}
          & &
        \begin{tabular}{@{}l@{}}
        \href{https://jira.lsstcorp.org/secure/Tests.jspa\#/testCase/LVV-T1334}{LVV-T1334} \\
        \vcdDocRef{LDM-540}
        \end{tabular} &
          \begin{tabular}{@{}l@{}}
          2019-12-02 \\
            \vcdDocRef{DMTR-161}
            {\scriptsize \href{https://jira.lsstcorp.org/secure/Tests.jspa\#/testPlan/LVV-P48}{LVV-P48} }
          \end{tabular} &
          \cndpass \\
  \midrule
  \begin{tabular}{@{}l@{}}
  DMS-PRTL-REQ-0025\\\vcdDocRef{LDM-554}~{\tiny
  }
  \end{tabular} &
    \begin{tabular}{@{}l@{}}
    \hypertarget{dms-prtl-req-0025-v-01}{DMS-PRTL-REQ-0025-V-01}
    \\\vcdJiraRef{LVV-9867}~{\tiny
    }
    \end{tabular} &
        \begin{tabular}{@{}l@{}}
        \href{https://jira.lsstcorp.org/secure/Tests.jspa\#/testCase/LVV-T660}{LVV-T660} \\
        \vcdDocRef{LDM-540}
        \end{tabular} &
          & \notexec{} \\
  \midrule
  \begin{tabular}{@{}l@{}}
  DMS-PRTL-REQ-0027\\\vcdDocRef{LDM-554}~{\tiny
  }
  \end{tabular} &
    \begin{tabular}{@{}l@{}}
    \hypertarget{dms-prtl-req-0027-v-01}{DMS-PRTL-REQ-0027-V-01}
    \\\vcdJiraRef{LVV-9868}~{\tiny
    }
    \end{tabular} &
        \begin{tabular}{@{}l@{}}
        \href{https://jira.lsstcorp.org/secure/Tests.jspa\#/testCase/LVV-T5}{LVV-T5} \\
        \vcdDocRef{LDM-540}
        \end{tabular} &
          \begin{tabular}{@{}l@{}}
          2019-05-20 \\
            \vcdDocRef{DMTR-52}
            {\scriptsize \href{https://jira.lsstcorp.org/secure/Tests.jspa\#/testPlan/LVV-P42}{LVV-P42} }
          \end{tabular} &
          \passed \\
          \cmidrule{3-5}
          & &
        \begin{tabular}{@{}l@{}}
        \href{https://jira.lsstcorp.org/secure/Tests.jspa\#/testCase/LVV-T662}{LVV-T662} \\
        \vcdDocRef{LDM-540}
        \end{tabular} &
          & \notexec{} \\
  \midrule
  \begin{tabular}{@{}l@{}}
  DMS-PRTL-REQ-0026\\\vcdDocRef{LDM-554}~{\tiny
  }
  \end{tabular} &
    \begin{tabular}{@{}l@{}}
    \hypertarget{dms-prtl-req-0026-v-01}{DMS-PRTL-REQ-0026-V-01}
    \\\vcdJiraRef{LVV-9869}~{\tiny
    }
    \end{tabular} &
        \begin{tabular}{@{}l@{}}
        \href{https://jira.lsstcorp.org/secure/Tests.jspa\#/testCase/LVV-T5}{LVV-T5} \\
        \vcdDocRef{LDM-540}
        \end{tabular} &
          \begin{tabular}{@{}l@{}}
          2019-05-20 \\
            \vcdDocRef{DMTR-52}
            {\scriptsize \href{https://jira.lsstcorp.org/secure/Tests.jspa\#/testPlan/LVV-P42}{LVV-P42} }
          \end{tabular} &
          \passed \\
          \cmidrule{3-5}
          & &
        \begin{tabular}{@{}l@{}}
        \href{https://jira.lsstcorp.org/secure/Tests.jspa\#/testCase/LVV-T661}{LVV-T661} \\
        \vcdDocRef{LDM-540}
        \end{tabular} &
          & \notexec{} \\
          \cmidrule{3-5}
          & &
        \begin{tabular}{@{}l@{}}
        \href{https://jira.lsstcorp.org/secure/Tests.jspa\#/testCase/LVV-T1334}{LVV-T1334} \\
        \vcdDocRef{LDM-540}
        \end{tabular} &
          \begin{tabular}{@{}l@{}}
          2019-12-02 \\
            \vcdDocRef{DMTR-161}
            {\scriptsize \href{https://jira.lsstcorp.org/secure/Tests.jspa\#/testPlan/LVV-P48}{LVV-P48} }
          \end{tabular} &
          \cndpass \\
  \midrule
  \begin{tabular}{@{}l@{}}
  DMS-PRTL-REQ-0019\\\vcdDocRef{LDM-554}~{\tiny
  }
  \end{tabular} &
    \begin{tabular}{@{}l@{}}
    \hypertarget{dms-prtl-req-0019-v-01}{DMS-PRTL-REQ-0019-V-01}
    \\\vcdJiraRef{LVV-9870}~{\tiny
    }
    \end{tabular} &
        \begin{tabular}{@{}l@{}}
        \href{https://jira.lsstcorp.org/secure/Tests.jspa\#/testCase/LVV-T663}{LVV-T663} \\
        \vcdDocRef{LDM-540}
        \end{tabular} &
          & \notexec{} \\
  \midrule
  \begin{tabular}{@{}l@{}}
  DMS-PRTL-REQ-0034\\\vcdDocRef{LDM-554}~{\tiny
  }
  \end{tabular} &
    \begin{tabular}{@{}l@{}}
    \hypertarget{dms-prtl-req-0034-v-01}{DMS-PRTL-REQ-0034-V-01}
    \\\vcdJiraRef{LVV-9871}~{\tiny
    }
    \end{tabular} &
        \begin{tabular}{@{}l@{}}
        \href{https://jira.lsstcorp.org/secure/Tests.jspa\#/testCase/LVV-T668}{LVV-T668} \\
        \vcdDocRef{LDM-540}
        \end{tabular} &
          & \notexec{} \\
  \midrule
  \begin{tabular}{@{}l@{}}
  DMS-PRTL-REQ-0033\\\vcdDocRef{LDM-554}~{\tiny
  }
  \end{tabular} &
    \begin{tabular}{@{}l@{}}
    \hypertarget{dms-prtl-req-0033-v-01}{DMS-PRTL-REQ-0033-V-01}
    \\\vcdJiraRef{LVV-9872}~{\tiny
    }
    \end{tabular} &
        \begin{tabular}{@{}l@{}}
        \href{https://jira.lsstcorp.org/secure/Tests.jspa\#/testCase/LVV-T667}{LVV-T667} \\
        \vcdDocRef{LDM-540}
        \end{tabular} &
          & \notexec{} \\
  \midrule
  \begin{tabular}{@{}l@{}}
  DMS-PRTL-REQ-0032\\\vcdDocRef{LDM-554}~{\tiny
  }
  \end{tabular} &
    \begin{tabular}{@{}l@{}}
    \hypertarget{dms-prtl-req-0032-v-01}{DMS-PRTL-REQ-0032-V-01}
    \\\vcdJiraRef{LVV-9873}~{\tiny
    }
    \end{tabular} &
        \begin{tabular}{@{}l@{}}
        \href{https://jira.lsstcorp.org/secure/Tests.jspa\#/testCase/LVV-T666}{LVV-T666} \\
        \vcdDocRef{LDM-540}
        \end{tabular} &
          & \notexec{} \\
  \midrule
  \begin{tabular}{@{}l@{}}
  DMS-PRTL-REQ-0031\\\vcdDocRef{LDM-554}~{\tiny
  }
  \end{tabular} &
    \begin{tabular}{@{}l@{}}
    \hypertarget{dms-prtl-req-0031-v-01}{DMS-PRTL-REQ-0031-V-01}
    \\\vcdJiraRef{LVV-9874}~{\tiny
    }
    \end{tabular} &
        \begin{tabular}{@{}l@{}}
        \href{https://jira.lsstcorp.org/secure/Tests.jspa\#/testCase/LVV-T664}{LVV-T664} \\
        \vcdDocRef{LDM-540}
        \end{tabular} &
          & \notexec{} \\
  \midrule
  \begin{tabular}{@{}l@{}}
  DMS-PRTL-REQ-0039\\\vcdDocRef{LDM-554}~{\tiny
  }
  \end{tabular} &
    \begin{tabular}{@{}l@{}}
    \hypertarget{dms-prtl-req-0039-v-01}{DMS-PRTL-REQ-0039-V-01}
    \\\vcdJiraRef{LVV-9875}~{\tiny
    }
    \end{tabular} &
        \begin{tabular}{@{}l@{}}
        \href{https://jira.lsstcorp.org/secure/Tests.jspa\#/testCase/LVV-T673}{LVV-T673} \\
        \vcdDocRef{LDM-540}
        \end{tabular} &
          & \notexec{} \\
  \midrule
  \begin{tabular}{@{}l@{}}
  DMS-PRTL-REQ-0037\\\vcdDocRef{LDM-554}~{\tiny
  }
  \end{tabular} &
    \begin{tabular}{@{}l@{}}
    \hypertarget{dms-prtl-req-0037-v-01}{DMS-PRTL-REQ-0037-V-01}
    \\\vcdJiraRef{LVV-9876}~{\tiny
    }
    \end{tabular} &
        \begin{tabular}{@{}l@{}}
        \href{https://jira.lsstcorp.org/secure/Tests.jspa\#/testCase/LVV-T671}{LVV-T671} \\
        \vcdDocRef{LDM-540}
        \end{tabular} &
          & \notexec{} \\
  \midrule
  \begin{tabular}{@{}l@{}}
  DMS-PRTL-REQ-0036\\\vcdDocRef{LDM-554}~{\tiny
  }
  \end{tabular} &
    \begin{tabular}{@{}l@{}}
    \hypertarget{dms-prtl-req-0036-v-01}{DMS-PRTL-REQ-0036-V-01}
    \\\vcdJiraRef{LVV-9877}~{\tiny
    }
    \end{tabular} &
        \begin{tabular}{@{}l@{}}
        \href{https://jira.lsstcorp.org/secure/Tests.jspa\#/testCase/LVV-T670}{LVV-T670} \\
        \vcdDocRef{LDM-540}
        \end{tabular} &
          & \notexec{} \\
  \midrule
  \begin{tabular}{@{}l@{}}
  DMS-PRTL-REQ-0035\\\vcdDocRef{LDM-554}~{\tiny
  }
  \end{tabular} &
    \begin{tabular}{@{}l@{}}
    \hypertarget{dms-prtl-req-0035-v-01}{DMS-PRTL-REQ-0035-V-01}
    \\\vcdJiraRef{LVV-9878}~{\tiny
    }
    \end{tabular} &
        \begin{tabular}{@{}l@{}}
        \href{https://jira.lsstcorp.org/secure/Tests.jspa\#/testCase/LVV-T669}{LVV-T669} \\
        \vcdDocRef{LDM-540}
        \end{tabular} &
          & \notexec{} \\
  \midrule
  \begin{tabular}{@{}l@{}}
  DMS-PRTL-REQ-0038\\\vcdDocRef{LDM-554}~{\tiny
  }
  \end{tabular} &
    \begin{tabular}{@{}l@{}}
    \hypertarget{dms-prtl-req-0038-v-01}{DMS-PRTL-REQ-0038-V-01}
    \\\vcdJiraRef{LVV-9879}~{\tiny
    }
    \end{tabular} &
        \begin{tabular}{@{}l@{}}
        \href{https://jira.lsstcorp.org/secure/Tests.jspa\#/testCase/LVV-T672}{LVV-T672} \\
        \vcdDocRef{LDM-540}
        \end{tabular} &
          & \notexec{} \\
  \midrule
  \begin{tabular}{@{}l@{}}
  DMS-PRTL-REQ-0041\\\vcdDocRef{LDM-554}~{\tiny
  }
  \end{tabular} &
    \begin{tabular}{@{}l@{}}
    \hypertarget{dms-prtl-req-0041-v-01}{DMS-PRTL-REQ-0041-V-01}
    \\\vcdJiraRef{LVV-9880}~{\tiny
    }
    \end{tabular} &
        \begin{tabular}{@{}l@{}}
        \href{https://jira.lsstcorp.org/secure/Tests.jspa\#/testCase/LVV-T7}{LVV-T7} \\
        \vcdDocRef{LDM-540}
        \end{tabular} &
          & \notexec{} \\
          \cmidrule{3-5}
          & &
        \begin{tabular}{@{}l@{}}
        \href{https://jira.lsstcorp.org/secure/Tests.jspa\#/testCase/LVV-T674}{LVV-T674} \\
        \vcdDocRef{LDM-540}
        \end{tabular} &
          & \notexec{} \\
  \midrule
  \begin{tabular}{@{}l@{}}
  DMS-PRTL-REQ-0040\\\vcdDocRef{LDM-554}~{\tiny
  }
  \end{tabular} &
    \begin{tabular}{@{}l@{}}
    \hypertarget{dms-prtl-req-0040-v-01}{DMS-PRTL-REQ-0040-V-01}
    \\\vcdJiraRef{LVV-9881}~{\tiny
    }
    \end{tabular} &
        \begin{tabular}{@{}l@{}}
        \href{https://jira.lsstcorp.org/secure/Tests.jspa\#/testCase/LVV-T7}{LVV-T7} \\
        \vcdDocRef{LDM-540}
        \end{tabular} &
          & \notexec{} \\
          \cmidrule{3-5}
          & &
        \begin{tabular}{@{}l@{}}
        \href{https://jira.lsstcorp.org/secure/Tests.jspa\#/testCase/LVV-T675}{LVV-T675} \\
        \vcdDocRef{LDM-540}
        \end{tabular} &
          & \notexec{} \\
  \midrule
  \begin{tabular}{@{}l@{}}
  DMS-PRTL-REQ-0044\\\vcdDocRef{LDM-554}~{\tiny
  }
  \end{tabular} &
    \begin{tabular}{@{}l@{}}
    \hypertarget{dms-prtl-req-0044-v-01}{DMS-PRTL-REQ-0044-V-01}
    \\\vcdJiraRef{LVV-9882}~{\tiny
    }
    \end{tabular} &
        \begin{tabular}{@{}l@{}}
        \href{https://jira.lsstcorp.org/secure/Tests.jspa\#/testCase/LVV-T679}{LVV-T679} \\
        \vcdDocRef{LDM-540}
        \end{tabular} &
          & \notexec{} \\
  \midrule
  \begin{tabular}{@{}l@{}}
  DMS-PRTL-REQ-0043\\\vcdDocRef{LDM-554}~{\tiny
  }
  \end{tabular} &
    \begin{tabular}{@{}l@{}}
    \hypertarget{dms-prtl-req-0043-v-01}{DMS-PRTL-REQ-0043-V-01}
    \\\vcdJiraRef{LVV-9883}~{\tiny
    }
    \end{tabular} &
        \begin{tabular}{@{}l@{}}
        \href{https://jira.lsstcorp.org/secure/Tests.jspa\#/testCase/LVV-T678}{LVV-T678} \\
        \vcdDocRef{LDM-540}
        \end{tabular} &
          & \notexec{} \\
  \midrule
  \begin{tabular}{@{}l@{}}
  DMS-PRTL-REQ-0042\\\vcdDocRef{LDM-554}~{\tiny
  }
  \end{tabular} &
    \begin{tabular}{@{}l@{}}
    \hypertarget{dms-prtl-req-0042-v-01}{DMS-PRTL-REQ-0042-V-01}
    \\\vcdJiraRef{LVV-9884}~{\tiny
    }
    \end{tabular} &
        \begin{tabular}{@{}l@{}}
        \href{https://jira.lsstcorp.org/secure/Tests.jspa\#/testCase/LVV-T677}{LVV-T677} \\
        \vcdDocRef{LDM-540}
        \end{tabular} &
          & \notexec{} \\
  \midrule
  \begin{tabular}{@{}l@{}}
  DMS-PRTL-REQ-0045\\\vcdDocRef{LDM-554}~{\tiny
  }
  \end{tabular} &
    \begin{tabular}{@{}l@{}}
    \hypertarget{dms-prtl-req-0045-v-01}{DMS-PRTL-REQ-0045-V-01}
    \\\vcdJiraRef{LVV-9885}~{\tiny
    }
    \end{tabular} &
        \begin{tabular}{@{}l@{}}
        \href{https://jira.lsstcorp.org/secure/Tests.jspa\#/testCase/LVV-T680}{LVV-T680} \\
        \vcdDocRef{LDM-540}
        \end{tabular} &
          & \notexec{} \\
  \midrule
  \begin{tabular}{@{}l@{}}
  DMS-PRTL-REQ-0046\\\vcdDocRef{LDM-554}~{\tiny
  }
  \end{tabular} &
    \begin{tabular}{@{}l@{}}
    \hypertarget{dms-prtl-req-0046-v-01}{DMS-PRTL-REQ-0046-V-01}
    \\\vcdJiraRef{LVV-9886}~{\tiny
    }
    \end{tabular} &
        \begin{tabular}{@{}l@{}}
        \href{https://jira.lsstcorp.org/secure/Tests.jspa\#/testCase/LVV-T681}{LVV-T681} \\
        \vcdDocRef{LDM-540}
        \end{tabular} &
          & \notexec{} \\
          \cmidrule{3-5}
          & &
        \begin{tabular}{@{}l@{}}
        \href{https://jira.lsstcorp.org/secure/Tests.jspa\#/testCase/LVV-T1818}{LVV-T1818} \\
        \vcdDocRef{LDM-540}
        \end{tabular} &
          \begin{tabular}{@{}l@{}}
          2020-05-11 \\
            \vcdDocRef{DMTR-211}
            {\scriptsize \href{https://jira.lsstcorp.org/secure/Tests.jspa\#/testPlan/LVV-P69}{LVV-P69} }
          \end{tabular} &
          \cndpass \\
  \midrule
  \begin{tabular}{@{}l@{}}
  DMS-PRTL-REQ-0048\\\vcdDocRef{LDM-554}~{\tiny
  }
  \end{tabular} &
    \begin{tabular}{@{}l@{}}
    \hypertarget{dms-prtl-req-0048-v-01}{DMS-PRTL-REQ-0048-V-01}
    \\\vcdJiraRef{LVV-9887}~{\tiny
    }
    \end{tabular} &
        \begin{tabular}{@{}l@{}}
        \href{https://jira.lsstcorp.org/secure/Tests.jspa\#/testCase/LVV-T683}{LVV-T683} \\
        \vcdDocRef{LDM-540}
        \end{tabular} &
          & \notexec{} \\
  \midrule
  \begin{tabular}{@{}l@{}}
  DMS-PRTL-REQ-0047\\\vcdDocRef{LDM-554}~{\tiny
  }
  \end{tabular} &
    \begin{tabular}{@{}l@{}}
    \hypertarget{dms-prtl-req-0047-v-01}{DMS-PRTL-REQ-0047-V-01}
    \\\vcdJiraRef{LVV-9888}~{\tiny
    }
    \end{tabular} &
        \begin{tabular}{@{}l@{}}
        \href{https://jira.lsstcorp.org/secure/Tests.jspa\#/testCase/LVV-T682}{LVV-T682} \\
        \vcdDocRef{LDM-540}
        \end{tabular} &
          & \notexec{} \\
  \midrule
  \begin{tabular}{@{}l@{}}
  DMS-PRTL-REQ-0050\\\vcdDocRef{LDM-554}~{\tiny
  }
  \end{tabular} &
    \begin{tabular}{@{}l@{}}
    \hypertarget{dms-prtl-req-0050-v-01}{DMS-PRTL-REQ-0050-V-01}
    \\\vcdJiraRef{LVV-9889}~{\tiny
    }
    \end{tabular} &
        \begin{tabular}{@{}l@{}}
        \href{https://jira.lsstcorp.org/secure/Tests.jspa\#/testCase/LVV-T6}{LVV-T6} \\
        \vcdDocRef{LDM-540}
        \end{tabular} &
          \begin{tabular}{@{}l@{}}
          2019-05-20 \\
            \vcdDocRef{DMTR-52}
            {\scriptsize \href{https://jira.lsstcorp.org/secure/Tests.jspa\#/testPlan/LVV-P42}{LVV-P42} }
          \end{tabular} &
          \cndpass \\
          \cmidrule{3-5}
          & &
        \begin{tabular}{@{}l@{}}
        \href{https://jira.lsstcorp.org/secure/Tests.jspa\#/testCase/LVV-T685}{LVV-T685} \\
        \vcdDocRef{LDM-540}
        \end{tabular} &
          & \notexec{} \\
  \midrule
  \begin{tabular}{@{}l@{}}
  DMS-PRTL-REQ-0052\\\vcdDocRef{LDM-554}~{\tiny
  }
  \end{tabular} &
    \begin{tabular}{@{}l@{}}
    \hypertarget{dms-prtl-req-0052-v-01}{DMS-PRTL-REQ-0052-V-01}
    \\\vcdJiraRef{LVV-9890}~{\tiny
    }
    \end{tabular} &
        \begin{tabular}{@{}l@{}}
        \href{https://jira.lsstcorp.org/secure/Tests.jspa\#/testCase/LVV-T687}{LVV-T687} \\
        \vcdDocRef{LDM-540}
        \end{tabular} &
          & \notexec{} \\
  \midrule
  \begin{tabular}{@{}l@{}}
  DMS-PRTL-REQ-0049\\\vcdDocRef{LDM-554}~{\tiny
  }
  \end{tabular} &
    \begin{tabular}{@{}l@{}}
    \hypertarget{dms-prtl-req-0049-v-01}{DMS-PRTL-REQ-0049-V-01}
    \\\vcdJiraRef{LVV-9891}~{\tiny
    }
    \end{tabular} &
        \begin{tabular}{@{}l@{}}
        \href{https://jira.lsstcorp.org/secure/Tests.jspa\#/testCase/LVV-T6}{LVV-T6} \\
        \vcdDocRef{LDM-540}
        \end{tabular} &
          \begin{tabular}{@{}l@{}}
          2019-05-20 \\
            \vcdDocRef{DMTR-52}
            {\scriptsize \href{https://jira.lsstcorp.org/secure/Tests.jspa\#/testPlan/LVV-P42}{LVV-P42} }
          \end{tabular} &
          \cndpass \\
          \cmidrule{3-5}
          & &
        \begin{tabular}{@{}l@{}}
        \href{https://jira.lsstcorp.org/secure/Tests.jspa\#/testCase/LVV-T684}{LVV-T684} \\
        \vcdDocRef{LDM-540}
        \end{tabular} &
          & \notexec{} \\
          \cmidrule{3-5}
          & &
        \begin{tabular}{@{}l@{}}
        \href{https://jira.lsstcorp.org/secure/Tests.jspa\#/testCase/LVV-T1334}{LVV-T1334} \\
        \vcdDocRef{LDM-540}
        \end{tabular} &
          \begin{tabular}{@{}l@{}}
          2019-12-02 \\
            \vcdDocRef{DMTR-161}
            {\scriptsize \href{https://jira.lsstcorp.org/secure/Tests.jspa\#/testPlan/LVV-P48}{LVV-P48} }
          \end{tabular} &
          \cndpass \\
  \midrule
  \begin{tabular}{@{}l@{}}
  DMS-PRTL-REQ-0051\\\vcdDocRef{LDM-554}~{\tiny
  }
  \end{tabular} &
    \begin{tabular}{@{}l@{}}
    \hypertarget{dms-prtl-req-0051-v-01}{DMS-PRTL-REQ-0051-V-01}
    \\\vcdJiraRef{LVV-9892}~{\tiny
    }
    \end{tabular} &
        \begin{tabular}{@{}l@{}}
        \href{https://jira.lsstcorp.org/secure/Tests.jspa\#/testCase/LVV-T686}{LVV-T686} \\
        \vcdDocRef{LDM-540}
        \end{tabular} &
          & \notexec{} \\
  \midrule
  \begin{tabular}{@{}l@{}}
  DMS-PRTL-REQ-0054\\\vcdDocRef{LDM-554}~{\tiny
  }
  \end{tabular} &
    \begin{tabular}{@{}l@{}}
    \hypertarget{dms-prtl-req-0054-v-01}{DMS-PRTL-REQ-0054-V-01}
    \\\vcdJiraRef{LVV-9893}~{\tiny
    }
    \end{tabular} &
        \begin{tabular}{@{}l@{}}
        \href{https://jira.lsstcorp.org/secure/Tests.jspa\#/testCase/LVV-T6}{LVV-T6} \\
        \vcdDocRef{LDM-540}
        \end{tabular} &
          \begin{tabular}{@{}l@{}}
          2019-05-20 \\
            \vcdDocRef{DMTR-52}
            {\scriptsize \href{https://jira.lsstcorp.org/secure/Tests.jspa\#/testPlan/LVV-P42}{LVV-P42} }
          \end{tabular} &
          \cndpass \\
          \cmidrule{3-5}
          & &
        \begin{tabular}{@{}l@{}}
        \href{https://jira.lsstcorp.org/secure/Tests.jspa\#/testCase/LVV-T689}{LVV-T689} \\
        \vcdDocRef{LDM-540}
        \end{tabular} &
          & \notexec{} \\
  \midrule
  \begin{tabular}{@{}l@{}}
  DMS-PRTL-REQ-0053\\\vcdDocRef{LDM-554}~{\tiny
  }
  \end{tabular} &
    \begin{tabular}{@{}l@{}}
    \hypertarget{dms-prtl-req-0053-v-01}{DMS-PRTL-REQ-0053-V-01}
    \\\vcdJiraRef{LVV-9894}~{\tiny
    }
    \end{tabular} &
        \begin{tabular}{@{}l@{}}
        \href{https://jira.lsstcorp.org/secure/Tests.jspa\#/testCase/LVV-T6}{LVV-T6} \\
        \vcdDocRef{LDM-540}
        \end{tabular} &
          \begin{tabular}{@{}l@{}}
          2019-05-20 \\
            \vcdDocRef{DMTR-52}
            {\scriptsize \href{https://jira.lsstcorp.org/secure/Tests.jspa\#/testPlan/LVV-P42}{LVV-P42} }
          \end{tabular} &
          \cndpass \\
          \cmidrule{3-5}
          & &
        \begin{tabular}{@{}l@{}}
        \href{https://jira.lsstcorp.org/secure/Tests.jspa\#/testCase/LVV-T688}{LVV-T688} \\
        \vcdDocRef{LDM-540}
        \end{tabular} &
          & \notexec{} \\
  \midrule
  \begin{tabular}{@{}l@{}}
  DMS-PRTL-REQ-0056\\\vcdDocRef{LDM-554}~{\tiny
  }
  \end{tabular} &
    \begin{tabular}{@{}l@{}}
    \hypertarget{dms-prtl-req-0056-v-01}{DMS-PRTL-REQ-0056-V-01}
    \\\vcdJiraRef{LVV-9895}~{\tiny
    }
    \end{tabular} &
        \begin{tabular}{@{}l@{}}
        \href{https://jira.lsstcorp.org/secure/Tests.jspa\#/testCase/LVV-T6}{LVV-T6} \\
        \vcdDocRef{LDM-540}
        \end{tabular} &
          \begin{tabular}{@{}l@{}}
          2019-05-20 \\
            \vcdDocRef{DMTR-52}
            {\scriptsize \href{https://jira.lsstcorp.org/secure/Tests.jspa\#/testPlan/LVV-P42}{LVV-P42} }
          \end{tabular} &
          \cndpass \\
          \cmidrule{3-5}
          & &
        \begin{tabular}{@{}l@{}}
        \href{https://jira.lsstcorp.org/secure/Tests.jspa\#/testCase/LVV-T691}{LVV-T691} \\
        \vcdDocRef{LDM-540}
        \end{tabular} &
          & \notexec{} \\
  \midrule
  \begin{tabular}{@{}l@{}}
  DMS-PRTL-REQ-0061\\\vcdDocRef{LDM-554}~{\tiny
  }
  \end{tabular} &
    \begin{tabular}{@{}l@{}}
    \hypertarget{dms-prtl-req-0061-v-01}{DMS-PRTL-REQ-0061-V-01}
    \\\vcdJiraRef{LVV-9896}~{\tiny
    }
    \end{tabular} &
        \begin{tabular}{@{}l@{}}
        \href{https://jira.lsstcorp.org/secure/Tests.jspa\#/testCase/LVV-T696}{LVV-T696} \\
        \vcdDocRef{LDM-540}
        \end{tabular} &
          & \notexec{} \\
  \midrule
  \begin{tabular}{@{}l@{}}
  DMS-PRTL-REQ-0059\\\vcdDocRef{LDM-554}~{\tiny
  }
  \end{tabular} &
    \begin{tabular}{@{}l@{}}
    \hypertarget{dms-prtl-req-0059-v-01}{DMS-PRTL-REQ-0059-V-01}
    \\\vcdJiraRef{LVV-9897}~{\tiny
    }
    \end{tabular} &
        \begin{tabular}{@{}l@{}}
        \href{https://jira.lsstcorp.org/secure/Tests.jspa\#/testCase/LVV-T694}{LVV-T694} \\
        \vcdDocRef{LDM-540}
        \end{tabular} &
          & \notexec{} \\
  \midrule
  \begin{tabular}{@{}l@{}}
  DMS-PRTL-REQ-0058\\\vcdDocRef{LDM-554}~{\tiny
  }
  \end{tabular} &
    \begin{tabular}{@{}l@{}}
    \hypertarget{dms-prtl-req-0058-v-01}{DMS-PRTL-REQ-0058-V-01}
    \\\vcdJiraRef{LVV-9898}~{\tiny
    }
    \end{tabular} &
        \begin{tabular}{@{}l@{}}
        \href{https://jira.lsstcorp.org/secure/Tests.jspa\#/testCase/LVV-T693}{LVV-T693} \\
        \vcdDocRef{LDM-540}
        \end{tabular} &
          & \notexec{} \\
  \midrule
  \begin{tabular}{@{}l@{}}
  DMS-PRTL-REQ-0060\\\vcdDocRef{LDM-554}~{\tiny
  }
  \end{tabular} &
    \begin{tabular}{@{}l@{}}
    \hypertarget{dms-prtl-req-0060-v-01}{DMS-PRTL-REQ-0060-V-01}
    \\\vcdJiraRef{LVV-9899}~{\tiny
    }
    \end{tabular} &
        \begin{tabular}{@{}l@{}}
        \href{https://jira.lsstcorp.org/secure/Tests.jspa\#/testCase/LVV-T695}{LVV-T695} \\
        \vcdDocRef{LDM-540}
        \end{tabular} &
          & \notexec{} \\
  \midrule
  \begin{tabular}{@{}l@{}}
  DMS-PRTL-REQ-0057\\\vcdDocRef{LDM-554}~{\tiny
  }
  \end{tabular} &
    \begin{tabular}{@{}l@{}}
    \hypertarget{dms-prtl-req-0057-v-01}{DMS-PRTL-REQ-0057-V-01}
    \\\vcdJiraRef{LVV-9900}~{\tiny
    }
    \end{tabular} &
        \begin{tabular}{@{}l@{}}
        \href{https://jira.lsstcorp.org/secure/Tests.jspa\#/testCase/LVV-T692}{LVV-T692} \\
        \vcdDocRef{LDM-540}
        \end{tabular} &
          & \notexec{} \\
  \midrule
  \begin{tabular}{@{}l@{}}
  DMS-PRTL-REQ-0055\\\vcdDocRef{LDM-554}~{\tiny
  }
  \end{tabular} &
    \begin{tabular}{@{}l@{}}
    \hypertarget{dms-prtl-req-0055-v-01}{DMS-PRTL-REQ-0055-V-01}
    \\\vcdJiraRef{LVV-9901}~{\tiny
    }
    \end{tabular} &
        \begin{tabular}{@{}l@{}}
        \href{https://jira.lsstcorp.org/secure/Tests.jspa\#/testCase/LVV-T6}{LVV-T6} \\
        \vcdDocRef{LDM-540}
        \end{tabular} &
          \begin{tabular}{@{}l@{}}
          2019-05-20 \\
            \vcdDocRef{DMTR-52}
            {\scriptsize \href{https://jira.lsstcorp.org/secure/Tests.jspa\#/testPlan/LVV-P42}{LVV-P42} }
          \end{tabular} &
          \cndpass \\
          \cmidrule{3-5}
          & &
        \begin{tabular}{@{}l@{}}
        \href{https://jira.lsstcorp.org/secure/Tests.jspa\#/testCase/LVV-T690}{LVV-T690} \\
        \vcdDocRef{LDM-540}
        \end{tabular} &
          & \notexec{} \\
  \midrule
  \begin{tabular}{@{}l@{}}
  DMS-PRTL-REQ-0067\\\vcdDocRef{LDM-554}~{\tiny
  }
  \end{tabular} &
    \begin{tabular}{@{}l@{}}
    \hypertarget{dms-prtl-req-0067-v-01}{DMS-PRTL-REQ-0067-V-01}
    \\\vcdJiraRef{LVV-9902}~{\tiny
    }
    \end{tabular} &
        \begin{tabular}{@{}l@{}}
        \href{https://jira.lsstcorp.org/secure/Tests.jspa\#/testCase/LVV-T701}{LVV-T701} \\
        \vcdDocRef{LDM-540}
        \end{tabular} &
          & \notexec{} \\
  \midrule
  \begin{tabular}{@{}l@{}}
  DMS-PRTL-REQ-0066\\\vcdDocRef{LDM-554}~{\tiny
  }
  \end{tabular} &
    \begin{tabular}{@{}l@{}}
    \hypertarget{dms-prtl-req-0066-v-01}{DMS-PRTL-REQ-0066-V-01}
    \\\vcdJiraRef{LVV-9903}~{\tiny
    }
    \end{tabular} &
        \begin{tabular}{@{}l@{}}
        \href{https://jira.lsstcorp.org/secure/Tests.jspa\#/testCase/LVV-T700}{LVV-T700} \\
        \vcdDocRef{LDM-540}
        \end{tabular} &
          & \notexec{} \\
  \midrule
  \begin{tabular}{@{}l@{}}
  DMS-PRTL-REQ-0065\\\vcdDocRef{LDM-554}~{\tiny
  }
  \end{tabular} &
    \begin{tabular}{@{}l@{}}
    \hypertarget{dms-prtl-req-0065-v-01}{DMS-PRTL-REQ-0065-V-01}
    \\\vcdJiraRef{LVV-9904}~{\tiny
    }
    \end{tabular} &
        \begin{tabular}{@{}l@{}}
        \href{https://jira.lsstcorp.org/secure/Tests.jspa\#/testCase/LVV-T699}{LVV-T699} \\
        \vcdDocRef{LDM-540}
        \end{tabular} &
          & \notexec{} \\
  \midrule
  \begin{tabular}{@{}l@{}}
  DMS-PRTL-REQ-0062\\\vcdDocRef{LDM-554}~{\tiny
  }
  \end{tabular} &
    \begin{tabular}{@{}l@{}}
    \hypertarget{dms-prtl-req-0062-v-01}{DMS-PRTL-REQ-0062-V-01}
    \\\vcdJiraRef{LVV-9905}~{\tiny
    }
    \end{tabular} &
        \begin{tabular}{@{}l@{}}
        \href{https://jira.lsstcorp.org/secure/Tests.jspa\#/testCase/LVV-T676}{LVV-T676} \\
        \vcdDocRef{LDM-540}
        \end{tabular} &
          & \notexec{} \\
  \midrule
  \begin{tabular}{@{}l@{}}
  DMS-PRTL-REQ-0063\\\vcdDocRef{LDM-554}~{\tiny
  }
  \end{tabular} &
    \begin{tabular}{@{}l@{}}
    \hypertarget{dms-prtl-req-0063-v-01}{DMS-PRTL-REQ-0063-V-01}
    \\\vcdJiraRef{LVV-9906}~{\tiny
    }
    \end{tabular} &
        \begin{tabular}{@{}l@{}}
        \href{https://jira.lsstcorp.org/secure/Tests.jspa\#/testCase/LVV-T697}{LVV-T697} \\
        \vcdDocRef{LDM-540}
        \end{tabular} &
          & \notexec{} \\
  \midrule
  \begin{tabular}{@{}l@{}}
  DMS-PRTL-REQ-0064\\\vcdDocRef{LDM-554}~{\tiny
  }
  \end{tabular} &
    \begin{tabular}{@{}l@{}}
    \hypertarget{dms-prtl-req-0064-v-01}{DMS-PRTL-REQ-0064-V-01}
    \\\vcdJiraRef{LVV-9907}~{\tiny
    }
    \end{tabular} &
        \begin{tabular}{@{}l@{}}
        \href{https://jira.lsstcorp.org/secure/Tests.jspa\#/testCase/LVV-T698}{LVV-T698} \\
        \vcdDocRef{LDM-540}
        \end{tabular} &
          & \notexec{} \\
  \midrule
  \begin{tabular}{@{}l@{}}
  DMS-PRTL-REQ-0068\\\vcdDocRef{LDM-554}~{\tiny
  }
  \end{tabular} &
    \begin{tabular}{@{}l@{}}
    \hypertarget{dms-prtl-req-0068-v-01}{DMS-PRTL-REQ-0068-V-01}
    \\\vcdJiraRef{LVV-9908}~{\tiny
    }
    \end{tabular} &
        \begin{tabular}{@{}l@{}}
        \href{https://jira.lsstcorp.org/secure/Tests.jspa\#/testCase/LVV-T702}{LVV-T702} \\
        \vcdDocRef{LDM-540}
        \end{tabular} &
          & \notexec{} \\
  \midrule
  \begin{tabular}{@{}l@{}}
  DMS-PRTL-REQ-0069\\\vcdDocRef{LDM-554}~{\tiny
  }
  \end{tabular} &
    \begin{tabular}{@{}l@{}}
    \hypertarget{dms-prtl-req-0069-v-01}{DMS-PRTL-REQ-0069-V-01}
    \\\vcdJiraRef{LVV-9909}~{\tiny
    }
    \end{tabular} &
        \begin{tabular}{@{}l@{}}
        \href{https://jira.lsstcorp.org/secure/Tests.jspa\#/testCase/LVV-T703}{LVV-T703} \\
        \vcdDocRef{LDM-540}
        \end{tabular} &
          & \notexec{} \\
  \midrule
  \begin{tabular}{@{}l@{}}
  DMS-PRTL-REQ-0074\\\vcdDocRef{LDM-554}~{\tiny
  }
  \end{tabular} &
    \begin{tabular}{@{}l@{}}
    \hypertarget{dms-prtl-req-0074-v-01}{DMS-PRTL-REQ-0074-V-01}
    \\\vcdJiraRef{LVV-9910}~{\tiny
    }
    \end{tabular} &
        \begin{tabular}{@{}l@{}}
        \href{https://jira.lsstcorp.org/secure/Tests.jspa\#/testCase/LVV-T708}{LVV-T708} \\
        \vcdDocRef{LDM-540}
        \end{tabular} &
          & \notexec{} \\
  \midrule
  \begin{tabular}{@{}l@{}}
  DMS-PRTL-REQ-0071\\\vcdDocRef{LDM-554}~{\tiny
  }
  \end{tabular} &
    \begin{tabular}{@{}l@{}}
    \hypertarget{dms-prtl-req-0071-v-01}{DMS-PRTL-REQ-0071-V-01}
    \\\vcdJiraRef{LVV-9911}~{\tiny
    }
    \end{tabular} &
        \begin{tabular}{@{}l@{}}
        \href{https://jira.lsstcorp.org/secure/Tests.jspa\#/testCase/LVV-T705}{LVV-T705} \\
        \vcdDocRef{LDM-540}
        \end{tabular} &
          & \notexec{} \\
  \midrule
  \begin{tabular}{@{}l@{}}
  DMS-PRTL-REQ-0072\\\vcdDocRef{LDM-554}~{\tiny
  }
  \end{tabular} &
    \begin{tabular}{@{}l@{}}
    \hypertarget{dms-prtl-req-0072-v-01}{DMS-PRTL-REQ-0072-V-01}
    \\\vcdJiraRef{LVV-9912}~{\tiny
    }
    \end{tabular} &
        \begin{tabular}{@{}l@{}}
        \href{https://jira.lsstcorp.org/secure/Tests.jspa\#/testCase/LVV-T706}{LVV-T706} \\
        \vcdDocRef{LDM-540}
        \end{tabular} &
          & \notexec{} \\
  \midrule
  \begin{tabular}{@{}l@{}}
  DMS-PRTL-REQ-0073\\\vcdDocRef{LDM-554}~{\tiny
  }
  \end{tabular} &
    \begin{tabular}{@{}l@{}}
    \hypertarget{dms-prtl-req-0073-v-01}{DMS-PRTL-REQ-0073-V-01}
    \\\vcdJiraRef{LVV-9913}~{\tiny
    }
    \end{tabular} &
        \begin{tabular}{@{}l@{}}
        \href{https://jira.lsstcorp.org/secure/Tests.jspa\#/testCase/LVV-T707}{LVV-T707} \\
        \vcdDocRef{LDM-540}
        \end{tabular} &
          & \notexec{} \\
  \midrule
  \begin{tabular}{@{}l@{}}
  DMS-PRTL-REQ-0070\\\vcdDocRef{LDM-554}~{\tiny
  }
  \end{tabular} &
    \begin{tabular}{@{}l@{}}
    \hypertarget{dms-prtl-req-0070-v-01}{DMS-PRTL-REQ-0070-V-01}
    \\\vcdJiraRef{LVV-9914}~{\tiny
    }
    \end{tabular} &
        \begin{tabular}{@{}l@{}}
        \href{https://jira.lsstcorp.org/secure/Tests.jspa\#/testCase/LVV-T704}{LVV-T704} \\
        \vcdDocRef{LDM-540}
        \end{tabular} &
          & \notexec{} \\
  \midrule
  \begin{tabular}{@{}l@{}}
  DMS-PRTL-REQ-0075\\\vcdDocRef{LDM-554}~{\tiny
  }
  \end{tabular} &
    \begin{tabular}{@{}l@{}}
    \hypertarget{dms-prtl-req-0075-v-01}{DMS-PRTL-REQ-0075-V-01}
    \\\vcdJiraRef{LVV-9915}~{\tiny
    }
    \end{tabular} &
        \begin{tabular}{@{}l@{}}
        \href{https://jira.lsstcorp.org/secure/Tests.jspa\#/testCase/LVV-T709}{LVV-T709} \\
        \vcdDocRef{LDM-540}
        \end{tabular} &
          & \notexec{} \\
  \midrule
  \begin{tabular}{@{}l@{}}
  DMS-PRTL-REQ-0077\\\vcdDocRef{LDM-554}~{\tiny
  }
  \end{tabular} &
    \begin{tabular}{@{}l@{}}
    \hypertarget{dms-prtl-req-0077-v-01}{DMS-PRTL-REQ-0077-V-01}
    \\\vcdJiraRef{LVV-9916}~{\tiny
    }
    \end{tabular} &
        \begin{tabular}{@{}l@{}}
        \href{https://jira.lsstcorp.org/secure/Tests.jspa\#/testCase/LVV-T711}{LVV-T711} \\
        \vcdDocRef{LDM-540}
        \end{tabular} &
          & \notexec{} \\
  \midrule
  \begin{tabular}{@{}l@{}}
  DMS-PRTL-REQ-0076\\\vcdDocRef{LDM-554}~{\tiny
  }
  \end{tabular} &
    \begin{tabular}{@{}l@{}}
    \hypertarget{dms-prtl-req-0076-v-01}{DMS-PRTL-REQ-0076-V-01}
    \\\vcdJiraRef{LVV-9917}~{\tiny
    }
    \end{tabular} &
        \begin{tabular}{@{}l@{}}
        \href{https://jira.lsstcorp.org/secure/Tests.jspa\#/testCase/LVV-T710}{LVV-T710} \\
        \vcdDocRef{LDM-540}
        \end{tabular} &
          & \notexec{} \\
  \midrule
  \begin{tabular}{@{}l@{}}
  DMS-PRTL-REQ-0078\\\vcdDocRef{LDM-554}~{\tiny
  }
  \end{tabular} &
    \begin{tabular}{@{}l@{}}
    \hypertarget{dms-prtl-req-0078-v-01}{DMS-PRTL-REQ-0078-V-01}
    \\\vcdJiraRef{LVV-9918}~{\tiny
    }
    \end{tabular} &
        \begin{tabular}{@{}l@{}}
        \href{https://jira.lsstcorp.org/secure/Tests.jspa\#/testCase/LVV-T712}{LVV-T712} \\
        \vcdDocRef{LDM-540}
        \end{tabular} &
          & \notexec{} \\
  \midrule
  \begin{tabular}{@{}l@{}}
  DMS-PRTL-REQ-0081\\\vcdDocRef{LDM-554}~{\tiny
  }
  \end{tabular} &
    \begin{tabular}{@{}l@{}}
    \hypertarget{dms-prtl-req-0081-v-01}{DMS-PRTL-REQ-0081-V-01}
    \\\vcdJiraRef{LVV-9919}~{\tiny
    }
    \end{tabular} &
        \begin{tabular}{@{}l@{}}
        \href{https://jira.lsstcorp.org/secure/Tests.jspa\#/testCase/LVV-T715}{LVV-T715} \\
        \vcdDocRef{LDM-540}
        \end{tabular} &
          & \notexec{} \\
  \midrule
  \begin{tabular}{@{}l@{}}
  DMS-PRTL-REQ-0080\\\vcdDocRef{LDM-554}~{\tiny
  }
  \end{tabular} &
    \begin{tabular}{@{}l@{}}
    \hypertarget{dms-prtl-req-0080-v-01}{DMS-PRTL-REQ-0080-V-01}
    \\\vcdJiraRef{LVV-9920}~{\tiny
    }
    \end{tabular} &
        \begin{tabular}{@{}l@{}}
        \href{https://jira.lsstcorp.org/secure/Tests.jspa\#/testCase/LVV-T714}{LVV-T714} \\
        \vcdDocRef{LDM-540}
        \end{tabular} &
          & \notexec{} \\
  \midrule
  \begin{tabular}{@{}l@{}}
  DMS-PRTL-REQ-0082\\\vcdDocRef{LDM-554}~{\tiny
  }
  \end{tabular} &
    \begin{tabular}{@{}l@{}}
    \hypertarget{dms-prtl-req-0082-v-01}{DMS-PRTL-REQ-0082-V-01}
    \\\vcdJiraRef{LVV-9921}~{\tiny
    }
    \end{tabular} &
        \begin{tabular}{@{}l@{}}
        \href{https://jira.lsstcorp.org/secure/Tests.jspa\#/testCase/LVV-T716}{LVV-T716} \\
        \vcdDocRef{LDM-540}
        \end{tabular} &
          & \notexec{} \\
  \midrule
  \begin{tabular}{@{}l@{}}
  DMS-PRTL-REQ-0079\\\vcdDocRef{LDM-554}~{\tiny
  }
  \end{tabular} &
    \begin{tabular}{@{}l@{}}
    \hypertarget{dms-prtl-req-0079-v-01}{DMS-PRTL-REQ-0079-V-01}
    \\\vcdJiraRef{LVV-9922}~{\tiny
    }
    \end{tabular} &
        \begin{tabular}{@{}l@{}}
        \href{https://jira.lsstcorp.org/secure/Tests.jspa\#/testCase/LVV-T713}{LVV-T713} \\
        \vcdDocRef{LDM-540}
        \end{tabular} &
          & \notexec{} \\
  \midrule
  \begin{tabular}{@{}l@{}}
  DMS-PRTL-REQ-0087\\\vcdDocRef{LDM-554}~{\tiny
  }
  \end{tabular} &
    \begin{tabular}{@{}l@{}}
    \hypertarget{dms-prtl-req-0087-v-01}{DMS-PRTL-REQ-0087-V-01}
    \\\vcdJiraRef{LVV-9923}~{\tiny
    }
    \end{tabular} &
        \begin{tabular}{@{}l@{}}
        \href{https://jira.lsstcorp.org/secure/Tests.jspa\#/testCase/LVV-T721}{LVV-T721} \\
        \vcdDocRef{LDM-540}
        \end{tabular} &
          & \notexec{} \\
  \midrule
  \begin{tabular}{@{}l@{}}
  DMS-PRTL-REQ-0083\\\vcdDocRef{LDM-554}~{\tiny
  }
  \end{tabular} &
    \begin{tabular}{@{}l@{}}
    \hypertarget{dms-prtl-req-0083-v-01}{DMS-PRTL-REQ-0083-V-01}
    \\\vcdJiraRef{LVV-9924}~{\tiny
    }
    \end{tabular} &
        \begin{tabular}{@{}l@{}}
        \href{https://jira.lsstcorp.org/secure/Tests.jspa\#/testCase/LVV-T717}{LVV-T717} \\
        \vcdDocRef{LDM-540}
        \end{tabular} &
          & \notexec{} \\
  \midrule
  \begin{tabular}{@{}l@{}}
  DMS-PRTL-REQ-0086\\\vcdDocRef{LDM-554}~{\tiny
  }
  \end{tabular} &
    \begin{tabular}{@{}l@{}}
    \hypertarget{dms-prtl-req-0086-v-01}{DMS-PRTL-REQ-0086-V-01}
    \\\vcdJiraRef{LVV-9925}~{\tiny
    }
    \end{tabular} &
        \begin{tabular}{@{}l@{}}
        \href{https://jira.lsstcorp.org/secure/Tests.jspa\#/testCase/LVV-T720}{LVV-T720} \\
        \vcdDocRef{LDM-540}
        \end{tabular} &
          & \notexec{} \\
  \midrule
  \begin{tabular}{@{}l@{}}
  DMS-PRTL-REQ-0085\\\vcdDocRef{LDM-554}~{\tiny
  }
  \end{tabular} &
    \begin{tabular}{@{}l@{}}
    \hypertarget{dms-prtl-req-0085-v-01}{DMS-PRTL-REQ-0085-V-01}
    \\\vcdJiraRef{LVV-9926}~{\tiny
    }
    \end{tabular} &
        \begin{tabular}{@{}l@{}}
        \href{https://jira.lsstcorp.org/secure/Tests.jspa\#/testCase/LVV-T719}{LVV-T719} \\
        \vcdDocRef{LDM-540}
        \end{tabular} &
          & \notexec{} \\
  \midrule
  \begin{tabular}{@{}l@{}}
  DMS-PRTL-REQ-0088\\\vcdDocRef{LDM-554}~{\tiny
  }
  \end{tabular} &
    \begin{tabular}{@{}l@{}}
    \hypertarget{dms-prtl-req-0088-v-01}{DMS-PRTL-REQ-0088-V-01}
    \\\vcdJiraRef{LVV-9927}~{\tiny
    }
    \end{tabular} &
        \begin{tabular}{@{}l@{}}
        \href{https://jira.lsstcorp.org/secure/Tests.jspa\#/testCase/LVV-T722}{LVV-T722} \\
        \vcdDocRef{LDM-540}
        \end{tabular} &
          & \notexec{} \\
  \midrule
  \begin{tabular}{@{}l@{}}
  DMS-PRTL-REQ-0084\\\vcdDocRef{LDM-554}~{\tiny
  }
  \end{tabular} &
    \begin{tabular}{@{}l@{}}
    \hypertarget{dms-prtl-req-0084-v-01}{DMS-PRTL-REQ-0084-V-01}
    \\\vcdJiraRef{LVV-9928}~{\tiny
    }
    \end{tabular} &
        \begin{tabular}{@{}l@{}}
        \href{https://jira.lsstcorp.org/secure/Tests.jspa\#/testCase/LVV-T718}{LVV-T718} \\
        \vcdDocRef{LDM-540}
        \end{tabular} &
          & \notexec{} \\
  \midrule
  \begin{tabular}{@{}l@{}}
  DMS-PRTL-REQ-0091\\\vcdDocRef{LDM-554}~{\tiny
  }
  \end{tabular} &
    \begin{tabular}{@{}l@{}}
    \hypertarget{dms-prtl-req-0091-v-01}{DMS-PRTL-REQ-0091-V-01}
    \\\vcdJiraRef{LVV-9929}~{\tiny
    }
    \end{tabular} &
        \begin{tabular}{@{}l@{}}
        \href{https://jira.lsstcorp.org/secure/Tests.jspa\#/testCase/LVV-T725}{LVV-T725} \\
        \vcdDocRef{LDM-540}
        \end{tabular} &
          & \notexec{} \\
  \midrule
  \begin{tabular}{@{}l@{}}
  DMS-PRTL-REQ-0093\\\vcdDocRef{LDM-554}~{\tiny
  }
  \end{tabular} &
    \begin{tabular}{@{}l@{}}
    \hypertarget{dms-prtl-req-0093-v-01}{DMS-PRTL-REQ-0093-V-01}
    \\\vcdJiraRef{LVV-9930}~{\tiny
    }
    \end{tabular} &
        \begin{tabular}{@{}l@{}}
        \href{https://jira.lsstcorp.org/secure/Tests.jspa\#/testCase/LVV-T727}{LVV-T727} \\
        \vcdDocRef{LDM-540}
        \end{tabular} &
          & \notexec{} \\
  \midrule
  \begin{tabular}{@{}l@{}}
  DMS-PRTL-REQ-0092\\\vcdDocRef{LDM-554}~{\tiny
  }
  \end{tabular} &
    \begin{tabular}{@{}l@{}}
    \hypertarget{dms-prtl-req-0092-v-01}{DMS-PRTL-REQ-0092-V-01}
    \\\vcdJiraRef{LVV-9931}~{\tiny
    }
    \end{tabular} &
        \begin{tabular}{@{}l@{}}
        \href{https://jira.lsstcorp.org/secure/Tests.jspa\#/testCase/LVV-T726}{LVV-T726} \\
        \vcdDocRef{LDM-540}
        \end{tabular} &
          & \notexec{} \\
  \midrule
  \begin{tabular}{@{}l@{}}
  DMS-PRTL-REQ-0095\\\vcdDocRef{LDM-554}~{\tiny
  }
  \end{tabular} &
    \begin{tabular}{@{}l@{}}
    \hypertarget{dms-prtl-req-0095-v-01}{DMS-PRTL-REQ-0095-V-01}
    \\\vcdJiraRef{LVV-9932}~{\tiny
    }
    \end{tabular} &
        \begin{tabular}{@{}l@{}}
        \href{https://jira.lsstcorp.org/secure/Tests.jspa\#/testCase/LVV-T729}{LVV-T729} \\
        \vcdDocRef{LDM-540}
        \end{tabular} &
          & \notexec{} \\
          \cmidrule{3-5}
          & &
        \begin{tabular}{@{}l@{}}
        \href{https://jira.lsstcorp.org/secure/Tests.jspa\#/testCase/LVV-T1334}{LVV-T1334} \\
        \vcdDocRef{LDM-540}
        \end{tabular} &
          \begin{tabular}{@{}l@{}}
          2019-12-02 \\
            \vcdDocRef{DMTR-161}
            {\scriptsize \href{https://jira.lsstcorp.org/secure/Tests.jspa\#/testPlan/LVV-P48}{LVV-P48} }
          \end{tabular} &
          \cndpass \\
          \cmidrule{3-5}
          & &
        \begin{tabular}{@{}l@{}}
        \href{https://jira.lsstcorp.org/secure/Tests.jspa\#/testCase/LVV-T1818}{LVV-T1818} \\
        \vcdDocRef{LDM-540}
        \end{tabular} &
          \begin{tabular}{@{}l@{}}
          2020-05-11 \\
            \vcdDocRef{DMTR-211}
            {\scriptsize \href{https://jira.lsstcorp.org/secure/Tests.jspa\#/testPlan/LVV-P69}{LVV-P69} }
          \end{tabular} &
          \cndpass \\
  \midrule
  \begin{tabular}{@{}l@{}}
  DMS-PRTL-REQ-0090\\\vcdDocRef{LDM-554}~{\tiny
  }
  \end{tabular} &
    \begin{tabular}{@{}l@{}}
    \hypertarget{dms-prtl-req-0090-v-01}{DMS-PRTL-REQ-0090-V-01}
    \\\vcdJiraRef{LVV-9933}~{\tiny
    }
    \end{tabular} &
        \begin{tabular}{@{}l@{}}
        \href{https://jira.lsstcorp.org/secure/Tests.jspa\#/testCase/LVV-T724}{LVV-T724} \\
        \vcdDocRef{LDM-540}
        \end{tabular} &
          & \notexec{} \\
  \midrule
  \begin{tabular}{@{}l@{}}
  DMS-PRTL-REQ-0089\\\vcdDocRef{LDM-554}~{\tiny
  }
  \end{tabular} &
    \begin{tabular}{@{}l@{}}
    \hypertarget{dms-prtl-req-0089-v-01}{DMS-PRTL-REQ-0089-V-01}
    \\\vcdJiraRef{LVV-9934}~{\tiny
    }
    \end{tabular} &
        \begin{tabular}{@{}l@{}}
        \href{https://jira.lsstcorp.org/secure/Tests.jspa\#/testCase/LVV-T723}{LVV-T723} \\
        \vcdDocRef{LDM-540}
        \end{tabular} &
          & \notexec{} \\
  \midrule
  \begin{tabular}{@{}l@{}}
  DMS-PRTL-REQ-0094\\\vcdDocRef{LDM-554}~{\tiny
  }
  \end{tabular} &
    \begin{tabular}{@{}l@{}}
    \hypertarget{dms-prtl-req-0094-v-01}{DMS-PRTL-REQ-0094-V-01}
    \\\vcdJiraRef{LVV-9935}~{\tiny
    }
    \end{tabular} &
        \begin{tabular}{@{}l@{}}
        \href{https://jira.lsstcorp.org/secure/Tests.jspa\#/testCase/LVV-T728}{LVV-T728} \\
        \vcdDocRef{LDM-540}
        \end{tabular} &
          & \notexec{} \\
  \midrule
  \begin{tabular}{@{}l@{}}
  DMS-PRTL-REQ-0096\\\vcdDocRef{LDM-554}~{\tiny
  }
  \end{tabular} &
    \begin{tabular}{@{}l@{}}
    \hypertarget{dms-prtl-req-0096-v-01}{DMS-PRTL-REQ-0096-V-01}
    \\\vcdJiraRef{LVV-9936}~{\tiny
    }
    \end{tabular} &
        \begin{tabular}{@{}l@{}}
        \href{https://jira.lsstcorp.org/secure/Tests.jspa\#/testCase/LVV-T730}{LVV-T730} \\
        \vcdDocRef{LDM-540}
        \end{tabular} &
          & \notexec{} \\
  \midrule
  \begin{tabular}{@{}l@{}}
  DMS-PRTL-REQ-0097\\\vcdDocRef{LDM-554}~{\tiny
  }
  \end{tabular} &
    \begin{tabular}{@{}l@{}}
    \hypertarget{dms-prtl-req-0097-v-01}{DMS-PRTL-REQ-0097-V-01}
    \\\vcdJiraRef{LVV-9937}~{\tiny
    }
    \end{tabular} &
        \begin{tabular}{@{}l@{}}
        \href{https://jira.lsstcorp.org/secure/Tests.jspa\#/testCase/LVV-T731}{LVV-T731} \\
        \vcdDocRef{LDM-540}
        \end{tabular} &
          & \notexec{} \\
  \midrule
  \begin{tabular}{@{}l@{}}
  DMS-PRTL-REQ-0105\\\vcdDocRef{LDM-554}~{\tiny
  }
  \end{tabular} &
    \begin{tabular}{@{}l@{}}
    \hypertarget{dms-prtl-req-0105-v-01}{DMS-PRTL-REQ-0105-V-01}
    \\\vcdJiraRef{LVV-9938}~{\tiny
    }
    \end{tabular} &
        \begin{tabular}{@{}l@{}}
        \href{https://jira.lsstcorp.org/secure/Tests.jspa\#/testCase/LVV-T739}{LVV-T739} \\
        \vcdDocRef{LDM-540}
        \end{tabular} &
          & \notexec{} \\
  \midrule
  \begin{tabular}{@{}l@{}}
  DMS-PRTL-REQ-0107\\\vcdDocRef{LDM-554}~{\tiny
  }
  \end{tabular} &
    \begin{tabular}{@{}l@{}}
    \hypertarget{dms-prtl-req-0107-v-01}{DMS-PRTL-REQ-0107-V-01}
    \\\vcdJiraRef{LVV-9939}~{\tiny
    }
    \end{tabular} &
        \begin{tabular}{@{}l@{}}
        \href{https://jira.lsstcorp.org/secure/Tests.jspa\#/testCase/LVV-T741}{LVV-T741} \\
        \vcdDocRef{LDM-540}
        \end{tabular} &
          & \notexec{} \\
  \midrule
  \begin{tabular}{@{}l@{}}
  DMS-PRTL-REQ-0102\\\vcdDocRef{LDM-554}~{\tiny
  }
  \end{tabular} &
    \begin{tabular}{@{}l@{}}
    \hypertarget{dms-prtl-req-0102-v-01}{DMS-PRTL-REQ-0102-V-01}
    \\\vcdJiraRef{LVV-9940}~{\tiny
    }
    \end{tabular} &
        \begin{tabular}{@{}l@{}}
        \href{https://jira.lsstcorp.org/secure/Tests.jspa\#/testCase/LVV-T736}{LVV-T736} \\
        \vcdDocRef{LDM-540}
        \end{tabular} &
          & \notexec{} \\
  \midrule
  \begin{tabular}{@{}l@{}}
  DMS-PRTL-REQ-0106\\\vcdDocRef{LDM-554}~{\tiny
  }
  \end{tabular} &
    \begin{tabular}{@{}l@{}}
    \hypertarget{dms-prtl-req-0106-v-01}{DMS-PRTL-REQ-0106-V-01}
    \\\vcdJiraRef{LVV-9941}~{\tiny
    }
    \end{tabular} &
        \begin{tabular}{@{}l@{}}
        \href{https://jira.lsstcorp.org/secure/Tests.jspa\#/testCase/LVV-T740}{LVV-T740} \\
        \vcdDocRef{LDM-540}
        \end{tabular} &
          & \notexec{} \\
  \midrule
  \begin{tabular}{@{}l@{}}
  DMS-PRTL-REQ-0098\\\vcdDocRef{LDM-554}~{\tiny
  }
  \end{tabular} &
    \begin{tabular}{@{}l@{}}
    \hypertarget{dms-prtl-req-0098-v-01}{DMS-PRTL-REQ-0098-V-01}
    \\\vcdJiraRef{LVV-9942}~{\tiny
    }
    \end{tabular} &
        \begin{tabular}{@{}l@{}}
        \href{https://jira.lsstcorp.org/secure/Tests.jspa\#/testCase/LVV-T732}{LVV-T732} \\
        \vcdDocRef{LDM-540}
        \end{tabular} &
          & \notexec{} \\
  \midrule
  \begin{tabular}{@{}l@{}}
  DMS-PRTL-REQ-0099\\\vcdDocRef{LDM-554}~{\tiny
  }
  \end{tabular} &
    \begin{tabular}{@{}l@{}}
    \hypertarget{dms-prtl-req-0099-v-01}{DMS-PRTL-REQ-0099-V-01}
    \\\vcdJiraRef{LVV-9943}~{\tiny
    }
    \end{tabular} &
        \begin{tabular}{@{}l@{}}
        \href{https://jira.lsstcorp.org/secure/Tests.jspa\#/testCase/LVV-T733}{LVV-T733} \\
        \vcdDocRef{LDM-540}
        \end{tabular} &
          & \notexec{} \\
  \midrule
  \begin{tabular}{@{}l@{}}
  DMS-PRTL-REQ-0100\\\vcdDocRef{LDM-554}~{\tiny
  }
  \end{tabular} &
    \begin{tabular}{@{}l@{}}
    \hypertarget{dms-prtl-req-0100-v-01}{DMS-PRTL-REQ-0100-V-01}
    \\\vcdJiraRef{LVV-9944}~{\tiny
    }
    \end{tabular} &
        \begin{tabular}{@{}l@{}}
        \href{https://jira.lsstcorp.org/secure/Tests.jspa\#/testCase/LVV-T734}{LVV-T734} \\
        \vcdDocRef{LDM-540}
        \end{tabular} &
          & \notexec{} \\
  \midrule
  \begin{tabular}{@{}l@{}}
  DMS-PRTL-REQ-0101\\\vcdDocRef{LDM-554}~{\tiny
  }
  \end{tabular} &
    \begin{tabular}{@{}l@{}}
    \hypertarget{dms-prtl-req-0101-v-01}{DMS-PRTL-REQ-0101-V-01}
    \\\vcdJiraRef{LVV-9945}~{\tiny
    }
    \end{tabular} &
        \begin{tabular}{@{}l@{}}
        \href{https://jira.lsstcorp.org/secure/Tests.jspa\#/testCase/LVV-T735}{LVV-T735} \\
        \vcdDocRef{LDM-540}
        \end{tabular} &
          & \notexec{} \\
  \midrule
  \begin{tabular}{@{}l@{}}
  DMS-PRTL-REQ-0104\\\vcdDocRef{LDM-554}~{\tiny
  }
  \end{tabular} &
    \begin{tabular}{@{}l@{}}
    \hypertarget{dms-prtl-req-0104-v-01}{DMS-PRTL-REQ-0104-V-01}
    \\\vcdJiraRef{LVV-9946}~{\tiny
    }
    \end{tabular} &
        \begin{tabular}{@{}l@{}}
        \href{https://jira.lsstcorp.org/secure/Tests.jspa\#/testCase/LVV-T738}{LVV-T738} \\
        \vcdDocRef{LDM-540}
        \end{tabular} &
          & \notexec{} \\
  \midrule
  \begin{tabular}{@{}l@{}}
  DMS-PRTL-REQ-0108\\\vcdDocRef{LDM-554}~{\tiny
  }
  \end{tabular} &
    \begin{tabular}{@{}l@{}}
    \hypertarget{dms-prtl-req-0108-v-01}{DMS-PRTL-REQ-0108-V-01}
    \\\vcdJiraRef{LVV-9947}~{\tiny
    }
    \end{tabular} &
        \begin{tabular}{@{}l@{}}
        \href{https://jira.lsstcorp.org/secure/Tests.jspa\#/testCase/LVV-T742}{LVV-T742} \\
        \vcdDocRef{LDM-540}
        \end{tabular} &
          & \notexec{} \\
  \midrule
  \begin{tabular}{@{}l@{}}
  DMS-PRTL-REQ-0103\\\vcdDocRef{LDM-554}~{\tiny
  }
  \end{tabular} &
    \begin{tabular}{@{}l@{}}
    \hypertarget{dms-prtl-req-0103-v-01}{DMS-PRTL-REQ-0103-V-01}
    \\\vcdJiraRef{LVV-9948}~{\tiny
    }
    \end{tabular} &
        \begin{tabular}{@{}l@{}}
        \href{https://jira.lsstcorp.org/secure/Tests.jspa\#/testCase/LVV-T737}{LVV-T737} \\
        \vcdDocRef{LDM-540}
        \end{tabular} &
          & \notexec{} \\
  \midrule
  \begin{tabular}{@{}l@{}}
  DMS-PRTL-REQ-0109\\\vcdDocRef{LDM-554}~{\tiny
  }
  \end{tabular} &
    \begin{tabular}{@{}l@{}}
    \hypertarget{dms-prtl-req-0109-v-01}{DMS-PRTL-REQ-0109-V-01}
    \\\vcdJiraRef{LVV-9949}~{\tiny
    }
    \end{tabular} &
        \begin{tabular}{@{}l@{}}
        \href{https://jira.lsstcorp.org/secure/Tests.jspa\#/testCase/LVV-T743}{LVV-T743} \\
        \vcdDocRef{LDM-540}
        \end{tabular} &
          & \notexec{} \\
  \midrule
  \begin{tabular}{@{}l@{}}
  DMS-PRTL-REQ-0113\\\vcdDocRef{LDM-554}~{\tiny
  }
  \end{tabular} &
    \begin{tabular}{@{}l@{}}
    \hypertarget{dms-prtl-req-0113-v-01}{DMS-PRTL-REQ-0113-V-01}
    \\\vcdJiraRef{LVV-9950}~{\tiny
    }
    \end{tabular} &
        \begin{tabular}{@{}l@{}}
        \href{https://jira.lsstcorp.org/secure/Tests.jspa\#/testCase/LVV-T747}{LVV-T747} \\
        \vcdDocRef{LDM-540}
        \end{tabular} &
          & \notexec{} \\
  \midrule
  \begin{tabular}{@{}l@{}}
  DMS-PRTL-REQ-0111\\\vcdDocRef{LDM-554}~{\tiny
  }
  \end{tabular} &
    \begin{tabular}{@{}l@{}}
    \hypertarget{dms-prtl-req-0111-v-01}{DMS-PRTL-REQ-0111-V-01}
    \\\vcdJiraRef{LVV-9951}~{\tiny
    }
    \end{tabular} &
        \begin{tabular}{@{}l@{}}
        \href{https://jira.lsstcorp.org/secure/Tests.jspa\#/testCase/LVV-T745}{LVV-T745} \\
        \vcdDocRef{LDM-540}
        \end{tabular} &
          & \notexec{} \\
          \cmidrule{3-5}
          & &
        \begin{tabular}{@{}l@{}}
        \href{https://jira.lsstcorp.org/secure/Tests.jspa\#/testCase/LVV-T1818}{LVV-T1818} \\
        \vcdDocRef{LDM-540}
        \end{tabular} &
          \begin{tabular}{@{}l@{}}
          2020-05-11 \\
            \vcdDocRef{DMTR-211}
            {\scriptsize \href{https://jira.lsstcorp.org/secure/Tests.jspa\#/testPlan/LVV-P69}{LVV-P69} }
          \end{tabular} &
          \cndpass \\
  \midrule
  \begin{tabular}{@{}l@{}}
  DMS-PRTL-REQ-0114\\\vcdDocRef{LDM-554}~{\tiny
  }
  \end{tabular} &
    \begin{tabular}{@{}l@{}}
    \hypertarget{dms-prtl-req-0114-v-01}{DMS-PRTL-REQ-0114-V-01}
    \\\vcdJiraRef{LVV-9952}~{\tiny
    }
    \end{tabular} &
        \begin{tabular}{@{}l@{}}
        \href{https://jira.lsstcorp.org/secure/Tests.jspa\#/testCase/LVV-T748}{LVV-T748} \\
        \vcdDocRef{LDM-540}
        \end{tabular} &
          & \notexec{} \\
  \midrule
  \begin{tabular}{@{}l@{}}
  DMS-PRTL-REQ-0112\\\vcdDocRef{LDM-554}~{\tiny
  }
  \end{tabular} &
    \begin{tabular}{@{}l@{}}
    \hypertarget{dms-prtl-req-0112-v-01}{DMS-PRTL-REQ-0112-V-01}
    \\\vcdJiraRef{LVV-9953}~{\tiny
    }
    \end{tabular} &
        \begin{tabular}{@{}l@{}}
        \href{https://jira.lsstcorp.org/secure/Tests.jspa\#/testCase/LVV-T746}{LVV-T746} \\
        \vcdDocRef{LDM-540}
        \end{tabular} &
          & \notexec{} \\
  \midrule
  \begin{tabular}{@{}l@{}}
  DMS-PRTL-REQ-0110\\\vcdDocRef{LDM-554}~{\tiny
  }
  \end{tabular} &
    \begin{tabular}{@{}l@{}}
    \hypertarget{dms-prtl-req-0110-v-01}{DMS-PRTL-REQ-0110-V-01}
    \\\vcdJiraRef{LVV-9954}~{\tiny
    }
    \end{tabular} &
        \begin{tabular}{@{}l@{}}
        \href{https://jira.lsstcorp.org/secure/Tests.jspa\#/testCase/LVV-T744}{LVV-T744} \\
        \vcdDocRef{LDM-540}
        \end{tabular} &
          & \notexec{} \\
          \cmidrule{3-5}
          & &
        \begin{tabular}{@{}l@{}}
        \href{https://jira.lsstcorp.org/secure/Tests.jspa\#/testCase/LVV-T1818}{LVV-T1818} \\
        \vcdDocRef{LDM-540}
        \end{tabular} &
          \begin{tabular}{@{}l@{}}
          2020-05-11 \\
            \vcdDocRef{DMTR-211}
            {\scriptsize \href{https://jira.lsstcorp.org/secure/Tests.jspa\#/testPlan/LVV-P69}{LVV-P69} }
          \end{tabular} &
          \cndpass \\
  \midrule
  \begin{tabular}{@{}l@{}}
  DMS-PRTL-REQ-0115\\\vcdDocRef{LDM-554}~{\tiny
  }
  \end{tabular} &
    \begin{tabular}{@{}l@{}}
    \hypertarget{dms-prtl-req-0115-v-01}{DMS-PRTL-REQ-0115-V-01}
    \\\vcdJiraRef{LVV-9955}~{\tiny
    }
    \end{tabular} &
        \begin{tabular}{@{}l@{}}
        \href{https://jira.lsstcorp.org/secure/Tests.jspa\#/testCase/LVV-T749}{LVV-T749} \\
        \vcdDocRef{LDM-540}
        \end{tabular} &
          & \notexec{} \\
  \midrule
  \begin{tabular}{@{}l@{}}
  DMS-PRTL-REQ-0117\\\vcdDocRef{LDM-554}~{\tiny
  }
  \end{tabular} &
    \begin{tabular}{@{}l@{}}
    \hypertarget{dms-prtl-req-0117-v-01}{DMS-PRTL-REQ-0117-V-01}
    \\\vcdJiraRef{LVV-9956}~{\tiny
    }
    \end{tabular} &
        \begin{tabular}{@{}l@{}}
        \href{https://jira.lsstcorp.org/secure/Tests.jspa\#/testCase/LVV-T751}{LVV-T751} \\
        \vcdDocRef{LDM-540}
        \end{tabular} &
          & \notexec{} \\
  \midrule
  \begin{tabular}{@{}l@{}}
  DMS-PRTL-REQ-0118\\\vcdDocRef{LDM-554}~{\tiny
  }
  \end{tabular} &
    \begin{tabular}{@{}l@{}}
    \hypertarget{dms-prtl-req-0118-v-01}{DMS-PRTL-REQ-0118-V-01}
    \\\vcdJiraRef{LVV-9957}~{\tiny
    }
    \end{tabular} &
        \begin{tabular}{@{}l@{}}
        \href{https://jira.lsstcorp.org/secure/Tests.jspa\#/testCase/LVV-T752}{LVV-T752} \\
        \vcdDocRef{LDM-540}
        \end{tabular} &
          & \notexec{} \\
  \midrule
  \begin{tabular}{@{}l@{}}
  DMS-PRTL-REQ-0116\\\vcdDocRef{LDM-554}~{\tiny
  }
  \end{tabular} &
    \begin{tabular}{@{}l@{}}
    \hypertarget{dms-prtl-req-0116-v-01}{DMS-PRTL-REQ-0116-V-01}
    \\\vcdJiraRef{LVV-9958}~{\tiny
    }
    \end{tabular} &
        \begin{tabular}{@{}l@{}}
        \href{https://jira.lsstcorp.org/secure/Tests.jspa\#/testCase/LVV-T750}{LVV-T750} \\
        \vcdDocRef{LDM-540}
        \end{tabular} &
          & \notexec{} \\
  \midrule
  \begin{tabular}{@{}l@{}}
  DMS-PRTL-REQ-0127\\\vcdDocRef{LDM-554}~{\tiny
  }
  \end{tabular} &
    \begin{tabular}{@{}l@{}}
    \hypertarget{dms-prtl-req-0127-v-01}{DMS-PRTL-REQ-0127-V-01}
    \\\vcdJiraRef{LVV-9959}~{\tiny
    }
    \end{tabular} &
        \begin{tabular}{@{}l@{}}
        \href{https://jira.lsstcorp.org/secure/Tests.jspa\#/testCase/LVV-T756}{LVV-T756} \\
        \vcdDocRef{LDM-540}
        \end{tabular} &
          & \notexec{} \\
  \midrule
  \begin{tabular}{@{}l@{}}
  DMS-PRTL-REQ-0119\\\vcdDocRef{LDM-554}~{\tiny
  }
  \end{tabular} &
    \begin{tabular}{@{}l@{}}
    \hypertarget{dms-prtl-req-0119-v-01}{DMS-PRTL-REQ-0119-V-01}
    \\\vcdJiraRef{LVV-9960}~{\tiny
    }
    \end{tabular} &
        \begin{tabular}{@{}l@{}}
        \href{https://jira.lsstcorp.org/secure/Tests.jspa\#/testCase/LVV-T753}{LVV-T753} \\
        \vcdDocRef{LDM-540}
        \end{tabular} &
          & \notexec{} \\
  \midrule
  \begin{tabular}{@{}l@{}}
  DMS-PRTL-REQ-0120\\\vcdDocRef{LDM-554}~{\tiny
  }
  \end{tabular} &
    \begin{tabular}{@{}l@{}}
    \hypertarget{dms-prtl-req-0120-v-01}{DMS-PRTL-REQ-0120-V-01}
    \\\vcdJiraRef{LVV-9961}~{\tiny
    }
    \end{tabular} &
        \begin{tabular}{@{}l@{}}
        \href{https://jira.lsstcorp.org/secure/Tests.jspa\#/testCase/LVV-T754}{LVV-T754} \\
        \vcdDocRef{LDM-540}
        \end{tabular} &
          & \notexec{} \\
  \midrule
  \begin{tabular}{@{}l@{}}
  DMS-PRTL-REQ-0121\\\vcdDocRef{LDM-554}~{\tiny
  }
  \end{tabular} &
    \begin{tabular}{@{}l@{}}
    \hypertarget{dms-prtl-req-0121-v-01}{DMS-PRTL-REQ-0121-V-01}
    \\\vcdJiraRef{LVV-9962}~{\tiny
    }
    \end{tabular} &
        \begin{tabular}{@{}l@{}}
        \href{https://jira.lsstcorp.org/secure/Tests.jspa\#/testCase/LVV-T755}{LVV-T755} \\
        \vcdDocRef{LDM-540}
        \end{tabular} &
          & \notexec{} \\
  \midrule
  \begin{tabular}{@{}l@{}}
  DMS-PRTL-REQ-0122\\\vcdDocRef{LDM-554}~{\tiny
  }
  \end{tabular} &
    \begin{tabular}{@{}l@{}}
    \hypertarget{dms-prtl-req-0122-v-01}{DMS-PRTL-REQ-0122-V-01}
    \\\vcdJiraRef{LVV-9963}~{\tiny
    }
    \end{tabular} &
        \begin{tabular}{@{}l@{}}
        \href{https://jira.lsstcorp.org/secure/Tests.jspa\#/testCase/LVV-T757}{LVV-T757} \\
        \vcdDocRef{LDM-540}
        \end{tabular} &
          & \notexec{} \\
  \midrule
  \begin{tabular}{@{}l@{}}
  DMS-PRTL-REQ-0124\\\vcdDocRef{LDM-554}~{\tiny
  }
  \end{tabular} &
    \begin{tabular}{@{}l@{}}
    \hypertarget{dms-prtl-req-0124-v-01}{DMS-PRTL-REQ-0124-V-01}
    \\\vcdJiraRef{LVV-9964}~{\tiny
    }
    \end{tabular} &
        \begin{tabular}{@{}l@{}}
        \href{https://jira.lsstcorp.org/secure/Tests.jspa\#/testCase/LVV-T759}{LVV-T759} \\
        \vcdDocRef{LDM-540}
        \end{tabular} &
          & \notexec{} \\
  \midrule
  \begin{tabular}{@{}l@{}}
  DMS-PRTL-REQ-0123\\\vcdDocRef{LDM-554}~{\tiny
  }
  \end{tabular} &
    \begin{tabular}{@{}l@{}}
    \hypertarget{dms-prtl-req-0123-v-01}{DMS-PRTL-REQ-0123-V-01}
    \\\vcdJiraRef{LVV-9965}~{\tiny
    }
    \end{tabular} &
        \begin{tabular}{@{}l@{}}
        \href{https://jira.lsstcorp.org/secure/Tests.jspa\#/testCase/LVV-T758}{LVV-T758} \\
        \vcdDocRef{LDM-540}
        \end{tabular} &
          & \notexec{} \\
  \midrule
  \begin{tabular}{@{}l@{}}
  DMS-PRTL-REQ-0126\\\vcdDocRef{LDM-554}~{\tiny
  }
  \end{tabular} &
    \begin{tabular}{@{}l@{}}
    \hypertarget{dms-prtl-req-0126-v-01}{DMS-PRTL-REQ-0126-V-01}
    \\\vcdJiraRef{LVV-9966}~{\tiny
    }
    \end{tabular} &
        \begin{tabular}{@{}l@{}}
        \href{https://jira.lsstcorp.org/secure/Tests.jspa\#/testCase/LVV-T761}{LVV-T761} \\
        \vcdDocRef{LDM-540}
        \end{tabular} &
          & \notexec{} \\
  \midrule
  \begin{tabular}{@{}l@{}}
  DMS-PRTL-REQ-0125\\\vcdDocRef{LDM-554}~{\tiny
  }
  \end{tabular} &
    \begin{tabular}{@{}l@{}}
    \hypertarget{dms-prtl-req-0125-v-01}{DMS-PRTL-REQ-0125-V-01}
    \\\vcdJiraRef{LVV-9967}~{\tiny
    }
    \end{tabular} &
        \begin{tabular}{@{}l@{}}
        \href{https://jira.lsstcorp.org/secure/Tests.jspa\#/testCase/LVV-T760}{LVV-T760} \\
        \vcdDocRef{LDM-540}
        \end{tabular} &
          & \notexec{} \\
  \midrule
  \begin{tabular}{@{}l@{}}
  DMS-NB-REQ-0010\\\vcdDocRef{LDM-554}~{\tiny
  }
  \end{tabular} &
    \begin{tabular}{@{}l@{}}
    \hypertarget{dms-nb-req-0010-v-01}{DMS-NB-REQ-0010-V-01}
    \\\vcdJiraRef{LVV-9968}~{\tiny
    }
    \end{tabular} &
        \begin{tabular}{@{}l@{}}
        \href{https://jira.lsstcorp.org/secure/Tests.jspa\#/testCase/LVV-T767}{LVV-T767} \\
        \vcdDocRef{LDM-540}
        \end{tabular} &
          & \notexec{} \\
  \midrule
  \begin{tabular}{@{}l@{}}
  DMS-NB-REQ-0009\\\vcdDocRef{LDM-554}~{\tiny
  }
  \end{tabular} &
    \begin{tabular}{@{}l@{}}
    \hypertarget{dms-nb-req-0009-v-01}{DMS-NB-REQ-0009-V-01}
    \\\vcdJiraRef{LVV-9969}~{\tiny
    }
    \end{tabular} &
        \begin{tabular}{@{}l@{}}
        \href{https://jira.lsstcorp.org/secure/Tests.jspa\#/testCase/LVV-T766}{LVV-T766} \\
        \vcdDocRef{LDM-540}
        \end{tabular} &
          & \notexec{} \\
  \midrule
  \begin{tabular}{@{}l@{}}
  DMS-NB-REQ-0014\\\vcdDocRef{LDM-554}~{\tiny
  }
  \end{tabular} &
    \begin{tabular}{@{}l@{}}
    \hypertarget{dms-nb-req-0014-v-01}{DMS-NB-REQ-0014-V-01}
    \\\vcdJiraRef{LVV-9970}~{\tiny
    }
    \end{tabular} &
        \begin{tabular}{@{}l@{}}
        \href{https://jira.lsstcorp.org/secure/Tests.jspa\#/testCase/LVV-T771}{LVV-T771} \\
        \vcdDocRef{LDM-540}
        \end{tabular} &
          & \notexec{} \\
  \midrule
  \begin{tabular}{@{}l@{}}
  DMS-NB-REQ-0005\\\vcdDocRef{LDM-554}~{\tiny
  }
  \end{tabular} &
    \begin{tabular}{@{}l@{}}
    \hypertarget{dms-nb-req-0005-v-01}{DMS-NB-REQ-0005-V-01}
    \\\vcdJiraRef{LVV-9971}~{\tiny
    }
    \end{tabular} &
        \begin{tabular}{@{}l@{}}
        \href{https://jira.lsstcorp.org/secure/Tests.jspa\#/testCase/LVV-T762}{LVV-T762} \\
        \vcdDocRef{LDM-540}
        \end{tabular} &
          & \notexec{} \\
          \cmidrule{3-5}
          & &
        \begin{tabular}{@{}l@{}}
        \href{https://jira.lsstcorp.org/secure/Tests.jspa\#/testCase/LVV-T1436}{LVV-T1436} \\
        \vcdDocRef{LDM-540}
        \end{tabular} &
          \begin{tabular}{@{}l@{}}
          2019-12-09 \\
            \vcdDocRef{DMTR-161}
            {\scriptsize \href{https://jira.lsstcorp.org/secure/Tests.jspa\#/testPlan/LVV-P48}{LVV-P48} }
          \end{tabular} &
          \cndpass \\
  \midrule
  \begin{tabular}{@{}l@{}}
  DMS-NB-REQ-0015\\\vcdDocRef{LDM-554}~{\tiny
  }
  \end{tabular} &
    \begin{tabular}{@{}l@{}}
    \hypertarget{dms-nb-req-0015-v-01}{DMS-NB-REQ-0015-V-01}
    \\\vcdJiraRef{LVV-9972}~{\tiny
    }
    \end{tabular} &
        \begin{tabular}{@{}l@{}}
        \href{https://jira.lsstcorp.org/secure/Tests.jspa\#/testCase/LVV-T772}{LVV-T772} \\
        \vcdDocRef{LDM-540}
        \end{tabular} &
          & \notexec{} \\
  \midrule
  \begin{tabular}{@{}l@{}}
  DMS-NB-REQ-0013\\\vcdDocRef{LDM-554}~{\tiny
  }
  \end{tabular} &
    \begin{tabular}{@{}l@{}}
    \hypertarget{dms-nb-req-0013-v-01}{DMS-NB-REQ-0013-V-01}
    \\\vcdJiraRef{LVV-9973}~{\tiny
    }
    \end{tabular} &
        \begin{tabular}{@{}l@{}}
        \href{https://jira.lsstcorp.org/secure/Tests.jspa\#/testCase/LVV-T770}{LVV-T770} \\
        \vcdDocRef{LDM-540}
        \end{tabular} &
          & \notexec{} \\
          \cmidrule{3-5}
          & &
        \begin{tabular}{@{}l@{}}
        \href{https://jira.lsstcorp.org/secure/Tests.jspa\#/testCase/LVV-T1436}{LVV-T1436} \\
        \vcdDocRef{LDM-540}
        \end{tabular} &
          \begin{tabular}{@{}l@{}}
          2019-12-09 \\
            \vcdDocRef{DMTR-161}
            {\scriptsize \href{https://jira.lsstcorp.org/secure/Tests.jspa\#/testPlan/LVV-P48}{LVV-P48} }
          \end{tabular} &
          \cndpass \\
  \midrule
  \begin{tabular}{@{}l@{}}
  DMS-NB-REQ-0007\\\vcdDocRef{LDM-554}~{\tiny
  }
  \end{tabular} &
    \begin{tabular}{@{}l@{}}
    \hypertarget{dms-nb-req-0007-v-01}{DMS-NB-REQ-0007-V-01}
    \\\vcdJiraRef{LVV-9974}~{\tiny
    }
    \end{tabular} &
        \begin{tabular}{@{}l@{}}
        \href{https://jira.lsstcorp.org/secure/Tests.jspa\#/testCase/LVV-T764}{LVV-T764} \\
        \vcdDocRef{LDM-540}
        \end{tabular} &
          & \notexec{} \\
  \midrule
  \begin{tabular}{@{}l@{}}
  DMS-NB-REQ-0008\\\vcdDocRef{LDM-554}~{\tiny
  }
  \end{tabular} &
    \begin{tabular}{@{}l@{}}
    \hypertarget{dms-nb-req-0008-v-01}{DMS-NB-REQ-0008-V-01}
    \\\vcdJiraRef{LVV-9975}~{\tiny
    }
    \end{tabular} &
        \begin{tabular}{@{}l@{}}
        \href{https://jira.lsstcorp.org/secure/Tests.jspa\#/testCase/LVV-T765}{LVV-T765} \\
        \vcdDocRef{LDM-540}
        \end{tabular} &
          & \notexec{} \\
  \midrule
  \begin{tabular}{@{}l@{}}
  DMS-NB-REQ-0006\\\vcdDocRef{LDM-554}~{\tiny
  }
  \end{tabular} &
    \begin{tabular}{@{}l@{}}
    \hypertarget{dms-nb-req-0006-v-01}{DMS-NB-REQ-0006-V-01}
    \\\vcdJiraRef{LVV-9976}~{\tiny
    }
    \end{tabular} &
        \begin{tabular}{@{}l@{}}
        \href{https://jira.lsstcorp.org/secure/Tests.jspa\#/testCase/LVV-T763}{LVV-T763} \\
        \vcdDocRef{LDM-540}
        \end{tabular} &
          & \notexec{} \\
          \cmidrule{3-5}
          & &
        \begin{tabular}{@{}l@{}}
        \href{https://jira.lsstcorp.org/secure/Tests.jspa\#/testCase/LVV-T1436}{LVV-T1436} \\
        \vcdDocRef{LDM-540}
        \end{tabular} &
          \begin{tabular}{@{}l@{}}
          2019-12-09 \\
            \vcdDocRef{DMTR-161}
            {\scriptsize \href{https://jira.lsstcorp.org/secure/Tests.jspa\#/testPlan/LVV-P48}{LVV-P48} }
          \end{tabular} &
          \cndpass \\
  \midrule
  \begin{tabular}{@{}l@{}}
  DMS-NB-REQ-0012\\\vcdDocRef{LDM-554}~{\tiny
  }
  \end{tabular} &
    \begin{tabular}{@{}l@{}}
    \hypertarget{dms-nb-req-0012-v-01}{DMS-NB-REQ-0012-V-01}
    \\\vcdJiraRef{LVV-9977}~{\tiny
    }
    \end{tabular} &
        \begin{tabular}{@{}l@{}}
        \href{https://jira.lsstcorp.org/secure/Tests.jspa\#/testCase/LVV-T769}{LVV-T769} \\
        \vcdDocRef{LDM-540}
        \end{tabular} &
          & \notexec{} \\
  \midrule
  \begin{tabular}{@{}l@{}}
  DMS-NB-REQ-0011\\\vcdDocRef{LDM-554}~{\tiny
  }
  \end{tabular} &
    \begin{tabular}{@{}l@{}}
    \hypertarget{dms-nb-req-0011-v-01}{DMS-NB-REQ-0011-V-01}
    \\\vcdJiraRef{LVV-9978}~{\tiny
    }
    \end{tabular} &
        \begin{tabular}{@{}l@{}}
        \href{https://jira.lsstcorp.org/secure/Tests.jspa\#/testCase/LVV-T768}{LVV-T768} \\
        \vcdDocRef{LDM-540}
        \end{tabular} &
          & \notexec{} \\
  \midrule
  \begin{tabular}{@{}l@{}}
  DMS-NB-REQ-0023\\\vcdDocRef{LDM-554}~{\tiny
  }
  \end{tabular} &
    \begin{tabular}{@{}l@{}}
    \hypertarget{dms-nb-req-0023-v-01}{DMS-NB-REQ-0023-V-01}
    \\\vcdJiraRef{LVV-9979}~{\tiny
    }
    \end{tabular} &
        \begin{tabular}{@{}l@{}}
        \href{https://jira.lsstcorp.org/secure/Tests.jspa\#/testCase/LVV-T780}{LVV-T780} \\
        \vcdDocRef{LDM-540}
        \end{tabular} &
          & \notexec{} \\
  \midrule
  \begin{tabular}{@{}l@{}}
  DMS-NB-REQ-0017\\\vcdDocRef{LDM-554}~{\tiny
  }
  \end{tabular} &
    \begin{tabular}{@{}l@{}}
    \hypertarget{dms-nb-req-0017-v-01}{DMS-NB-REQ-0017-V-01}
    \\\vcdJiraRef{LVV-9980}~{\tiny
    }
    \end{tabular} &
        \begin{tabular}{@{}l@{}}
        \href{https://jira.lsstcorp.org/secure/Tests.jspa\#/testCase/LVV-T774}{LVV-T774} \\
        \vcdDocRef{LDM-540}
        \end{tabular} &
          & \notexec{} \\
          \cmidrule{3-5}
          & &
        \begin{tabular}{@{}l@{}}
        \href{https://jira.lsstcorp.org/secure/Tests.jspa\#/testCase/LVV-T1436}{LVV-T1436} \\
        \vcdDocRef{LDM-540}
        \end{tabular} &
          \begin{tabular}{@{}l@{}}
          2019-12-09 \\
            \vcdDocRef{DMTR-161}
            {\scriptsize \href{https://jira.lsstcorp.org/secure/Tests.jspa\#/testPlan/LVV-P48}{LVV-P48} }
          \end{tabular} &
          \cndpass \\
  \midrule
  \begin{tabular}{@{}l@{}}
  DMS-NB-REQ-0021\\\vcdDocRef{LDM-554}~{\tiny
  }
  \end{tabular} &
    \begin{tabular}{@{}l@{}}
    \hypertarget{dms-nb-req-0021-v-01}{DMS-NB-REQ-0021-V-01}
    \\\vcdJiraRef{LVV-9981}~{\tiny
    }
    \end{tabular} &
        \begin{tabular}{@{}l@{}}
        \href{https://jira.lsstcorp.org/secure/Tests.jspa\#/testCase/LVV-T778}{LVV-T778} \\
        \vcdDocRef{LDM-540}
        \end{tabular} &
          & \notexec{} \\
  \midrule
  \begin{tabular}{@{}l@{}}
  DMS-NB-REQ-0022\\\vcdDocRef{LDM-554}~{\tiny
  }
  \end{tabular} &
    \begin{tabular}{@{}l@{}}
    \hypertarget{dms-nb-req-0022-v-01}{DMS-NB-REQ-0022-V-01}
    \\\vcdJiraRef{LVV-9982}~{\tiny
    }
    \end{tabular} &
        \begin{tabular}{@{}l@{}}
        \href{https://jira.lsstcorp.org/secure/Tests.jspa\#/testCase/LVV-T779}{LVV-T779} \\
        \vcdDocRef{LDM-540}
        \end{tabular} &
          & \notexec{} \\
  \midrule
  \begin{tabular}{@{}l@{}}
  DMS-NB-REQ-0016\\\vcdDocRef{LDM-554}~{\tiny
  }
  \end{tabular} &
    \begin{tabular}{@{}l@{}}
    \hypertarget{dms-nb-req-0016-v-01}{DMS-NB-REQ-0016-V-01}
    \\\vcdJiraRef{LVV-9983}~{\tiny
    }
    \end{tabular} &
        \begin{tabular}{@{}l@{}}
        \href{https://jira.lsstcorp.org/secure/Tests.jspa\#/testCase/LVV-T773}{LVV-T773} \\
        \vcdDocRef{LDM-540}
        \end{tabular} &
          & \notexec{} \\
  \midrule
  \begin{tabular}{@{}l@{}}
  DMS-NB-REQ-0020\\\vcdDocRef{LDM-554}~{\tiny
  }
  \end{tabular} &
    \begin{tabular}{@{}l@{}}
    \hypertarget{dms-nb-req-0020-v-01}{DMS-NB-REQ-0020-V-01}
    \\\vcdJiraRef{LVV-9984}~{\tiny
    }
    \end{tabular} &
        \begin{tabular}{@{}l@{}}
        \href{https://jira.lsstcorp.org/secure/Tests.jspa\#/testCase/LVV-T777}{LVV-T777} \\
        \vcdDocRef{LDM-540}
        \end{tabular} &
          & \notexec{} \\
  \midrule
  \begin{tabular}{@{}l@{}}
  DMS-NB-REQ-0018\\\vcdDocRef{LDM-554}~{\tiny
  }
  \end{tabular} &
    \begin{tabular}{@{}l@{}}
    \hypertarget{dms-nb-req-0018-v-01}{DMS-NB-REQ-0018-V-01}
    \\\vcdJiraRef{LVV-9985}~{\tiny
    }
    \end{tabular} &
        \begin{tabular}{@{}l@{}}
        \href{https://jira.lsstcorp.org/secure/Tests.jspa\#/testCase/LVV-T775}{LVV-T775} \\
        \vcdDocRef{LDM-540}
        \end{tabular} &
          & \notexec{} \\
  \midrule
  \begin{tabular}{@{}l@{}}
  DMS-NB-REQ-0019\\\vcdDocRef{LDM-554}~{\tiny
  }
  \end{tabular} &
    \begin{tabular}{@{}l@{}}
    \hypertarget{dms-nb-req-0019-v-01}{DMS-NB-REQ-0019-V-01}
    \\\vcdJiraRef{LVV-9986}~{\tiny
    }
    \end{tabular} &
        \begin{tabular}{@{}l@{}}
        \href{https://jira.lsstcorp.org/secure/Tests.jspa\#/testCase/LVV-T776}{LVV-T776} \\
        \vcdDocRef{LDM-540}
        \end{tabular} &
          & \notexec{} \\
  \midrule
  \begin{tabular}{@{}l@{}}
  DMS-NB-REQ-0025\\\vcdDocRef{LDM-554}~{\tiny
  }
  \end{tabular} &
    \begin{tabular}{@{}l@{}}
    \hypertarget{dms-nb-req-0025-v-01}{DMS-NB-REQ-0025-V-01}
    \\\vcdJiraRef{LVV-9987}~{\tiny
    }
    \end{tabular} &
        \begin{tabular}{@{}l@{}}
        \href{https://jira.lsstcorp.org/secure/Tests.jspa\#/testCase/LVV-T782}{LVV-T782} \\
        \vcdDocRef{LDM-540}
        \end{tabular} &
          & \notexec{} \\
  \midrule
  \begin{tabular}{@{}l@{}}
  DMS-NB-REQ-0024\\\vcdDocRef{LDM-554}~{\tiny
  }
  \end{tabular} &
    \begin{tabular}{@{}l@{}}
    \hypertarget{dms-nb-req-0024-v-01}{DMS-NB-REQ-0024-V-01}
    \\\vcdJiraRef{LVV-9988}~{\tiny
    }
    \end{tabular} &
        \begin{tabular}{@{}l@{}}
        \href{https://jira.lsstcorp.org/secure/Tests.jspa\#/testCase/LVV-T781}{LVV-T781} \\
        \vcdDocRef{LDM-540}
        \end{tabular} &
          & \notexec{} \\
  \midrule
  \begin{tabular}{@{}l@{}}
  DMS-NB-REQ-0026\\\vcdDocRef{LDM-554}~{\tiny
  }
  \end{tabular} &
    \begin{tabular}{@{}l@{}}
    \hypertarget{dms-nb-req-0026-v-01}{DMS-NB-REQ-0026-V-01}
    \\\vcdJiraRef{LVV-9989}~{\tiny
    }
    \end{tabular} &
        \begin{tabular}{@{}l@{}}
        \href{https://jira.lsstcorp.org/secure/Tests.jspa\#/testCase/LVV-T783}{LVV-T783} \\
        \vcdDocRef{LDM-540}
        \end{tabular} &
          & \notexec{} \\
  \midrule
  \begin{tabular}{@{}l@{}}
  DMS-NB-REQ-0032\\\vcdDocRef{LDM-554}~{\tiny
  }
  \end{tabular} &
    \begin{tabular}{@{}l@{}}
    \hypertarget{dms-nb-req-0032-v-01}{DMS-NB-REQ-0032-V-01}
    \\\vcdJiraRef{LVV-9990}~{\tiny
    }
    \end{tabular} &
        \begin{tabular}{@{}l@{}}
        \href{https://jira.lsstcorp.org/secure/Tests.jspa\#/testCase/LVV-T784}{LVV-T784} \\
        \vcdDocRef{LDM-540}
        \end{tabular} &
          & \notexec{} \\
  \midrule
  \begin{tabular}{@{}l@{}}
  DMS-NB-REQ-0033\\\vcdDocRef{LDM-554}~{\tiny
  }
  \end{tabular} &
    \begin{tabular}{@{}l@{}}
    \hypertarget{dms-nb-req-0033-v-01}{DMS-NB-REQ-0033-V-01}
    \\\vcdJiraRef{LVV-9991}~{\tiny
    }
    \end{tabular} &
        \begin{tabular}{@{}l@{}}
        \href{https://jira.lsstcorp.org/secure/Tests.jspa\#/testCase/LVV-T785}{LVV-T785} \\
        \vcdDocRef{LDM-540}
        \end{tabular} &
          & \notexec{} \\
  \midrule
  \begin{tabular}{@{}l@{}}
  DMS-NB-REQ-0035\\\vcdDocRef{LDM-554}~{\tiny
  }
  \end{tabular} &
    \begin{tabular}{@{}l@{}}
    \hypertarget{dms-nb-req-0035-v-01}{DMS-NB-REQ-0035-V-01}
    \\\vcdJiraRef{LVV-9992}~{\tiny
    }
    \end{tabular} &
        \begin{tabular}{@{}l@{}}
        \href{https://jira.lsstcorp.org/secure/Tests.jspa\#/testCase/LVV-T787}{LVV-T787} \\
        \vcdDocRef{LDM-540}
        \end{tabular} &
          & \notexec{} \\
  \midrule
  \begin{tabular}{@{}l@{}}
  DMS-NB-REQ-0034\\\vcdDocRef{LDM-554}~{\tiny
  }
  \end{tabular} &
    \begin{tabular}{@{}l@{}}
    \hypertarget{dms-nb-req-0034-v-01}{DMS-NB-REQ-0034-V-01}
    \\\vcdJiraRef{LVV-9993}~{\tiny
    }
    \end{tabular} &
        \begin{tabular}{@{}l@{}}
        \href{https://jira.lsstcorp.org/secure/Tests.jspa\#/testCase/LVV-T786}{LVV-T786} \\
        \vcdDocRef{LDM-540}
        \end{tabular} &
          & \notexec{} \\
  \midrule
  \begin{tabular}{@{}l@{}}
  DMS-NB-REQ-0036\\\vcdDocRef{LDM-554}~{\tiny
  }
  \end{tabular} &
    \begin{tabular}{@{}l@{}}
    \hypertarget{dms-nb-req-0036-v-01}{DMS-NB-REQ-0036-V-01}
    \\\vcdJiraRef{LVV-9994}~{\tiny
    }
    \end{tabular} &
        \begin{tabular}{@{}l@{}}
        \href{https://jira.lsstcorp.org/secure/Tests.jspa\#/testCase/LVV-T788}{LVV-T788} \\
        \vcdDocRef{LDM-540}
        \end{tabular} &
          & \notexec{} \\
  \midrule
  \begin{tabular}{@{}l@{}}
  DMS-NB-REQ-0030\\\vcdDocRef{LDM-554}~{\tiny
  }
  \end{tabular} &
    \begin{tabular}{@{}l@{}}
    \hypertarget{dms-nb-req-0030-v-01}{DMS-NB-REQ-0030-V-01}
    \\\vcdJiraRef{LVV-9995}~{\tiny
    }
    \end{tabular} &
        \begin{tabular}{@{}l@{}}
        \href{https://jira.lsstcorp.org/secure/Tests.jspa\#/testCase/LVV-T790}{LVV-T790} \\
        \vcdDocRef{LDM-540}
        \end{tabular} &
          & \notexec{} \\
  \midrule
  \begin{tabular}{@{}l@{}}
  DMS-NB-REQ-0029\\\vcdDocRef{LDM-554}~{\tiny
  }
  \end{tabular} &
    \begin{tabular}{@{}l@{}}
    \hypertarget{dms-nb-req-0029-v-01}{DMS-NB-REQ-0029-V-01}
    \\\vcdJiraRef{LVV-9996}~{\tiny
    }
    \end{tabular} &
        \begin{tabular}{@{}l@{}}
        \href{https://jira.lsstcorp.org/secure/Tests.jspa\#/testCase/LVV-T789}{LVV-T789} \\
        \vcdDocRef{LDM-540}
        \end{tabular} &
          & \notexec{} \\
          \cmidrule{3-5}
          & &
        \begin{tabular}{@{}l@{}}
        \href{https://jira.lsstcorp.org/secure/Tests.jspa\#/testCase/LVV-T1436}{LVV-T1436} \\
        \vcdDocRef{LDM-540}
        \end{tabular} &
          \begin{tabular}{@{}l@{}}
          2019-12-09 \\
            \vcdDocRef{DMTR-161}
            {\scriptsize \href{https://jira.lsstcorp.org/secure/Tests.jspa\#/testPlan/LVV-P48}{LVV-P48} }
          \end{tabular} &
          \cndpass \\
  \midrule
  \begin{tabular}{@{}l@{}}
  DMS-NB-REQ-0031\\\vcdDocRef{LDM-554}~{\tiny
  }
  \end{tabular} &
    \begin{tabular}{@{}l@{}}
    \hypertarget{dms-nb-req-0031-v-01}{DMS-NB-REQ-0031-V-01}
    \\\vcdJiraRef{LVV-9997}~{\tiny
    }
    \end{tabular} &
        \begin{tabular}{@{}l@{}}
        \href{https://jira.lsstcorp.org/secure/Tests.jspa\#/testCase/LVV-T791}{LVV-T791} \\
        \vcdDocRef{LDM-540}
        \end{tabular} &
          & \notexec{} \\
  \midrule
  \begin{tabular}{@{}l@{}}
  DMS-NB-REQ-0002\\\vcdDocRef{LDM-554}~{\tiny
  }
  \end{tabular} &
    \begin{tabular}{@{}l@{}}
    \hypertarget{dms-nb-req-0002-v-01}{DMS-NB-REQ-0002-V-01}
    \\\vcdJiraRef{LVV-9998}~{\tiny
    }
    \end{tabular} &
        \begin{tabular}{@{}l@{}}
        \href{https://jira.lsstcorp.org/secure/Tests.jspa\#/testCase/LVV-T793}{LVV-T793} \\
        \vcdDocRef{LDM-540}
        \end{tabular} &
          & \notexec{} \\
          \cmidrule{3-5}
          & &
        \begin{tabular}{@{}l@{}}
        \href{https://jira.lsstcorp.org/secure/Tests.jspa\#/testCase/LVV-T1436}{LVV-T1436} \\
        \vcdDocRef{LDM-540}
        \end{tabular} &
          \begin{tabular}{@{}l@{}}
          2019-12-09 \\
            \vcdDocRef{DMTR-161}
            {\scriptsize \href{https://jira.lsstcorp.org/secure/Tests.jspa\#/testPlan/LVV-P48}{LVV-P48} }
          \end{tabular} &
          \cndpass \\
  \midrule
  \begin{tabular}{@{}l@{}}
  DMS-NB-REQ-0003\\\vcdDocRef{LDM-554}~{\tiny
  }
  \end{tabular} &
    \begin{tabular}{@{}l@{}}
    \hypertarget{dms-nb-req-0003-v-01}{DMS-NB-REQ-0003-V-01}
    \\\vcdJiraRef{LVV-9999}~{\tiny
    }
    \end{tabular} &
        \begin{tabular}{@{}l@{}}
        \href{https://jira.lsstcorp.org/secure/Tests.jspa\#/testCase/LVV-T794}{LVV-T794} \\
        \vcdDocRef{LDM-540}
        \end{tabular} &
          & \notexec{} \\
  \midrule
  \begin{tabular}{@{}l@{}}
  DMS-NB-REQ-0001\\\vcdDocRef{LDM-554}~{\tiny
  }
  \end{tabular} &
    \begin{tabular}{@{}l@{}}
    \hypertarget{dms-nb-req-0001-v-01}{DMS-NB-REQ-0001-V-01}
    \\\vcdJiraRef{LVV-10000}~{\tiny
    }
    \end{tabular} &
        \begin{tabular}{@{}l@{}}
        \href{https://jira.lsstcorp.org/secure/Tests.jspa\#/testCase/LVV-T792}{LVV-T792} \\
        \vcdDocRef{LDM-540}
        \end{tabular} &
          & \notexec{} \\
          \cmidrule{3-5}
          & &
        \begin{tabular}{@{}l@{}}
        \href{https://jira.lsstcorp.org/secure/Tests.jspa\#/testCase/LVV-T1436}{LVV-T1436} \\
        \vcdDocRef{LDM-540}
        \end{tabular} &
          \begin{tabular}{@{}l@{}}
          2019-12-09 \\
            \vcdDocRef{DMTR-161}
            {\scriptsize \href{https://jira.lsstcorp.org/secure/Tests.jspa\#/testPlan/LVV-P48}{LVV-P48} }
          \end{tabular} &
          \cndpass \\
  \midrule
  \begin{tabular}{@{}l@{}}
  DMS-NB-REQ-0004\\\vcdDocRef{LDM-554}~{\tiny
  }
  \end{tabular} &
    \begin{tabular}{@{}l@{}}
    \hypertarget{dms-nb-req-0004-v-01}{DMS-NB-REQ-0004-V-01}
    \\\vcdJiraRef{LVV-10001}~{\tiny
    }
    \end{tabular} &
        \begin{tabular}{@{}l@{}}
        \href{https://jira.lsstcorp.org/secure/Tests.jspa\#/testCase/LVV-T795}{LVV-T795} \\
        \vcdDocRef{LDM-540}
        \end{tabular} &
          & \notexec{} \\
  \midrule
  \begin{tabular}{@{}l@{}}
  DMS-API-REQ-0023\\\vcdDocRef{LDM-554}~{\tiny
  }
  \end{tabular} &
    \begin{tabular}{@{}l@{}}
    \hypertarget{dms-api-req-0023-v-01}{DMS-API-REQ-0023-V-01}
    \\\vcdJiraRef{LVV-10002}~{\tiny
    }
    \end{tabular} &
        \begin{tabular}{@{}l@{}}
        \href{https://jira.lsstcorp.org/secure/Tests.jspa\#/testCase/LVV-T798}{LVV-T798} \\
        \vcdDocRef{LDM-540}
        \end{tabular} &
          & \notexec{} \\
          \cmidrule{3-5}
          & &
        \begin{tabular}{@{}l@{}}
        \href{https://jira.lsstcorp.org/secure/Tests.jspa\#/testCase/LVV-T1437}{LVV-T1437} \\
        \vcdDocRef{LDM-540}
        \end{tabular} &
          \begin{tabular}{@{}l@{}}
          2019-12-09 \\
            \vcdDocRef{DMTR-161}
            {\scriptsize \href{https://jira.lsstcorp.org/secure/Tests.jspa\#/testPlan/LVV-P48}{LVV-P48} }
          \end{tabular} &
          \cndpass \\
  \midrule
  \begin{tabular}{@{}l@{}}
  DMS-API-REQ-0022\\\vcdDocRef{LDM-554}~{\tiny
  }
  \end{tabular} &
    \begin{tabular}{@{}l@{}}
    \hypertarget{dms-api-req-0022-v-01}{DMS-API-REQ-0022-V-01}
    \\\vcdJiraRef{LVV-10003}~{\tiny
    }
    \end{tabular} &
        \begin{tabular}{@{}l@{}}
        \href{https://jira.lsstcorp.org/secure/Tests.jspa\#/testCase/LVV-T797}{LVV-T797} \\
        \vcdDocRef{LDM-540}
        \end{tabular} &
          & \notexec{} \\
  \midrule
  \begin{tabular}{@{}l@{}}
  DMS-API-REQ-0028\\\vcdDocRef{LDM-554}~{\tiny
  }
  \end{tabular} &
    \begin{tabular}{@{}l@{}}
    \hypertarget{dms-api-req-0028-v-01}{DMS-API-REQ-0028-V-01}
    \\\vcdJiraRef{LVV-10004}~{\tiny
    }
    \end{tabular} &
        \begin{tabular}{@{}l@{}}
        \href{https://jira.lsstcorp.org/secure/Tests.jspa\#/testCase/LVV-T803}{LVV-T803} \\
        \vcdDocRef{LDM-540}
        \end{tabular} &
          & \notexec{} \\
  \midrule
  \begin{tabular}{@{}l@{}}
  DMS-API-REQ-0024\\\vcdDocRef{LDM-554}~{\tiny
  }
  \end{tabular} &
    \begin{tabular}{@{}l@{}}
    \hypertarget{dms-api-req-0024-v-01}{DMS-API-REQ-0024-V-01}
    \\\vcdJiraRef{LVV-10005}~{\tiny
    }
    \end{tabular} &
        \begin{tabular}{@{}l@{}}
        \href{https://jira.lsstcorp.org/secure/Tests.jspa\#/testCase/LVV-T799}{LVV-T799} \\
        \vcdDocRef{LDM-540}
        \end{tabular} &
          & \notexec{} \\
  \midrule
  \begin{tabular}{@{}l@{}}
  DMS-API-REQ-0026\\\vcdDocRef{LDM-554}~{\tiny
  }
  \end{tabular} &
    \begin{tabular}{@{}l@{}}
    \hypertarget{dms-api-req-0026-v-01}{DMS-API-REQ-0026-V-01}
    \\\vcdJiraRef{LVV-10006}~{\tiny
    }
    \end{tabular} &
        \begin{tabular}{@{}l@{}}
        \href{https://jira.lsstcorp.org/secure/Tests.jspa\#/testCase/LVV-T801}{LVV-T801} \\
        \vcdDocRef{LDM-540}
        \end{tabular} &
          & \notexec{} \\
  \midrule
  \begin{tabular}{@{}l@{}}
  DMS-API-REQ-0027\\\vcdDocRef{LDM-554}~{\tiny
  }
  \end{tabular} &
    \begin{tabular}{@{}l@{}}
    \hypertarget{dms-api-req-0027-v-01}{DMS-API-REQ-0027-V-01}
    \\\vcdJiraRef{LVV-10007}~{\tiny
    }
    \end{tabular} &
        \begin{tabular}{@{}l@{}}
        \href{https://jira.lsstcorp.org/secure/Tests.jspa\#/testCase/LVV-T802}{LVV-T802} \\
        \vcdDocRef{LDM-540}
        \end{tabular} &
          & \notexec{} \\
  \midrule
  \begin{tabular}{@{}l@{}}
  DMS-API-REQ-0030\\\vcdDocRef{LDM-554}~{\tiny
  }
  \end{tabular} &
    \begin{tabular}{@{}l@{}}
    \hypertarget{dms-api-req-0030-v-01}{DMS-API-REQ-0030-V-01}
    \\\vcdJiraRef{LVV-10008}~{\tiny
    }
    \end{tabular} &
        \begin{tabular}{@{}l@{}}
        \href{https://jira.lsstcorp.org/secure/Tests.jspa\#/testCase/LVV-T805}{LVV-T805} \\
        \vcdDocRef{LDM-540}
        \end{tabular} &
          & \notexec{} \\
  \midrule
  \begin{tabular}{@{}l@{}}
  DMS-API-REQ-0025\\\vcdDocRef{LDM-554}~{\tiny
  }
  \end{tabular} &
    \begin{tabular}{@{}l@{}}
    \hypertarget{dms-api-req-0025-v-01}{DMS-API-REQ-0025-V-01}
    \\\vcdJiraRef{LVV-10009}~{\tiny
    }
    \end{tabular} &
        \begin{tabular}{@{}l@{}}
        \href{https://jira.lsstcorp.org/secure/Tests.jspa\#/testCase/LVV-T800}{LVV-T800} \\
        \vcdDocRef{LDM-540}
        \end{tabular} &
          & \notexec{} \\
  \midrule
  \begin{tabular}{@{}l@{}}
  DMS-API-REQ-0029\\\vcdDocRef{LDM-554}~{\tiny
  }
  \end{tabular} &
    \begin{tabular}{@{}l@{}}
    \hypertarget{dms-api-req-0029-v-01}{DMS-API-REQ-0029-V-01}
    \\\vcdJiraRef{LVV-10010}~{\tiny
    }
    \end{tabular} &
        \begin{tabular}{@{}l@{}}
        \href{https://jira.lsstcorp.org/secure/Tests.jspa\#/testCase/LVV-T804}{LVV-T804} \\
        \vcdDocRef{LDM-540}
        \end{tabular} &
          & \notexec{} \\
  \midrule
  \begin{tabular}{@{}l@{}}
  DMS-API-REQ-0021\\\vcdDocRef{LDM-554}~{\tiny
  }
  \end{tabular} &
    \begin{tabular}{@{}l@{}}
    \hypertarget{dms-api-req-0021-v-01}{DMS-API-REQ-0021-V-01}
    \\\vcdJiraRef{LVV-10011}~{\tiny
    }
    \end{tabular} &
        \begin{tabular}{@{}l@{}}
        \href{https://jira.lsstcorp.org/secure/Tests.jspa\#/testCase/LVV-T796}{LVV-T796} \\
        \vcdDocRef{LDM-540}
        \end{tabular} &
          & \notexec{} \\
  \midrule
  \begin{tabular}{@{}l@{}}
  DMS-API-REQ-0009\\\vcdDocRef{LDM-554}~{\tiny
  }
  \end{tabular} &
    \begin{tabular}{@{}l@{}}
    \hypertarget{dms-api-req-0009-v-01}{DMS-API-REQ-0009-V-01}
    \\\vcdJiraRef{LVV-10012}~{\tiny
    }
    \end{tabular} &
        \begin{tabular}{@{}l@{}}
        \href{https://jira.lsstcorp.org/secure/Tests.jspa\#/testCase/LVV-T809}{LVV-T809} \\
        \vcdDocRef{LDM-540}
        \end{tabular} &
          & \notexec{} \\
          \cmidrule{3-5}
          & &
        \begin{tabular}{@{}l@{}}
        \href{https://jira.lsstcorp.org/secure/Tests.jspa\#/testCase/LVV-T1437}{LVV-T1437} \\
        \vcdDocRef{LDM-540}
        \end{tabular} &
          \begin{tabular}{@{}l@{}}
          2019-12-09 \\
            \vcdDocRef{DMTR-161}
            {\scriptsize \href{https://jira.lsstcorp.org/secure/Tests.jspa\#/testPlan/LVV-P48}{LVV-P48} }
          \end{tabular} &
          \cndpass \\
  \midrule
  \begin{tabular}{@{}l@{}}
  DMS-API-REQ-0008\\\vcdDocRef{LDM-554}~{\tiny
  }
  \end{tabular} &
    \begin{tabular}{@{}l@{}}
    \hypertarget{dms-api-req-0008-v-01}{DMS-API-REQ-0008-V-01}
    \\\vcdJiraRef{LVV-10013}~{\tiny
    }
    \end{tabular} &
        \begin{tabular}{@{}l@{}}
        \href{https://jira.lsstcorp.org/secure/Tests.jspa\#/testCase/LVV-T808}{LVV-T808} \\
        \vcdDocRef{LDM-540}
        \end{tabular} &
          & \notexec{} \\
          \cmidrule{3-5}
          & &
        \begin{tabular}{@{}l@{}}
        \href{https://jira.lsstcorp.org/secure/Tests.jspa\#/testCase/LVV-T1437}{LVV-T1437} \\
        \vcdDocRef{LDM-540}
        \end{tabular} &
          \begin{tabular}{@{}l@{}}
          2019-12-09 \\
            \vcdDocRef{DMTR-161}
            {\scriptsize \href{https://jira.lsstcorp.org/secure/Tests.jspa\#/testPlan/LVV-P48}{LVV-P48} }
          \end{tabular} &
          \cndpass \\
  \midrule
  \begin{tabular}{@{}l@{}}
  DMS-API-REQ-0007\\\vcdDocRef{LDM-554}~{\tiny
  }
  \end{tabular} &
    \begin{tabular}{@{}l@{}}
    \hypertarget{dms-api-req-0007-v-01}{DMS-API-REQ-0007-V-01}
    \\\vcdJiraRef{LVV-10014}~{\tiny
    }
    \end{tabular} &
        \begin{tabular}{@{}l@{}}
        \href{https://jira.lsstcorp.org/secure/Tests.jspa\#/testCase/LVV-T807}{LVV-T807} \\
        \vcdDocRef{LDM-540}
        \end{tabular} &
          \begin{tabular}{@{}l@{}}
          2019-11-25 \\
            \vcdDocRef{DMTR-161}
            {\scriptsize \href{https://jira.lsstcorp.org/secure/Tests.jspa\#/testPlan/LVV-P48}{LVV-P48} }
          \end{tabular} &
          \passed \\
          \cmidrule{3-5}
          & &
        \begin{tabular}{@{}l@{}}
        \href{https://jira.lsstcorp.org/secure/Tests.jspa\#/testCase/LVV-T1437}{LVV-T1437} \\
        \vcdDocRef{LDM-540}
        \end{tabular} &
          \begin{tabular}{@{}l@{}}
          2019-12-09 \\
            \vcdDocRef{DMTR-161}
            {\scriptsize \href{https://jira.lsstcorp.org/secure/Tests.jspa\#/testPlan/LVV-P48}{LVV-P48} }
          \end{tabular} &
          \cndpass \\
  \midrule
  \begin{tabular}{@{}l@{}}
  DMS-API-REQ-0006\\\vcdDocRef{LDM-554}~{\tiny
  }
  \end{tabular} &
    \begin{tabular}{@{}l@{}}
    \hypertarget{dms-api-req-0006-v-01}{DMS-API-REQ-0006-V-01}
    \\\vcdJiraRef{LVV-10015}~{\tiny
    }
    \end{tabular} &
        \begin{tabular}{@{}l@{}}
        \href{https://jira.lsstcorp.org/secure/Tests.jspa\#/testCase/LVV-T806}{LVV-T806} \\
        \vcdDocRef{LDM-540}
        \end{tabular} &
          & \notexec{} \\
          \cmidrule{3-5}
          & &
        \begin{tabular}{@{}l@{}}
        \href{https://jira.lsstcorp.org/secure/Tests.jspa\#/testCase/LVV-T1437}{LVV-T1437} \\
        \vcdDocRef{LDM-540}
        \end{tabular} &
          \begin{tabular}{@{}l@{}}
          2019-12-09 \\
            \vcdDocRef{DMTR-161}
            {\scriptsize \href{https://jira.lsstcorp.org/secure/Tests.jspa\#/testPlan/LVV-P48}{LVV-P48} }
          \end{tabular} &
          \cndpass \\
  \midrule
  \begin{tabular}{@{}l@{}}
  DMS-API-REQ-0016\\\vcdDocRef{LDM-554}~{\tiny
  }
  \end{tabular} &
    \begin{tabular}{@{}l@{}}
    \hypertarget{dms-api-req-0016-v-01}{DMS-API-REQ-0016-V-01}
    \\\vcdJiraRef{LVV-10016}~{\tiny
    }
    \end{tabular} &
        \begin{tabular}{@{}l@{}}
        \href{https://jira.lsstcorp.org/secure/Tests.jspa\#/testCase/LVV-T810}{LVV-T810} \\
        \vcdDocRef{LDM-540}
        \end{tabular} &
          & \notexec{} \\
  \midrule
  \begin{tabular}{@{}l@{}}
  DMS-API-REQ-0018\\\vcdDocRef{LDM-554}~{\tiny
  }
  \end{tabular} &
    \begin{tabular}{@{}l@{}}
    \hypertarget{dms-api-req-0018-v-01}{DMS-API-REQ-0018-V-01}
    \\\vcdJiraRef{LVV-10017}~{\tiny
    }
    \end{tabular} &
        \begin{tabular}{@{}l@{}}
        \href{https://jira.lsstcorp.org/secure/Tests.jspa\#/testCase/LVV-T812}{LVV-T812} \\
        \vcdDocRef{LDM-540}
        \end{tabular} &
          & \notexec{} \\
  \midrule
  \begin{tabular}{@{}l@{}}
  DMS-API-REQ-0017\\\vcdDocRef{LDM-554}~{\tiny
  }
  \end{tabular} &
    \begin{tabular}{@{}l@{}}
    \hypertarget{dms-api-req-0017-v-01}{DMS-API-REQ-0017-V-01}
    \\\vcdJiraRef{LVV-10018}~{\tiny
    }
    \end{tabular} &
        \begin{tabular}{@{}l@{}}
        \href{https://jira.lsstcorp.org/secure/Tests.jspa\#/testCase/LVV-T811}{LVV-T811} \\
        \vcdDocRef{LDM-540}
        \end{tabular} &
          & \notexec{} \\
  \midrule
  \begin{tabular}{@{}l@{}}
  DMS-API-REQ-0039\\\vcdDocRef{LDM-554}~{\tiny
  }
  \end{tabular} &
    \begin{tabular}{@{}l@{}}
    \hypertarget{dms-api-req-0039-v-01}{DMS-API-REQ-0039-V-01}
    \\\vcdJiraRef{LVV-10019}~{\tiny
    }
    \end{tabular} &
        \begin{tabular}{@{}l@{}}
        \href{https://jira.lsstcorp.org/secure/Tests.jspa\#/testCase/LVV-T814}{LVV-T814} \\
        \vcdDocRef{LDM-540}
        \end{tabular} &
          & \notexec{} \\
          \cmidrule{3-5}
          & &
        \begin{tabular}{@{}l@{}}
        \href{https://jira.lsstcorp.org/secure/Tests.jspa\#/testCase/LVV-T1437}{LVV-T1437} \\
        \vcdDocRef{LDM-540}
        \end{tabular} &
          \begin{tabular}{@{}l@{}}
          2019-12-09 \\
            \vcdDocRef{DMTR-161}
            {\scriptsize \href{https://jira.lsstcorp.org/secure/Tests.jspa\#/testPlan/LVV-P48}{LVV-P48} }
          \end{tabular} &
          \cndpass \\
  \midrule
  \begin{tabular}{@{}l@{}}
  DMS-API-REQ-0038\\\vcdDocRef{LDM-554}~{\tiny
  }
  \end{tabular} &
    \begin{tabular}{@{}l@{}}
    \hypertarget{dms-api-req-0038-v-01}{DMS-API-REQ-0038-V-01}
    \\\vcdJiraRef{LVV-10020}~{\tiny
    }
    \end{tabular} &
        \begin{tabular}{@{}l@{}}
        \href{https://jira.lsstcorp.org/secure/Tests.jspa\#/testCase/LVV-T813}{LVV-T813} \\
        \vcdDocRef{LDM-540}
        \end{tabular} &
          & \notexec{} \\
  \midrule
  \begin{tabular}{@{}l@{}}
  DMS-API-REQ-0040\\\vcdDocRef{LDM-554}~{\tiny
  }
  \end{tabular} &
    \begin{tabular}{@{}l@{}}
    \hypertarget{dms-api-req-0040-v-01}{DMS-API-REQ-0040-V-01}
    \\\vcdJiraRef{LVV-10021}~{\tiny
    }
    \end{tabular} &
        \begin{tabular}{@{}l@{}}
        \href{https://jira.lsstcorp.org/secure/Tests.jspa\#/testCase/LVV-T815}{LVV-T815} \\
        \vcdDocRef{LDM-540}
        \end{tabular} &
          & \notexec{} \\
  \midrule
  \begin{tabular}{@{}l@{}}
  DMS-API-REQ-0034\\\vcdDocRef{LDM-554}~{\tiny
  }
  \end{tabular} &
    \begin{tabular}{@{}l@{}}
    \hypertarget{dms-api-req-0034-v-01}{DMS-API-REQ-0034-V-01}
    \\\vcdJiraRef{LVV-10022}~{\tiny
    }
    \end{tabular} &
        \begin{tabular}{@{}l@{}}
        \href{https://jira.lsstcorp.org/secure/Tests.jspa\#/testCase/LVV-T816}{LVV-T816} \\
        \vcdDocRef{LDM-540}
        \end{tabular} &
          & \notexec{} \\
  \midrule
  \begin{tabular}{@{}l@{}}
  DMS-API-REQ-0019\\\vcdDocRef{LDM-554}~{\tiny
  }
  \end{tabular} &
    \begin{tabular}{@{}l@{}}
    \hypertarget{dms-api-req-0019-v-01}{DMS-API-REQ-0019-V-01}
    \\\vcdJiraRef{LVV-10023}~{\tiny
    }
    \end{tabular} &
        \begin{tabular}{@{}l@{}}
        \href{https://jira.lsstcorp.org/secure/Tests.jspa\#/testCase/LVV-T817}{LVV-T817} \\
        \vcdDocRef{LDM-540}
        \end{tabular} &
          & \notexec{} \\
  \midrule
  \begin{tabular}{@{}l@{}}
  DMS-API-REQ-0020\\\vcdDocRef{LDM-554}~{\tiny
  }
  \end{tabular} &
    \begin{tabular}{@{}l@{}}
    \hypertarget{dms-api-req-0020-v-01}{DMS-API-REQ-0020-V-01}
    \\\vcdJiraRef{LVV-10024}~{\tiny
    }
    \end{tabular} &
        \begin{tabular}{@{}l@{}}
        \href{https://jira.lsstcorp.org/secure/Tests.jspa\#/testCase/LVV-T818}{LVV-T818} \\
        \vcdDocRef{LDM-540}
        \end{tabular} &
          & \notexec{} \\
  \midrule
  \begin{tabular}{@{}l@{}}
  DMS-API-REQ-0014\\\vcdDocRef{LDM-554}~{\tiny
  }
  \end{tabular} &
    \begin{tabular}{@{}l@{}}
    \hypertarget{dms-api-req-0014-v-01}{DMS-API-REQ-0014-V-01}
    \\\vcdJiraRef{LVV-10025}~{\tiny
    }
    \end{tabular} &
        \begin{tabular}{@{}l@{}}
        \href{https://jira.lsstcorp.org/secure/Tests.jspa\#/testCase/LVV-T823}{LVV-T823} \\
        \vcdDocRef{LDM-540}
        \end{tabular} &
          & \notexec{} \\
  \midrule
  \begin{tabular}{@{}l@{}}
  DMS-API-REQ-0013\\\vcdDocRef{LDM-554}~{\tiny
  }
  \end{tabular} &
    \begin{tabular}{@{}l@{}}
    \hypertarget{dms-api-req-0013-v-01}{DMS-API-REQ-0013-V-01}
    \\\vcdJiraRef{LVV-10026}~{\tiny
    }
    \end{tabular} &
        \begin{tabular}{@{}l@{}}
        \href{https://jira.lsstcorp.org/secure/Tests.jspa\#/testCase/LVV-T822}{LVV-T822} \\
        \vcdDocRef{LDM-540}
        \end{tabular} &
          & \notexec{} \\
  \midrule
  \begin{tabular}{@{}l@{}}
  DMS-API-REQ-0015\\\vcdDocRef{LDM-554}~{\tiny
  }
  \end{tabular} &
    \begin{tabular}{@{}l@{}}
    \hypertarget{dms-api-req-0015-v-01}{DMS-API-REQ-0015-V-01}
    \\\vcdJiraRef{LVV-10027}~{\tiny
    }
    \end{tabular} &
        \begin{tabular}{@{}l@{}}
        \href{https://jira.lsstcorp.org/secure/Tests.jspa\#/testCase/LVV-T824}{LVV-T824} \\
        \vcdDocRef{LDM-540}
        \end{tabular} &
          & \notexec{} \\
  \midrule
  \begin{tabular}{@{}l@{}}
  DMS-API-REQ-0012\\\vcdDocRef{LDM-554}~{\tiny
  }
  \end{tabular} &
    \begin{tabular}{@{}l@{}}
    \hypertarget{dms-api-req-0012-v-01}{DMS-API-REQ-0012-V-01}
    \\\vcdJiraRef{LVV-10028}~{\tiny
    }
    \end{tabular} &
        \begin{tabular}{@{}l@{}}
        \href{https://jira.lsstcorp.org/secure/Tests.jspa\#/testCase/LVV-T821}{LVV-T821} \\
        \vcdDocRef{LDM-540}
        \end{tabular} &
          & \notexec{} \\
  \midrule
  \begin{tabular}{@{}l@{}}
  DMS-API-REQ-0010\\\vcdDocRef{LDM-554}~{\tiny
  }
  \end{tabular} &
    \begin{tabular}{@{}l@{}}
    \hypertarget{dms-api-req-0010-v-01}{DMS-API-REQ-0010-V-01}
    \\\vcdJiraRef{LVV-10029}~{\tiny
    }
    \end{tabular} &
        \begin{tabular}{@{}l@{}}
        \href{https://jira.lsstcorp.org/secure/Tests.jspa\#/testCase/LVV-T819}{LVV-T819} \\
        \vcdDocRef{LDM-540}
        \end{tabular} &
          & \notexec{} \\
  \midrule
  \begin{tabular}{@{}l@{}}
  DMS-API-REQ-0011\\\vcdDocRef{LDM-554}~{\tiny
  }
  \end{tabular} &
    \begin{tabular}{@{}l@{}}
    \hypertarget{dms-api-req-0011-v-01}{DMS-API-REQ-0011-V-01}
    \\\vcdJiraRef{LVV-10030}~{\tiny
    }
    \end{tabular} &
        \begin{tabular}{@{}l@{}}
        \href{https://jira.lsstcorp.org/secure/Tests.jspa\#/testCase/LVV-T820}{LVV-T820} \\
        \vcdDocRef{LDM-540}
        \end{tabular} &
          & \notexec{} \\
  \midrule
  \begin{tabular}{@{}l@{}}
  DMS-API-REQ-0033\\\vcdDocRef{LDM-554}~{\tiny
  }
  \end{tabular} &
    \begin{tabular}{@{}l@{}}
    \hypertarget{dms-api-req-0033-v-01}{DMS-API-REQ-0033-V-01}
    \\\vcdJiraRef{LVV-10031}~{\tiny
    }
    \end{tabular} &
        \begin{tabular}{@{}l@{}}
        \href{https://jira.lsstcorp.org/secure/Tests.jspa\#/testCase/LVV-T827}{LVV-T827} \\
        \vcdDocRef{LDM-540}
        \end{tabular} &
          & \notexec{} \\
  \midrule
  \begin{tabular}{@{}l@{}}
  DMS-API-REQ-0031\\\vcdDocRef{LDM-554}~{\tiny
  }
  \end{tabular} &
    \begin{tabular}{@{}l@{}}
    \hypertarget{dms-api-req-0031-v-01}{DMS-API-REQ-0031-V-01}
    \\\vcdJiraRef{LVV-10032}~{\tiny
    }
    \end{tabular} &
        \begin{tabular}{@{}l@{}}
        \href{https://jira.lsstcorp.org/secure/Tests.jspa\#/testCase/LVV-T825}{LVV-T825} \\
        \vcdDocRef{LDM-540}
        \end{tabular} &
          & \notexec{} \\
  \midrule
  \begin{tabular}{@{}l@{}}
  DMS-API-REQ-0032\\\vcdDocRef{LDM-554}~{\tiny
  }
  \end{tabular} &
    \begin{tabular}{@{}l@{}}
    \hypertarget{dms-api-req-0032-v-01}{DMS-API-REQ-0032-V-01}
    \\\vcdJiraRef{LVV-10033}~{\tiny
    }
    \end{tabular} &
        \begin{tabular}{@{}l@{}}
        \href{https://jira.lsstcorp.org/secure/Tests.jspa\#/testCase/LVV-T826}{LVV-T826} \\
        \vcdDocRef{LDM-540}
        \end{tabular} &
          & \notexec{} \\
  \midrule
  \begin{tabular}{@{}l@{}}
  DMS-API-REQ-0003\\\vcdDocRef{LDM-554}~{\tiny
  }
  \end{tabular} &
    \begin{tabular}{@{}l@{}}
    \hypertarget{dms-api-req-0003-v-01}{DMS-API-REQ-0003-V-01}
    \\\vcdJiraRef{LVV-10034}~{\tiny
    }
    \end{tabular} &
        \begin{tabular}{@{}l@{}}
        \href{https://jira.lsstcorp.org/secure/Tests.jspa\#/testCase/LVV-T829}{LVV-T829} \\
        \vcdDocRef{LDM-540}
        \end{tabular} &
          & \notexec{} \\
          \cmidrule{3-5}
          & &
        \begin{tabular}{@{}l@{}}
        \href{https://jira.lsstcorp.org/secure/Tests.jspa\#/testCase/LVV-T1437}{LVV-T1437} \\
        \vcdDocRef{LDM-540}
        \end{tabular} &
          \begin{tabular}{@{}l@{}}
          2019-12-09 \\
            \vcdDocRef{DMTR-161}
            {\scriptsize \href{https://jira.lsstcorp.org/secure/Tests.jspa\#/testPlan/LVV-P48}{LVV-P48} }
          \end{tabular} &
          \cndpass \\
  \midrule
  \begin{tabular}{@{}l@{}}
  DMS-API-REQ-0004\\\vcdDocRef{LDM-554}~{\tiny
  }
  \end{tabular} &
    \begin{tabular}{@{}l@{}}
    \hypertarget{dms-api-req-0004-v-01}{DMS-API-REQ-0004-V-01}
    \\\vcdJiraRef{LVV-10035}~{\tiny
    }
    \end{tabular} &
        \begin{tabular}{@{}l@{}}
        \href{https://jira.lsstcorp.org/secure/Tests.jspa\#/testCase/LVV-T830}{LVV-T830} \\
        \vcdDocRef{LDM-540}
        \end{tabular} &
          & \notexec{} \\
          \cmidrule{3-5}
          & &
        \begin{tabular}{@{}l@{}}
        \href{https://jira.lsstcorp.org/secure/Tests.jspa\#/testCase/LVV-T1437}{LVV-T1437} \\
        \vcdDocRef{LDM-540}
        \end{tabular} &
          \begin{tabular}{@{}l@{}}
          2019-12-09 \\
            \vcdDocRef{DMTR-161}
            {\scriptsize \href{https://jira.lsstcorp.org/secure/Tests.jspa\#/testPlan/LVV-P48}{LVV-P48} }
          \end{tabular} &
          \cndpass \\
  \midrule
  \begin{tabular}{@{}l@{}}
  DMS-API-REQ-0005\\\vcdDocRef{LDM-554}~{\tiny
  }
  \end{tabular} &
    \begin{tabular}{@{}l@{}}
    \hypertarget{dms-api-req-0005-v-01}{DMS-API-REQ-0005-V-01}
    \\\vcdJiraRef{LVV-10036}~{\tiny
    }
    \end{tabular} &
        \begin{tabular}{@{}l@{}}
        \href{https://jira.lsstcorp.org/secure/Tests.jspa\#/testCase/LVV-T831}{LVV-T831} \\
        \vcdDocRef{LDM-540}
        \end{tabular} &
          & \notexec{} \\
  \midrule
  \begin{tabular}{@{}l@{}}
  DMS-API-REQ-0001\\\vcdDocRef{LDM-554}~{\tiny
  }
  \end{tabular} &
    \begin{tabular}{@{}l@{}}
    \hypertarget{dms-api-req-0001-v-01}{DMS-API-REQ-0001-V-01}
    \\\vcdJiraRef{LVV-10037}~{\tiny
    }
    \end{tabular} &
        \begin{tabular}{@{}l@{}}
        \href{https://jira.lsstcorp.org/secure/Tests.jspa\#/testCase/LVV-T828}{LVV-T828} \\
        \vcdDocRef{LDM-540}
        \end{tabular} &
          & \notexec{} \\
          \cmidrule{3-5}
          & &
        \begin{tabular}{@{}l@{}}
        \href{https://jira.lsstcorp.org/secure/Tests.jspa\#/testCase/LVV-T1437}{LVV-T1437} \\
        \vcdDocRef{LDM-540}
        \end{tabular} &
          \begin{tabular}{@{}l@{}}
          2019-12-09 \\
            \vcdDocRef{DMTR-161}
            {\scriptsize \href{https://jira.lsstcorp.org/secure/Tests.jspa\#/testPlan/LVV-P48}{LVV-P48} }
          \end{tabular} &
          \cndpass \\
  \midrule
  \begin{tabular}{@{}l@{}}
  DMS-API-REQ-0035\\\vcdDocRef{LDM-554}~{\tiny
  }
  \end{tabular} &
    \begin{tabular}{@{}l@{}}
    \hypertarget{dms-api-req-0035-v-01}{DMS-API-REQ-0035-V-01}
    \\\vcdJiraRef{LVV-10038}~{\tiny
    }
    \end{tabular} &
        \begin{tabular}{@{}l@{}}
        \href{https://jira.lsstcorp.org/secure/Tests.jspa\#/testCase/LVV-T832}{LVV-T832} \\
        \vcdDocRef{LDM-540}
        \end{tabular} &
          & \notexec{} \\
  \midrule
  \begin{tabular}{@{}l@{}}
  DMS-API-REQ-0037\\\vcdDocRef{LDM-554}~{\tiny
  }
  \end{tabular} &
    \begin{tabular}{@{}l@{}}
    \hypertarget{dms-api-req-0037-v-01}{DMS-API-REQ-0037-V-01}
    \\\vcdJiraRef{LVV-10039}~{\tiny
    }
    \end{tabular} &
        \begin{tabular}{@{}l@{}}
        \href{https://jira.lsstcorp.org/secure/Tests.jspa\#/testCase/LVV-T835}{LVV-T835} \\
        \vcdDocRef{LDM-540}
        \end{tabular} &
          & \notexec{} \\
  \midrule
  \begin{tabular}{@{}l@{}}
  DMS-API-REQ-0002\\\vcdDocRef{LDM-554}~{\tiny
  }
  \end{tabular} &
    \begin{tabular}{@{}l@{}}
    \hypertarget{dms-api-req-0002-v-01}{DMS-API-REQ-0002-V-01}
    \\\vcdJiraRef{LVV-10040}~{\tiny
    }
    \end{tabular} &
        \begin{tabular}{@{}l@{}}
        \href{https://jira.lsstcorp.org/secure/Tests.jspa\#/testCase/LVV-T833}{LVV-T833} \\
        \vcdDocRef{LDM-540}
        \end{tabular} &
          & \notexec{} \\
  \midrule
  \begin{tabular}{@{}l@{}}
  DMS-API-REQ-0036\\\vcdDocRef{LDM-554}~{\tiny
  }
  \end{tabular} &
    \begin{tabular}{@{}l@{}}
    \hypertarget{dms-api-req-0036-v-01}{DMS-API-REQ-0036-V-01}
    \\\vcdJiraRef{LVV-10041}~{\tiny
    }
    \end{tabular} &
        \begin{tabular}{@{}l@{}}
        \href{https://jira.lsstcorp.org/secure/Tests.jspa\#/testCase/LVV-T834}{LVV-T834} \\
        \vcdDocRef{LDM-540}
        \end{tabular} &
          & \notexec{} \\
  \midrule
\label{tab:dmvcd}
\end{longtable}
}

\subsection{LSE-400 Requirements Coverage}

\setlength\LTleft{-0.25in}
\setlength\LTright{-0.5in}
{\small
\begin{longtable}{lllll}
\caption{ DM LSE-400 Requirements.} \\
\toprule
\textbf{Requirement} & \textbf{Verification Element} & \textbf{Test Case} & \textbf{Last Run} & \textbf{Test Status} \\
\toprule
\endhead
  \begin{tabular}{@{}l@{}}
  OCS-EFD-HS-0001\\\vcdDocRef{LSE-400}~{\tiny
  }
  \end{tabular} &
    \begin{tabular}{@{}l@{}}
    \hypertarget{ocs-efd-hs-0001-v-01}{OCS-EFD-HS-0001-V-01}
    \\\vcdJiraRef{LVV-18271}~{\tiny
    }
    \end{tabular} &
        & & \\
  \midrule
  \begin{tabular}{@{}l@{}}
  OCS-EFD-HS-0002\\\vcdDocRef{LSE-400}~{\tiny
  }
  \end{tabular} &
    \begin{tabular}{@{}l@{}}
    \hypertarget{ocs-efd-hs-0002-v-01}{OCS-EFD-HS-0002-V-01}
    \\\vcdJiraRef{LVV-18272}~{\tiny
    }
    \end{tabular} &
        & & \\
  \midrule
  \begin{tabular}{@{}l@{}}
  OCS-EFD-HS-0003\\\vcdDocRef{LSE-400}~{\tiny
  }
  \end{tabular} &
    \begin{tabular}{@{}l@{}}
    \hypertarget{ocs-efd-hs-0003-v-01}{OCS-EFD-HS-0003-V-01}
    \\\vcdJiraRef{LVV-18273}~{\tiny
    }
    \end{tabular} &
        & & \\
  \midrule
  \begin{tabular}{@{}l@{}}
  OCS-EFD-HS-0004\\\vcdDocRef{LSE-400}~{\tiny
  }
  \end{tabular} &
    \begin{tabular}{@{}l@{}}
    \hypertarget{ocs-efd-hs-0004-v-01}{OCS-EFD-HS-0004-V-01}
    \\\vcdJiraRef{LVV-18274}~{\tiny
    }
    \end{tabular} &
        & & \\
  \midrule
  \begin{tabular}{@{}l@{}}
  OCS-EFD-HS-0005\\\vcdDocRef{LSE-400}~{\tiny
  }
  \end{tabular} &
    \begin{tabular}{@{}l@{}}
    \hypertarget{ocs-efd-hs-0005-v-01}{OCS-EFD-HS-0005-V-01}
    \\\vcdJiraRef{LVV-18275}~{\tiny
    }
    \end{tabular} &
        & & \\
  \midrule
  \begin{tabular}{@{}l@{}}
  OCS-EFD-HS-0006\\\vcdDocRef{LSE-400}~{\tiny
  }
  \end{tabular} &
    \begin{tabular}{@{}l@{}}
    \hypertarget{ocs-efd-hs-0006-v-01}{OCS-EFD-HS-0006-V-01}
    \\\vcdJiraRef{LVV-18276}~{\tiny
    }
    \end{tabular} &
        & & \\
  \midrule
  \begin{tabular}{@{}l@{}}
  OCS-EFD-HS-0007\\\vcdDocRef{LSE-400}~{\tiny
  }
  \end{tabular} &
    \begin{tabular}{@{}l@{}}
    \hypertarget{ocs-efd-hs-0007-v-01}{OCS-EFD-HS-0007-V-01}
    \\\vcdJiraRef{LVV-18277}~{\tiny
    }
    \end{tabular} &
        & & \\
  \midrule
  \begin{tabular}{@{}l@{}}
  OCS-EFD-HS-0008\\\vcdDocRef{LSE-400}~{\tiny
  }
  \end{tabular} &
    \begin{tabular}{@{}l@{}}
    \hypertarget{ocs-efd-hs-0008-v-01}{OCS-EFD-HS-0008-V-01}
    \\\vcdJiraRef{LVV-18278}~{\tiny
    }
    \end{tabular} &
        & & \\
  \midrule
  \begin{tabular}{@{}l@{}}
  OCS-EFD-HS-0009\\\vcdDocRef{LSE-400}~{\tiny
  }
  \end{tabular} &
    \begin{tabular}{@{}l@{}}
    \hypertarget{ocs-efd-hs-0009-v-01}{OCS-EFD-HS-0009-V-01}
    \\\vcdJiraRef{LVV-18279}~{\tiny
    }
    \end{tabular} &
        & & \\
  \midrule
  \begin{tabular}{@{}l@{}}
  OCS-EFD-HS-0010\\\vcdDocRef{LSE-400}~{\tiny
  }
  \end{tabular} &
    \begin{tabular}{@{}l@{}}
    \hypertarget{ocs-efd-hs-0010-v-01}{OCS-EFD-HS-0010-V-01}
    \\\vcdJiraRef{LVV-18280}~{\tiny
    }
    \end{tabular} &
        & & \\
  \midrule
  \begin{tabular}{@{}l@{}}
  OCS-EFD-HS-0011\\\vcdDocRef{LSE-400}~{\tiny
  }
  \end{tabular} &
    \begin{tabular}{@{}l@{}}
    \hypertarget{ocs-efd-hs-0011-v-01}{OCS-EFD-HS-0011-V-01}
    \\\vcdJiraRef{LVV-18281}~{\tiny
    }
    \end{tabular} &
        & & \\
  \midrule
  \begin{tabular}{@{}l@{}}
  OCS-EFD-HS-0012\\\vcdDocRef{LSE-400}~{\tiny
  }
  \end{tabular} &
    \begin{tabular}{@{}l@{}}
    \hypertarget{ocs-efd-hs-0012-v-01}{OCS-EFD-HS-0012-V-01}
    \\\vcdJiraRef{LVV-18282}~{\tiny
    }
    \end{tabular} &
        & & \\
  \midrule
  \begin{tabular}{@{}l@{}}
  OCS-EFD-HS-0013\\\vcdDocRef{LSE-400}~{\tiny
  }
  \end{tabular} &
    \begin{tabular}{@{}l@{}}
    \hypertarget{ocs-efd-hs-0013-v-01}{OCS-EFD-HS-0013-V-01}
    \\\vcdJiraRef{LVV-18283}~{\tiny
    }
    \end{tabular} &
        & & \\
  \midrule
  \begin{tabular}{@{}l@{}}
  OCS-EFD-HS-0014\\\vcdDocRef{LSE-400}~{\tiny
  }
  \end{tabular} &
    \begin{tabular}{@{}l@{}}
    \hypertarget{ocs-efd-hs-0014-v-01}{OCS-EFD-HS-0014-V-01}
    \\\vcdJiraRef{LVV-18284}~{\tiny
    }
    \end{tabular} &
        & & \\
  \midrule
  \begin{tabular}{@{}l@{}}
  OCS-EFD-HS-0015\\\vcdDocRef{LSE-400}~{\tiny
  }
  \end{tabular} &
    \begin{tabular}{@{}l@{}}
    \hypertarget{ocs-efd-hs-0015-v-01}{OCS-EFD-HS-0015-V-01}
    \\\vcdJiraRef{LVV-18285}~{\tiny
    }
    \end{tabular} &
        & & \\
  \midrule
\label{tab:dmvcd}
\end{longtable}
}




\appendix
\newpage
\section{References\label{sect:references}}
\renewcommand{\refname}{}
\bibliography{lsst,refs,books,refs_ads,local.bib}

\section{Acronyms \label{sect:acronyms}} % include acronyms.tex generated by the generateAcronyms.py (in texmf/scripts)
\addtocounter{table}{-1}
\begin{longtable}{|l|p{0.8\textwidth}|}\hline
\textbf{Acronym} & \textbf{Description}  \\\hline

API & Application Programming Interface \\\hline
CA-DM-CON-ICD & Requirements for Interface between the Camera and Data Management (\citeds{LSE-69}) \\\hline
CA-DM-DAQ-ICD & Requirements for Camera Data Acquisition Interface (\citeds{LSE-68}) \\\hline
CA-DM-SUP-ICD & Requirements for Support-Data Exchanges between Data Management and Camera (\citeds{LSE-130}) \\\hline
CPT-OCS-INT-ICD & Summit Computer Room Requirements (\citeds{LSE-209}) \\\hline
DAQ & Data Acquisition System \\\hline
DM & Data Management \\\hline
DM-TS-AUX-ICD & Requirements for Auxiliary Instrumentation Interface between Data Management and Telescope (\citeds{LSE-140}) \\\hline
DM-TS-CON-ICD & Requirements for Control System Interfaces between the Telescope \& Data Management (\citeds{LSE-75}) \\\hline
DMS & Data Management Subsystem \\\hline
DMS-API-REQ & LSP API Aspect requirements (\citeds{LDM-554}) \\\hline
DMS-LSP-REQ & Top level LSP requirements (\citeds{LDM-554}) \\\hline
DMS-NB-REQ & LSP Notebook Aspect requirements (\citeds{LDM-554}) \\\hline
DMS-PRTL-REQ & LSP Portal Aspect requirements (\citeds{LDM-554}) \\\hline
DMS-REQ & Data Management top level requirements (\citeds{LSE-61}) \\\hline
DMTR & DM Test (Plan and) Report \\\hline
Document & Any object (in any application supported by DocuShare or design archives such as PDMWorks or GIT) that supports project management or records milestones and deliverables of the LSST Project \\\hline
EP-DM-CON-ICD & Requirements for Interface between Data Management and EPO (\citeds{LSE-131}) \\\hline
ICD & Interface Control Document \\\hline
LDM & LSST Data Management (Document Handle) \\\hline
LSE & LSST Systems Engineering (Document Handle) \\\hline
LSP & LSST Science Platform \\\hline
LSST & Large Synoptic Survey Telescope \\\hline
LVV & LSST Verification and Validation (Jira project) \\\hline
OCS & Observatory Control System \\\hline
OCS-DM-COM-ICD & Requirements for OCS - Data Management Software Communication Interface (\citeds{LSE-72}) \\\hline
Requirement & A declaration of a specified function or quantitative performance that the delivered system or subsystem must meet.  It is a statement that identifies a necessary attribute, capability, characteristic, or quality of a system in order for the delivered system or subsystem to meet a derived or higher requirement, constraint, or function. \\\hline
SYS-ALL-COM-ICD & Requirements for LSST Observatory Control System Communication Architecture and Protocol (\citeds{LSE-70}) \\\hline
Specification & One or more performance parameter(s) being established by a requirement that the delivered system or subsystem must meet \\\hline
Subsystem & A set of elements comprising a system within the larger LSST system that is responsible for a key technical deliverable of the project. \\\hline
TS & Test Specification \\\hline
VCD & Verification Control Document \\\hline
Verification & The process of evaluating the design, including hardware and software - to ensure the requirements have been met;  verification (of requirements) is performed by test, analysis, inspection, and/or demonstration \\\hline
\end{longtable}


%\section{Glossary \label{sect:glossary}}
%\renewcommand{\refname}{}
%\printglossaries

\end{document}
