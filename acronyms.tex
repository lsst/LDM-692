\addtocounter{table}{-1}
\begin{longtable}{|p{0.145\textwidth}|p{0.8\textwidth}|}\hline
\textbf{Acronym} & \textbf{Description}  \\\hline

API & Application Programming Interface \\\hline
CA-DM-CON-ICD & Requirements for Interface between the Camera and Data Management (\citeds{LSE-69}) \\\hline
CA-DM-DAQ-ICD & Requirements for Camera Data Acquisition Interface (\citeds{LSE-68}) \\\hline
CA-DM-SUP-ICD & Requirements for Support-Data Exchanges between Data Management and Camera (\citeds{LSE-130}) \\\hline
CPT-OCS-INT-ICD & Summit Computer Room Requirements (\citeds{LSE-209}) \\\hline
DAQ & Data Acquisition System \\\hline
DM & Data Management \\\hline
DM-TS-AUX-ICD & Requirements for Auxiliary Instrumentation Interface between Data Management and Telescope (\citeds{LSE-140}) \\\hline
DM-TS-CON-ICD & Requirements for Control System Interfaces between the Telescope \& Data Management (\citeds{LSE-75}) \\\hline
DMS & Data Management Subsystem \\\hline
DMS-API-REQ & LSP API Aspect requirements (\citeds{LDM-554}) \\\hline
DMS-LSP-REQ & Top level LSP requirements (\citeds{LDM-554}) \\\hline
DMS-NB-REQ & LSP Notebook Aspect requirements (\citeds{LDM-554}) \\\hline
DMS-PRTL-REQ & LSP Portal Aspect requirements (\citeds{LDM-554}) \\\hline
DMS-REQ & Data Management top level requirements (\citeds{LSE-61}) \\\hline
DMTR & DM Test (Plan and) Report \\\hline
Data Management Subsystem & The subsystems within Data Management may contain a defined combination of hardware, a software stack, a set of running processes, and the people who manage them: they are a major component of the DM System operations. Examples include the 'Archive Operations Subsystem' and the 'Data Processing Subsystem'"." \\\hline
Data Management System & The computing infrastructure, middleware, and applications that process, store, and enable information extraction from the LSST dataset; the DMS will process peta-scale data volume, convert raw images into a faithful representation of the universe, and archive the results in a useful form. The infrastructure layer consists of the computing, storage, networking hardware, and system software. The middleware layer handles distributed processing, data access, user interface, and system operations services. The applications layer includes the data pipelines and the science data archives' products and services. \\\hline
Document & Any object (in any application supported by DocuShare or design archives such as PDMWorks or GIT) that supports project management or records milestones and deliverables of the LSST Project \\\hline
EP-DM-CON-ICD & Requirements for Interface between Data Management and EPO (\citeds{LSE-131}) \\\hline
ICD & Interface Control Document \\\hline
LDM & LSST Data Management (Document Handle) \\\hline
LSE & LSST Systems Engineering (Document Handle) \\\hline
LSP & LSST Science Platform \\\hline
LSST & Large Synoptic Survey Telescope \\\hline
LVV & LSST Verification and Validation (Jira project) \\\hline
OCS & Observatory Control System \\\hline
OCS-DM-COM-ICD & Requirements for OCS - Data Management Software Communication Interface (\citeds{LSE-72}) \\\hline
Requirement & A declaration of a specified function or quantitative performance that the delivered system or subsystem must meet.  It is a statement that identifies a necessary attribute, capability, characteristic, or quality of a system in order for the delivered system or subsystem to meet a derived or higher requirement, constraint, or function. \\\hline
SYS-ALL-COM-ICD & Requirements for LSST Observatory Control System Communication Architecture and Protocol (\citeds{LSE-70}) \\\hline
Specification & One or more performance parameter(s) being established by a requirement that the delivered system or subsystem must meet \\\hline
Subsystem & A set of elements comprising a system within the larger LSST system that is responsible for a key technical deliverable of the project. \\\hline
TS & Test Specification \\\hline
VCD & Verification Control Document \\\hline
Verification & The process of evaluating the design, including hardware and software - to ensure the requirements have been met;  verification (of requirements) is performed by test, analysis, inspection, and/or demonstration \\\hline
\end{longtable}
